\begin{itemize}
    \item State jurisdiction refers to the power or competence of a state to prescribe or enforce its laws
    \item State jurisdiction includes:
    \begin{itemize}
        \item \textbf{Prescriptive jurisdiction}, which is the power to enact laws/assert jurisdiction by legislation (legislative power)
        \item \textbf{Adjudicative jurisdiction}, which is the power to apply laws and decide disputes (judicial power)
        \item \textbf{Enforcement jurisdiction}, which is the ability to enforce laws within a state's territory (executive power)
    \end{itemize}
    \item Enforcement jurisdiction is almost always confined to a state's territory, but prescriptive jurisdiction is becoming increasingly extraterritorial
    \item The litmus test for the limits of state jurisdiction is \textbf{protest by other states}
    \item The rules around custom in relation to jurisdiction have been formed by states asserting jurisdiction, and by some states resisting/opposing jurisdiction
\end{itemize}

\section{Civil Jurisdiction}
\begin{itemize}
    \item Civil jurisdiction refers to the exercise by states of power (prescriptive and adjudicative) over persons, matters or things in private disputes (e.g., torts, contracts, family law, etc.)
    \item International law places very few limits on the assertion of civil jurisdiction, taking a hands-off approach to the power of states
    \item This has resulted in many situations where states have been extravagant in exercising civil powers internationally
    \item For example, the \statute{\textit{Alien Tort Claims Act 1789} (US)} conferred jurisdiction on US federal courts in civil actions by non-nationals for violations of international law
    \begin{itemize}
        \item Following a revitalisation of this law in the 1980s, it has now been interpreted restrictively by the US Supreme Court (see \case{\textit{Sosa v Alvare-Machain} (SCOTUS, 2004)} and \case{\textit{Kiobel v Royal Dutch Petroleum} (SCOTUS, 2013)})
        \item In \case{\textit{Kiobel v Royal Dutch Petroleum}}, Nigerian plaintiffs were injured by Royal Dutch Petroleum
        \item In \case{\textit{Sosa v Alvare-Machain}}, a Mexican national was kidnapped by US agents in Mexico and brought to the US for trial
    \end{itemize}
\end{itemize}

\section{Criminal Jurisdiction}
\begin{itemize}
    \item International law sets limits on the assertion of of criminal jurisdiction by states
    \item States can only assert criminal jurisdiction when they can establish that their jurisdiction is supported by one or more of the following principles:
    \begin{itemize}
        \item Territorial principle
        \item Nationality principle
        \item Protective principle (also known as the security principle)
        \item Passive personality principle (also known as the passive nationality principle)
        \item Universality principle
    \end{itemize}
    \item These principles are not mutually exclusive, and states can assert jurisdiction on the basis of more than one principle (e.g., a crime on the territory of one state committed against citizens of another state, which becomes so severe that it shocks all humans (a crime against humanity) would invoke the territorial principle, the nationality principle, the passive personality principle and the universality principle)
    \item If there is an overlap in jurisdiction, or there is otherwise concurrent jurisdiction, the state with the most significant connection to the crime will usually be the one to prosecute (generally, the state with custody has the first opportunity to prosecute)
    \begin{itemize}
        \item To exercise jurisdiction, a state must have custody of the offender
    \end{itemize}
\end{itemize}

\subsection{Territorial Principle}
\begin{itemize}
    \item The territorial principle holds that states may exercise criminal jurisdiction where an element of a criminal offence takes place within its territory
    \begin{itemize}
        \item Criminal laws are normally expressed to apply in the territories of a state
    \end{itemize}
    \item This applies where an element of an offence occurs within the jurisdiction of a state, or where an effect of the offence was felt within the state's jurisdiction
    \item The territorial principle requires some sort of geographical nexus, but not necessarily that the entirety of an offence is committed within the territory of the forum asserting jurisdiction
\end{itemize}

\begin{casedetails}{\textit{SS Lotus (France v Turkey)} (1927) PCIJ}
    \flushleft
    This case involved the collision of two vessels on the high seas, one French and one Turkish. The collision had occurred as a result of the criminal negligence of the master of the watch on the French vessel. The Turkish courts asserted criminal jurisdiction over the French master of the watch, and sought to prosecute him in the Turkish courts. France argued that Turkey could not exercise jurisdiction, instead arguing in favour of the customary rule of exclusive jurisdiction of the flag state over collisions on the high seas; i.e., that France, as the flag state of the offending vessel and of the master of the watch, had sole jurisdiction.

    \tcblower
    \flushleft

    The PCIJ held that there was no such rule, and that Turkey was entitled to assert criminal jurisdiction. It did so on what is now a highly-contended basis that ``anything which international law does not expressly prohibit, it permits'' (i.e., a permissive view of state jurisdiction). This view is no longer subscribed to, as the prevailing view now is that states when asserting criminal jurisdiction need to identify  valid basis for doing so (in this instance, the valid basis was held to be the territorial principle).

    \vspace{\baselineskip}

    The PCIJ held that the territorial principle of jurisdiction meant that if at least one of the constituent elements of an offence took place in a state's territory, that is sufficient to enliven the territorial principle. Here, as some effects were felt in Turkey as a result of the damage to the Turkish vessel, the PCIJ held that as the vessel was akin to the floating territory of Turkey (this is not generally correct), the territorial principle was enlivened in manner that held that not all elements of the offence had to be in a state so long as some effect was felt by the state.

    \vspace{\baselineskip}

    ``The territoriality of all criminal law ... is not an absolute principle." States may exercise jurisdiction on a territorial basis if `if one of the constituent elements of the offence, and more especially its effects, have taken place there'. `[O]nce it is admitted that the effects of the offence were produced on the Turkish vessel, it becomes impossible to hold that there is a rule of international law which prohibits Turkey from prosecuting...' `... in this case, a prosecution may also be justified from the point of view of the so-called territorial principle' (this statement embodies the now-discredited `floating territory' principle). Moreover, `there is no rule of international law in regard to collision cases to the effect that criminal proceedings are exclusively within the jurisdiction of the State whose flag is flown.'

    \vspace{\baselineskip}

    The rules created by this case around jurisdiction of vessels on the high seas has now been overturned by \convention{\textit{UNCLOS} Art 97} (the exclusive flag state criminal jurisdiction for high seas collisions), which holds that if there is a collision in the high seas between two different nationalities, the victim vessel and thus state has no jurisdiction over the perpetrator vessel (which is exclusively under the jurisdiction of the flag vessel).
\end{casedetails}

\begin{itemize}
    \item The territory of a state is a nuanced issue
    \begin{itemize}
        \item It is clear that vessels bearing a state's flag are not considered to be part of the territory of that state
        \item Following \case{\textit{R v Turnbull; ex parte Petroff} (ACTSC, 1971)}, the grounds of an embassy remain part of the ACT's territory for the purposes of jurisdiction (but enforcement of that jurisdiction is a separate issue, due to diplomatic inviolability)
        \item Following \case{\textit{R v Disun; R v Nardin} (WASC, 2003)}, the territory of a state includes its territorial seas
    \end{itemize}
\end{itemize}

\begin{casedetails}{\textit{R v Turnbull; ex parte Petroff} (ACTSC, 1971)}
    \flushleft
    Petroff and another defendant were charged under ACT law for attempted bombing of the USSR embassy in Canberra. The defendant argued that he did not commit a crime within the ACT (and thus within Australia), but rather did so within the USSR. The ACT Supreme Court held that ``the premises of a foreign embassy are not outside the territory to which the criminal law applies'', and as such, the criminal law of the ACT applied throughout the ACT; the missions of foreign states do not become foreign territory, despite being subject to inviolable protections.
\end{casedetails}

\begin{casedetails}{\textit{R v Disun; R v Nardin} (WASC, 2003)}
    \flushleft
    The MV Tampa, a Norwegian vessel, had rescued 400 asylum seekers from a sinking vessel in the Indian Ocean. They sought to enter the Australia territorial sea to offload the asylum seekers so that they could be offloaded, looked after and then make claims for protection. The defendants were subsequently charged with people-smuggling offences, as they had not been given the requisite permission but nonetheless still steered the ship into Australia's territorial waters.

    \vspace{\baselineskip}

    In their defence, the defendants argued that they were arrested on a Norwegian vessel within the Australian territorial sea, and that this meant that they had been arrested on Norwegian territory. Consequently, they argued that they should be governed by extradition laws before they could be prosecuted. The Western Australian Supreme Court rejected this, holding that the territory of Australia included all of Australia's land territory and also its territorial sea. It is moreover a general rule of international law that a state possesses jurisdiction in virtue of its sovereignty over persons and property found in its territory.
\end{casedetails}

\begin{itemize}
    \item If a criminal offence has connections with several states, the territorial principle will be enlivened in those states, with two approaches taken to resolve this conflict:
    \begin{itemize}
        \item \textbf{Subjective territorial jurisdiction} - this is the exercise of prescriptive jurisdiction by the state in which the criminal offence originated but was completed outside its territory (i.e., look at the start place of the offence)
        \item \textbf{Objective territorial jurisdiction} - this is the exercise of prescriptive jurisdiction by the state in which the criminal offence is completed, even if the offence was initiated outside its territory (i.e., look at the end place of the offence)
    \end{itemize}
    \item Following \case{\textit{Ward v R} (1980) 142 CLR 308}, the High Court held that Victoria tended to follow objective territorial jurisdiction
\end{itemize}

\begin{casedetails}{\textit{Ward v R} (1980) 142 CLR 308}
    \flushleft
    This was a case of murder, where the accused (Ward) was standing on of of the bank of the Murray River in Victoria, where he fired a gun and killed a victim who was standing by the water of the river. The High Court faced the factual issue of deciding whether the victim was in NSW or in Victoria at the time he was struck; upon closer examination, the High Court held that the victim was in NSW at the point he was killed. The High Court examined the relevant Victorian statute, and found that it adhered to the objective/terminatory theory, and as such, the venue of the crime was where the act took effect upon the victim (which was NSW in this case).
\end{casedetails}

\begin{itemize}
    \item This is a principle that has been codified in NSW through \statute{\textit{Crimes Act 1900} ss 10A, 10C}, where the geographic nexus of a crime is within NSW if the offence was committed wholly or partly within NSW, or has an effect in NSW (i.e., was committed wholly outside NSW, but has a material effect in NSW)
    \begin{itemize}
        \item s 10A holds that it will also apply to an offence beyond the territorial limits of NSW if there is a geographic nexus between the offence and the state
        \item s 10C delineates how to determine if there is a geographic nexus
    \end{itemize}
\end{itemize}

\begin{statutedetails}{\textit{Crimes Act 1900} ss 10A, 10C}
    \flushleft
    \textbf{Application and effect of Part}
    \begin{enumerate}[label=(\arabic*)]
        \item This Part applies to all offences.
        \item This Part extends, beyond the territorial limits of the State, the application of a law of the State that creates an offence if there is the nexus required by this Part between the State and the offence.
        \item If the law that creates an offence makes provision with respect to any geographical consideration concerning the offence, that provision prevails over any inconsistent provision of this Part.
        \item This Part is in addition to and does not derogate from any other basis on which the courts of the State may exercise criminal jurisdiction.
    \end{enumerate}
\end{statutedetails}

\begin{statutedetails}{\textit{Crimes Act 1900} s 10C}
    \flushleft
    \textbf{Extension of offences if there is a geographical nexus}
    \begin{enumerate}[label=(\arabic*)]
        \item If—
        \begin{enumerate}[label=(\alph*)]
            \item all elements necessary to constitute an offence against a law of the State exist (disregarding geographical considerations), and
            \item a geographical nexus exists between the State and the offence,
        \end{enumerate}
        the person alleged to have committed the offence is guilty of an offence against that law.
        \item A geographical nexus exists between the State and an offence if—
        \begin{enumerate}[label=(\alph*)]
            \item the offence is committed wholly or partly in the State (whether or not the offence has any effect in the State), or
            \item the offence is committed wholly outside the State, but the offence has an effect in the State.
        \end{enumerate}
    \end{enumerate}
\end{statutedetails}

\subsection{Nationality Principle}
\begin{itemize}
    \item The nationality principle refers to the application of criminal laws to nationals (i.e., citizens) of states
    \item Determining the nationality of a person is a matter left to municipal law; it is up to states to confer or rescind nationality (an individual cannot make a unilateral declaration of renunciation, for example), with international law setting very few limits on this matter
    \item An example of the nationality principle is \statute{\textit{Criminal Code 1995} (Cth) Div. 272 (`Child Sex Offences Outside Australia')}, which was challenged in \case{\textit{XYZ v Commonwealth} (2006) 227 CLR 532}
    \begin{itemize}
        \item In this case, Gleeson CJ held that ``[t]he assertion of extra-territorial criminal jurisdiction is not, in itself, contrary to the principles of international law ... The territorial principle of legislative jurisdiction over crime is not the exclusive source of competence recognised by international law. Of primary relevance to the present case is the nationality principle, which covers conduct abroad by citizens or residents of a state''
        \item In this case, the High Court considered whether the Div. 272 provisions were supported by the external affairs power
        \item Gleeson CJ noted that there was no problem with the principle of extraterritorial jurisdiction; the relevant principle was the nationality principle, which applies to both citizens and residents of states (suggesting a broader application of this principle, rather than just states)
    \end{itemize} 
\end{itemize}

\subsection{Protective (Security) Principle}
\begin{itemize}
    \item The protective (security) principle is concerned less with where an offence is committed (territorial) and who committed the offence (nationality), but rather with the nature of the offence, and whether it impairs some kind of fundamental/vital state security interest
    \item Jurisdiction is exercised over persons (including non-nationals) who have committed acts abroad prejudicial to the security of the state
    \item States can take action against nationals/non-nationals where the offence was serious enough to violate state security
    \item This is a controversial basis of state jurisdiction, and is generally limited to offences infringing vital state interests, and as such, is not invoked very often
    \item Examples include:
    \begin{itemize}
        \item Genocide of the Jewish people - \case{\textit{A-G v Eichmann} (Dist Ct. of Jerusalem, 1961)}
        \item Attempted murder of government agents - \case{\textit{US v Benitez} (US Ct. of A for 11\textsuperscript{th} Cir, 1984)}
        \item Propaganda - \case{\textit{Joyce v DPP} (HoL, 1946)}; \case{\textit{R v Casement} (1917) Eng Ct. of Crim. A.}
    \end{itemize}
\end{itemize}

\begin{casedetails}{\textit{A-G v Eichmann} (Dist Ct. of Jerusalem, 1961)}\label{case:Eichmann (protective principle)}
    \flushleft
    Eichmann was abducted by the Israeli secret service from Buenos Aires, where he had been living under a different name. He was taken to Israel to face trial for carrying out the so-called Holocaust final solution in Nazi Germany. He faced trial in Israel, and was ultimately convicted. The Israeli prosecutors presented multiple bases upon which they asserted that Israel possessed jurisdiction:
    \begin{itemize}
        \item The character of the offence as involving genocide of the Jewish people (enlivening the universal principle of jurisdiction)
        \item Invoking jurisdiction to protect the very existence of the Jewish state of Israel (enlivening the protective principle)
    \end{itemize}
\end{casedetails}

\begin{casedetails}{\textit{US v Benitez} (US Ct. of A for 11\textsuperscript{th} Cir, 1984)}
    \flushleft
    This case involved the attempted murder of two government agents, with the US authorities prosecuting a Colombian national. The Court justified this by saying that the crime would have potential ramifications for the broader security of the United States.
\end{casedetails}

\begin{casedetails}{\textit{Joyce v DPP} (HoL, 1946)}
    \flushleft
    The defendant was born in the US, and had fraudulently obtained a UK passport by claiming he was born in Ireland. In 1939, when WW2 began, he moved to Germany and began to work for German radio services, broadcasting propaganda to the British people. He was tried for treason by the British authorities, which was controversial as it was unclear as to whether he had British nationality to begin with (he had acquired his British passport fraudulently). Jallett LJ proposed an alternate basis for jurisdiction, stating that ``no principle of comity demands that a state should ignore the crime of treason committed against it outside its own territory...'' This case is often cited in support of the protective principle.
\end{casedetails}

\begin{casedetails}{\textit{R v Casement} (1917) Eng Ct. of Crim. A.}
    \flushleft
    This case involved a British subject who in Germany was trying to persuade British prisoners of war to give up their allegiance to Britain and join the German military. He was put on trial in the UK, and was convicted of treason. This case is often cited in support of the protective principle, and for its enablement of the prosecution of nationals and non-nationals where there has been some infringement of a vital state security interest
\end{casedetails}

\subsection{Passive Personality Principle}
\begin{itemize}
    \item The passive personality principle allows for criminal jurisdiction to be asserted anywhere in the world to citizens of a state, enabling the national state of the victim of an offence to assert jurisdiction over the offender
    \item The \case{\textit{SS Lotus Case} (1927) PCIJ} expressly reserved the question as to whether this was a valid basis of jurisdiction
    \item Examples of this principle being exercised include
    \begin{itemize}
        \item Taking hostages during the hijacking of an aircraft - \case{\textit{US v Yunis} (US Ct. of A for 11\textsuperscript{th} Cir, 1991)}
        \item Harming Australians abroad - \statute{\textit{Criminal Code 1995} (Cth) Div 115}
    \end{itemize}
\end{itemize}

\begin{casedetails}{\textit{US v Yunis} (US Ct. of A for 11\textsuperscript{th} Cir, 1991)}
    \flushleft
    This case involved the defendant and four other men who boarded a Royal Jordanian flight in Beirut and then hijacked the flight. A number of passengers were American citizens, and two of them had been detained (but otherwise unharmed). On the basis of the passive personality principle, the US Court held that it had jurisdiction of this matter.
\end{casedetails}

\begin{itemize}
    \item Under \statute{\textit{Criminal Code 1995} (Cth) Div 115}, Australian nationals are protected by Australian criminal law as a national of Australia in the event that they are subject to harm abroad
\end{itemize}

\begin{statutedetails}{\textit{Criminal Code 1995} (Cth) Div 115 - Harming Australians}
    \flushleft
    \tcbsubtitle{115.1  Murder of an Australian citizen or a resident of Australia}
    \begin{enumerate}[label=(\arabic*)]
        \item A person commits an offence if:
        \begin{enumerate}[label=(\alph*)]
            \item the person engages in conduct outside Australia (whether before or after 1 October 2002 or the commencement of this Code); and
            \item the conduct causes the death of another person; and
            \item the other person is an Australian citizen or a resident of Australia; and
            \item the first-mentioned person intends to cause, or is reckless as to causing, the death of the Australian citizen or resident of Australia or any other person by the conduct; and
            \item if the conduct was engaged in before 1 October 2002—at the time the conduct was engaged in, the conduct constituted an offence against a law of the foreign country, or the part of the foreign country, in which the conduct was engaged.
        \end{enumerate}
        \par \textit{Note: This section commenced on 1 October 2002.}
        \item[(1A)] If the conduct constituting an offence against subsection (1) was engaged in before 1 October 2002, the offence is punishable on conviction by:
        \begin{enumerate}[label=(\alph*)]
            \item if, at the time the conduct was engaged in, the offence mentioned in paragraph (1)(e) was punishable on conviction by a term of imprisonment (other than imprisonment for life)—a maximum penalty of imprisonment for a term of not more than that term; or
            \item otherwise—a maximum penalty of imprisonment for life.
        \end{enumerate}
        \item[(1B)] If the conduct constituting an offence against subsection (1) was engaged in on or after 1 October 2002, the offence is punishable on conviction by a maximum penalty of imprisonment for life.
        \item Absolute liability applies to paragraphs (1)(c) and (e).
        \item If:
        \begin{enumerate}[label=(\alph*)]
            \item a person has been convicted or acquitted of an offence in respect of conduct under a law of a foreign country or a part of a foreign country; and
            \item the person engaged in the conduct before 1 October 2002;
        \end{enumerate}
        the person cannot be convicted of an offence against this section in respect of that conduct.
    \end{enumerate}

    \tcbsubtitle{115.2  Manslaughter of an Australian citizen or a resident of Australia}
    \begin{enumerate}[label=(\arabic*)]
        \item A person commits an offence if:
        \begin{enumerate}[label=(\alph*)]
            \item the person engages in conduct outside Australia (whether before or after 1 October 2002 or the commencement of this Code); and
            \item the conduct causes the death of another person; and
            \item the other person is an Australian citizen or a resident of Australia; and
            \item the first-mentioned person intends that the conduct will cause serious harm, or is reckless as to a risk that the conduct will cause serious harm, to the Australian citizen or resident of Australia or any other person; and
            \item if the conduct was engaged in before 1 October 2002—at the time the conduct was engaged in, the conduct constituted an offence against a law of the foreign country, or the part of the foreign country, in which the conduct was engaged.
        \end{enumerate}
        \par \textit{Note: This section commenced on 1 October 2002.}
        \item[(1A)] If the conduct constituting an offence against subsection (1) was engaged in before 1 October 2002, the offence is punishable on conviction by:
        \begin{enumerate}[label=(\alph*)]
            \item if, at the time the conduct was engaged in, the offence mentioned in paragraph (1)(e) was punishable on conviction by imprisonment for a term of less than 25 years—a maximum penalty of imprisonment for a term of not more than that term; or
            \item otherwise—a maximum penalty of imprisonment for a term of not more than 25 years.
        \end{enumerate}
        \item[(1B)] If the conduct constituting an offence against subsection (1) was engaged in on or after 1 October 2002, the offence is punishable on conviction by a maximum penalty of imprisonment for a term of not more than 25 years.
        \item Absolute liability applies to paragraphs (1)(b), (c), and (e).
        \item If:
        \begin{enumerate}[label=(\alph*)]
            \item a person has been convicted or acquitted of an offence in respect of conduct under a law of a foreign country or a part of a foreign country; and
            \item the person engaged in the conduct before 1 October 2002;
        \end{enumerate}
        the person cannot be convicted of an offence against this section in respect of that conduct.
    \end{enumerate}

    \tcbsubtitle{\flushleft 115.3  Intentionally causing serious harm to an Australian citizen or a resident of Australia}
    \begin{enumerate}[label=(\arabic*)]
        \item A person commits an offence if:
        \begin{enumerate}[label=(\alph*)]
            \item the person engages in conduct outside Australia; and
            \item the conduct causes serious harm to another person; and
            \item the other person is an Australian citizen or a resident of Australia; and
            \item the first-mentioned person intends to cause serious harm to the Australian citizen or resident of Australia or any other person by the conduct.
        \end{enumerate}
        \par \textbf{Penalty:} Imprisonment for 20 years.
        \item Absolute liability applies to paragraph (1)(c).
    \end{enumerate}

    \tcbsubtitle{115.4  Recklessly causing serious harm to an Australian citizen or a resident of Australia}
    \begin{enumerate}[label=(\arabic*)]
        \item A person commits an offence if:
        \begin{enumerate}[label=(\alph*)]
            \item the person engages in conduct outside Australia; and
            \item the conduct causes serious harm to another person; and
            \item the other person is an Australian citizen or a resident of Australia; and
            \item the first-mentioned person is reckless as to causing serious harm to the Australian citizen or resident of Australia or any other person by the conduct.
        \end{enumerate}
        \par \textbf{Penalty:} Imprisonment for 15 years.
        \item Absolute liability applies to paragraph (1)(c).
    \end{enumerate}

    \tcbsubtitle{115.5  Saving of other laws}
    This Division is not intended to exclude or limit the operation of any other law of the Commonwealth or of a State or Territory.

    \tcbsubtitle{115.6  Bringing proceedings under this Division}
    \begin{enumerate}[label=(\arabic*)]
        \item Proceedings for an offence under this Division must not be commenced without the Attorney-General's written consent.
        \item However, a person may be arrested, charged, remanded in custody, or released on bail, in connection with an offence under this Division before the necessary consent has been given.
    \end{enumerate}

    \tcbsubtitle{115.7  Ministerial certificates relating to proceedings}
    \begin{enumerate}[label=(\arabic*)]
        \item A Minister who administers one or more of the following Acts:
        \begin{enumerate}[label=(\alph*)]
            \item the Australian Citizenship Act 2007;
            \item the Migration Act 1958;
            \item the Australian Passports Act 2005;
        \end{enumerate}
        may issue a certificate stating that a person is or was an Australian citizen or a resident of Australia at a particular time.
        \item In any proceedings, a certificate under this section is \textit{prima facie} evidence of the matters in the certificate.
    \end{enumerate}

    \tcbsubtitle{115.8  Geographical Jurisdiction}
    Each offence against this Division applies:
    \begin{enumerate}[label=(\alph*)]
        \item whether or not a result of the conduct constituting the alleged offence occurs in Australia; and
        \item if the alleged offence is an ancillary offence and the conduct to which the ancillary offence relates occurs outside Australia—whether or not the conduct constituting the ancillary offence occurs in Australia.
    \end{enumerate}

    \tcbsubtitle{115.9  Meaning of \textit{causes} death or harm}
    In this Division, a person's conduct causes death or harm if it substantially contributes to the death or harm.
\end{statutedetails}

\subsection{Universality Principle}
\begin{itemize}
    \item The universality principle is the exercise of jurisdiction over particular offences because of their seriousness (e.g., genocide), or because they may otherwise go unpunished (e.g., piracy, crimes committed on the high seas, crimes beyond territorial jurisdiction)
    \item There is no need to establish a link or nexus between the offender or offence and the prosecuting state other than custody
    \item The basis of this is that it is the duty of the international community to repress the crimes to which it applies (because it is the type of crimes that can only effectively be dealt with in universal jurisdiction (e.g., piracy), or because they are particularly heinous (e.g., genocide))
    \begin{itemize}
        \item The exercise of universal jurisdiction without custody (i.e., \textit{in absentia}) is controversial; usually, custody is the only necessary link for a state to exercise their powers under the universality principle
    \end{itemize}
    \item It is difficult to discern a coherent theory for those offences to which universal jurisdiction has applied; it has been developed on a case-by-case basis
    \item Crimes over which universal jurisdiction may be exercised include:
    \begin{itemize}
        \item Piracy (attack on a vessel on the high seas for private ends)
        \item Slavery (exercise of powers of ownership over persons)
        \item Genocide (certain acts including murder committed with intent to destroy in whole or in part, a national, ethnical, racial or religious group)
        \item War crimes (serious breaches of the laws of war, such as intentional attacks on civilians)
        \item Crimes against humanity (certain acts including murder committed as part of a widespread or systematic attack directed against any civilian population)
        \item Torture (state infliction of pain/suffering to obtain information)
    \end{itemize}
\end{itemize}

\subsubsection{Piracy and Marine Terrorism}
\begin{itemize}
    \item Piracy was the first offence over which universal jurisdiction was accepted to apply, as it has a functional aspect to it which enables for the law to be enforced against pirates, which is otherwise very difficult as it occurs beyond the jurisdiction of any state
    \item International law has come to view that any state with custody of the offending pirate can exercise jurisdiction over them
    \item Piracy can only be committed on the high seas; if it is conducted in the territorial seas, it would not be piracy and would likely be subject to the national law of where the sea is located
    \item Under \convention{\textit{UNCLOS} Art 101}, piracy is defined in \convention{Art 101(a)}
    \item It requires two vessels (one of which is attacking the other), that such an attack takes place on the high seas, and that it is done for private ends (i.e., government ships cannot commit piracy)
\end{itemize}

\begin{conventiondetails}{\textit{UNCLOS} Article 101}
    \flushleft
    \textit{Definition of Piracy}

    \vspace{\baselineskip}

    Piracy consists of any of the following acts:
    \begin{enumerate}[label=(\alph*)]
        \item any illegal acts of violence or detention, or any act of depredation, committed for private ends by the crew or the passengers of a private ship or a private aircraft, and directed:
        \begin{enumerate}[label=(\roman*)]
            \item on the high seas, against another ship or aircraft, or against persons or property on board such ship or aircraft;
            \item against a ship, aircraft, persons or property in a place outside the jurisdiction of any State;
        \end{enumerate}
        \item any act of voluntary participation in the operation of a ship or of an aircraft with knowledge of facts making it a pirate ship or aircraft;
        \item any act of inciting or of intentionally facilitating an act described in subparagraph (a) or (b).
    \end{enumerate}
\end{conventiondetails}

\begin{casedetails}{\textit{US v Dire} (Court of Appeal for the 4\textsuperscript{th} Circuit, 2012)}
    \flushleft
    This case involved the prosecution of Somali pirates, who attacked a vessel off the coast of Somalia on the high seas. Whilst the pirates thought they were attacking a merchant vessel, it was a US frigate disguised as a merchant ship. The US chased the pirates, arrested them and took them to the US for trial. The jurisdiction of the court was asserted on the basis of universality. Dire argued that there was no true offence of piracy having being committed here, but rather it was at most attempted piracy; there was no act of violence or degradation against the vessel, but rather an attempted act that was thwarted. The US Courts rejected this logic, holding that the \convention{\textit{UNCLOS}} definition of piracy was broad enough to involve attempted acts of piracy. This case held that actual robbery was \textit{not} an essential element of piracy, and that attempted/frustrated robbery on the high seas constitutes piracy \textit{jure gentium}.
\end{casedetails}

\begin{itemize}
    \item In the 1960s-1980s, it was clear that the law of piracy was not being applied to acts of marine terrorism
\end{itemize}

\begin{casedetails}{\textit{Achille Lauro Incident} (1985)}
    \flushleft
    In 1985, four armed men who claimed to represent the Palestinian Liberation Front took control of an Italian cruiseliner on the high seas. However, they had laid on board after boarding in Genova, rather than boarding the ship from another vessel on the high seas, preventing the definition of piracy from being enlivened. Moreover, they killed a Jewish-American prisoner by shooting him dead and pushing him over the side of the vessel, and did this to make a political point, rather than for a commercial advantage or to otherwise loot the ship. From this incident, it was evident that piracy did not apply to incidents that commenced purely on board one vessel.
\end{casedetails}

\begin{itemize}
    \item The \convention{\textit{1988 Convention for the Suppression of Unlawful Acts against the Safety of Maritime Navigation}} remedied the situation in the \case{\textit{Achille Lauro Incident} (1985)}, by establishing international criminal offences relating to the safety of maritime navigation, including the seizing of control of a ship by force
    \item Maritime terrorism and other terrorism are now the subject of a treaty-based system of quasi-universal jurisdiction; this is not true jurisdiction, as jurisdiction may only be exercised over crimes committed on the territory of parties or over nationals of parties to the convention
    \begin{itemize}
        \item A sufficient number of states have ratified these conventions, providing a broad enough basis for states to take action against a whole host of terrorist attacks (e.g., financing terrorists, bombing, aircraft hijacks, etc.)
    \end{itemize}
    \item These conventions require some sort of connection between the state party to the convention and the offender/the place of the offence
    \item ``Unlike those offenses supporting universal jurisdiction under customary international law - that is, piracy, war crimes, and crimes against humanity - that now have fairly precise definitions and that have achieved universal condemnation, `terrorism' is a term as loosely deployed as it is powerfully charged.'' - \case{\textit{US v Yousef} (2003) US ct of Ap, 2nd Circuit (contra \textit{US v Yunis} (1991))}
\end{itemize}

\subsubsection{Genocide}
\begin{itemize}
    \item Genocide refers to the killing or other crimes with the intent to destroy, in part or in whole, a national, ethnical, racial or religious group
    \item Genocide requires an additional fault element (\textit{dolus specialis})
    \begin{itemize}
        \item If there was a special intention to destroy a group that was manifest even through only one murder, that will amount to genocide, making it subject to universal jurisdiction
    \end{itemize}
\end{itemize}

\begin{casedetails}{\textit{A-G v Eichmann} (Dist Ct of Jerusalem, 1961)}
    \flushleft
    See the existing note for this case under the Protective Principle on Page \pageref{case:Eichmann (protective principle)}.

    \vspace{\baselineskip}

    ``These crimes…are grave offences against the law of nations itself … international law, in the absence of an International [Criminal] Court, is in need of the judicial and legislative organs of every country to give effect to its criminal interdictions and bring the criminals to trial. The jurisdiction to try crimes under international law is universal.'' This case was held to have three bases for jurisdiction:
    \begin{itemize}
        \item Passive personality principle
        \item Protective principle
        \item Universality principle (the crimes committed were grave offences against the laws of natures themselves)
    \end{itemize}
    This case also demonstrated that the ICC has limitations, and continues to have salience today for national courts to take actions in cases of genocide.
\end{casedetails}

\begin{casedetails}{\textit{Nulyarimma v Thompson} (1999) FCAFC}
    \flushleft
    See the existing note for this case on Page \pageref{case: Nulyarimanna v Thompson}.

    \vspace{\baselineskip}

    The FCAFC was not prepared to accepted that genocide comprised part of Australian common law, but noted that the court did accept that genocide could be subject to universal jurisdiction and that it was up to each state to determine themselves how they will exercise that jurisdiction. Wilcox J noted that ``[u]niversal jurisdiction conferred by the principles of international law is a component of sovereignty ... and the way in which sovereignty is exercised will depend on each common law country's peculiar constitutional arrangements.''
\end{casedetails}

\subsubsection{War Crimes and Crimes Against Humanity}
\begin{itemize}
    \item The \convention{\textit{1945 Charter of the International Military Tribunal at Nuremberg}}, created after the Second World War, defines crimes against peace (i.e., aggression), war crimes and crimes against humanity
    \item In Australia, Brennan J in \case{\textit{Polyukhovich v Commonwealth} (War Crimes Act Case) (1991) HCA at [33]} held that ``a law which vested in an Australian court a jurisdiction recognized by international law as a universal jurisdiction is a law with respect to Australia's external affairs...international law recognizes a State to have universal jurisdiction to try suspected war criminals''
    \begin{itemize}
        \item This case concerned the constitutionality of legislation passed in Australia that had retrospective effect vested in Australia's criminal jurisdiction with respect to crimes committed in Europe in WW2
        \item Following Brennan J, the Court held that this legislation was valid, and that states have rights to exercise universal jurisdiction over war criminals
    \end{itemize}
    \item Presently, the \statute{\textit{Criminal Code 1995} (Cth)} implements the \statute{\textit{1998 Rome Statute of the International Criminal Court}} into Australian law, which defines war crimes and crimes against humanity in more detail (the \textit{War Crimes Act} is now repealed)
\end{itemize}

\begin{casedetails}{\textit{Arrest Warrant (Democratic Republic of the Congo v Belgium)} (2002) ICJ}
    \flushleft
    This case addresses the immunity of foreign affairs ministers and the concept of universal jurisdiction over war crimes and crimes against humanity. The case arose when Belgium initiated a prosecution against the Foreign Affairs Minister of the Democratic Republic of the Congo, asserting universal jurisdiction \textit{in absentia} despite the lack of any connection between Belgium and the offence; neither the offence nor the perpetrator had links to Belgium, and there were no Belgian victims involved. This represented a bold application of universal jurisdiction, as the accused was not in custody and had no ties to the prosecuting state.

    \vspace{\baselineskip}
    
    The primary issue in the case was the immunity of the Congolese Foreign Affairs Minister, but a subsidiary question emerged: can a state exercise universal jurisdiction over an offence or offender with no connection to that state, particularly in absentia? The ICJ ultimately did not address this question directly, as it ruled that the minister was protected by immunity. However, the case sparked important discussions on universal jurisdiction in absentia through separate opinions by several judges. Judges Higgins, Kooijmans, and Buergenthal opined that ``a State may choose to exercise a universal criminal jurisdiction in absentia [provided] certain safeguards are in place ... [i.e. no violation of immunity, jurisdiction only over `most heinous' crimes, including war crimes and crimes against humanity]''. In contrast, Judge Guillaume expressed a more restrictive view, arguing that customary international law recognises only one true case of universal jurisdiction - piracy. These views, while influential, were not part of the court's official judgment but were expressed in separate opinions, highlighting the ongoing debate over the scope and application of universal jurisdiction in international law.
\end{casedetails}

\begin{casedetails}{\textit{Criminal Complaint against Donald Rumsfeld} (2007) German Prosecutor-General}
    \flushleft
    The case was brought before the German Prosecutor-General, targeting the former U.S. Secretary of Defense for alleged war crimes involving acts of torture during the Iraq War, specifically at Abu Ghraib prison in Iraq and Guantanamo Bay. The complaint accused Rumsfeld of responsibility for these acts, which were described as some of the most severe instances of torture within the U.S. prison system in Iraq. The prosecution was initiated in Germany, despite Rumsfeld neither being present in the country nor expected to be, raising questions about the application of universal jurisdiction over such crimes.

    \vspace{\baselineskip}

    The German Prosecutor-General, however, decided not to proceed with the case and provided a detailed explanation for this decision. Under German law, prosecutors have the discretion to refuse to institute proceedings when the defendants and their acts lack any connection to Germany. The Prosecutor-General determined that the alleged wrongdoing had no link to Germany whatsoever, and thus, despite the technical ability to exercise universal jurisdiction over such war crimes, it was deemed more appropriate for the U.S. justice system to handle the matter. The Prosecutor-General emphasised that a ``legitimizing domestic linkage" was necessary to justify German jurisdiction over crimes committed by foreigners against foreigners outside the country, a linkage that was absent in this case. This case highlights the permissive nature of universal jurisdiction, meaning it allows states to exercise jurisdiction over certain international crimes but does not mandate them to do so, although in some circumstances there may be an obligation to exercise such jurisdiction.
\end{casedetails}

\subsubsection{Torture}
\begin{itemize}
    \item Torture is defined in \convention{\textit{1984 Convention Against Torture} Art 1(1)}
    \item For the purposes of international law, torture refers to state torture (i.e., only at the hands of a state official)
    \item It is widely accepted that torture is subject to universal jurisdiction
\end{itemize}
    
\begin{conventiondetails}{\textit{1984 Convention Against Torture} Article 1}
    \flushleft
    \begin{enumerate}
        \item For the purposes of this Convention, the term "torture" means any act by which severe pain or suffering, whether physical or mental, is intentionally inflicted on a person for such purposes as obtaining from him or a third person information or a confession, punishing him for an act he or a third person has committed or is suspected of having committed, or intimidating or coercing him or a third person, or for any reason based on discrimination of any kind, when such pain or suffering is inflicted by or at the instigation of or with the consent or acquiescence of a public official or other person acting in an official capacity. It does not include pain or suffering arising only from, inherent in or incidental to lawful sanctions.
        \item This article is without prejudice to any international instrument or national legislation which does or may contain provisions of wider application.
    \end{enumerate}
\end{conventiondetails}

\begin{casedetails}{\textit{Pinochet (No 3)} (2000) HoL}
    \flushleft
    Between 1973 and 1990, Pinochet oversaw a regime that violated the rights of people following a coup in 1973 that deposed the democratically-elected government. After leaving office, Pinochet in 1998 travelled to the UK for medical treatment, where he was made the subject of an extradition request by the Spanish government to face criminal charges in Spain. The UK courts had to determine if they had jurisdiction over Pinochet as a former head of state of Chile. In their judgement, the Law Lords noted that torture may attract universal jurisdiction. Lord Brown-Wilkinson stated that the ``The jus cogens nature of the international crime of torture justifies states in taking universal jurisdiction over torture wherever committed. International law provides that offences jus cogens may be punished by any state...''
\end{casedetails}

\begin{itemize}
    \item There have been attempts to invoke state torture against people who are absent from the forum exercising jurisdiction; different approaches are taken as to whether such jurisdiction can be maintained in absentia
\end{itemize}

\begin{casedetails}{\textit{National Commissioner of the South African Police Service v Southern African Human Rights Litigation Centre} (2013) SA Const Court}
    \flushleft
    This case involved an investigation into alleged acts of torture committed in Zimbabwe by officials of the Zimbabwe state. The South African Constitutional Court held that ``it would appear that the predominant international position is that presence of a suspect is required at a more advanced stage of criminal proceedings...''. Moreover, they held that ``the exercise of universal jurisdiction for purposes of the investigation of an international crime committed outside our territory, may occur in the absence of a suspect without offending our Constitution or international law.''
\end{casedetails}

\subsection{Prosecution and Extradition}
\begin{itemize}
    \item A question arises whether there is a duty or prosecute or extradite a defendant relation to these severe crimes (\textit{aut dedere aut judicare}?)
    \item Generally, there is an obligation to do so under several treaties (and potentially at customary international law)
    \item The adoption of a permissive jurisdiction means that if a crime can be identified, a state can assert jurisdiction over the offender
    \item There are some crimes subject to universal jurisdiction, which are subject to a mandatory requirement to exercise jurisdiction
    \begin{itemize}
        \item If a state has custody of such a defendant, they must either prosecute them, or extradite them to a state where they will face prosecution (this is found arguably at customary international law, and is certainly a matter of treaty law in relation to a number of offences)
    \end{itemize}
\end{itemize}

\begin{casedetails}{\textit{Belgium v Senegal} (2012) ICJ}
    \flushleft
    Belgium accused Senegal of violating the \convention{\textit{Convention Against Torture}} by failing to prosecute or extradite the former President of Chad, who was accused of torture and present in Senegal. Senegal, as a party to the Convention, was bound by its obligations, particularly under Article 7, which mandates that states either prosecute or extradite individuals accused of torture to prevent them from escaping the consequences of their actions. Belgium, despite having no direct involvement in the matter, argued that Senegal's inaction contravened the Convention, and the Court agreed, emphasising that Article 7 serves as a critical mechanism to ensure suspects face accountability for their criminal responsibility.

    \vspace{\baselineskip}

    The Court highlighted that Senegal was obligated to submit the case to its competent authorities for prosecution within a reasonable time, or, if an extradition request was made, it could fulfill its obligation by extraditing the accused. However, Senegal failed to take either action in a timely manner, thereby breaching its duties under the Convention. This case underscores the importance of Article 7 as a core provision of the \convention{\textit{Convention Against Torture}}, designed to prevent impunity by ensuring that states actively address allegations of torture through prosecution or extradition.
\end{casedetails}

\section{Jurisdiction of the ICJ}
\begin{itemize}
    \item The \statute{\textit{Rome Statute of the International Criminal Court}} provides for the broad jurisdiction of the ICC, and has been broadly adopted (but not universally adopted)
    \item The ICJ maintains subject matter jurisdiction over genocide, war crimes, crimes against humanity, and aggression (including the planning of, preparation, initiation or execution of an act of aggression)
    \item Personal jurisdiction, governed by \statute{\textit{Rome Statute of the International Criminal Court} Article 12}, delineates which offenders can be subjected to the ICJ's jurisdiction; it has two general criteria:
    \begin{itemize}
        \item The crime is committed in the territory of a party to the Statute, or by a national or a party (i.e., the offender is a national of the state)
        \item The state is a party to the \statute{\textit{Rome Statute of the International Criminal Court}}
    \end{itemize}
\end{itemize}

\begin{statutedetails}{\textit{Rome Statute of the International Criminal Court} Article 12}
    \flushleft
    \textbf{Preconditions to the exercise of jurisdiction}
    \begin{enumerate}
        \item A State which becomes a Party to this Statute thereby accepts the jurisdiction of the Court with respect to the crimes referred to in article 5. 
        \item In the case of article 13, paragraph (a) or (c), the Court may exercise its jurisdiction if one or more of the following States are Parties to this Statute or have accepted the jurisdiction of the Court in accordance with paragraph 3: 
        \begin{enumerate}[label=(\alph*)]
            \item The State on the territory of which the conduct in question occurred or, if the crime was committed on board a vessel or aircraft, the State of registration of that vessel or aircraft; 
            \item The State of which the person accused of the crime is a national.
        \end{enumerate}
        \item If the acceptance of a State which is not a Party to this Statute is required under paragraph 2, that State may, by declaration lodged with the Registrar, accept the exercise of jurisdiction by the Court with respect to the crime in question. The accepting State shall cooperate with the Court without any delay or exception in accordance with Part 9.
    \end{enumerate}
\end{statutedetails}

\begin{itemize}
    \item Complementarity under \statute{\textit{Rome Statute of the International Criminal Court} Art 17} refers to the notion that if a state cannot prosecute a crime, the ICC has jurisdiction to do so
    \item This echoes \case{\textit{A-G v Eichmann} (Dist Ct. of Jerusalem, 1961)}, and supports the notion that whilst states have primary responsibility for enforcing criminal law, the ICC is a fallback; i.e., it complements jurisdiction, rather than replacing it
\end{itemize}

\begin{statutedetails}{\textit{Rome Statute of the International Criminal Court} Article 17}
    \flushleft
    \textbf{Issues of admissibility}
    \begin{enumerate}
        \item Having regard to paragraph 10 of the Preamble and article 1, the Court shall determine that a case is inadmissible where:
        \begin{enumerate}[label=(\alph*)]
            \item The case is being investigated or prosecuted by a State which has jurisdiction over it, unless the State is unwilling or unable genuinely to carry out the investigation or prosecution; 
            \item The case has been investigated by a State which has jurisdiction over it and the State has decided not to prosecute the person concerned, unless the decision resulted from the unwillingness or inability of the State genuinely to prosecute; 
            \item The person concerned has already been tried for conduct which is the subject of the complaint, and a trial by the Court is not permitted under article 20, paragraph 3; 
            \item The case is not of sufficient gravity to justify further action by the Court.
        \end{enumerate}
        \item In order to determine unwillingness in a particular case, the Court shall consider, having regard to the principles of due process recognized by international law, whether one or more of the following exist, as applicable:
        \begin{enumerate}[label=(\alph*)]
            \item The proceedings were or are being undertaken or the national decision was made for the purpose of shielding the person concerned from criminal responsibility for crimes within the jurisdiction of the Court referred to in article 5; 
            \item There has been an unjustified delay in the proceedings which in the circumstances is inconsistent with an intent to bring the person concerned to justice; 
            \item The proceedings were not or are not being conducted independently or impartially, and they were or are being conducted in a manner which, in the circumstances, is inconsistent with an intent to bring the person concerned to justice.
        \end{enumerate}
        \item In order to determine inability in a particular case, the Court shall consider whether, due to a total or substantial collapse or unavailability of its national judicial system, the State is unable to obtain the accused or the necessary evidence and testimony or otherwise unable to carry out its proceedings.
    \end{enumerate}
\end{statutedetails}

\begin{itemize}
    \item Under \statute{\textit{Rome Statute of the International Criminal Court} Art 25}, individuals can be held responsible and liable for the commission of crimes
\end{itemize}

\begin{statutedetails}{\textit{Rome Statute of the International Criminal Court} Article 25}
    \flushleft
    \textbf{Individual Criminal Responsibility}
    \begin{enumerate}
        \item The Court shall have jurisdiction over natural persons pursuant to this Statute.
        \item A person who commits a crime within the jurisdiction of the Court shall be individually responsible and liable for punishment in accordance with this Statute.
        \item In accordance with this Statute, a person shall be criminally responsible and liable for punishment for a crime within the jurisdiction of the Court if that person:
        \begin{enumerate}[label=(\alph*)]
            \item Commits such a crime, whether as an individual, jointly with another or through another person, regardless of whether that other person is criminally responsible; 
            \item Orders, solicits or induces the commission of such a crime which in fact occurs or is attempted;
            \item For the purpose of facilitating the commission of such a crime, aids, abets or otherwise assists in its commission or its attempted commission, including providing the means for its commission; 
            \item In any other way contributes to the commission or attempted commission of such a crime by a group of persons acting with a common purpose. Such contribution shall be intentional and shall either:
            \begin{enumerate}[label=(\roman*)]
                \item Be made with the aim of furthering the criminal activity or criminal purpose of the group, where such activity or purpose involves the commission of a crime within the jurisdiction of the Court; or 
                \item Be made in the knowledge of the intention of the group to commit the crime;
            \end{enumerate}
            \item In respect of the crime of genocide, directly and publicly incites others to commit genocide; 
            \item Attempts to commit such a crime by taking action that commences its execution by means of a substantial step, but the crime does not occur because of circumstances independent of the person's intentions. However, a person who abandons the effort to commit the crime or otherwise prevents the completion of the crime shall not be liable for punishment under this Statute for the attempt to commit that crime if that person completely and voluntarily gave up the criminal purpose.
        \end{enumerate}
        \item[3 \textit{bis}.] In respect of the crime of aggression, the provisions of this article shall apply only to persons in a position effectively to exercise control over or to direct the political or military action of a State.
        \item No provision in this Statute relating to individual criminal responsibility shall affect the responsibility of States under international law.
    \end{enumerate}
\end{statutedetails}

\begin{itemize}
    \item Under \statute{\textit{Rome Statute of the International Criminal Court} Art 27}, the official capacity of a person shall not be a defence to the commission of a crime under the jurisdiction of the ICC (i.e., a cleaner and a head of state are equally liable for the commission of a crime)
    \item If the ICC has assumed jurisdiction over an individual who has committed a crime in a territory party to the \statute{\textit{Rome Statute}}, or of a national of a state party to the \statute{\textit{Rome Statute}}, immunity cannot be raised; individuals will be treated as the same, and cannot invoke their official status to escape individual criminal responsibility
\end{itemize}

\begin{statutedetails}{\textit{Rome Statute of the International Criminal Court} Article 27}
    \flushleft
    \textbf{Irrelevance of official capacity}
    \begin{enumerate}
        \item This Statute shall apply equally to all persons without any distinction based on official capacity. In particular, official capacity as a Head of State or Government, a member of a Government or parliament, an elected representative or a government official shall in no case exempt a person from criminal responsibility under this Statute, nor shall it, in and of itself, constitute a ground for reduction of sentence. 
        \item Immunities or special procedural rules which may attach to the official capacity of a person, whether under national or international law, shall not bar the Court from exercising its jurisdiction over such a person.
    \end{enumerate}
\end{statutedetails}

\subsection{Relevance of Illegally Obtained Custody}
\begin{itemize}
    \item Sometimes, an individual may have been brought into custody in breach of the law of one or more jurisdictions
    \item There is a divergence of views as to whether illegally obtained custody renders the exercise of criminal jurisdiction by a domestic or international criminal court unlawful; there is moreover no consensus on this point, as various courts have taken different approaches
\end{itemize}

\subsubsection{Domestic Courts}
\begin{itemize}
    \item \case{\textit{A-G v Eichmann (Dist Ct of Jerusalem, 1961)}}
    \begin{itemize}
        \item This case is a common law authority for the proposition that the illegality of arrest irrelevant, and in any event Argentina waived all claims to assert jurisdiction over Eichmann
        \item Whilst this ordinarily would have comprised a breach of Argentine sovereignty, it was not an impediment to the Court exercising jurisdiction, as Argentina had later waived any claim concerning the violation of its sovereignty
    \end{itemize}
    \item \case{\textit{State v Ebrahim (SupCTSA, 1992)}}
    \begin{itemize}
        \item In this case, a member of the ANC was abducted by South Africa from Swaziland
        \item The SA court was held to have no jurisdiction over this member, serving to protect the rights of the individual and also to respect state sovereignty
    \end{itemize}
    \item \case{\textit{Moti v R (2011, HCA)}}
    \begin{itemize}
        \item In this case, the HCA permanently stayed the prosecution of the former Attorney-General of the Solomon Islands, who had been removed unlawfully from Solomon Islands to Australia to face charges for child sex offences
        \item The former Attorney-General successfully raised illegality obtained custody
        \item He was now an Australian citizen, but had conducted sex offences in Vanuatu
        \item He had been removed to Australia in 2007, but this removal was unlawful under Solomon Islands law
        \item The HCA held that after stating that the end of criminal prosecution does not justify securing the presence of the accused, his prosecution had to be permanently stayed
    \end{itemize}
\end{itemize}

\subsubsection{International Criminal Courts}
\begin{itemize}
    \item `Universally condemned offences are a matter of concern to the international community as a whole. ... There is a legitimate expectation that those accused of these crimes will be brought to justice swiftly. Accountability for these crimes is a necessary condition for the achievement of international justice, which plays a critical role in the reconciliation and rebuilding based on the rule of law of countries and societies torn apart by international and internecine conflicts ... This legitimate expectation needs to be weighed against the principle of State sovereignty and the fundamental human rights of the accused.' - \case{\textit{Prosecutor v Nikolić} (CITY Appeals Chamber, 2003)}
    \begin{itemize}
        \item This case involved unlawful prosecution in Bosnia, as the ICC had custody of an offender that was acquired unlawfully
        \item The defendant challenged the ICC by claiming that his unlawful detention made the proceedings unlawful
        \item This case holds that for more serious charges, the more likely that any irregularities in jurisdiction (in terms of custody of the offender) would be overlooked, in pursuit of justice
    \end{itemize}
\end{itemize}