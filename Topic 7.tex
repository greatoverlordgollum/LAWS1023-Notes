\begin{itemize}
    \item State jurisdiction refers to the power or competence of a state to prescribe or enforce its laws
    \item State jurisdiction includes:
    \begin{itemize}
        \item \textbf{Prescriptive jurisdiction}, which is the power to enact laws/assert jurisdiction by legislation (legislative power)
        \item \textbf{Adjudicative jurisdiction}, which is the power to apply laws and decide disputes (judicial power)
        \item \textbf{Enforcement jurisdiction}, which is the ability to enforce laws within a state's territory (executive power)
    \end{itemize}
    \item Enforcement jurisdiction is almost always confined to a state's territory, but prescriptive jurisdiction is becoming increasingly extraterritorial
    \item The litmus test for the limits of state jurisdiction is \textbf{protest by other states}
    \item The rules around custom in relation to jurisdiction have been formed by states asserting jurisdiction, and by some states resisting/opposing jurisdiction
\end{itemize}

\section{Civil Jurisdiction}
\begin{itemize}
    \item Civil jurisdiction refers to the exercise by states of power (prescriptive and adjudicative) over persons, matters or things in private disputes (e.g., torts, contracts, family law, etc.)
    \item International law places very few limits on the assertion of civil jurisdiction, taking a hands-off approach to the power of states
    \item This has resulted in many situations where states have been extravagant in exercising civil powers internationally
    \item For example, the \statute{\textit{Alien Tort Claims Act 1789} (US)} conferred jurisdiction on US federal courts in civil actions by non-nationals for violations of international law
    \begin{itemize}
        \item Following a revitalisation of this law in the 1980s, it has now been interpreted restrictively by the US Supreme Court (see \case{\textit{Sosa v Alvare-Machain} (SCOTUS, 2004)} and \case{\textit{Kiobel v Royal Dutch Petroleum} (SCOTUS, 2013)})
        \item In \case{\textit{Kiobel v Royal Dutch Petroleum}}, Nigerian plaintiffs were injured by Royal Dutch Petroleum
        \item In \case{\textit{Sosa v Alvare-Machain}}, a Mexican national was kidnapped by US agents in Mexico and brought to the US for trial
    \end{itemize}
\end{itemize}

\section{Criminal Jurisdiction}

\subsection{Territorial Principle}

\subsection{Nationality Principle}

\subsection{Protective Principle}

\subsection{Passive Personality Principle}

\subsection{Universality Principle}

\section{Jurisdiction of the ICJ}

\subsection{Relevance of Illegally Obtained Custody}