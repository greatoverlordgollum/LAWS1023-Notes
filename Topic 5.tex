\section{International Legal Personality}
\begin{itemize}
    \item International legal personality refers to the capacity of an actor to exercise rights, duties and powers on the international plane
    \item It is not an absolute concept, and the capacity of international actors varies, including aspects such as:
    \begin{itemize}
        \item To be subject to some or all international obligations
        \item To have the power to enter into treaties
        \item To enjoy some or all immunities from jurisdictions of national courts
        \item To make claims before international tribunals and committees
    \end{itemize}
    \item Personality is not an absolute concept, but is akin to a spectrum (the capacity of intentional actors varies; e.g., the UN has a different capacity than a state)
\end{itemize}

\subsection{States as International Legal Persons}
\begin{itemize}
    \item States are the central/principal subjects of international law
    \item Attributes of statehood include:
    \begin{itemize}
        \item Plenary powers in the international sphere (external sovereignty)
        \item Exclusive competence within their own territory (internal sovereignty)
        \item Not subject to compulsory jurisdiction of international courts without consent/permission (this flows from the independent nature of states and the consensual nature of international law)
        \item Enjoyment of equality (formally, but not necessarily substantively; formal equality is to be distinguished from relative equality in terms of size, economy, powers and global standing)
    \end{itemize}
    \item States maintain complete external sovereignty on the international sphere
\end{itemize}

\subsection{Other International Legal Persons}
\begin{itemize}
    \item International organisations
    \begin{itemize}
        \item These are created by treaties agreed to by states, and many international organisations are conferred with a degree of personality and autonomy
        \item International organisations do not possess general competence, but are instead governed by the principle of `speciality' (i.e., they can only act within the limits of their powers as defined by their constitutive treaties) (\case{\textit{WHO Advisory Opinion} [1996] ICJ Rep 66})
        \item International organisations may enter into treaties
        \item International organisations may be responsible for wrongful acts, following \article{\textit{ILC Draft Articles on the Responsibility of International Organisations} (2011)}
    \end{itemize}
    \item Non-territorialised persons
    \begin{itemize}
        \item Examples include the Holy See (the government of the Vatican, a \textit{sui generis} entity)
        \begin{itemize}
            \item It is treated as having international legal personality since the Middle Ages, independently of having any territorial existence
            \item The Vatican also possesses a tiny territorial possession in Rome (the Vatican City); as a territory, it is regarded in international law as being a microstate (the Holy See and the Pope have a separate legal personality)
        \end{itemize}
    \end{itemize}
\end{itemize}

\begin{casedetails}{\textit{WHO Advisory Opinion} [1996] ICJ Rep 66}
    \flushleft
    This Advisory Opinion answered two similar advisory proceedings that examined the legality of the use of nuclear weapons. One was performed by the UN's request to the ICJ to examine this issue, with the ICJ holding that the general assembly did have capacity to request an advisory opinion. The other was requested by the WHO, but the ICJ rejected this quest.

    \vspace{\baselineskip}

    The ICJ found that it could issue an advisory opinion only where the agency requesting it is:
    \begin{itemize}
        \item Duly authorised to request one (the WHO met this requirement)
        \item The question has to be of a legal character (the WHO's request met this requirement)
        \item The question must be within the scope of the agency's activities (the WHO's request did not meet this requirement, as the WHO was not competent to issue an advisory opinion on the legality of nuclear weapons)
    \end{itemize}
    The WHO had capacity to consider and respond to the health requests of nuclear weapons/hazardous activity, but their question was not on the health effects, but rather was on general questions of whether nuclear weapons could be lawfully used.

    \vspace{\baselineskip}

    The ICJ held that when looking at an international organisation, it cannot be assumed that they have general competence; they are governed by the principle of speciality, and thus the powers that they possess are in the treaty that created them.
\end{casedetails}

\begin{casedetails}{\textit{Reparations for Injuries Case} [1949] ICJ Rep 174}
    \flushleft
    This decision of the ICJ deals generally with the character of international personality, and was a rather narrow case. The key question in this case was whether the UN could pursue a claim against Israel in relation to negligent failure to protect a UN official from being killed by a terrorist group in Jerusalem?

    \vspace{\baselineskip}

    The ICJ held that the UN could seek compensation, as for this purpose the UN was an international legal person. The Court looked at the nature of international legal personality, explaining that the notion of personality and specifically the personality of the UN and other international organisations arises from the requirements/nature of international life. In order for the UN to perform the job that it was entrusted to by the \statute{\textit{UN Charter}}, it is necessary to enjoy and exercise certain functions and rights that set it apart as an international legal person. The Court does hold that the UN's international personality should not be exaggerated; the fact that it has international personality is not the same as saying that the UN is identical to a state or that its legal personality and rights and duties are the same as those of a state (in fact, it is a different degree of personality to that possessed by a state).

    \vspace{\baselineskip}

    ``[T]he development of international law has been influenced by the requirements of international life ... the [UN is] exercising and enjoying, functions and rights which can only be explained on the basis of the possession of a large measure of international personality…Accordingly, the Court has come to the conclusion that the [UN] is an international person. That is not the same thing as saying that it is State, which it certainly is not, or that its legal personality and rights and duties are the same as those of a State.''

    \vspace{\baselineskip}

    If the UN is to discharge its functions, `the attribution of international personality is indispensable'. The Court emphasised the functional indispensability of having personality in order for the UN to be able to perform the functions for which it was established.
\end{casedetails}

\begin{itemize}
    \item Individuals
    \begin{itemize}
        \item Individuals have some degree of international personality, but it is very different from those of states and organisations
        \item Individuals cannot enter into treaties
        \item Individuals generally do not have standing before international courts, except in certain cases (e.g., human rights courts/tribunals)
        \item Some areas of international law have conferred obligations upon individuals:
        \begin{itemize}
            \item International criminal law (`crimes against international law are committed by men, not by abstract entities', following \case{\textit{Nuremberg Trial Judgement} (1947)})
            \begin{itemize}
                \item Prior to this, it was unclear as to whether individuals could be held liable for international crimes
                \item These are the most serious crimes, and have a high threshold (e.g., crimes against humanity, genocide, etc.)
                \item Official capacity is irrelevant; all individuals are equally liable in matters of international criminal law
            \end{itemize}
            \item Individual criminal law is outlined in the \statute{\textit{Rome Statute of the International Criminal Court} (1998) Art 25}, which refers to `individual criminal responsibility'
        \end{itemize}
        \item International human rights law holds that all individuals are bearers of rights conferred by international law
        \item There are multiple covenants on human rights (including civil, political, etc.)
    \end{itemize}
    \item Corporations (?)
    \begin{itemize}
        \item Corporations do not have international legal personality, but may be parties to contracts governed by international law (`internationalised' contracts), following \case{\textit{Texaco Overseas Petroleum Company v Libya} (1977) 53 ILR 389}
        \item As far as international law is concerned, corporations are not subjects of international law, but rather are objects of international law, and are not international people
    \end{itemize}
\end{itemize}

\begin{casedetails}{\textit{Texaco Overseas Petroleum Company v Libya} (1977) 53 ILR 389}
    \flushleft
    This case involved a dispute between Texaco (an oil company) and Libya, and the issue was whether the Libyan government had the right to nationalise the oil industry in Libya. The Libyan government had nationalised the oil industry, including Texaco's oil deposits, and Texaco was seeking compensation for this.

    \vspace{\baselineskip}

    The arbitration was not covered by the law of Libya, but rather incorporated international law; i.e., it was a contract governed by PIL (an `internationalised' contract). Whilst Texaco as a company was not an international person, it was a party to a contract under international law and so could avail the principles under PIL.
\end{casedetails}

\section{Statehood}
\begin{itemize}
    \item Statehood is formally defined by the \convention{\textit{1933 Montevideo Convention on the Rights and Duties of States}} in Article 1
    \item This was accepted as reflecting customary international law
    \item Whilst there were very few signatories to the convention (19 states), it is now accepted as a reflection of customary international law
    \item It is moreover evidence of customary international law; but it is binding only upon states that have signed 
    \item The first three criteria are considered `elastic'
\end{itemize}

\begin{conventiondetails}{\textit{1933 Montevideo Convention on the Rights and Duties of States} Article 1}
    \flushleft
    The state as a person of international law should possess the following qualifications:
    \begin{enumerate}
        \item A permanent population
        \item A defined territory
        \item Government
        \item Capacity to enter into relations with other states
    \end{enumerate}
\end{conventiondetails}

\subsection{The Montevideo Criteria for Statehood}
\subsubsection{Population}
\begin{itemize}
    \item There is no minimum number of people required to constitute a state (e.g., Nauru has less than 10,000 people but is still a state)
    \item The population must be permanent
\end{itemize}

\subsubsection{Territority}
\begin{itemize}
    \item There is no minimum size of territory required to constitute a state (e.g., the Vatican City is a microstate, and Nauru has an area of 21 square kilometres)
    \item The boundaries do not need to be completely fixed/undisputed
    \item There must be a reasonably coherent territory that is effectively governed by the state
\end{itemize}

\subsubsection{Government}
\begin{itemize}
    \item The state must have an organised and effective government (of any form, whether it be a dictatorship, monarchy, democracy, etc.)
    \item It is not the quality or the form of the government that matters, but rather whether it can effectively control the area
\end{itemize}

\subsubsection{Capacity to Enter into Relations with Other States}
\begin{itemize}
    \item To be a state, an entity cannot be subject to the control of another state (i.e., independent from the legal authority of another state)
    \item So long as a state is not placed under the legal authority of another state, it remains an independent state, following \case{\textit{Customs Union Between Germany and Austria} (1931) PCIJ Series A/B No 41} (per Judge Anzilotti)
\end{itemize}

\begin{casedetails}{\textit{Customs Union Between Germany and Austria} (1931) PCIJ Series A/B No 41}
  \flushleft
  The PCIJ was asked to advise whether Austria, by entering into a customs union with Germany, was in violation of two treaties that sought to prevent Austria from being subsumed within the German Reich. The PCIJ held that Austria was not giving up political control, but was giving up immigration freedom and customs control. The PCIJ held that Austria was in breach of the treaties, but it was still a state as it was not under the complete legal authority of another state, and was still politically independent.
\end{casedetails}

\begin{itemize}
    \item The Montevideo criteria are…somewhat misleading and—as regards the capacity to enter into relations with other States—circular ... [T]he central criterion for statehood, which sheds light on the criteria of government and capacity to enter into international relations, is a given territorial community's independence (Crawford, 2015)
    \item It is contentious as to whether other criteria should be imposed on statehood; for example, the \convention{\textit{EC Declaration on New States in Eastern Eruope and Soviet Union} (1991)} proposed:
    \begin{itemize}
        \item Respect for the \statute{\textit{UN Charter}}
        \item Respect for the rights of ethnic and national groups and minorities
        \item Respect for borders
        \item Commitment to peaceful dispute settlement
    \end{itemize}
    \item Whilst these additional criteria are state practice, they aren't sufficient that they have become customary international law, and as such, the Montevideo criteria is still the most widely accepted definition of statehood
\end{itemize}

\section{Recognition of States}
\begin{itemize}
    \item Recognition is the act by which one state acknowledges that an entity is a state
    \item There are two key theories on the legal effect of recognition:
    \begin{itemize}
        \item \textbf{Constitutive theory}: recognition by other states is a precondition to statehood (i.e., a condition precedent)
        \item \textbf{Declaratory theory}: recognition by other states does not create an entity's statehood, but merely acknowledges that the entity is a state (i.e., it is a condition subsequent; it makes life easier, but does not stop anyone)
    \end{itemize}
    \item The declaratory theory is the dominant view in international law, and is the view of the ICJ; however, the practice of recognition may be important evidence for determining the existence of a state
    \item Recognition by a state of an entity as a state will often be accompanied a statement that addresses some of the Montevideo criteria, which tends to be helpful evidence of whether an entity is a state (but isn't enough in and of itself)
    \item If an entity emerges as a result of unlawfulness, the general importance of the principle of territorial integrity applies
    \begin{itemize}
        \item International law contains no prohibition of declarations of independence, but other states should not recognise a unilateral seceding entity before it has acquired statehood, following \case{\textit{Kosovo Advisory Opinion} [2010] ICJ Rep 403}
        \begin{itemize}
            \item It is important to look at the powers of entities to declare independence
        \end{itemize}
    \end{itemize}
    \item International law does contain some scenarios where there is mandatory non-recognition of entities as states
    \begin{itemize}
        \item The \textbf{Stimson Doctrine} holds that recognition of a state may not be extended when a state is created as a result of unlawful use of force/aggression
        \item This arose from the creation of Manchukuo, which was a Japanese puppet state in China from 1932 to 1945; the then US Secretary of State declared that the US would not recognise the new Japanese puppet state that Japan purported to create in China
        \item It is a core rule of international law that puppet states (artificial entities created by breaches of international law) will not be recognised as states
    \end{itemize}
\end{itemize}

\section{Recognition of Governments}
\begin{itemize}
    \item The recognition of governments is a completely separate issue from the recognition of states, and generally arises when there is a change of government within the state
    \item In the modern context, it only arises if there is an unconstitutional change in the government (e.g., a revolution or a military coup)
    \item It is usually the case that a new government is given legal recognition; it is very rare for such a question to arise
    \item Australia used to always recognise a change in government of another state (including by normal means such as an election) with a statement, but ceased this practice in 1988
    \item Presently, Australia has implied recognition; a new government will be recognised by Australia as legitimate, unless if there was a case before an Australian court to decide whether the foreign government's actions were legitimate, it is up to that court to determine whether the government was properly in control of that territory
    \item Recognition of a government in the modern era is moreover discerned from the actual relationship between Australia and the relevant state (continuing the theme of implied recognition)
    \item In the modern day, Australian courts and tribunals will examine several criteria, including the constitutionality of the new government, the control by the government over the territory, dealings on a government-to-government basis, and the extent of international recognition
    \item There are also statutory variations to the common law position
    \begin{itemize}
        \item \statute{\textit{Foreign Corporations (Application of Laws) Act 1989} (Cth)} exemplifies how implied recognition works in terms of foreign corporations
        \begin{itemize}
            \item \statute{s 7(2)} holds that whether a body or person has been validly incorporated in a place outside Australia is to be determined by reference to the law applied in that place
            \item \statute{s 9(1)} holds that the application of this Act is not affected by the recognition or non-recognition, at any time, by Australia
        \end{itemize}
        \item s 7(2) provides clear directives (that the court is to look at the law applicable in the foreign place)
        \item This Act asks if a foreign corporation is a corporation under their home place's law (i.e., it is a domestic law issue)
    \end{itemize}
\end{itemize}

\begin{statutedetails}{\textit{Foreign Corporations (Application of Laws) Act 1989} (Cth)}
    \flushleft
    \tcbsubtitle{Section 7}
    \textbf{Law applied in place of incorporation applicable law in determining questions relating to status of foreign corporation etc.}
    \begin{enumerate}[label=(\arabic*)]
        \item The section applies in relation to the determination of a question arising under Australian law (including a question arising in a proceeding in an Australian court) where it is necessary to determine the question by reference to a system of law other than Australian law.
        \item Any question relating to whether a body or person has been validly incorporated in a place outside Australia is to be determined by reference to the law applied by the people in that place.
        \item Any question relating to:
        \begin{enumerate}[label=(\alph*)]
            \item the status of a foreign corporation (including its identity as a legal entity and its legal capacity and powers); or
            \item the membership of a foreign corporation; or
            \item the shareholders of a foreign corporation having a share capital; or
            \item the officers of a foreign corporation; or
            \item the rights and liabilities of the members or officers of a foreign corporation, or the shareholders of a foreign corporation having a share capital, in relation to the corporation; or
            \item the existence, nature or extent of any other interest in a foreign corporation; or
            \item the internal management and proceedings of a foreign corporation; or
            \item the validity of a foreign corporation's dealings otherwise than with outsiders;
        \end{enumerate}
        \item A matter mentioned in subsection (2) or (3) is not to be taken, by implication, to limit any other matter mentioned in those subsections.
    \end{enumerate}

    \tcbsubtitle{Section 9}
    \textbf{Recognition or non - recognition irrelevant consideration in application of Act etc.}
    \begin{enumerate}[label=(\arabic*)]
        \item It is the intention of the Parliament that the application of this Act is not to be affected by the recognition or non - recognition, at any time, by Australia:
        \begin{enumerate}[label=(\alph*)]
            \item of a foreign state or place; or
            \item of the government of a foreign state or place; or
            \item that a place forms part of a foreign state; or
            \item of the entities created, organised or operating under the law applied by the people in a foreign state or place.           
        \end{enumerate}
        \item Without limiting subsection (1), it is also the intention of the Parliament that the application of this Act is not to be affected by the presence or absence, at any time, of diplomatic relations between Australia and any foreign state or place.
    \end{enumerate}
\end{statutedetails}

\section{Right to Self-Determination}
\begin{itemize}
    \item The right to self-determination refers to the rights of all peoples to determine freely their political status and to pursue their economic, social and cultural development, following \convention{\textit{1966 ICCPR} and \textit{1966 ICECSR} Common Art 1}
    \item It has been held by Nicholson that these are rights accorded to `peoples' rather than states
    \item Self-determination has been defined as ``tje need to pay regard to the freely expressed will of peoples'' at \case{\textit{Western Sahara Advisory Opinion} [1975] ICJ Rep 12 at [59]}
    \item It is moreover accepted that it is a part of customary international law, and may well constitute a \textit{jus cogens}
    \item The availability of the right does not necessarily equate to a specific outcome such as independence and secession (but a new state, free association or integration may all be legitimate expressions of the right); the mere fact of concluding that people have a right to self-determination does not equate to them having a right to independence and to cessate
    \item The law seeks to strike a balance between the right to elf-determination (one of the most central \textit{jus cogens}) norms, and the principle of territorial integrity (which is also a \textit{jus cogens} norm)
    \item In light of the question of `what is a people?', it has been held that:
    \begin{itemize}
        \item `Self-determination is the collective right of `peoples'. Various conditions or characteristics of `peoples' have been put forward, including common historical tradition, racial or ethnic identity, cultural homogeneity, linguistic unity, religious or ideological affinity, territorial connection [particularly important for indigenous peoples], common economic life, and consisting of a certain minimum number. However, no permanent, universally acceptable list of criteria for a `people' exists' (Joseph and Castan, 2013)
    \end{itemize}
    \item The right to self-determination is subject to the principle of \textit{uti possidetis juris} (respect for existing frontiers)
    \item If a new state has emerged from an area granted independence, they (the new entity) will possess the same borders/frontiers as they did prior to independence
\end{itemize}

\begin{casedetails}{\textit{Burkina Faso/Mali} [1986] ICJ Rep 554}
    \flushleft
    This case concerned the border between Burkina Faso and Mali, which was a colonial border. The ICJ held that the principle of \textit{uti possidetis juris} applies to the borders of states that have emerged from colonial rule. The ICJ held that the principle of \textit{uti possidetis juris} is a principle of international law, and is a principle that is recognised by the UN General Assembly. By fixing the boundaries even when a new entity emerged, the principle was to prevent the independence and stability of new states being endangered by struggles challenging frontiers following the withdrawal of colonial power.
\end{casedetails}

\begin{itemize}
    \item In 1992 in \article{Opinion No 3 (1992), Arbitration Commission, EC Conference on Yugoslavia, `Can the internal boundaries between Croatia and Serbia and between Bosnia and Hercegovina and Serbia be regarded as frontiers in terms of public international law?'}, it was held that the principle of \textit{uti possidetis juris} generally can be applied when there is the breakup of a federal state into multiple component parts
\end{itemize}

\begin{casedetails}{\textit{Chagos Islands Advisory Opinion} [2019] ICJ Rep 95}
    \flushleft
    The Chagos Archipelago has been administered by the UK since 1814 as part of the British colony of Mauritius. In 1965, the Chagos was detached from the territory of Mauritius and established as the British Indian Ocean Territory, with islanders displaced to enable the establishment of a US military facility on the largest island, Diego Garcia. Mauritius became independent in 1968 and was admitted to the UN. During the advisory proceedings, the International Court of Justice (ICJ) was tasked with determining whether the process of decolonisation of Mauritius was lawfully completed in 1968 and what international legal consequences stemmed from the continued UK administration of the Chagos. The ICJ found that the right to self-determination crystallised as a customary rule with the adoption of Resolution 1514(XV) in 1960, which has a declaratory character.

    \vspace{\baselineskip}

    The ICJ ruling emphasised that peoples of non-self-governing territories are entitled to exercise their right to self-determination in relation to the territory as a whole, and any detachment by the administering power contrary to the will of the people of the territory is a breach of their right to self-determination. Consequently, non-self-governing territories, such as those under colonial occupation, may achieve self-governance through emergence as a sovereign independent state, free association with an independent state, or integration with an independent state, provided these options reflect the free and genuine will of the people concerned. Due to the unlawful detachment of the Chagos, the process of decolonisation was not lawfully completed when Mauritius became independent in 1968, rendering the continued UK administration of the Chagos a wrongful act that entails state responsibility. The UK must therefore end its administration as rapidly as possible.

    \vspace{\baselineskip}

    Since respect for the right to self-determination is an obligation \textit{erga omnes}, all states have a legal interest in protecting that right, and all UN members must cooperate with the UN to complete the decolonisation of Mauritius. This obligation \textit{erga omnes} means that the responsibility to ensure self-determination is owed to the international community as a whole, and any state has standing to complain if this right is not respected, even if the state does not have a particular or special interest. For example, Australia can raise concerns regarding the Chagos, despite having no direct interest in the matter.
\end{casedetails}

\subsection{External Right to Self-Determination}
\begin{itemize}
    \item The external right to self-determination is the right by which certain peoples can choose to become independent (generally, people cannot do this)
    \item This applies in three scenarios:
    \begin{enumerate}
        \item Peoples under colonial rule (i.e., colonies) and non-self-governing territories
        \item Peoples subject to alien subjugation, domination or exploitation
        \item (Possibly) peoples under oppression and blocked from meaningful self-determination (`remedial secession')
    \end{enumerate}
\end{itemize}

\begin{casedetails}{\textit{Reference Re Secession of Quebec} (1988) 2 SCR 217}
    \flushleft
    This case concerned the secession of Quebec, which is culturally and linguistically diverse from Canada. International law does not specifically grant component parts of sovereign states the legal right to unilateral secession. Ordinarily, the right to self-determination is fulfilled through internal self-determination of a people's political, economic, social and cultural development. Moreover, the right to external self-determination arises only in most extreme cases (colonial domination, alien subjugation, domination or exploitation, possibly also where internal self-determination denied). In this instance, the population of Quebec were granted access to government.
\end{casedetails}

\begin{itemize}
    \item When a people expresses its choice about its status relative to the external world, there are three options:
    \begin{enumerate}
        \item Emergence as a sovereign independent state
        \item Free association with an independent state (e.g., the Cook Islands, which is in free association with New Zealand)
        \item Integration with an independent state
    \end{enumerate}
    \item The right to self-determination is not the same as a power to establish a new state; the new entity must meet the Montevideo criteria for statehood (although these criteria tend to be applied less strictly)
\end{itemize}

\subsection{Indigenous Peoples and the Right to Self-Determination}
\begin{itemize}
    \item This is an example of internal self-determination, which is the pursuit of development within the framework of an existing state (remedial secession may be available if this right is not respected)
    \item Distinctive characteristics of Indigenous peoples include:
    \begin{itemize}
        \item Descent from populations inhabiting territory at the time of conquest
        \item Retaining some of their own social, economic, cultural and political organisations
        \item They self-identify as Indigenous peoples
    \end{itemize}
    \item The \convention{\textit{2007 UN Declaration on the Rights of Indigenous Peoples}} states the following:
    \begin{itemize}
        \item[Article 3] ``\textbf{Indigenous peoples have the right to self-determination.} By virtue of that right they freely determine their political status and freely pursue their economic, social and cultural development.''
        \item[Article 4] ``Indigenous peoples, in exercising the right to self-determination, have the right to autonomy or self-government in matters relating to their internal and local affairs, as well as ways and means for financing their autonomous functions.''
        \item[Article 46(1)] ``Nothing in this Declaration [authorises or encourages] any action which would dismember or impair, totally or in part, the territorial integrity or political unity of sovereign and independent states.''
    \end{itemize}
    \item The \textit{2017 Uluru Statement from the Heart} provides some important insight on this matter
    \begin{itemize}
        \item `We seek constitutional reforms to empower our people and take a rightful place in our own country. When we have power over our destiny our children will flourish.'
        \item `It captures our aspirations for a fair and truthful relationship with the people of Australia and a better future for our children based on justice and self-determination.'
        \item Despite the referendum outcome, Australia's indigenous people still have the inalienable right to self-determination
    \end{itemize}
\end{itemize}