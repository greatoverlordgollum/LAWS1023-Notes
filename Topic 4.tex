\section{Role of International Law in Domestic Law}
\begin{itemize}
    \item Domestic law can be taken as a source of public international law, as evidence of custom and/or the general principles of public international law
    \item International law may recognise institutions of domestic law that have an important/extensive role in international law (e.g., in the case of \case{\textit{Barcelona Traction (Belgium v Spain)} [1970] ICJ Rep 3}, the ICJ recognised that corporations can be recognised within international law)
    \item States cannot invoke absent/inconsistent domestic law as an excuse for failing to meet their obligations under international law
    \begin{itemize}
        \item In \case{\textit{Alabama Claims Arbitration (US/Britain)} (1872)}, it was held that Britain could not ``justify itself for a failure in due diligence on the plea of insufficiency of the legal means of action which it possessed''
        \begin{itemize}
            \item Here, the US was successful in pursuing compensation for Britain's failure to perform its obligations as a neutral party during the civil war, by claiming Britain had failed to stop the construction of Confederate ports
            \item Britain claimed that they did not have executive permission to do so
            \item The court held that in matters of international law, the British government cannot justify itself by reference to insufficient or absence of domestic law
        \end{itemize}
        \item In \case{\textit{Sandline Arbitration} (1998)}, Papua New Guinea (PNG) could not rely on internal law to support their plea that an international contract was invalid
        \begin{itemize}
            \item This case was a commercial arbitration between PNG and Sandline (which was a mercenary company). The government of PNG entered into a\$36m contract for Sandline to supply mercenaries to assist the PNG defence forces in their fight against the boganville revolutionary army
            \item PNG made a payment of \$18m, but declined to make the second half of the payment, claiming that the agreement had been reached contrary to the PNG constitution (i.e., they didn't have approval from Parliament for the hiring of external military forces)
            \item The tribunal held that the contract was governed by international law, and applied the principle that a state cannot rely on its own internal laws for the basis that the claim was wrong/illegal
            \item This case reinforces the principle that a state cannot cite inconsistent/absent domestic law to escape their obligations
        \end{itemize}
    \end{itemize}
    \item In an Australian Court, public international law, like Australian law, cannot be proved law by expert evidence, following \case{\textit{ACCC v PT Garuda (No 9)} [2013] FCA 323}
    \begin{itemize}
        \item Generally, to refer to the law of another country, expert evidence can be called upon to give context and content of the other country's law
        \item This is not the case for international law, as it is treated as being the same as Australian law for the purposes of interpreting it
    \end{itemize}
\end{itemize}

\begin{casedetails}{\textit{Australian Competition and Consumer Commission v PT Garuda (No 9)} [2013] FCA 323}\label{case:ACCC v Garuda}
    \flushleft
    This case concerned the ACCC's claim that PT Garuda had engaged in price fixing in relation to air cargo services. The ACCC sought to rely on expert evidence to prove the existence of public international law, which the court rejected.

    \vspace{\baselineskip}

    The court held that public international law, like Australian law, cannot be proved by expert evidence. This is because the court is the ultimate arbiter of the law, and so it is the court's responsibility to determine the law, rather than an expert witness.

    \begin{longtable}{p{0.1\textwidth}|>{\raggedright\arraybackslash}p{0.85\textwidth}}
        Perram J at [31] & In truth, opinion evidence is not receivable on an issue of domestic law because the law is not a matter for proof or disproof. It is for this reason, as pointed out in Cross on Evidence at [3075], that a judge is not obliged to accept a proposition of law agreed upon by the parties: cf \textit{Damberg v Damberg} (2001) 52 NSWLR 492 at [149].
    \end{longtable}  
\end{casedetails}

\section{Monism and Dualism}\label{sec:Monism and Dualism}
\begin{itemize}
    \item \textbf{Monism} refers to the idea that international law and domestic law are part of a single legal system, and that international law is automatically incorporated into domestic law
    \item Incorporation refers to the notion that international law is automatically a part of domestic law, and there are several variations of this notion:
    \begin{enumerate}[label=(\alph*)]
        \item The courts are to apply international law unless it is inconsistent with statute (i.e., apply international law over common law)
        \item The courts are to apply international law unless it is inconsistent with statute or common law
    \end{enumerate}
    \item It is relatively rare to find a state that automatically accepts international law; it is much more likely that the process of incorporation will be mediated by the courts
    \item \textbf{Dualism} refers to the notion that there are two independent systems of law; international law has no direct impact upon municipal law, and must be implemented into domestic law through executive order, legislation or judicial decision (i.e., the opposite of monism)
    \begin{itemize}
        \item If there is an interaction between domestic and international law, it mus tbe governed by either international law or domestic law; they cannot just freely interact
    \end{itemize}
    \item Transformation refers to the notion that international law must be transformed into domestic law before it can be applied, and has several variations:
    \begin{enumerate}[label=(\alph*)]
        \item Only legislation may implement the provisions of international law
        \item Legislation or court decisions may implement the provisions of international law
    \end{enumerate}
\end{itemize}

\section{Customary International Law in Australian Law}\label{sec:Customary International Law in Australian Law}
\begin{itemize}
    \item In Australia, the automatic incorporation of customary international law has been rejected
    \begin{itemize}
        \item There is no clear authority on this, but it can be said to a high degree of confidence that the Courts are not happy with custom becoming an automatic part of Australian common law
    \end{itemize}
    \item However, custom can influence courts, and be a source of common law, which is known as the soft transformation approach
    \begin{itemize}
        \item This approach has not been clearly endorsed by the courts (i.e., the notion that custom may be adopted by the courts, and not exclusively left to the parliament to implement)
    \end{itemize}
    \item This is in contrast to the UK, where they are more open to the incorporation approach with the exception for international crimes
    \item The case of \case{\textit{Trendex Trading} [1977] QB 529}, which was a case involving foreign state immunity, approached this issue
    \begin{itemize}
        \item This case concerned a contract for the purchase of cement by the Nigerian government; the terms of the contract were governed by English law and gave jurisdiction to the courts of England and Wales
        \item The case looked at the extent to which the rules of immunity under international law could apply in English common law
        \item The shipments of concrete were clogging up the port of Lagos
        \item The Central Bank of Nigeria cancelled these contracts, and Trendex sued the Central Bank of Nigeria under the contract, and the Court found that the bank was separate from the state, and could not claim immunity
        \item This case demonstrated that when the rules of international law had changed, the UK courts were justified in applying these new rules
        \item Per Lord Denning MR at page 544, ``[i]ntl. law does change: and the courts have applied the changes without…any Act of Parliament. In a sense, the doctrine of incorporation admits to the reality of international law''
    \end{itemize}
    \item Likewise, the recent example of \case{\textit{Law Debenture Trust v Ukraine} [2023] UKSC 11} held that it was English, and not international, law which was to be applied to ascertain whether the defence of duress applies to an English contractual dispute
    \begin{itemize}
        \item The facts of this case relate to a loan made by Russia to Ukraine (who, at the time of writing, remain engaged in armed conflict)
        \item As part of this conflict, Ukraine stopped paying moneys owed under this loan, and raised various justifications for doing so, under both English contract law and under PIL
        \item Ukraine claimed economic and military duress as to being forced into the threat, arguing that they could rely on the doctrine of countermeasures, which allowed for them to take retaliatory measures against Russia in response to Russia's 2014 annexation of Crimea
        \item The Court held thtat the relationship between domestic and international law was far more complex than as suggested in \textit{Trendex}
        \item The UK Supreme Court adopted PIL as a source of law, so long as it was not inconsistent with English law
        \item At [204], the court held that ``It seems preferable, therefore, to regard customary international law not as automatically a part of the common law but as a source of the common law on which courts in this jurisdiction may draw as appropriate."
    \end{itemize}
    \item The cases of \case{\textit{Chow Hung Ching v R} (1949) 77 CLR 449} and \case{\textit{Mabo v Queensland (No 2)} (1992) 175 CLR 1} provide some insight into the influence of public international law in Australian law
    \item Additionally, in \case{\textit{Habib v Commonwealth} (2010) 183 FCR 62}, for some claims surrounding fundamental human rights, the common law should reflect universal norms
\end{itemize}

\begin{casedetails}{\textit{Chow Hung Ching v R} (1949) 77 CLR 449}\label{case:Chow Hung Ching}
    \flushleft
    This was a pivotal case concerning Chinese army labourers who had been convicted of assault in Papua New Guinea (which was then under Australian UN mandate), with the central question being whether they enjoyed immunity as `visiting armed forces'. The Chinese government claimed that the labourers were immune from prosecution under international law, as they were part of the Chinese army. The HCA held that no immunity applied as they were in PNG as civilians, not in their capacity as members of the military forces of China. Generally, when foreign armed forces are present in a country, they are protected by a status of forces treaty, which affords them certain types of immunities. 

    \begin{longtable}{p{0.1\textwidth}|>{\raggedright\arraybackslash}p{0.85\textwidth}}
        Latham CJ at Page 462 & International law is not as such part of the law of Australia (\textit{Chung Chi Cheung v. The King}, and see \textit{Polites v. The Commonwealth}), but a universally recognized principle of international law would be applied by our courts: \textit{West Rand Central Gold Mining Co. v. The King}. \\\hline
        Dixon J at Page 477 & The theory of Blackstone (automatic incorporation) is `regarded as without foundation' and the `true view' is that of Brierly `that international law is not part, but is one of the sources' of Australian law. The immunity of foreign armed forces held to be part of the common law.\\\hline
        Starke J at Page 471 & The Courts acknowledge the existence of a body of rules which nations accept amongst themselves. On any judicial issue they seek to ascertain what the relevant rule is, and, having found it, they will treat it as incorporated into the domestic law, so far as it is not inconsistent with rules enacted by statutes or finally declared by their tribunals." What then are the immunities of arms and military forces of other nations accepted by our courts? It is by no means easy to answer that question, for in modern times those immunities are settled by conventions between the nations
    \end{longtable}

    This case held that the common law can be developed by regard to customary international law where it is not inconsistent with domestic law (i.e., it opens up potential for development, but does not automatically give it status). Here, there was no relevant treaty, and so the issue was of common law and of customary international law
\end{casedetails}

\begin{casedetails}{\textit{Mabo v Queensland (No 2)} (1992) 175 CLR 1}\label{case: Mabo}
    \flushleft
    This was a landmark case that recognised the native title of Indigenous Australians to the lands of Australia. Additionally, it provides some insight into the influence of public international law. In it, the High Court rejects the automatic inclusion of international law in the Australian legal system, but holds that it is still a good influence. This is particularly the case when referring to aspects of international law that touch on universal human values (e.g., areas like international human rights law may be more amenable to being relevant in terms of incorporating international law).

    \begin{longtable}{p{0.1\textwidth}|>{\raggedright\arraybackslash}p{0.85\textwidth}}
        Brennan J at Page 42 & ``The common law does not necessarily conform with international law, but international law is a legitimate and important influence on the development of the common law, especially when international law declares the existence of universal human values.''
    \end{longtable}
\end{casedetails}

\begin{casedetails}{\textit{Habib v Commonwealth} (2010) 183 FCR 62}\label{case: Habib v Commonwealth}
    \flushleft
    This case was a civil claim for torture committed overseas. Habib was an Australian citizen, and was accused of being involved in various terrorist offences for which he was never proven to have committed. He was kept in detention overseas; moreover, the Australian authorities knew he was being detained and was being seriously mistreated. He brought a civil claim for tort against the Australian government seeking damages for what he alleged was torture. Ordinary, this is a type of case that is hard to win in an Australian Court as they will not decide on matters that happened by other governments in their country. However, Black CJ of the federal court held that the foreign act of state doctrine must yield when we are looking at a case involving torture (which is one of the most serious international crimes) - i.e., the doctrine could be modified to take note of the international prohibition of torture. The overarching principle of this case is that in a civil claim for torture (or any serious crime forbidden under international norms), the common law of state doctrine should reflect universal norms.

    \begin{longtable}{p{0.1\textwidth}|>{\raggedright\arraybackslash}p{0.85\textwidth}}
        Black CJ at [7] & I agree with Jagot J that the common law has evolved such that the authorities do not support the application of the act of state doctrine in the present case. If, however, the choice were finely balanced, the same conclusion should be reached. When the common law, in its development, confronts a choice properly open to it, the path chosen should not be in disconformity with moral choices made on behalf of the people by the Parliament reflecting and seeking to enforce universally accepted aspirations about the behaviour of people one to another.
    \end{longtable} 
\end{casedetails}

\subsection{Criminal Law and Customary International Law}
\begin{itemize}
    \item The courts have held that customary/international criminal law established by custom can \textbf{never} be part of the Australian common law, following \case{\textit{Nulyarimanna v Thompson} (1999) 165 ALR 621}, and \case{\textit{R v Jones} [2006] 1 All ER 741}
\end{itemize}

\begin{conventiondetails}{\textit{1948 Convention on the Prevention and Punishment of the Crime of Genocide} Article 2}
    \flushleft
    In the present Convention, genocide means any of the following acts committed with intent to destroy, in whole or in part, a national, ethnical, racial or religious group, as such: 
    \begin{enumerate}[label=(\alph*)]
        \item Killing members of the group; 
        \item Causing serious bodily or mental harm to members of the group; 
        \item Deliberately inflicting on the group conditions of life calculated to bring about its physical destruction in whole or in part; 
        \item Imposing measures intended to prevent births within the group; 
        \item Forcibly transferring children of the group to another group.
    \end{enumerate}
\end{conventiondetails}

\begin{casedetails}{\textit{Nulyarimanna v Thompson} (1999) 165 ALR 621}\label{case: Nulyarimanna v Thompson}
    \flushleft
    This case questioned whether genocide was an offence under criminal law (this case was decided before Australia became a party to the 1998 Rome Statute of the International Criminal Court and implemented the Statute in legislation). At the time, Australia was a party to the genocide convention, but it was yet to implement the crime of genocide as a matter of legislation.

    \vspace{\baselineskip}

    A number of indigenous people had argued that the Commonwealth had committed genocide against their people (by extinguishing their native title, and failing to apply for UNESCO for their lands). They alleged that Commonwealth ministers and certain others had committed genocide by:
    \begin{enumerate}[label=(\alph*)]
        \item Adopting laws and policies that extinguished native title; and
        \item Not applying for World Heritage listing of certain Indigenous lands
    \end{enumerate}

    The Court held that genocide was not part of Australian common law, and so it never decided. Wilcox J and Whitlam J held that a \textit{jus cogens} prohibition of genocide was not automatically part of Australian common law, and that criminal offences must be created by statute, not by the courts. Moreover, it is for Australian parliaments to create criminal law, not for the common law to decide criminal law. Wilcox J held that if custom could create common law crime, `it would lead to the curious result that an international obligation incurred pursuant to customary law has greater domestic consequences than an obligation incurred, expressly and voluntarily, by Australia signing and ratifying an international convention'. It is a point of the treaty process in Australia that mere ratification isn't sufficient to make it law; it must be placed into legislation by parliament. Merkel J (dissenting) held that the offence of genocide is an offence under Australian common law, and that the Australian approach is the `common law adoption approach'; a rule of international law is to be adopted by a court so long as it is not inconsistent with legislation or public policy.

    \vspace{\baselineskip}

    Wilcox J held that if domestic criminal law could be influenced by customary/international criminal law, it would lead to the position where international obligations have greater obligations than domestic consequences, sidelining domestic law and thus a state's independence to make its own criminal laws.

    \begin{longtable}{p{0.1\textwidth}|>{\raggedright\arraybackslash}p{0.85\textwidth}}
        Wilcox J at [20] & If this were the position, it would lead to the curious result that an international obligation incurred pursuant to customary law has greater domestic consequences than an obligation incurred, expressly and voluntarily, by Australia signing and ratifying an international convention. \\[0.5cm]\hline
        McHugh J at [66] & ``This Court has never accepted that the Constitution contains an implication ... that it should be interpreted to conform with the rules of international law ... If the rule were applicable to a Constitution, it would operate as a restraint on the grants of power conferred".
    \end{longtable} 

\end{casedetails}

\begin{casedetails}{\textit{R v Jones} [2006] 2 All ER 741}\label{case: R v Jones}
    \flushleft
    In 2003, Margaret Jones and others broke into a RAF base, and caused damage to fuel tankers and bomb trailers at the beginning of the second Iraq war. They were subsequently charged with conspiracy to cause criminal damage contrary to the UK's \textit{Criminal Law Act 1967}. The defendant sought to rely on the legal justification that she had acted to impede the commission of the customary international law crime of aggression by the UK and the US (i.e., they should not be culpable because they broke the law to prevent the worse crime of aggression).

    \vspace{\baselineskip}

    The House of Lords held that whilst the crime of aggression was part of customary international law, it was not a crime under English law in the absence of any specific statutory authority saying otherwise - no such authority existed. Lord Bingham held that automatic incorporation of common law crimes would unjustifiably usurp the legislature. Likewise, Lord Mance held that `even crimes under public international law can no longer be, if they ever were, the subject of any automatic reception or recognition in domestic law by the courts'. Moreover, Lord Hoffman held that new domestic offences `should not creep into existence as a result of an international consensus to which only the executive of this country is a party', emphasising the concerns surrounding the separation of powers.
\end{casedetails}

\section{Treaties in Australian Law}
\begin{itemize}
    \item The power to enter into treaties is an exclusively Executive prerogative power under s 61 of the Constitution
    \begin{itemize}
        \item This power was inherited from the UK Imperial government, who initially negotiated and entered into treaties on Australia's behalf
        \item From 1926, Australia began to enter into treaties on its own behalf
    \end{itemize}
    \item ``The federal executive, though the Crown's representative, possessed exclusive and unfettered treaty-making power'' - \case{\textit{Koowarta v Bjelke-Petersen} (1982) 153 CLR 168 at [215]}, per Stephen J
\end{itemize}

\begin{statutedetails}{\textit{Constitution} s 61}\label{Constitution s 61}
    \flushleft
    The executive power of the Commonwealth is vested in the Queen and is exercisable by the Governor-General as the Queen's representative, and extends to the execution and maintenance of this Constitution, and of the laws of the Commonwealth.
\end{statutedetails}

\begin{itemize}
    \item The power to implement treaties is a legislative power, and is vested in the Parliament under s 51(xxix) of the Constitution (``The Parliament shall, subject to this Constitution, have power to make laws for the peace, order, and good government of the Commonwealth with respect to: ... external affairs'')
    \item The provisions of a treaty do not form part of Australian law, unless they have been implemented by statute, which was determined in \case{\textit{Dietrich v R} [1992] HCA 57}
    \item The same principle applies to implementing the resolutions of international organisations (such as those of the United Nations' Security Council), following \case{\textit{Bradley v Commonwealth} (1973) 128 CLR 557}
    \item This approach arises as a result of the separation of powers doctrine (treaty-making is for the Executive, law-making is for the Parliament); there are limited exceptions for peace treaties and maritime boundary agreements, although these have never been tested
\end{itemize}

\begin{casedetails}{\textit{Dietrich v R} [1992] HCA 57}\label{case: Dietrich v R}
    \flushleft
    In this case, the accused made an argument that he was entitled to publicly-funded legal representation in a criminal case under art 14 of the \textit{International Covenant on Civil and Political Rights} (of which Australia was a party). The High Court held that this was not the case, as the ratification of the covenant as an executive act did not affect Australian law, as its provisions had not been legislated and thus implemented by the Parliament, following Brennan CJ, and Mason and McHugh JJ. \textbf{The Court held that the provisions of a treaty do not form part of Australian law unless they have been implemented by statute.}
\end{casedetails}

\begin{casedetails}{\textit{Bradley v Commonwealth} (1973) 128 CLR 557}\label{case:Bradley v Commonwealth}
    \flushleft
    In this case, the executive was concerned about the activities of a place in Crows Nest known as the Rhodesian Information Centre, which was an agent of the illegal Southern Rhodesian regime. The UN Security Council passed a binding resolution on all members, requiring them not to recognise the illegal Rhodesian regime, and to take action against them in their own jurisdictions. In line with this, the Australian government shut down all communications to this centre, but as the resolution had not been implemented in Australian law, it was found that the government did not have any legislative authority to do what it had done.

    \begin{longtable}{p{0.1\textwidth}|>{\raggedright\arraybackslash}p{0.85\textwidth}}
        Barwick CJ and Gibbs J at Page 582 & Two matters were suggested as justifying an exercise of discretion in the defendants' favour. First, reliance was placed upon the resolutions of the Security Council to which reference has already been made. These resolutions are, in their terms, addressed to Member States who, by Art. 25 of the Charter, have agreed ``to accept and carry out the decisions of the Security Council in accordance with the present Charter". However, resolutions of the Security Council neither form part of the law of the Commonwealth nor by their own force confer any power on the Executive Government of the Commonwealth which it would not otherwise possess. The Parliament has passed the Charter of the United Nations Act 1945 (Cth), s 3 of which provides that ``The Charter of the United Nations (a copy of which is set out in the Schedule to this Act) is approved". That provision does not make the Charter itself binding on individuals within Australia as part of the law of the Commonwealth.
    \end{longtable} 
    
\end{casedetails}

\section{Treaty Making Process}
\begin{itemize}
    \item Australia can enter into two different types of treaties:
    \begin{itemize}
        \item Bilateral treaties, which enter into force for Australia after
        \begin{enumerate}
            \item Signature
            \item Subsequent exchange of notes stating that the constitutional process is completed
        \end{enumerate}
        \item Multilateral treaties, which enter into force for Australia after
        \begin{enumerate}
            \item Signature
            \item Subsequent ratification (or accession if there was no previous signature)
        \end{enumerate}
        \item This process for multilateral treaties allows for the Commonwealth to implement any legislation to allow for the treaty's provisions to be enlivened in domestic law (i.e., sign $\rightarrow$ prepare domestic law for the treaty's provisions $\rightarrow$ ratify)
    \end{itemize}
    \item There is no constitutional requirement for the Parliament to be involved in the treaty-making process
    \begin{itemize}
        \item However, only Parliament can pass legislation to implement treaties
        \item The Commonwealth can enter into any treaty that it wishes to, but it cannot implement the treaty without the Parliament's approval
    \end{itemize}
    \item As a matter of policy, since 1996 Parliament has been consulted on the treaty-making process
    \begin{itemize}
        \item However, they are not given a veto, but rather are provided a capacity to provide input into this process
        \item This was the result of the 1995 report of the Senate Legal and Constitutional References Committee (`Trick or Treaty?')
        \item It is now `required' that all proposed treaty actions are tabled in Parliament at least 15 sitting days prior to any binding action being undertaken (with exemptions for urgent or sensitive treaties)
    \end{itemize}
    \item To implement a treaty, a National Interest Analysis (NIA) must be prepared, which is akin to an explanatory memoranda for a treaty, and outlines why Australia has entered into a treaty
    \item The treaty should also be reviewed by the Joint Standing Committee on Treaties (JSCOT), which is a parliamentary committee that reviews Australia's participation in treaties
\end{itemize}

\section{Implementing Treaties}
\subsection{Constitutional Considerations}
\begin{itemize}
    \item The constitution enables the executive to enter into treaties as part of the `external affairs' provision in s 51(xxix) of the \textit{Constitution}
    \begin{itemize}
        \item This provision governs the relations between Australia and other countries/international organisations, matters external to Australia, and the implementation of international law (including custom, treaties, international recommendations, etc.)
    \end{itemize}
    \item The Commonwealth parliament does not have plenary power to legislate on whatever it wants to, but only to legislate with respect to matters conferred on it by the \textit{Constitution} (s 51)
    \item The external affairs power will support legislation applicable to matters geographically external to Australia (\case{\textit{Horta v Commonwealth} (1994) 181 CLR 183})
\end{itemize}

\begin{casedetails}{\textit{Horta v Commonwealth} (1994) 181 CLR 183}\label{case:Horta v Commonwealth}
    \flushleft
    In 1995, Indonesia invaded East Timor, and remained in occupation of it until the early 2000s; this occupation was held to be unlawful as a matter of PIL. However, this did not stop Australia from accepting Indonesia sovereignty over East Timor, and subsequently, Australia concluded a treaty with Indonesia which allowed for them to access oil in the Timor Gap. The plaintiff argued that the law implementing the 1989 Timor Gap Treaty was invalid as the treaty itself was void (by virtue of recognising Indonesia's unlawful occupation of East Timor). The High Court said that they do not have to address that issue, as there is an element of the external affairs power that says that they have to only look at whether the law applies geographically externally to Australia; the law applied to the Timor sea, which was valid. Moreover, even if the treaty was void as a matter of PIL, that didn't undermine or impugn the character of this law as one with respect to external affairs.

    \vspace{\baselineskip}

    Per Curiam, `the area of the Timor Gap and the exploration...and exploitation of, petroleum resources ... [are] matters ... geographically external to Australia. There is an obvious and substantial nexus between each of them and Australia' `[E]ven if the Treaty were void or unlawful under international law ... the [impugned Acts] would not thereby be deprived of their character as laws with respect to ``External Affairs"'. Moreover, this case leaves unresolved the question as to whether the executive can enter into treaties that are unlawful; the HCA did not deal with it as they held that it was still lawful in a way.
\end{casedetails}

\begin{itemize}
    \item It has been reinforced that the external affairs power will support legislation that implements treaties in Australian law
\end{itemize}

\begin{casedetails}{\textit{Koowarta v Bjelke-Petersen} (1982) 153 CLR 168}\label{case:Koowarta v Bjelke-Petersen}
    \flushleft
    The Aboriginal Land Fund Commission had entered into a contract to purchase a pastoral lease in Queensland. The Queensland government refused to consent to the transfer as the purchaser was Aboriginal. The Commission sued under the \textit{Racial Discrimination Act 1975} (Cth), and the Queensland government challenged the validity of the legislation. The Court found that the Act was valid as it implements the \textit{1969 International Convention on the Elimination of All Forms of Racial Discrimination (ICERD)}. Mason J held that a law implementing custom would be a law with respect to external affairs, indeed `any matter which has 'become the topic of international debate, discussion and negotiation constitutes an external affair before Australia enters a treaty relating to it'.
\end{casedetails}

\begin{casedetails}{\textit{Commonwealth v Tasmania} (1983) 158 CLR 1}\label{case:Commonwealth v Tasmania}
    \flushleft
    This case concerned the validity of the \textit{World Heritage Conservation Act 1983} (Cth), which implemented the \textit{1972 Convention Concerning the Protection of the World Cultural and Natural Heritage}. The majority of the Court held that most of the legislation was valid in respect to external affairs and the external affairs power; Tasmania had challenged the validity of the legislation. The majority additionally held that the Commonwealth can legislate to implement a treaty, but that power is not unlimited.

    \vspace{\baselineskip}

    Deane J held that the law under s 51(xxix) of the \textit{Constitution} must carry into effect treaty obligations, and be reasonably considered to be appropriate and adapted to achieving this objective (i.e., reasonable proportionality between the designated object and the means for achieving it). That is, the legislation must bear some relation to the treaty, and must be reasonably appropriate and adapted to achieving the objective of the treaty.
\end{casedetails}

\subsection{Legislative Concerns}
\begin{itemize}
    \item Australia will generally not ratify a treaty until the legislation to domestically implement the provisions of the treaty is in place
    \item Legislation will be needed to implement a treaty if the treaty creates rights for or imposes obligations upon individuals; however, it will not be very detailed as often there is existing legislation or common law at the state or federal levels that allows Australia to comply with the terms of the treaties
    \item Existing legislation can often be used to make the necessary regulations to implement international provisions
    \begin{itemize}
        \item For example, the \textit{Charter of the United Nations Act 1945} (Cth) was used to implement UNSC resolutions dealing with sanctions and the listing of terrorist organisations and the subsequent freezing of assets; generally, this legislation is used to implement UNSC regulations (i.e., it is delegation legislation)
    \end{itemize}
    \item Legislation may `give a treaty the force of law'
    \begin{itemize}
        \item This generally occurs where the treaty has been drafted with domestic incorporation in mind (e.g., the \statute{\textit{Diplomatic Privileges and Immunities Act 1967} (Cth) s 7}, which holds that the Vienna Convention on Diplomatic Relations is to have force of law)
    \end{itemize}
    \item Generally, there is the translation of treaty provisions into domestic legislation
    \begin{itemize}
        \item This is the most common practice
        \item It avoids uncertainty by directly translating the terms of the treaty into Australian law
        \item It may refer to terms of a treaty (e.g., \statute{\textit{Migration Act 1958} (Cth) s 4} refers to the definition of a `refugee')
    \end{itemize}
    \item If legislation `approves' a treaty, it is not binding
    \begin{itemize}
        \item This merely notes that the terms of the treaty are acceptable to Australia, but does not implement them (i.e., it is not sufficient to be binding)
        \item The mere approval of Parliament does not give a treaty the force of law, as discussed in \case{\textit{Bradley v Commonwealth (1973) 128 CLR 557}} (which discussed the \statute{\textit{Charter of the UN Act 1945} (Cth)})
        \item The practice of approving the provisions of treaties has since lapsed
    \end{itemize}
\end{itemize}

\section{Statutory Interpretation and International Law}
\begin{itemize}
    \item There are several scenarios where international law can be used to interpret domestic law
    \item International law can be used as extrinsic material when interpreting legislation which refers to a treaty, following the \statute{\textit{Acts Interpretation Act 1901} (Cth) ss 15AB(1) and (2)(d)}
    \item International law can be used to interpret a legislative provision that incorporates a treaty provision (here, the rules of treaty interpretation (i.e., the VCLT) are applied, rather than the rules of statutory interpretation)
    \item If its language permits, a legislative provision is interpreted to avoid placing Australia in breach of its international obligations, following the Polites principle
\end{itemize}

\begin{statutedetails}{\textit{Acts Interpretation Act 1901} (Cth) s 15AB(1)-(2)}\label{Acts Interpretation Act s 15AB}
    \flushleft
    \begin{enumerate}[label=(\arabic*)]
        \item Subject to subsection (3), in the interpretation of a provision of an Act, if any material not forming part of the Act is capable of assisting in the ascertainment of the meaning of the provision, consideration may be given to that material:
        \begin{enumerate}[label=(\alph*)]
            \item to confirm that the meaning of the provision is the ordinary meaning conveyed by the text of the provision taking into account its context in the Act and the purpose or object underlying the Act; or
            \item to determine the meaning of the provision when:
            \begin{enumerate}[label=(\roman*)]
                \item the provision is ambiguous or obscure; or
                \item the ordinary meaning conveyed by the text of the provision taking into account its context in the Act and the purpose or object underlying the Act leads to a result that is manifestly absurd or is unreasonable.
            \end{enumerate}
        \end{enumerate}
        \item Without limiting the generality of subsection (1), the material that may be considered in accordance with that subsection in the interpretation of a provision of an Act includes:
        \begin{enumerate}[label=(\alph*)]
            \item all matters not forming part of the Act that are set out in the document containing the text of the Act as printed by the Government Printer;
            \item any relevant report of a Royal Commission, Law Reform Commission, committee of inquiry or other similar body that was laid before either House of the Parliament before the time when the provision was enacted;
            \item any relevant report of a committee of the Parliament or of either House of the Parliament that was made to the Parliament or that House of the Parliament before the time when the provision was enacted;
            \item any treaty or other international agreement that is referred to in the Act;
            \item any explanatory memorandum relating to the Bill containing the provision, or any other relevant document, that was laid before, or furnished to the members of, either House of the Parliament by a Minister before the time when the provision was enacted;
            \item the speech made to a House of the Parliament by a Minister on the occasion of the moving by that Minister of a motion that the Bill containing the provision be read a second time in that House;
            \item any document (whether or not a document to which a preceding paragraph applies) that is declared by the Act to be a relevant document for the purposes of this section; and
            \item any relevant material in the Journals of the Senate, in the Votes and Proceedings of the House of Representatives or in any official record of debates in the Parliament or either House of the Parliament.
        \end{enumerate}
    \end{enumerate}
\end{statutedetails}

\subsection{Polites Principle}
\begin{itemize}
    \item The Polites principle refers to the presumption that Parliament intends to give effect to Australia's obligations under international law (following \case{\textit{Polites v Commonwealth} (1945) 70 CLR 60})
\end{itemize}

\begin{casedetails}{\textit{Polites v Commonwealth} (1945) 70 CLR 60}\label{case:Polites v Commonwealth}
    \flushleft
    In this case, Mr Polites (a Greek national), was given notice under regulations requiring him to serve in the Australian Defence Force (i.e., a situation where a foreign national was being conscripted, which is explicitly prohibited by international law). This case resulted in the High Court reviewing the legislation, and held that the legislation was valid. This was despite an established rule of international law that aliens may not be required to serve in armed forces.

    \vspace{\baselineskip}

    From Latham CJ's statement (below), it is the case that the Commonwealth parliament can still pass legislation that is inconsistent with public international law, but every effort should be made to avoid this where possible.

    \begin{longtable}{p{0.1\textwidth}|>{\raggedright\arraybackslash}p{0.85\textwidth}}
        Latham CH at Page 69 & It must be held that legislation otherwise within the power of the Commonwealth Parliament does not become invalid because it conflicts with a rule of international law, though every effort should be made to construe Commonwealth statutes so as to avoid breaches of international law and of international comity. \\\hline
        Dixon J at Page 77 & It is a rule of construction that, unless a contrary intention appear, general words occurring in a statute are to be read subject to the established rules of international law and not as intended to apply to persons or subjects which, according to those rules, a national law of the kind in question ought not to include.
    \end{longtable}  
\end{casedetails}

\begin{itemize}
    \item The Polites principle does not apply for constitutional interpretation, following \case{\textit{Al-Kateb v Godwin} (2004) 208 ALR 124}, as it would violate s 128 of the Constitution otherwise, which explicitly requires a referendum to amend it
\end{itemize}

\begin{casedetails}{\textit{Al-Kateb v Godwin} (2004) 219 CLR 562}\label{case:Al-Kateb v Godwin}
    \flushleft
    This case concerned whether a stateless Palestinian man could be subject to indefinite detention as a result of a lack of a state to which he could be deported to. Whilst this case has been overturned, the discussion between Kirby J and McHugh J is still relevant, especially in the context of the Polites principle. Note that McHugh J's position reflect that of the majority.

    \begin{longtable}{p{0.1\textwidth}|>{\raggedright\arraybackslash}p{0.85\textwidth}}
        Kirby J at [175] & Whatever may have been possible in the world of 1945, the complete isolation of constitutional law from the dynamic impact of international law is neither possible nor desirable today. That is why national courts, and especially national constitutional courts such as this, have a duty, so far as possible, to interpret their constitutional texts in a way that is generally harmonious with the basic principles of international law, including as that law states human rights and fundamental freedoms. \\[3cm]\hline
        McHugh J at [66] & ``This Court has never accepted that the Constitution contains an implication ... that it should be interpreted to conform with the rules of international law ... If the rule were applicable to a Constitution, it would operate as a restraint on the grants of power conferred".
    \end{longtable}  
\end{casedetails}

\begin{itemize}
    \item The majority held that the constitution is fundamentally different from statute, and so it is not to be interpreted in a way to conform to the rules of public international law, as to do so would violate the independence of the constitution under s 128
\end{itemize}

\begin{statutedetails}{\textit{Australian Constitution} s 128}\label{Constitution s 128}
    \flushleft
    This Constitution shall not be altered except in the following manner:

    \vspace{\baselineskip}

    The proposed law for the alteration thereof must be passed by an absolute majority of each House of the Parliament, and not less than two nor more than six months after its passage through both Houses the proposed law shall be submitted in each State and Territory to the electors qualified to vote for the election of members of the House of Representatives.

    \vspace{\baselineskip}

    But if either House passes any such proposed law by an absolute majority, and the other House rejects or fails to pass it, or passes it with any amendment to which the first-mentioned House will not agree, and if after an interval of three months the first-mentioned House in the same or the next session again passes the proposed law by an absolute majority with or without any amendment which has been made or agreed to by the other House, and such other House rejects or fails to pass it or passes it with any amendment to which the first-mentioned House will not agree, the Governor-General may submit the proposed law as last proposed by the first-mentioned House, and either with or without any amendments subsequently agreed to by both Houses, to the electors in each State and Territory qualified to vote for the election of the House of Representatives.

    \vspace{\baselineskip}

    When a proposed law is submitted to the electors the vote shall be taken in such manner as the Parliament prescribes. But until the qualification of electors of members of the House of Representatives becomes uniform throughout the Commonwealth, only one-half the electors voting for and against the proposed law shall be counted in any State in which adult suffrage prevails.

    \vspace{\baselineskip}

    And if in a majority of the States a majority of the electors voting approve the proposed law, and if a majority of all the electors voting also approve the proposed law, it shall be presented to the Governor-General for the Queen's assent.

    \vspace{\baselineskip}

    No alteration diminishing the proportionate representation of any State in either House of the Parliament, or the minimum number of representatives of a State in the House of Representatives, or increasing, diminishing, or otherwise altering the limits of the State, or in any manner affecting the provisions of the Constitution in relation thereto, shall become law unless the majority of the electors voting in that State approve the proposed law.

    \vspace{\baselineskip}

    In this section, Territory means any territory referred to in section one hundred and twenty-two of this Constitution in respect of which there is in force a law allowing its representation in the House of Representatives.
\end{statutedetails}