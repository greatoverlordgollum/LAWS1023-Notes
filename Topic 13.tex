\begin{conventiondetails}{\textit{UN Charter} Article 2(3)}
    \flushleft
    All Members shall settle their international disputes by peaceful means in such a manner that international peace and security, and justice, are not endangered.
\end{conventiondetails}

\begin{conventiondetails}{\textit{UN Charter} Article 33}
    \flushleft
    \begin{enumerate}
        \item The parties to any dispute, the continuance of which is likely to endanger the maintenance of international peace and security, shall, first of all, seek a solution by negotiation, enquiry, mediation, conciliation, arbitration, judicial settlement, resort to regional agencies or arrangements, or other peaceful means of their own choice.
        \item The Security Council shall, when it deems necessary, call upon the parties to settle their dispute by such means.
    \end{enumerate}
\end{conventiondetails}

\begin{itemize}
    \item \convention{\textit{UN Charter} Article 33} lists the various options made available for states to peacefully resolve their disputes
    \item ``A dispute may be defined as a specific disagreement concerning a matter of fact, law or policy in which a claim or assertion of one party is met with refusal, counter-claim or denial by another'' (John Merrills); from this, it is important for any set of facts of be exactly clear, and for the nature of the dispute to be defined, as without this, parties are unable to utilise dispute resolution methods
\end{itemize}

\begin{table}[H]
    \centering
    \begin{tabular}{|p{0.12\textwidth}|p{0.12\textwidth}|p{0.12\textwidth}|p{0.12\textwidth}|p{0.12\textwidth}|p{0.12\textwidth}|}
        \hline
        \textbf{Negotiation} & \textbf{Mediation} & \textbf{Conciliation} & \textbf{Inquiry} & \textbf{Arbitration\footnote{Ad-hoc and institutional}} & \textbf{Judicial Settlement} \\\hline
        Informal (unstructured) & \multicolumn{3}{l|}{More formal (more structured)} & \multicolumn{2}{p{0.25\textwidth}|}{Formal (highly structured)} \\\hline 
        Resolution by the parties themselves & \multicolumn{3}{p{0.35\textwidth}|}{Resort to third party for guidance} & \multicolumn{2}{l|}{Resort to third party for determination} \\\hline 
        \multicolumn{5}{|l|}{Flexible (high degree of control by the parties)} & Fixed (less control by the parties) \\\hline
        \multicolumn{4}{|l|}{Non-binding (advisory)} & \multicolumn{2}{l|}{Binding (mandatory)} \\\hline
        \multicolumn{4}{|p{0.4\textwidth}|}{Resolution of dispute according to the criteria agreed to by the parties (`diplomatic method of dispute settlement')} & \multicolumn{2}{p{0.3\textwidth}|}{Resolution of dispute according to the law (`legal method of dispute settlement')} \\\hline
        Most common & \multicolumn{3}{p{0.35\textwidth}|}{Much less common} & \multicolumn{2}{p{0.25\textwidth}|}{Historically uncommon; recently more common} \\\hline
    \end{tabular}
    \caption{Summary of dispute resolution methods in international law}
    \label{tab:dispute-resolution-methods}
\end{table}

\section{Negotiation}
\begin{itemize}
    \item This is used more frequently than any other method of dispute resolution in international law
    \item A state will negotiate with another state to resolve a dispute, and this is often done through diplomatic channels
    \item Parties themselves retain maximum control over the process and outcome
    \item Negotiation may be used prior to, during, or after a dispute arises, and may be used once a dispute is resolved (in order to agree what the meaning of the court's decision is and how it should be implemented)
    \item Modes of negotiation include through normal diplomatic channels, through high-level officials, through international forums/organisations, etc.
    \item Negotiation may sometimes be a procedural precondition for jurisdiction of an international court (i.e., some treaties will require negotiation as a precondition to formal dispute settlement by a court or tribunal)
\end{itemize}

\begin{casedetails}{\textit{Ukraine v Russia} [2019] ICJ Rep 558}
    \flushleft
    This case concerned alleged violations by Russia of the 1999 \textit{International Convention for the Suppression of the Financing of Terrorism} and the 2003 \textit{International Convention on the Elimination of All Forms of Racial Discrimination}, following events in eastern Ukraine and the invasion of Crimea. There were objections set out in the preliminary issues stage of dispute resolution.

    \tcbsubtitle{Judgement on Preliminary Objections in 2019}

    Under CERD, Art. 22, ``a State [must] make a genuine attempt to settle through negotiation the dispute in question with the other State''. As such, a court will not have jurisdiction until it can be established that the parties have sought to negotiate the dispute. The Court then described what was required under Art. 22:

    \begin{quote}
        Negotiations are distinct from mere protests or disputations. Negotiations entail more than the plain opposition of legal views or interests between two parties, or the existence of a series of accusations and rebuttals, or even the exchange of claims and directly opposed counter-claims.
    \end{quote}

    \begin{quote}
        [Negotiation] requires — at the very least — a genuine attempt by one of the disputing parties to engage in discussions with the other disputing party, with a view to resolving the dispute.
    \end{quote}

    \begin{quote}
        The precondition of negotiation is met only when there has been a failure of negotiations [tried and don't get anywhere], or when negotiations have become futile or deadlocked.
    \end{quote}

    Under the last paragraph, if a state attempts to negotiate but is met by resistance from the other state, then the precondition of negotiation is met, and the court can exercise jurisdiction. In this case, the Court held that the exchanges of \textit{notes verbales} (diplomatic notes) between the parties which included specific mentions of the CERD, negotiations over two years (including face to face meetings), and the breakdown/deadlocking of those negotiations by the time that Ukraine had filed its application meant that the ICJ had jurisdiction here.

    \tcbsubtitle{Judgement on Merits (January 2024)}
    The Russian Federation was found to have violated its obligations under both the 1999 Convention and the CERD as a result of its failure to maintain Ukrainian language education in Crimea.
\end{casedetails}

\begin{casedetails}{\textit{South West Africa} [1962] ICJ 319}
    \flushleft
    South West Africa (which was a German colony) was placed under the mandate of South Africa by the League of Nations from 1915 (this became Namibia in 1990). Ethiopia and Liberia brought a case against South Africa in the ICJ, alleging that South Africa's racist policy of apartheid violated its obligations under the League of Nations mandate to manage and protect South West Africa.

    \vspace{\baselineskip}

    This case held that it was sufficient for parties to hold discussions at international organisations, and that this was sufficient to satisfy the requirement of negotiation. South Africa held that any dispute with Ethiopia and Liberia was outside the scope of the ICJ's jurisdiction, because of the supposed lack of direct negotiation. On the basis that there had been extensive discussions on this issue within the UN on the topic of South West Africa, the ICJ rejected South Africa's argument.

    \vspace{\baselineskip}

    In the second phase of these proceedings, the ICJ in \case{[1966] ICJ Rep 6} found that (by the casting vote), the applicants did not have any standing, and in an Advisory Opinion (\case{[1971] ICJ Rep 16}), found that the continued presence of South Africa in Namibia was illegal.

    \vspace{\baselineskip}

    Hence, to determine if negotiation is a precondition, look at the instrument conferring jurisdiction on the court (e.g., a relevant treaty, or a treaty, or something entailing compulsory acceptance of a jurisdiction).
\end{casedetails}

\section{Mediation}
\begin{itemize}
    \item Mediation is a process that involves a third party (the mediator) who assists in reaching a settlement (the parties remain in control over the process)
    \item Mediators might be invited by the parties to propose a solution, or otherwise might take a more engaged role in the process
    \item Third parties may perform various functions, from mere `good offices' (communication channels between the parties) to proposing a solution (i.e., conciliation)
    \item Mediators can include other governments, international organisations (e.g., the UN), private individuals (e.g., Pope John Paul II in the Beagle Channel dispute), NGOs (e.g., the International Committee of the Red Cross), or other parties
    \item Parties can choose anyone to be a mediator, depending on their agreement
\end{itemize}

\section{Inquiry}
\begin{itemize}
    \item Inquiry is a process that involves an objective, disinterested assessment of the evidence and finding of facts (i.e., it is a factually focused method of dispute settlement)
    \begin{itemize}
        \item It is not necessarily concerned with the legal characteristics of the dispute, but rather resolves differences between facts held by the parties
    \end{itemize}
    \item Sometimes, it can be a precursor to other forms of dispute settlement, but at other times can be standalone; in either case, the parties retain control of the parameters of how the process is conducted
    \item Inquiry is governed by the \convention{\textit{1899 Hague Convention for the Pacific Settlement of International Disputes}}, which were concluded in response to the Spanish-American War
    \begin{itemize}
        \item Articles 9-14 hold that inquiry requires special agreement between the parties, its procedure is determined by the parties, it is limited to statements of fact, and it is left to the parties to determine the effect of the proceedings and how its results are to be used
    \end{itemize}
\end{itemize}

\begin{conventiondetails}{\textit{1899 Hague Convention for the Pacific Settlement of International Disputes} Article 9}
    \flushleft
    In differences of an international nature involving neither honour nor vital interests, and arising from a difference of opinion on points of fact, the Signatory Powers recommend that the parties, who have not been able to come to an agreement by means of diplomacy, should, as far as circumstances allow, institute an International Commission of Inquiry, to facilitate a solution of these differences by elucidating the facts by means of an impartial and conscientious investigation.
\end{conventiondetails}

\begin{conventiondetails}{\textit{1899 Hague Convention for the Pacific Settlement of International Disputes} Article 10}
    \flushleft
    The International Commissions of Inquiry are constituted by special agreement between the parties in conflict.

    \vspace{\baselineskip}

    The Convention for an inquiry defines the facts to be examined and the extent of the Commissioners' powers.

    \vspace{\baselineskip}

    It settles the procedure.

    \vspace{\baselineskip}

    On the inquiry both sides must be heard.

    \vspace{\baselineskip}

    The form and the periods to be observed, if not stated in the Inquiry Convention, are decided by the Commission itself.
\end{conventiondetails}

\begin{conventiondetails}{\textit{1899 Hague Convention for the Pacific Settlement of International Disputes} Article 11}
    \flushleft
    The International Commissions of Inquiry are formed, unless otherwise stipulated, in the manner fixed by Article 32 of the present Convention.
\end{conventiondetails}

\begin{conventiondetails}{\textit{1899 Hague Convention for the Pacific Settlement of International Disputes} Article 12}
    \flushleft
    The Powers in dispute engage to supply the International Commission of Inquiry, as fully as they may think possible, with all means and facilities necessary to enable it to be completely acquainted with and to accurately understand the facts in question.
\end{conventiondetails}

\begin{conventiondetails}{\textit{1899 Hague Convention for the Pacific Settlement of International Disputes} Article 13}
    \flushleft
    The International Commission of Inquiry communicates its Report to the conflicting Powers, signed by all the members of the Commission.
\end{conventiondetails}

\begin{conventiondetails}{\textit{1899 Hague Convention for the Pacific Settlement of International Disputes} Article 14}
    \flushleft
    The Report of the International Commission of Inquiry is limited to a statement of facts, and has in no way the character of an Arbitral Award. It leaves the conflicting Powers entire freedom as to the effect to be given to this statement.
\end{conventiondetails}

\begin{casedetails}{\textit{Dogger Bank Inquiry} (1904)}
    \flushleft
    This case involved Russian warships steaming from the Baltic to the Russo-Japanese War. It encountered British fishing vessels, and attacked and sank them, believing to be Japanese torpedo boats. France persuaded Britain and Russia to establish a Commission of Inquiry under the \convention{\textit{1899 Hague Convention}}, which found that the Russian Admiral had fired in error, but not ``of a nature to cast any discredit on [his] military qualities ... or humanity''.

    \vspace{\baselineskip}

    This case, and the broader time period it was in, is symbolic of the prominence of dispute settlement entities (e.g., mediators) trying to find a quasi-settlement resolution to disputes, as opposed to the modern tendency to turn to international courts.
\end{casedetails}

\section{Conciliation}
\begin{itemize}
    \item Conciliation is a process that involves a third party (the conciliator) who investigates the dispute and proposes a solution to the parties
    \item The involvement of a third party makes it more formal, and gives conciliation an institutional role
    \item Its functions are to investigate the dispute and recommend terms of settlement (i.e., it is investigatory and recommendatory)
    \item It is usually a confidential process, and the recommendations of the conciliator are provided as a draft report to which parties can respond to; a statement of reasons is also issued by the conciliator to the parties
    \item Conciliation may consider non-legal factors (i.e., it is not confined only to applying international law)
    \item \convention{\textit{1928 General Act for the Pacific Settlement of International Disputes} Articles 1-15} holds that there must be compulsory conciliation for all disputes between parties to the Convention if the issue is not resolved by diplomacy
    \item The \convention{\textit{1948 Pact of Botogá} Articles XV-XXX} holds that conciliation is one method of dispute settlement, as well as setting out the requirements for a panel of conciliators, the rules of procedure, and the publication of the report or summary thereof of the conciliatory proceedings
    \item The \convention{\textit{1928 UN Convention on the Law of the Sea} (UNCLOS)} contains provisions around conciliation, as explored in \case{\textit{Australia-Timor Leste Conciliation} (2016)}
\end{itemize}

\begin{casedetails}{\textit{Australia-Timor Leste Conciliation} (2016)}
    \flushleft
    See \case{\textit{Horta v Commonwealth} (1994) 181 CLR 183} on Page \pageref{case:Horta v Commonwealth} for the background to this case.

    \vspace{\baselineskip}

    Once Timor Leste had gained independence from Indonesia, they had to resolve their maritime boundary in the Timor Sea. Initially, they agreed to continue the Timor Gap treaty that had been enforced between Australia and Indonesia (which was the subject of \case{\textit{Horta v Commonwealth} (1994) 181 CLR 183}). However, in 2016, Timor Leste argued that Australia was in breach of several conventions in the treaty, and that Australia was under an obligation to enter into a negotiated agreement on the maritime boundaries.

    \vspace{\baselineskip}

    Australia tried to avoid dispute settlement in a number of ways. It first submitted an adjusted declaration accepting the jurisdiction of the ICJ to exclude maritime boundary disputes (as this was done prior to the crystallisation of the dispute, it prevented Timor Leste from taking action in the ICJ). Additionally, Australia submitted a reservation under \convention{\textit{UNCLOS}} opting out of \convention{\textit{UNCLOS}}'s compulsory dispute resolution processes. \convention{\textit{UNCLOS}} provides for optional (Article 284) and compulsory (Article 298) conciliation. Under \convention{\textit{UNCLOS} Article 298}, a party can opt out of Part XV compulsory settlement for certain disputes, including maritime boundary disputes, but compulsory conciliation was preserved, and as such, Timor Leste could go to compulsory conciliation, which they did in 2016.

    \vspace{\baselineskip}

    This was the first-ever conciliation held under \convention{\textit{UNCLOS} Article 298 and Annex V}. Annex V and the Rules of Procedure adopted by the commission held that the parties are largely in control of the process. Moreover, they agreed on the constitution of the commission, and that among other things, the commission would deal with issues of jurisdiction in an adversarial and adjudicative manner. The substantive phase of the proceedings was more flexible, informal and agile (if jurisdiction could be established, then a diplomatic means of dispute resolution could be used), thereby ensuring the confidentiality of proceedings (subject to certain exceptions in the interest of transparency).

    \vspace{\baselineskip}

    Australia unsuccessfully challenged the competence of the Commission. As such, the Commission then engaged with all of the issues, including the political and economic factors at stake, and not only the issues of maritime boundary delineation. Australia then engaged and determined a Maritime Boundary treaty in 2018.
\end{casedetails}

\section{Arbitration}
\begin{itemize}
    \item This is tending towards a more formalised method of dispute settlement
    \item It involves a third party, it is a legal form of dispute settlement (i.e., the arbitrators (generally an odd number of them to prevent deadlocks) will apply international law to create \textbf{awards} (which are binding decisions on the parties))
    \item Arbitration can be pursued in many different ways, and there are two key types of it:
    \begin{itemize}
        \item Inter-state (i.e., government vs government)
        \begin{itemize}
            \item This is a legal form of dispute settlement (there are independent arbitrators, who apply international law and make a legally binding decision)
            \item This may be ad-hoc (i.e., to resolve a specific dispute, and so the arbitration tribunal is established only when there is a specific issue), or institutional (i.e., arbitration is established as a standing, go-to system of dispute resolution)
            \item An example includes the \case{\textit{South China Sea Arbitration} (2016)}
        \end{itemize}
        \item Individual/corporation vs state (e.g., investment arbitration)
        \begin{itemize}
            \item Bilateral investment treaties (BITs) often provide for arbitration between an individual/corporation and a state, which is a legal form of dispute settlement
            \item An example is the \case{\textit{Philip Morris v Australia} (2015)} case under the Hong Kong-Australia BIT, which concerned Australia's plain packaging laws for tobacco products; Philip Morris challenged this but was ultimately unsuccessful
        \end{itemize}
    \end{itemize}
\end{itemize}

\section{Judicial Settlement}
\begin{itemize}
    \item Judicial settlement is the most formal process of dispute settlement
    \item It is institutionalised (i.e., consists of permanent courts and tribunals) by definition, and as opposed to arbitration (where the parties can choose the arbitrators as they want), turning to judicial settlement means accepting the current compositions of the courts as they stand at that point in time
    \item Judicial settlement consists of fair and impartial judges who render a decision according to law (i.e., a legal situation)
    \item This is an open process, and the decisions are binding upon the parties
    \item However, the decisions are not binding on non-parties of that case, and so there is no system of precedent
    \item From \statute{\textit{ICJ Statute} Article 38(1)(d)}, judgements of international courts and tribunals may influence the development of international law
    \item There are a range of international courts and tribunals, including:
    \begin{itemize}
        \item The International Court of Justice (ICJ)
        \item The International Criminal Court (ICC)
        \item The International Tribunal for the Law of the Sea (ITLOS)
        \item World Trade Organization (WTO) Dispute Settlement Body
        \item Regional courts (e.g., the European Court of Human Rights, the European Court of Justice)
        \item International criminal tribunals
        \item Human rights courts
    \end{itemize}
\end{itemize}

\section{International Court of Justice}
\begin{itemize}
    \item The ICJ is governed by the \statute{\textit{Statute of the International Court of Justice}}, and the \convention{\textit{Rules of Court}}
    \item It is composed of 15 judges elected by the UN General Assembly for 9-year terms; whilst their election is a political process, once they are elected they are operate impartially and independently of their state
\end{itemize}

\subsection{Jurisdiction}
\begin{itemize}
    \item The ICJ has \textbf{contentious jurisdiction}, which enables it to hear cases between two or more states (this is jurisdiction \textit{ratione personae})
    \item Under \statute{\textit{ICJ Statute} Article 34(1)}, only states may be parties to ICJ proceedings
    \item Under \statute{\textit{ICJ Statute} Article 59}, the decisions of the ICJ are only binding on the parties to the case, and not on third parties (i.e., there is no system of precedent)
\end{itemize}

\begin{statutedetails}{\textit{Statute of the ICJ} Article 34}
    \flushleft
    \begin{enumerate}
        \item Only states may be parties in cases before the Court.
        \item The Court, subject to and in conformity with its Rules, may request of public international organizations information relevant to cases before it, and shall receive such information presented by such organizations on their own initiative.
        \item Whenever the construction of the constituent instrument of a public international organization or of an international convention adopted thereunder is in question in a case before the Court, the Registrar shall so notify the public international organization concerned and shall communicate to it copies of all the written proceedings.
    \end{enumerate}
\end{statutedetails}

\begin{statutedetails}{\textit{Statute of the ICJ} Article 59}
    \flushleft
    The decision of the Court has no binding force except between the parties and in respect of that particular case.
\end{statutedetails}

\begin{statutedetails}{\textit{Statute of the ICJ} Article 36}
    \flushleft
    \begin{enumerate}
        \item The jurisdiction of the Court comprises all cases which the parties refer to it and all matters specially provided for in the Charter of the United Nations or in treaties and conventions in force.
        \item The states parties to the present Statute may at any time declare that they recognize as compulsory \textit{ipso facto} and without special agreement, in relation to any other state accepting the same obligation, the jurisdiction of the Court in all legal disputes concerning:
        \begin{enumerate}
            \item the interpretation of a treaty;
            \item any question of international law;
            \item the existence of any fact which, if established, would constitute a breach of an international obligation;
            \item the nature or extent of the reparation to be made for the breach of an international obligation.
        \end{enumerate}
        \item The declarations referred to above may be made unconditionally or on condition of reciprocity on the part of several or certain states, or for a certain time.
        \item Such declarations shall be deposited with the Secretary-General of the United Nations, who shall transmit copies thereof to the parties to the Statute and to the Registrar of the Court.
        \item Declarations made under Article 36 of the Statute of the Permanent Court of International Justice and which are still in force shall be deemed, as between the parties to the present Statute, to be acceptances of the compulsory jurisdiction of the International Court of Justice for the period which they still have to run and in accordance with their terms.
        \item In the event of a dispute as to whether the Court has jurisdiction, the matter shall be settled by the decision of the Court.
    \end{enumerate}
\end{statutedetails}

\begin{itemize}
    \item The ICJ's contentious jurisdiction is based on state consent (this is a core tenet), which can be expressed in a number of ways:
    \begin{itemize}
        \item Special agreement (compromis), per \statute{\textit{ICJ Statute} Article 36(1)}
        \item Jurisdictional clause in a treaty (compromissory clause), per \statute{\textit{ICJ Statute} Article 36(1)} (this is a provision that allows the parties to refer a dispute to the ICJ, but the general treaty is not a compromis as it is not completely about dispute resolution, but rather contains some clauses which enable disputes to be resolved in the ICJ)
        \item Jurisdiction under the optional clause (compulsory jurisdiction), per \statute{\textit{ICJ Statute} Article 36(2)} (this is a declaration deposited by a state that it accepts the ICJ's jurisdiction in all disputes with other states that have deposited the same declaration or similar declarations)
        \item Forum prorogatum (i.e., the parties agree to submit a dispute to the ICJ after the dispute has arisen, per \statute{\textit{ICJ Rules} Article 38(5)})
    \end{itemize}
    \item 74 states have accepted the ICJ's compulsory jurisdiction under \statute{\textit{ICJ Statute} Article 36(2)}, but many include reservations, and so the ICJ's compulsory jurisdiction is not all-encompassing
    \item Australia's 2002 declaration holds as follows (noting that (b) precluded Timor Leste from bringing a case against Australia in the ICJ, but didn't halt conciliation):
    \begin{quote}
        Australia ``declares that it recognises as compulsory ipso facto and without special agreement, in relation to any other State accepting the same obligation, the jurisdiction of the [ICJ] ... This declaration does not apply to: (a) any dispute in regard to which the parties thereto have agreed or shall agree to have recourse to some other method of peaceful settlement; (b) any dispute concerning or relating to the delimitation of maritime zones ... (c) any dispute in respect of which any other party to the dispute has accepted the compulsory jurisdiction of the Court only in relation to or for the purpose of the dispute ..."
    \end{quote}
    \item The \case{\textit{Whaling in the Antarctic Case} [2014] ICJ Rep 226} briefly considered Australia's reservation (in that case, it was held that (b) of Australia's reservation did not apply as the dispute was not about the delimitation of maritime borders; moreover, the ICJ held that Australia's and Japan's declarations and reservations combined comprised the parameters of jurisdiction)
\end{itemize}

\begin{table}[H]
    \centering
    \begin{tabular}{|p{0.2\textwidth}|p{0.2\textwidth}|p{0.2\textwidth}|p{0.2\textwidth}|}
        \hline & \textbf{When consent is given} & \textbf{The scope of the consent given} & \textbf{How proceedings are commenced} \\\hline
        \textbf{Special Agreement} (\textit{compromis}) & After the dispute has arisen & A specific dispute & Bilaterally \\\hline
        \textbf{\textit{Forum Prorogatum}} & After the dispute has arisen & A specific dispute & Unilaterally \\\hline
        \textbf{Compromissory Clause} & Before the dispute has arisen & Usually a category of disputes & Unilaterally \\\hline
        \textbf{Optional Clause} & Before the dispute has arisen & Disputes generally, subject to reservations & Unilaterally \\\hline
    \end{tabular}
    \caption{Types of Jurisdiction of the ICJ}
    \label{tab:icj-jurisdiction-types}
\end{table}

\begin{itemize}
    \item Under compromissory jurisdiction, states cannot take disputes over different categories (e.g., if a compromissory clause is made over technology rights, then one cannot be made over human rights, as it is a different category)
    \item Under optional jurisdiction, the subject matter jurisdiction/jurisdiction \textit{ratione materiae} is broad
\end{itemize}

\subsubsection{Reciprocity of Jurisdiction}

\begin{itemize}
    \item Reservations to the ICJ's jurisdiction have a reciprocal nature; the ICJ will always look at the declarations of both countries, and determine the overlap between them (i.e., which set of parameters satisfies both declarations)
    \item E.g., if Country A accepts 100\% of the ICJ's jurisdiction, and Country B accepts 50\% of the ICJ's jurisdiction, then the ICJ will only have jurisdiction over the 50\% that both countries accept
\end{itemize}

\begin{casedetails}{\textit{Nicaragua v United States} [1984] ICJ Rep 392}
    \flushleft
    ``Declarations of acceptance of the compulsory jurisdiction of the Court are facultative, unilateral engagements, that States are absolutely free to make or not to make. In making the declaration a State is equally free either to do so unconditionally and without limit of time for its duration, or to qualify it with conditions or reservations.''

    \vspace{\baselineskip}

    This case affirmed that declarations and reservations can be done in terms of subject matter and other issues (e.g., time); it is completely up to the state to determine the parameters of its acceptance of the ICJ's jurisdiction, and the ICJ will not question the validity of those reservations or declarations unless it is fundamentally incompatible with the purpose of the ICJ.
\end{casedetails}

\begin{casedetails}{\textit{Norwegian Loans Case (France v Norway)} [1957] ICJ Rep 9}
    \flushleft
    ``The French Declaration accepts the Court's jurisdiction within narrower limits than the Norwegian Declaration; consequently the common will of the Parties, which is the basis of the Court's jurisdiction, exists within these narrower limits indicated by the French reservation."

    \vspace{\baselineskip}

    The ICJ noted here that the French reservation accepts the ICJ's jurisdiction in a narrower manner than the Norwegian declaration, and so the common will of the parties is found through the narrower French reservation.
\end{casedetails}

\begin{casedetails}{\textit{Interhandel Case (Switzerland v United States)} [1959] ICJ Rep 6}
    \flushleft
    In this case, a Swiss company was caught up in war measures in the US, and the Swiss government sued the US in a diplomatic protection's claim. The US sought to rely on certain reservations in the Swiss acceptance of the ICJ's jurisdiction.

    \begin{quote}
        Reciprocity enables the State which has made the wider acceptance of the jurisdiction of the Court to rely upon the reservations to the acceptance laid down by the other Party. There the effect of reciprocity ends. It cannot justify a State, in this instance the United States, in relying upon a restriction which the other Party, Switzerland, has not included in its own Declaration.
    \end{quote}
\end{casedetails}

\subsubsection{Existence of a Dispute}
\begin{casedetails}{\textit{Marshall Islands Case} [2016] ICJ Rep 833}
    \flushleft
    Here, the ICJ held that even if the existence of jurisdiction is satisfied, the existence of a dispute must also be satisfied.

    \vspace{\baselineskip}

    Here, the Marshall Islands contended that the UK and other states were in breach of their nuclear disarmament obligations. The Marshall Islands was the subject of substantial nuclear testing with long-standing effects. The UK contended that the Marshall Islands had brought the case before it could even be established that there was a dispute or what the parameters of the dispute were.

    \vspace{\baselineskip}
    
    The existence of a dispute is a condition of the ICJ's jurisdiction, and it must be showed that the claim of a party is positively opposed by the other. Prior negotiations were not required where jurisdiction was established under \statute{\textit{ICJ Statute} Article 36(2)} (unless one of the relevant declarations required it). However, whether a dispute exists is a matter for objective determination, with the ICJ considering any bilateral or multilateral exchanges between parties (without this, jurisdiction cannot be positively established).

    \vspace{\baselineskip}

    Here, the ICJ found that there was no evidence that the Marshall Islands had made a concrete claim such that it crystallised a dispute between the parties; whilst they had for a period of time raised general concerns, they had never raised their case prior to the submission of their application. On the President's casting vote (the ICJ was otherwise tied 8-8), the ICJ held that there was no dispute present, and hence they did not have jurisdiction to hear the case.

    \tcbsubtitle{Dissenting Opinion of Crawford J}
    In his dissenting opinion, Crawford J held that there was no basis for an objective awareness test for the existence of a dispute, and that the ICJ has traditionally exercised flexibility in determining the existence of a dispute. He held that the dispute here should have been characterised as a multilateral dispute, relying on \case{\textit{South West Africa (Preliminary Objections)}} as authority that disputes may crystallise in multilateral fora. From this, he held that there was an incident dispute between the Marshall Islands and the respondent, given that the Marshall Islands had associated itself with one side of a multilateral disagreement with nuclear weapons states. 
\end{casedetails}

\subsection{Admissibility}
\begin{itemize}
    \item Even when the ICJ has jurisdiction, it may still be unable to hear a case if it is inadmissible, which can arise for a number of reasons:
    \begin{itemize}
        \item An indispensable third party is not present
        \item The dispute is moot or hypothetical
        \item The dispute is non-justiciable (i.e., it is a political question, rather than a legal dispute - this is very rare)
        \item The applicant does not have standing (i.e., there is no injury, the nationality of claims rule is breached, the rule on exhaustion of local remedies is breached, etc.)
    \end{itemize}
\end{itemize}

\subsubsection{Indispensable Third Party}
\begin{itemize}
    \item A claim may be inadmissible as a third party integral to the proceeding of a case is not present
    \item This doctrine was set out in \case{\textit{Monetary Gold (Italy v France; UK and US)} [1954] ICJ Rep 19}, and as such, this doctrine is known as the \textit{Monetary Gold principle}
\end{itemize}
\begin{casedetails}{\textit{Monetary Gold (Italy v France; UK and US)} [1954] ICJ Rep 19}
    \flushleft
    This case was set out to recover gold that was said to be owned by Albania to Italy, and had been recovered in Germany at the end of World War II. It was a complex dispute, as Italy wanted access to the gold in satisfaction of the reparations it was owed by Albania, the UK wanted access by way of reparations for interests suffered in the \case{\textit{Corfu Channel Case} [1949] ICJ Rep 4} (where UK ships struck mines in Albanian waters).

    \vspace{\baselineskip}

    The Court held that whilst the claims were not invalid, Albania was missing from the proceedings. Since the ICJ's jurisdiction operates on a consensual basis, it could not decide on a case involving Albania if Albania was missing from the proceedings and had not consented to jurisdiction otherwise, as Albania was a central party of this case. The Court also stated that:

    \begin{quote}
        ``Where ... the vital issue to be settled concerns ... a third State, the Court cannot, without the consent of that third State, give a decision.''
    \end{quote}
\end{casedetails}

\begin{casedetails}{\textit{East Timor (Portugal v Australia)} [1955] ICJ Rep 6}
    \flushleft
    This case, which can be described as the equivalent of \case{\textit{Horta v Commonwealth} (1994) 181 CLR 183} (Page \pageref{case:Horta v Commonwealth}), saw Portugal seeking to take action against Australia, holding that the treaty that had been negotiated between Australia and Indonesia regarding the Timor Gap was invalid. The ICJ held that Portugal and Australia had accepted the jurisdiction of the ICJ, but Indonesia had not, and as such, the validity of the treaty could not be determined as it would then require the ICJ to determine the lawfulness of Indonesia's 1975 invasion of East Timor, which was a matter that could not be determined without Indonesia's consent. As such, the ICJ held that it could not hear the case, as Indonesia was an indispensable third party given its interests would constitute the core of the dispute.

    \begin{quote}
        In this case, the effects of the judgment ... would amount to a determination that Indonesia's entry into and continued presence in East Timor are unlawful ... Indonesia's rights and obligations would thus constitute the very subject-matter of such a judgment made in the absence of that State's consent.
    \end{quote}
\end{casedetails}

\begin{casedetails}{\textit{Nauru v Australia} [1992] ICJ Rep 240}
    \flushleft
    Nauru initiated legal proceedings against Australia at the International Court of Justice (ICJ), seeking reparations for environmental damage caused by phosphate mining on its territory. Australia objected to the court's jurisdiction, arguing that the United Kingdom and New Zealand, two other administering authorities of Nauru, were not parties to the case due to their reservations about the court's jurisdiction. The ICJ ultimately determined that it had jurisdiction, as the resolution of the dispute did not require passing judgment on the rights or obligations of the UK or New Zealand, which were not vital to the case. While a decision might have implications for the legal situation of these two states, their interests were not central to the subject matter of Nauru's application, allowing the court to proceed without their involvement.
\end{casedetails}

\subsubsection{Non-Justiciable Dispute}
\begin{casedetails}{\textit{Legal Consequences of the Construction of a Wall in Occupied Palestinian Territory} [2004] ICJ 136}
    \flushleft
    ``Given its responsibilities as the ``principal judicial organ of the UN" (Article 92 of the \convention{\textit{UN Charter}}), the Court should in principle not decline to give an advisory opinion. In accordance with its consistent jurisprudence, only ``compelling reasons" should lead the Court to refuse its opinion [citations omitted].''

    \vspace{\baselineskip}

    Here, the ICJ has interpreted its role as the principal judicial organ of the UN to mean that it should not decline to give a judgement/advisory opinion on the basis that a dispute has some political aspects; so long as there is a legal question to be answered, the ICJ will answer it, and set aside the political aspect of it.
\end{casedetails}

\subsection{Provisional Measures}
\begin{statutedetails}{\textit{Statute of the ICJ} Article 41}
    \flushleft
    \begin{enumerate}
        \item The Court shall have the power to indicate, if it considers that circumstances so require, any provisional measures which ought to be taken to preserve the respective rights of either party.
        \item Pending the final decision, notice of the measures suggested shall forthwith be given to the parties and to the Security Council.
    \end{enumerate}
\end{statutedetails}

\begin{itemize}
    \item Provisional measures are akin to interim/interlocutory relief to protect the rights of the parties prior to the final determination of the substance of the case (they are generally granted to prevent irreparable harm to the parties)
    \item Provisional measures are governed under \statute{\textit{ICJ Statute} Article 41}
    \item Provisional measures orders are legally binding, per \case{\textit{La Grand (Germany v United States)} [2001] ICJ Rep 466}
\end{itemize}

\begin{casedetails}{\textit{La Grand (Germany v United States)} [2001] ICJ Rep 466}
    \flushleft
    This case reinforced that provisional measures are legally binding and that states must comply, as opposed to a case of states being encouraged to comply. This case related to the \convention{\textit{Vienna Convention on Consular Relations}}, which includes a compromissory clause, where disputes on this convention can be referred to the ICJ. The US withdrew its acceptance of the ICJ's jurisdiction halfway through the \case{\textit{Nicaragua case}}. \case{\textit{La Grand}} did not involve an optional clause, but instead involved a compromissory clause.

    \vspace{\baselineskip}

    A lot of foreign nationals had been executed in the US; in this instance, there were two German nationals (who were born in Germany but had moved to the US when they were little) who were involved in a bank robbery that had gone very wrong. They were sentenced to death, and Germany took up a diplomatic protection claim against the US, claiming that it had violated its obligation under the \convention{\textit{Vienna Convention on Consular Relations}} to allow Germany to provide consular assistance (and hence, they had not been provided proper legal representation). Pending their execution, Germany sought provisional measures before the ICJ to prevent the executions, and the ICJ granted this. The US executive claimed they could not enforce this, as they were unable to order the state of Arizona (which was holding the German nationals).
\end{casedetails}

\begin{itemize}
    \item Key preconditions that give rise to the ICJ issuing provisional measures include:
    \begin{itemize}
        \item \Gls{prima facie} jurisdiction (i.e., it is not conclusively satisfied that the ICJ has jurisdiction, but there is a reasonable basis to believe that it does; if the court later holds that it does not have jurisdiction, the provisional measures will immediately cease to have effect, but will continue to hold until that time)
        \item Existence of a dispute
        \item Applicant must have standing
        \item The rights to be protected are plausible and are linked to the requested measures
        \item Risk of irreparable damage (i.e., there is the chance that any damage that occurs before the ICJ can hear the case will not be reparable)
        \item Urgency (i.e., there is a real and imminent risk that irreparable prejudice will be caused to the rights claimed)
        \begin{itemize}
            \item Provisional measures can only be granted where there is a real and imminent risk that rights will be violated
        \end{itemize}
    \end{itemize}
\end{itemize}

\begin{casedetails}{\textit{Application of the Convention on the Prevention and Punismnent of Crime of Genocide in the Gaza Strip (South Africa v Israel)} [2024] ICJ Rep 1}
    \flushleft
    \begin{itemize}
        \item The ICJ has now issued three sets of provisional measures
    \end{itemize}
    \tcbsubtitle{Provisional Measures 26 January 2024}
    In 2024, the court determined that it possessed prima facie jurisdiction over the case, recognising the existence of a dispute under the Genocide Convention. The court further established that South Africa had standing to bring the case, as all states share an erga omnes obligation to uphold the provisions of the Geneva Convention. This obligation underscores the collective responsibility of all states to ensure compliance with the Genocide Convention, reflecting a common interest in its enforcement.

    \vspace{\baselineskip}
    
    The court also addressed the issue of provisional measures, which may be ordered to preserve the rights in question, provided those rights are plausible. South Africa sought to protect the plausible rights of Palestinians in Gaza to be safeguarded from acts of genocide. While the court found no plausible proof of genocide occurring, it held that South Africa had plausible rights to seek protection for the affected population, thereby satisfying the requirement of plausible rights for the imposition of provisional measures.

    \vspace{\baselineskip}
    
    Furthermore, the court considered the risk of irreparable prejudice to the claimed rights, determining that there was a real and imminent risk that such prejudice would occur. This assessment was based on the catastrophic humanitarian situation in the Gaza Strip, which the court noted was at serious risk of deteriorating further before a final judgment could be rendered.

    \tcbsubtitle{Provisional Measures 28 March 2024}
    
    South Africa requested a modification of the provisional measures, citing a change in the situation regarding Israel's conduct in Gaza. The court determined that a modification of provisional measures requires a demonstrated change in the situation, a condition that was met due to the deterioration of conditions in Gaza.

    \vspace{\baselineskip}

    The court expressed regret, observing that ``the catastrophic living conditions of the Palestinians in the Gaza Strip have deteriorated further, in particular in view of the prolonged and widespread deprivation of food and other basic necessities to which the Palestinians in the Gaza Strip have been subjected". In light of this, the court indicated the need for additional provisional measures to address the worsening circumstances.
    
    \tcbsubtitle{Provisional Measures 24 May 2024}
    South Africa requested a modification of the provisional measures on the basis of changes in the situation. They held that ``the humanitarian situation is now to be characterised as disastrous ...'' The ICJ stated that:
    \begin{quote}
        The Court considers that, in conformity with its obligations under the Genocide Convention, Israel must immediately halt its military offensive, and any other action in the Rafah Governorate, which may inflict on the Palestinian group in Gaza conditions of life that could bring about its physical destruction in whole or in part
    \end{quote}
\end{casedetails}

\subsection{Advisory Jurisdiction}
\begin{itemize}
    \item Usually, the ICJ does not have jurisdiction to issue opinions on issues, which are known as advisory opinions
    \item However, the ICJ can provide an advisory opinion on any legal question at the request of authorised UN bodies (e.g., the UNGA, UNSC, and some other UN bodies)
    \item Advisory opinions are not binding, but they tend to be extremely influential
    \item Subjects of advisory opinions include:
    \begin{itemize}
        \item Questions regarding UN law, per \case{\textit{Certain Expenses of the United Nations} [1962] ICJ Rep 151}
        \item Decolonisation, per \case{\textit{Chagos Islands} [2019] ICJ Rep 95}
        \item Unlawful occupation, per \case{\textit{Legal Consequences Arising from the Policies and Practices of Israel in the Occupied Palestinian Territory} [2024]}
        \item Climate change, per \case{\textit{Obligations of States in Respect of Climate Change} [pending]}
    \end{itemize}
\end{itemize}