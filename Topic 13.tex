\begin{conventiondetails}{\textit{UN Charter} Article 2(3)}
    \flushleft
    All Members shall settle their international disputes by peaceful means in such a manner that international peace and security, and justice, are not endangered.
\end{conventiondetails}

\begin{conventiondetails}{\textit{UN Charter} Article 33}
    \flushleft
    \begin{enumerate}
        \item The parties to any dispute, the continuance of which is likely to endanger the maintenance of international peace and security, shall, first of all, seek a solution by negotiation, enquiry, mediation, conciliation, arbitration, judicial settlement, resort to regional agencies or arrangements, or other peaceful means of their own choice.
        \item The Security Council shall, when it deems necessary, call upon the parties to settle their dispute by such means.
    \end{enumerate}
\end{conventiondetails}

\begin{itemize}
    \item \convention{\textit{UN Charter} Article 33} lists the various options made available for states to peacefully resolve their disputes
    \item ``A dispute may be defined as a specific disagreement concerning a matter of fact, law or policy in which a claim or assertion of one party is met with refusal, counter-claim or denial by another'' (John Merrills); from this, it is important for any set of facts of be exactly clear, and for the nature of the dispute to be defined, as without this, parties are unable to utilise dispute resolution methods
\end{itemize}

\begin{table}[H]
    \centering
    \begin{tabular}{|p{0.12\textwidth}|p{0.12\textwidth}|p{0.12\textwidth}|p{0.12\textwidth}|p{0.12\textwidth}|p{0.12\textwidth}|}
        \hline
        \textbf{Negotiation} & \textbf{Mediation} & \textbf{Conciliation} & \textbf{Inquiry} & \textbf{Arbitration\footnote{Ad-hoc and institutional}} & \textbf{Judicial Settlement} \\\hline
        Informal (unstructured) & \multicolumn{3}{l|}{More formal (more structured)} & \multicolumn{2}{p{0.25\textwidth}|}{Formal (highly structured)} \\\hline 
        Resolution by the parties themselves & \multicolumn{3}{p{0.35\textwidth}|}{Resort to third party for guidance} & \multicolumn{2}{l|}{Resort to third party for determination} \\\hline 
        \multicolumn{5}{|l|}{Flexible (high degree of control by the parties)} & Fixed (less control by the parties) \\\hline
        \multicolumn{4}{|l|}{Non-binding (advisory)} & \multicolumn{2}{l|}{Binding (mandatory)} \\\hline
        \multicolumn{4}{|p{0.4\textwidth}|}{Resolution of dispute according to the criteria agreed to by the parties (`diplomatic method of dispute settlement')} & \multicolumn{2}{p{0.3\textwidth}|}{Resolution of dispute according to the law (`legal method of dispute settlement')} \\\hline
        Most common & \multicolumn{3}{p{0.35\textwidth}|}{Much less common} & \multicolumn{2}{p{0.25\textwidth}|}{Historically uncommon; recently more common} \\\hline
    \end{tabular}
    \caption{Summary of dispute resolution methods in international law}
    \label{tab:dispute-resolution-methods}
\end{table}

\section{Negotiation}
\begin{itemize}
    \item This is used more frequently than any other method of dispute resolution in international law
    \item A state will negotiate with another state to resolve a dispute, and this is often done through diplomatic channels
    \item Parties themselves retain maximum control over the process and outcome
    \item Negotiation may be used prior to, during, or after a dispute arises, and may be used once a dispute is resolved (in order to agree what the meaning of the court's decision is and how it should be implemented)
    \item Modes of negotiation include through normal diplomatic channels, through high-level officials, through international forums/organisations, etc.
    \item Negotiation may sometimes be a procedural precondition for jurisdiction of an international court (i.e., some treaties will require negotiation as a precondition to formal dispute settlement by a court or tribunal)
\end{itemize}

\begin{casedetails}{\textit{Ukraine v Russia} [2019] ICJ Rep 558}
    \flushleft
    This case concerned alleged violations by Russia of the 1999 \textit{International Convention for the Suppression of the Financing of Terrorism} and the 2003 \textit{International Convention on the Elimination of All Forms of Racial Discrimination}, following events in eastern Ukraine and the invasion of Crimea. There were objections set out in the preliminary issues stage of dispute resolution.

    \tcbsubtitle{Judgement on Preliminary Objections in 2019}

    Under CERD, Art. 22, ``a State [must] make a genuine attempt to settle through negotiation the dispute in question with the other State''. As such, a court will not have jurisdiction until it can be established that the parties have sought to negotiate the dispute. The Court then described what was required under Art. 22:

    \begin{quote}
        Negotiations are distinct from mere protests or disputations. Negotiations entail more than the plain opposition of legal views or interests between two parties, or the existence of a series of accusations and rebuttals, or even the exchange of claims and directly opposed counter-claims.
    \end{quote}

    \begin{quote}
        [Negotiation] requires — at the very least — a genuine attempt by one of the disputing parties to engage in discussions with the other disputing party, with a view to resolving the dispute.
    \end{quote}

    \begin{quote}
        The precondition of negotiation is met only when there has been a failure of negotiations [tried and don't get anywhere], or when negotiations have become futile or deadlocked.
    \end{quote}

    Under the last paragraph, if a state attempts to negotiate but is met by resistance from the other state, then the precondition of negotiation is met, and the court can exercise jurisdiction. In this case, the Court held that the exchanges of \textit{notes verbales} (diplomatic notes) between the parties which included specific mentions of the CERD, negotiations over two years (including face to face meetings), and the breakdown/deadlocking of those negotiations by the time that Ukraine had filed its application meant that the ICJ had jurisdiction here.

    \tcbsubtitle{Judgement on Merits (January 2024)}
    The Russian Federation was found to have violated its obligations under both the 1999 Convention and the CERD as a result of its failure to maintain Ukrainian language education in Crimea.
\end{casedetails}

\begin{casedetails}{\textit{South West Africa} [1962] ICJ 319}
    \flushleft
    South West Africa (which was a German colony) was placed under the mandate of South Africa by the League of Nations from 1915 (this became Namibia in 1990). Ethiopia and Liberia brought a case against South Africa in the ICJ, alleging that South Africa's racist policy of apartheid violated its obligations under the League of Nations mandate to manage and protect South West Africa.

    \vspace{\baselineskip}

    This case held that it was sufficient for parties to hold discussions at international organisations, and that this was sufficient to satisfy the requirement of negotiation. South Africa held that any dispute with Ethiopia and Liberia was outside the scope of the ICJ's jurisdiction, because of the supposed lack of direct negotiation. On the basis that there had been extensive discussions on this issue within the UN on the topic of South West Africa, the ICJ rejected South Africa's argument.

    \vspace{\baselineskip}

    In the second phase of these proceedings, the ICJ in \case{[1966] ICJ Rep 6} found that (by the casting vote), the applicants did not have any standing, and in an Advisory Opinion (\case{[1971] ICJ Rep 16}), found that the continued presence of South Africa in Namibia was illegal.

    \vspace{\baselineskip}

    Hence, to determine if negotiation is a precondition, look at the instrument conferring jurisdiction on the court (e.g., a relevant treaty, or a treaty, or something entailing compulsory acceptance of a jurisdiction).
\end{casedetails}

\section{Mediation}
\begin{itemize}
    \item Mediation is a process that involves a third party (the mediator) who assists in reaching a settlement (the parties remain in control over the process)
    \item Mediators might be invited by the parties to propose a solution, or otherwise might take a more engaged role in the process
    \item Third parties may perform various functions, from mere `good offices' (communication channels between the parties) to proposing a solution (i.e., conciliation)
    \item Mediators can include other governments, international organisations (e.g., the UN), private individuals (e.g., Pope John Paul II in the Beagle Channel dispute), NGOs (e.g., the International Committee of the Red Cross), or other parties
    \item Parties can choose anyone to be a mediator, depending on their agreement
\end{itemize}

\section{Inquiry}
\begin{itemize}
    \item Inquiry is a process that involves an objective, disinterested assessment of the evidence and finding of facts (i.e., it is a factually focused method of dispute settlement)
    \begin{itemize}
        \item It is not necessarily concerned with the legal characteristics of the dispute, but rather resolves differences between facts held by the parties
    \end{itemize}
    \item Sometimes, it can be a precursor to other forms of dispute settlement, but at other times can be standalone; in either case, the parties retain control of the parameters of how the process is conducted
    \item Inquiry is governed by the \convention{\textit{1899 Hague Convention for the Pacific Settlement of International Disputes}}, which were concluded in response to the Spanish-American War
    \begin{itemize}
        \item Articles 9-14 hold that inquiry requires special agreement between the parties, its procedure is determined by the parties, it is limited to statements of fact, and it is left to the parties to determine the effect of the proceedings and how its results are to be used
    \end{itemize}
\end{itemize}

\begin{conventiondetails}{\textit{1899 Hague Convention for the Pacific Settlement of International Disputes} Article 9}
    \flushleft
    In differences of an international nature involving neither honour nor vital interests, and arising from a difference of opinion on points of fact, the Signatory Powers recommend that the parties, who have not been able to come to an agreement by means of diplomacy, should, as far as circumstances allow, institute an International Commission of Inquiry, to facilitate a solution of these differences by elucidating the facts by means of an impartial and conscientious investigation.
\end{conventiondetails}

\begin{conventiondetails}{\textit{1899 Hague Convention for the Pacific Settlement of International Disputes} Article 10}
    \flushleft
    The International Commissions of Inquiry are constituted by special agreement between the parties in conflict.

    \vspace{\baselineskip}

    The Convention for an inquiry defines the facts to be examined and the extent of the Commissioners' powers.

    \vspace{\baselineskip}

    It settles the procedure.

    \vspace{\baselineskip}

    On the inquiry both sides must be heard.

    \vspace{\baselineskip}

    The form and the periods to be observed, if not stated in the Inquiry Convention, are decided by the Commission itself.
\end{conventiondetails}

\begin{conventiondetails}{\textit{1899 Hague Convention for the Pacific Settlement of International Disputes} Article 11}
    \flushleft
    The International Commissions of Inquiry are formed, unless otherwise stipulated, in the manner fixed by Article 32 of the present Convention.
\end{conventiondetails}

\begin{conventiondetails}{\textit{1899 Hague Convention for the Pacific Settlement of International Disputes} Article 12}
    \flushleft
    The Powers in dispute engage to supply the International Commission of Inquiry, as fully as they may think possible, with all means and facilities necessary to enable it to be completely acquainted with and to accurately understand the facts in question.
\end{conventiondetails}

\begin{conventiondetails}{\textit{1899 Hague Convention for the Pacific Settlement of International Disputes} Article 13}
    \flushleft
    The International Commission of Inquiry communicates its Report to the conflicting Powers, signed by all the members of the Commission.
\end{conventiondetails}

\begin{conventiondetails}{\textit{1899 Hague Convention for the Pacific Settlement of International Disputes} Article 14}
    \flushleft
    The Report of the International Commission of Inquiry is limited to a statement of facts, and has in no way the character of an Arbitral Award. It leaves the conflicting Powers entire freedom as to the effect to be given to this statement.
\end{conventiondetails}

\begin{casedetails}{\textit{Dogger Bank Inquiry} (1904)}
    \flushleft
    This case involved Russian warships steaming from the Baltic to the Russo-Japanese War. It encountered British fishing vessels, and attacked and sank them, believing to be Japanese torpedo boats. France persuaded Britain and Russia to establish a Commission of Inquiry under the \convention{\textit{1899 Hague Convention}}, which found that the Russian Admiral had fired in error, but not ``of a nature to cast any discredit on [his] military qualities ... or humanity''.

    \vspace{\baselineskip}

    This case, and the broader time period it was in, is symbolic of the prominence of dispute settlement entities (e.g., mediators) trying to find a quasi-settlement resolution to disputes, as opposed to the modern tendency to turn to international courts.
\end{casedetails}

\section{Conciliation}
\begin{itemize}
    \item Conciliation is a process that involves a third party (the conciliator) who investigates the dispute and proposes a solution to the parties
    \item The involvement of a third party makes it more formal, and gives conciliation an institutional role
    \item Its functions are to investigate the dispute and recommend terms of settlement (i.e., it is investigatory and recommendatory)
    \item It is usually a confidential process, and the recommendations of the conciliator are provided as a draft report to which parties can respond to; a statement of reasons is also issued by the conciliator to the parties
    \item Conciliation may consider non-legal factors (i.e., it is not confined only to applying international law)
    \item \convention{\textit{1928 General Act for the Pacific Settlement of International Disputes} Articles 1-15} holds that there must be compulsory conciliation for all disputes between parties to the Convention if the issue is not resolved by diplomacy
    \item The \convention{\textit{1948 Pact of Botogá} Articles XV-XXX} holds that conciliation is one method of dispute settlement, as well as setting out the requirements for a panel of conciliators, the rules of procedure, and the publication of the report or summary thereof of the conciliatory proceedings
    \item The \convention{\textit{1928 UN Convention on the Law of the Sea} (UNCLOS)} contains provisions around conciliation, as explored in \case{\textit{Australia-Timor Leste Conciliation} (2016)}
\end{itemize}

\begin{casedetails}{\textit{Australia-Timor Leste Conciliation} (2016)}
    \flushleft
    See \case{\textit{Horta v Commonwealth} (1994) 181 CLR 183} on Page \pageref{case:Horta v Commonwealth} for the background to this case.

    \vspace{\baselineskip}

    Once Timor Leste had gained independence from Indonesia, they had to resolve their maritime boundary in the Timor Sea. Initially, they agreed to continue the Timor Gap treaty that had been enforced between Australia and Indonesia (which was the subject of \case{\textit{Horta v Commonwealth} (1994) 181 CLR 183}). However, in 2016, Timor Leste argued that Australia was in breach of several conventions in the treaty, and that Australia was under an obligation to enter into a negotiated agreement on the maritime boundaries.

    \vspace{\baselineskip}

    Australia tried to avoid dispute settlement in a number of ways. It first submitted an adjusted declaration accepting the jurisdiction of the ICJ to exclude maritime boundary disputes (as this was done prior to the crystallisation of the dispute, it prevented Timor Leste from taking action in the ICJ). Additionally, Australia submitted a reservation under \convention{\textit{UNCLOS}} opting out of \convention{\textit{UNCLOS}}'s compulsory dispute resolution processes. \convention{\textit{UNCLOS}} provides for optional (Article 284) and compulsory (Article 298) conciliation. Under \convention{\textit{UNCLOS} Article 298}, a party can opt out of Part XV compulsory settlement for certain disputes, including maritime boundary disputes, but compulsory conciliation was preserved, and as such, Timor Leste could go to compulsory conciliation, which they did in 2016.

    \vspace{\baselineskip}

    This was the first-ever conciliation held under \convention{\textit{UNCLOS} Article 298 and Annex V}. Annex V and the Rules of Procedure adopted by the commission held that the parties are largely in control of the process. Moreover, they agreed on the constitution of the commission, and that among other things, the commission would deal with issues of jurisdiction in an adversarial and adjudicative manner. The substantive phase of the proceedings was more flexible, informal and agile (if jurisdiction could be established, then a diplomatic means of dispute resolution could be used), thereby ensuring the confidentiality of proceedings (subject to certain exceptions in the interest of transparency).

    \vspace{\baselineskip}

    Australia unsuccessfully challenged the competence of the Commission. As such, the Commission then engaged with all of the issues, including the political and economic factors at stake, and not only the issues of maritime boundary delineation. Australia then engaged and determined a Maritime Boundary treaty in 2018.
\end{casedetails}

\section{Arbitration}
\begin{itemize}
    \item This is tending towards a more formalised method of dispute settlement
    \item It involves a third party, it is a legal form of dispute settlement (i.e., the arbitrators (generally an odd number of them to prevent deadlocks) will apply international law to create \textbf{awards} (which are binding decisions on the parties))
    \item Arbitration can be pursued in many different ways, and there are two key types of it:
    \begin{itemize}
        \item Inter-state (i.e., government vs government)
        \begin{itemize}
            \item This is a legal form of dispute settlement (there are independent arbitrators, who apply international law and make a legally binding decision)
            \item This may be ad-hoc (i.e., to resolve a specific dispute, and so the arbitration tribunal is established only when there is a specific issue), or institutional (i.e., arbitration is established as a standing, go-to system of dispute resolution)
            \item An example includes the \case{\textit{South China Sea Arbitration} (2016)}
        \end{itemize}
        \item Individual/corporation vs state (e.g., investment arbitration)
        \begin{itemize}
            \item Bilateral investment treaties (BITs) often provide for arbitration between an individual/corporation and a state, which is a legal form of dispute settlement
            \item An example is the \case{\textit{Philip Morris v Australia} (2015)} case under the Hong Kong-Australia BIT, which concerned Australia's plain packaging laws for tobacco products; Philip Morris challenged this but was ultimately unsuccessful
        \end{itemize}
    \end{itemize}
\end{itemize}

\section{Judicial Settlement}
\begin{itemize}
    \item Judicial settlement is the most formal process of dispute settlement
    \item It is institutionalised (i.e., consists of permanent courts and tribunals) by definition, and as opposed to arbitration (where the parties can choose the arbitrators as they want), turning to judicial settlement means accepting the current compositions of the courts as they stand at that point in time
    \item Judicial settlement consists of fair and impartial judges who render a decision according to law (i.e., a legal situation)
    \item This is an open process, and the decisions are binding upon the parties
    \item However, the decisions are not binding on non-parties of that case, and so there is no system of precedent
    \item From \statute{\textit{ICJ Statute} Article 38(1)(d)}, judgements of international courts and tribunals may influence the development of international law
    \item There are a range of international courts and tribunals, including:
    \begin{itemize}
        \item The International Court of Justice (ICJ)
        \item The International Criminal Court (ICC)
        \item The International Tribunal for the Law of the Sea (ITLOS)
        \item World Trade Organization (WTO) Dispute Settlement Body
        \item Regional courts (e.g., the European Court of Human Rights, the European Court of Justice)
        \item International criminal tribunals
        \item Human rights courts
    \end{itemize}
\end{itemize}

\section{International Court of Justice}

\subsection{Jurisdiction}

\subsection{Admissibility}

\subsection{Provisional Measures}

\subsection{Advisory Jurisdiction}