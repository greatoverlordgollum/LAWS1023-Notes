\section{Development, Nature and Scope of Public International Law}

\section{Sources of Public International Law}
\begin{enumerate}
    \item Is the document a source of public international law?
    \begin{enumerate}
        \item The ICJ is the principal judicial organ of the UN
        \item Under Art 38(1) of the \convention{\textit{1969 Vienna Convention on the Law of Treaties}} (Page \pageref{ICJ Statute Art 38f}), the sources of public international law are:
        \begin{enumerate}
            \item Treaties
            \item Custom
            \item General principles of law
            \item Judicial decisions and the teachings of publicists
        \end{enumerate}
        \item Art 38(1) is `generally regarded as a complete statement of the sources of international law' -- \case{\textit{Ure v Commonwealth} (2016) 329 ALR 452 at [15]} (Page \pageref{case:Ure v Commonwealth})
        \item Did the state consent to be bound to the jurisdiction of the ICJ/to the terms of the source?
    \end{enumerate}
    \item Was the source a treaty?
    \begin{enumerate}
        \item Under Art 38(1)(a) of the \statute{\textit{ICJ Statute}}, treaties are a source of international law
        \item Was the document a bilateral and/or multilateral convention between two or more states?
        \item See Topic 3 scaffolds (\ref{scaffold:Topic 3} on Page \pageref{scaffold:Topic 3}) for a detailed analysis
        \item Note that if there are a number of instances of states contravening a treaty, it is not necessary for the states to provide consistently correct conduct and that some variations in practice are acceptable, and they do not form a new rule -- \case{\textit{Military and Paramilitary Activities in and against Nicaragua} [1986] ICJ Rep 14}
    \end{enumerate}
    \item Was there international custom involved?
    \begin{enumerate}
        \item Under Art 38(1)(b) of the \statute{\textit{ICJ Statute}}, custom is a source of international law, and requires two elements: -- \case{\textit{North Sea Continental Shelf Cases (Germany v Denmark; Germany v Netherlands)} [1969] ICJ Rep 3} (Page \pageref{case:North Sea Continental Shelf})
        \begin{enumerate}
            \item State practice - objective evidence that the custom is practiced widely
            \item \textit{Opinio juris} - the belief that the practice is legally required
        \end{enumerate}
        \item Unless they are persistent objectors, all states are bound to customary international law
        \item Certain customary norms are \textit{jus cogens}, which are peremptory norms from which no derogation is permitted
        \item Was there state practice?
        \begin{enumerate}
            \item State practice can be evidenced by materials that demonstrate the activities and views of states and state officials
            \item State practice can generate custom if the following requirements are met: -- \case{\textit{North Sea Continental Shelf Cases} (1969) ICJ Rep 3}
            \begin{enumerate}
                \item The practice was consistent over time (but not necessarily entirely uniform) -- \case{\textit{Military and Paramilitary Activities in and against Nicaragua} [1986] ICJ Rep 14} (Page \pageref{case:Military and paramilitary activities in Nicaragua})
                \item The practice was widespread
                \item The practice was representative of multiple states (especially those who are most likely affected by it)
                \item The practice was developed over a lengthy period of time (this is not a steadfast requirement; customary norms may still emerge rapidly if there is an overwhelming practice of it)
            \end{enumerate}
            \item State practice can be shown through government legal opinions, treaty provisions and conduct in connection with resolutions that the country makes -- UN \textit{Draft conclusions on identification of customary international law} Conclusion 10(2)
            \item Under Conclusion 10(3), failure to react over time to a practice may serve as evidence of acceptance provided that the State was in a position to react and the circumstances called for some reaction
        \end{enumerate}
        \item Was there \textit{opinio juris}?
        \begin{enumerate}
            \item \textit{Opinio juris} refers to the belief that the practice is legally required
            \item \textit{Opinio juris} is hard to show, and generally can be shown through statements made by countries (\case{\textit{North Sea Continental Shelf Cases (Germany v Denmark; Germany v Netherlands)} [1969] ICJ Rep 3}), although it can also be shown through an omission of a state, which evinces a belief that the said State is obligated by law to refrain from acting in a particular way (\case{\textit{The Lotus Case (France v Turkey)} PCIJ Series A No 10})
            \item If there is extensive state practice, then \textit{opinio juris} tends to be less important, and vice-versa
        \end{enumerate}
        \item It is possible for treaty norms to become custom, and for treaty provisions to become customary international law -- \case{\textit{North Sea Continental Shelf Cases} (1969) ICJ Rep 3}
        \begin{enumerate}
            \item However, the custom exists independently of the treaty -- \case{\textit{Military and Paramilitary Activities in and against Nicaragua} [1986] ICJ Rep 14}
        \end{enumerate}
    \end{enumerate}
    \item Was there regional custom involved?
    \begin{enumerate}
        \item The ICJ has recognised that it is possible for regional custom to exist, but invoking it requires a higher standard than general international custom -- \case{\textit{Asylum Case (Colombia v Peru)} [1950] ICJ Rep 226}
        \begin{itemize}
            \item Regional custom must have a higher degree of stability and continuity to apply as international law in that area -- \case{\textit{Asylum Case (Colombia v Peru)} [1950] ICJ Rep 226}
            \item Such an example was made out in the English Court of Appeal in \case{\textit{R (app. Al-Saadoon v Sec. of Defence)} [2010] 1 All ER 271}, where rules of regional custom were found to exist, but had not met the high threshold to be invoked (Page \pageref{case:R (Al-Saadoon v Sec. of Defence)})
        \end{itemize}
    \end{enumerate}
    \item If there was custom involved, was the party a persistent objector?
    \begin{enumerate}
        \item The doctrine of a persistent objector is fairly narrow, and enunciates that states which consistently object to the emergence of a rule from its earliest point of gestation will not be bound by it -- \case{\textit{Anglo Norwegian Fisheries Case (UK v Norway)} [1951] ICJ Rep 116}
        \item A state cannot be a persistent objector to a \textit{jus cogens} principle -- \article{\textit{International Law Commission 2019 Report} Chapter V Conclusion 14} (Page \pageref{report:2019 ILC Conc. 14})
    \end{enumerate}
    \item Was the source a general principle of international law?
    \begin{enumerate}
        \item Under Art 38(1)(c) of the \statute{\textit{ICJ Statute}}, general principles of law recognised by civilised nations form a source of PIL, with the objective of avoiding the \textit{non liquet} (the situation where `it is not clear' by enabling the ICJ to look at different legal systems for inspiration)
        \item General principles of international law and municipal law are included in this provision
        \item General principles of law may be implicitly adopted in judicial decisions to enable a conclusion to be made -- \case{\textit{Bay of Bengal (Bangladesh/Myanmar)} [2012] ILTOS 12}
        \item For example, various domestic legal systems were examined in relation to the issue of estoppel to aid the Tribunal in its decision -- \case{\textit{Chagos Marine Protected Area Arbitration (Mauritius v United Kingdom)} (2015) XXXI RIAA 359}
    \end{enumerate}
    \item Was there a judicial decision and/or the teachings of a publicist?
    \begin{enumerate}
        \item Whilst Art 38(1)(d) of the \statute{\textit{ICJ Statute}} enables judicial decisions and the work of publicists to be considered as sources of PIL, they are subsidiary means for the determinations of the rules of law, and are treated as having lesser significance than other sources
        \item Decisions taken by the ICJ do not constitute binding precedent in future decisions, but remain merely persuasive -- \statute{\textit{Statute of the International Court of Justice} Art 59}
        \begin{enumerate}
            \item These sources are ``resorted to by judicial tribunals not for the
            speculations of their authors concerning what the law ought to be, but for trustworthy
            evidence of what the law really is'' -- \case{\textit{The Paquete Habana} 175 US 677 (1900)}
        \end{enumerate}
        \item Was this a UN General Assembly Resolution?
        \begin{enumerate}
            \item The UN General Assembly (UNGA) is the plenary body of the UN, and as all UN members have a seat, it has become a great forum for state practice and \textit{opinio juris}
            \item Decisions of the UNGA are not binding, except in the key areas of (without these areas, the UN could not function):
            \begin{enumerate}
                \item Admission of member states
                \item Suspension of member states
                \item Matters related to the UN budget
            \end{enumerate}
            \item Resolutions of the UNGA can provide evidence for state practice
            \item UNGA resolutions can influence international law in three key ways
            \begin{enumerate}
                \item Interpreting the \textit{Charter of the United Nations}
                \item Affirming recognised customary norms (through passing a resolution)
                \item Influencing the creation of new customary norms
            \end{enumerate}
            \item UNGA resolutions, whilst normally not binding, may have normative value, and can provide ``evidence important for the establishing the existence of a rule or the emergence of a \textit{opinion juris}'' -- \case{\textit{Legality of the Threat or Use of Nuclear Weapons} [1996] ICJ Rep 254 at [70] - [73]} (Page \pageref{case:Legality of Nuclear Weapons [1996] ICJ Rep 254}); such evidence can include:
            \begin{enumerate}
                \item The voting records of the UNGA
                \item Transcripts of what was said on the floor of the UNGA
                \item Margins of the votes undertaken in the UNGA
            \end{enumerate}
        \end{enumerate}
        \item Was this a UN Security Council Resolution?
        \begin{enumerate}
            \item The UN Security Council (UNSC) has limited law-making capacity, but can adopt certain binding resolutions
            \item UNSC resolutions are binding only on the members of the UN -- \statute{\textit{Charter of the United Nations} Art 25}
        \end{enumerate}
    \end{enumerate}
    \item Was there a measure of soft law involved?
    \begin{itemize}
        \item Soft law refers to rules that are binding but vague, and/or `rules' that are clear but are not binding
        \item They can articulate standards/norms that will, over time, become binding, and can also be used to interpret other sources of international law
    \end{itemize}
\end{enumerate}

\section{The Law of Treaties}\label{scaffold:Topic 3}
\begin{enumerate}
    \item Was there a treaty involved?
    \begin{enumerate}
        \item ``Treaty'' means an international agreement concluded between States in written form and governed by international law, whether embodies in a single instrument or in two or more related instruments and whatever its particular designation -- \convention{\textit{VCLT} Art 2(1)(a)} (Page \pageref{VCLT Art 2})
        \item It has been accepted that a treaty may be written across multiple documents -- \case{\textit{Maritime Delimitation and Territorial Questions (Qatar v Bahrain)} (1994) ICJ Rep 112} (Page \pageref{case:Qatar v Bahrain})
        \item A unilateral declaration can be considered to have binding effect -- \case{\textit{Nuclear Test Cases (Australia v France)} (1974) ICJ Rep 253}
        \begin{enumerate}
            \item ``An undertaking ... if given publicly with an intent to be bound, even though not made within the context of international negotiations, is binding''
            \item This case outlines four key characteristics for a unilateral declaration to be binding:
            \begin{enumerate}
                \item The undertaking is made publicly 
            \end{enumerate}
        \end{enumerate}
    \end{enumerate}
    \item Does the \convention{\textit{1969 Vienna Convention on the Law of Treaties} (\textit{VCLT})} apply?
    \begin{enumerate}
        \item Was the treaty between two or more states? -- \convention{\textit{VCLT Art 3}} (Page \pageref{VCLT Art 3})
        \begin{enumerate}
            \item Under Article 3, the VCLT does not influence agreements between states and other subjects or between other subjects of international law
        \end{enumerate}
        \item Was the treaty in writing? -- \convention{\textit{VCLT Art 3}} (Page \pageref{VCLT Art 3})
        \begin{enumerate}
            \item The VCLT applies to written treaties only, but as many of its provisions are now customary law, those provisions may still apply to non-written treaties (see Table \ref{tab:VCLT Articles that can apply as customary international law} on Page \pageref{tab:VCLT Articles that can apply as customary international law})
        \end{enumerate}
        \item Had the treaty commenced after 1980 (when the \convention{\textit{VCLT}} entered into force)?
        \begin{itemize}
            \item As the \convention{\textit{VCLT}} entered into force in 1980, it only applies to treaties concluded after 1980, but many of its provisions can apply to treaties concluded before 1980 as provisions of general international law
        \end{itemize}
    \end{enumerate}
    \item Was the treaty registered with the United Nations?
    \begin{itemize}
        \item A treaty must be registered with the UN in order to be used as a binding instrument in proceedings before the UN -- \convention{\textit{Charter of the United Nations} Art 102}; \convention{\textit{VCLT} Art 80} (Page \pageref{convention:UN Charter Art 102})
        \item This is not a requirement for a treaty to be binding in general, but is a requirement for the treaty to be recognised before the UN
    \end{itemize}
    \item Was the treaty signed by an appropriate authority/representative?
    \begin{enumerate}
        \item Was the party entering into the treaty a state, an international organisation or an international entity with capacity to enter into the treaty?
        \begin{enumerate}
            \item Every state possesses capacity to conclude treaties -- \convention{\textit{VCLT} Art 6} (Page \pageref{VCLT Art 6})
        \end{enumerate}
        \item Has the individual representing the party produced full powers evincing their authority to enter into the treaty? -- \convention{\textit{VCLT} Art 7(1)(a)} (Page \pageref{VCLT Art 2})
        \begin{enumerate}
            \item ``Full powers'' refers to a document emanating from the competent authority of a State designating a person or persons to represent the State for negotiating, adopting or authenticating the text of a treaty, for expressing the consent of the State to be bound by a treaty, or for accomplishing any other act with respect to a treaty -- \convention{\textit{VCLT} Art 2(1)(d)} (Page \pageref{VCLT Art 2})
            \item Heads of State, Heads of Government and Ministers of Foreign Affairs are taken to have the capacity to conclude treaties without producing full powers -- \convention{\textit{VCLT} Art 7(2)(a)} (Page \pageref{VCLT Art 2})
            \item Heads of diplomatic missions will likewise not need to produce full powers if they are accredited to adopt treaties in that area -- \convention{\textit{VCLT} Art 7(2)(b)} (Page \pageref{VCLT Art 2})
            \item A representative of a state will not need to produce full powers if they have been sent to a conference/organisation with the purpose of adopting the text of a treaty at that conference/organisation -- \convention{\textit{VCLT} Art 7(2)(c)} (Page \pageref{VCLT Art 2})
        \end{enumerate}
        \item If the individual has not produced full powers, is it evident from the practice of the States concerned or from other circumstances that the person is representing the State? -- \convention{\textit{VCLT} Art 7(1)(b)} (Page \pageref{VCLT Art 2})
    \end{enumerate}
    \item Did the state enter into the treaty?
    \begin{enumerate}
        \item Signing is a two step process, entailing signature, and either ratification or accession
        \item Upon \textbf{signing} a treaty, the state expresses a willingness to continue the treaty-making process, and agrees with the treaty in principle
        \begin{enumerate}
            \item However, the state is not bound by the treaty at this point
        \end{enumerate}
        \item If the treaty is a new one, was it \textbf{ratified} by the party?
        \begin{enumerate}
            \item Upon ratification, the party indicates that it has consented to be bound by the treaty once it enters into force
        \end{enumerate}
        \item If the treaty is an existing one, was it \textbf{accessioned} by the party?
        \begin{enumerate}
            \item This only applies if a state is becoming party to a treat that is already negotiated and signed by other states
            \item This has the same legal effect as ratification
        \end{enumerate}
    \end{enumerate}
    \item Was the treaty in force at the time of contention?
    \begin{enumerate}
        \item A treaty enters into force in accordance with the relevant provisions in the treaty -- \convention{\textit{VCLT} Art 24(1)} (Page \pageref{VCLT Art 24})
        \item If the treaty is silent on this point, it will enter into force when all parties have consented to being bound by it -- \convention{\textit{VCLT} Art 24(2)} (Page \pageref{VCLT Art 24})
        \item If a party signs a treaty after its formation, it will be binding upon that state on the day that consent to being bound is established -- \convention{\textit{VCLT} Art 24(3)} (Page \pageref{VCLT Art 24})
    \end{enumerate}
    \item Does the treaty apply to the present scenario?
    \begin{enumerate}
        \item The principle of \textit{pacta sunt servanda} requires that ``every treaty in force is binding upon the parties to it, and must be performed by them in good faith'' -- \convention{\textit{VCLT} Art 26} (Page \pageref{VCLT Art 26})
        \item A party may not invoke the provisions of its internal law as justification for failing to perform its obligations -- \convention{\textit{VCLT} Art 27} (Page \pageref{VCLT Art 27})
        \begin{enumerate}
            \item However, they may do so if the other party was aware of that law, and the law was not contrary to the treaty -- \convention{\textit{VCLT} Art 46} (Page \pageref{VCLT Art 46})
        \end{enumerate}
        \item If the treaty has been signed but not ratified/approved/accepted, a state is obliged to not undermine the spirit of the treaty, and moreover is required to refrain from acts that would defeat the object and purpose of the treaty -- \convention{\textit{VCLT} Art 18(a)} (Page \pageref{VCLT Art 18})
        \begin{itemize}
            \item The same principle also applies where a state has expressed its consent to be bound by the treaty, pending the entry into force of the treaty -- \convention{\textit{VCLT} Art 18(b)} (Page \pageref{VCLT Art 18})
        \end{itemize}
        \item Treaties do not impose obligations or create rights for third states in the absence of their consent (\textit{pacta tertiss nex nocent nec prosunt}) -- \convention{\textit{VCLT} Art 34} (Page \pageref{VCLT Art 34})
    \end{enumerate}
    \item Was there any reservation to the treaty?
    \begin{enumerate}
        \item Was there a reservation or an interpretive declaration?
        \begin{enumerate}
            \item A reservation is a unilateral statement, however phrased or named, made by a State, when signing, ratifying, accepting, approving or acceding to a treaty, whereby it purports to exclude or to modify the legal effect of certain provisions of the treaty in their application to that State -- \convention{\textit{VCLT} Art 2(1)(d)} (Page \pageref{VCLT Art 2})
            \item Interpretive declarations are statements made by a state to clarify its understanding of a treaty; it does not affect the legal effect of a treaty
        \end{enumerate}
        \item Was the reservation permissible?
        \begin{enumerate}
            \item By default, a reservation is permissible, unless: -- \convention{\textit{VCLT} Art 19} (Page \pageref{VCLT Art 19})
            \begin{enumerate}
                \item The reservation is prohibited by the treaty -- \convention{\textit{VCLT} Art 19(a)} (Page \pageref{VCLT Art 19})
                \item The treaty provides that only specified reservations may be made and the reservation in question is not in that list -- \convention{\textit{VCLT} Art 19(b)} (Page \pageref{VCLT Art 19})
                \item The reservation is otherwise incompatible with the object and purpose of the treaty -- \convention{\textit{VCLT} Art 19(c)} (Page \pageref{VCLT Art 19})
            \end{enumerate}
            \item Incompatibility hinges on whether it “affects an essential element of the treaty that is necessary to its general tenor, in such a way that the reservation impairs the raison d'être [the most important reason] of the treaty” -- \article{\textit{ILC Guide to Practice on Reservations} Art 3.1.5}
            \item If a reservation is impermissible:
            \begin{enumerate}
                \item Traditionally, this vitiates the consent of the state to the treaty as a whole and results in the state not being a party to the treaty (this is the predominant view) -- \case{\textit{Reservations to Genocide Convention} [1951] ICJ Rep 15}
                \item More recently, the offending reservation will be held void, with the state being bound without the protection of the reservation (i.e., it is cut out), unless consent is conditional on reservation, in which case they are not bound to the treaty at all
            \end{enumerate}
        \end{enumerate}
        \item Was the reservation accepted or objected to? -- \convention{\textit{VCLT} Art 20} (Page \pageref{VCLT Art 20})
        \begin{enumerate}
            \item If a treaty expressly allows for reservations, then no acceptance of a reservation is required by other parties -- \convention{\textit{VCLT} Art 20(1)} (Page \pageref{VCLT Art 20})
            \item If a treaty has a small number of parties and the application of the treaty in its entirety is an essential condition of signing, acceptance by all parties is required -- \convention{\textit{VCLT} Art 20(2)} (Page \pageref{VCLT Art 20})
            \item If a treaty is a constituent instrument of an international organisation, and unless it otherwise provides, a reservation requires the acceptance of the competent organ of that organisation -- \convention{\textit{VCLT} Art 20(3)} (Page \pageref{VCLT Art 20})
            \item Acceptance by the other contracting state(s) of the reservation results in the reserving state being bound by the treaty (with the reservation incorporated) -- \convention{\textit{VCLT} Art 20(4)(a)} (Page \pageref{VCLT Art 20})
            \item Objection to a reservation does not prevent entry into force of a treaty between the objecting state and the reserving state, unless the objecting state says otherwise -- \convention{\textit{VCLT} Art 20(4)(b)} (Page \pageref{VCLT Art 20})
            \item An act indicating consent to being bound by the treaty that contains a reservation is effective as soon as at least one other state has accepted the reservation -- \convention{\textit{VCLT} Art 20(4)(c)} (Page \pageref{VCLT Art 20})
        \end{enumerate}
        \item What is the legal effect of the reservation?
        \begin{enumerate}
            \item If State A accepts State R's reservation, then the treaty is modified only between States A and R, to the the extent of the reservation -- \convention{\textit{VCLT} Art 21(1) and (2)} (Page \pageref{VCLT Art 21}); \case{\textit{Republic of India v CCDM Holdings, LLC} [2025] FCAFC 2 at [63]} (Page \pageref{case:India v CCDM})
            \begin{enumerate}
                \item However, other states will not be bound by this reservation; it acts as a side agreement between State A and State R -- \convention{\textit{VCLT} Art 21(2)} (Page \pageref{VCLT Art 21})
            \end{enumerate}
            \item If State B objects to State R's reservation, and says the treaty is not to apply, then there is no treaty between them at all -- \convention{\textit{VCLT} Art 20(4)(b)} (Page \pageref{VCLT Art 20})
            \item If State C objects to State R's reservation but does not say that the treaty is not to apply, then the treaty applies, but “the provisions to which the reservation applies do not apply ... to the extent of the reservation” -- \convention{\textit{VCLT} Art 21(3)} (Page \pageref{VCLT Art 21})
        \end{enumerate}
        \item Was the state a persistent objector?
        \begin{enumerate}
            \item States which consistently object to the emergence of a rule of custom from its earliest point of gestation will not be bound by it -- \case{\textit{Anglo Norwegian Fisheries Case (UK v Norway)} [1951] ICJ Rep 116} (Page \pageref{case:UK v Norway Fisheries})
            \item A state cannot be a persistent objector to a \textit{jus cogens} principle -- \article{\textit{International Law Commission 2019 Report} Chapter V Conclusion 14} (Page \pageref{report:2019 ILC Conc. 14})
        \end{enumerate}
    \end{enumerate}
    \item How was the treaty interpreted by the state?
    \begin{enumerate}
        \item There are a number of different approaches to treaty interpretation:
        \begin{enumerate}
            \item Formalist/Textual (formal adherence to the terms of the treaty)
            \item Restrictive (deference to state sovereignty)
            \item Teleological (to give effect to the object and purpose of the treaty)
            \item Effectiveness (to ensure the treaty regime remains as effective as possible)
            \item Originalist (to focus on the original purpose of the treaty)
        \end{enumerate}
        \item The Australian courts will apply the VCLT when interpreting a treaty that has been incorporated into Australian law -- \case{\textit{DHI22 v Qatar Airways} [2024] FCA 348} (see Section \ref{sec:Interpretation of Treaties} on Page \pageref{sec:Interpretation of Treaties})
        \item The VCLT contains rules on how to interpret treaties -- \convention{\textit{VCLT} Art 31} (Page \pageref{VCLT Art 31})
        \item Under the VCLT, instruments used in treaty interpretation must have been adopted by all states -- \case{\textit{Whaling in the Antarctic Case} [2014] ICJ Rep 226 at [83]} (Page \pageref{case:Antarctic Whaling})
        \item As a last resort, supplementary means of interpretation can be used to interpret the provisions of a treaty under Art 31 -- \convention{\textit{VCLT} Art 32} (Page \pageref{VCLT Art 32})
    \end{enumerate}
    \item Is the treaty void or otherwise invalidated?
    \begin{enumerate}
        \item Is the treaty void?
        \begin{enumerate}
            \item If the State's representative had been coerced into entering the treaty, or there were acts or threats directed against the representative, a State's consent will not be made out and so the treaty will be void -- \convention{\textit{VCLT} Art 51} (Page \pageref{VCLT Art 51})
            \item If a State's consent was obtained through a threat or the use of force, it is void -- \convention{\textit{VCLT} Art 51} (Page \pageref{VCLT Art 52})
            \item If a treaty conflicts with a \textit{jus cogens} norm, it is void -- \convention{\textit{VCLT} Art 51} (Page \pageref{VCLT Art 53})
            \item If a new \textit{jus cogens} norm has emerged since the ratification of a treaty and the treaty conflicts with that \textit{jus cogens} norm, the treaty is void -- \convention{\textit{VCLT} Art 51} (Page \pageref{VCLT Art 64})
        \end{enumerate}
        \item Is the treaty invalid?
        \begin{enumerate}
            \item Did the state's consent to a treaty involve a violation of an internal law of fundamental importance? -- \convention{\textit{VCLT} Art 46(1)} (Page \pageref{VCLT Art 46})
            \begin{enumerate}
                \item A state may not invoke inconsistent internal law as a basis on which it could not sign a treaty, unless that rule is of manifest importance
            \end{enumerate}
            \item If a representative of a state had gone beyond what he was authorised to do so in signing the treaty, their omission to observe their limitations will not constitute an invalidation of the treaty, unless the restriction was notified to other states prior to the expression of consent -- \convention{\textit{VCLT} Art 47} (Page \pageref{VCLT Art 47})
        \end{enumerate}
        \item Was there an error of fact that formed the essential basis of consent? -- \convention{\textit{VCLT} Art 48} (Page \pageref{VCLT Art 48})
        \begin{enumerate}
            \item Consent may be validated by means of an error if the error relates to a fact or situation assumed by the state that existed at the time when the treaty was concluded, and forms an essential basis of its consent to be bound by the treaty -- \convention{\textit{VCLT} Art 48(1)} (Page \pageref{VCLT Art 48})
            \item An error of fact cannot be plead by a party if they contributed to it, could have avoided it, or were otherwise put on notice of a possible error -- \convention{\textit{VCLT} Art 48(2)} (Page \pageref{VCLT Art 48}); \case{\textit{Temple of Preah Vihear (Cambodia v Thailand)} [1962] ICJ Rep 6}
            \item If there is an error relating to only the wording of the treaty's text, its validity is not affected, and \convention{Art 79} is enlivened -- \convention{\textit{VCLT} Art 48(3)} (Page \pageref{VCLT Art 48})
        \end{enumerate}
        \item Had the state been induced to conclude the treaty by the fraudulent conduct of another state? -- \convention{\textit{VCLT} Art 49} (Page \pageref{VCLT Art 49})
    \end{enumerate}
    \item Are there grounds to terminate, withdraw or suspend the treaty? \\\vspace{8pt}
    \textit{The following constitute internal grounds of termination, withdrawal or suspension.}
    \begin{enumerate}
        \item Was the treaty terminated or withdrawn from under:
        \begin{enumerate}
            \item Its provisions? -- \convention{\textit{VCLT} Art 54(a)} (Page \pageref{VCLT Art 54})
            \item By consent of all of the parties after consultation with the other contracting states? -- \convention{\textit{VCLT} Art 54(b)} (Page \pageref{VCLT Art 54})
        \end{enumerate}
        \item Was the treaty suspended under:
        \begin{enumerate}
            \item Its provisions? -- \convention{\textit{VCLT} Art 57(a)} (Page \pageref{VCLT Art 57})
            \item By consent of all of the parties after consultation with the other contracting states? -- \convention{\textit{VCLT} Art 57(b)} (Page \pageref{VCLT Art 57})
        \end{enumerate}
    \end{enumerate}
    \textit{The following constitute external grounds of termination, withdrawal or suspension.}
    \begin{enumerate}[resume]
        \item Was there a denunciation or withdrawal from the treaty when there is no provision to do so? -- \convention{\textit{VCLT} Art 56} (Page \pageref{VCLT Art 56})
        \begin{enumerate}
            \item There is generally no right of denunciation, except where: -- \convention{\textit{VCLT} Art 56(1)} (Page \pageref{VCLT Art 56})
            \begin{enumerate}
                \item It is established that the parties intended to admit the possibility of denunciation or withdrawal -- \convention{\textit{VCLT} Art 56(1)(a)} (Page \pageref{VCLT Art 56})
                \item A right of denunciation or withdrawal may be implied by the nature of the treaty -- \convention{\textit{VCLT} Art 56(1)(b)} (Page \pageref{VCLT Art 56})
            \end{enumerate}
            \item Under this provision, a party must give at least 12 months' notice of its intention to denounce/withdrawal from the treaty -- \convention{\textit{VCLT} Art 56(2)} (Page \pageref{VCLT Art 56})
        \end{enumerate}
        \item Was there a material breach? -- \convention{\textit{VCLT} Art 60} (Page \pageref{VCLT Art 60})
        \begin{enumerate}
            \item A material breach involves a wrongful act being intentionally committed by a party
            \item In a bilateral treaty, this entitles the other party to terminate the treaty or suspend its operation, in whole or in part -- \convention{\textit{VCLT} Art 60(1)} (Page \pageref{VCLT Art 60})
            \item If there was a breach in a multilateral treaty: -- \convention{\textit{VCLT} Art 60(2)} (Page \pageref{VCLT Art 60})
            \begin{enumerate}
                \item The other parties can unanimously opt to suspend or terminate the treaty either (i) between themselves and the defaulting state, or (ii) between all parties -- \convention{\textit{VCLT} Art 60(2)(a)} (Page \pageref{VCLT Art 60})
                \item A party who has been especially affected has grounds to suspend the treaty in whole or in part between itself and the defaulting state -- \convention{\textit{VCLT} Art 60(2)(b)} (Page \pageref{VCLT Art 60})
                \item Any party other than the defaulting party may suspend the treaty in whole or in part if the breach is such that it radically changes the position of every party with respect to the further performance of its obligations under the treaty -- \convention{\textit{VCLT} Art 60(2)(c)} (Page \pageref{VCLT Art 60})
            \end{enumerate}
            \item Moreover, a material breach entails:
            \begin{enumerate}
                \item A repudiation of the treaty not sanctioned by the present Convention -- \convention{\textit{VCLT} Art 60(3)(a)} (Page \pageref{VCLT Art 60})
                \item The violation of a provision essential to the accomplishment of the object or purpose of this treaty -- \convention{\textit{VCLT} Art 60(3)(b)} (Page \pageref{VCLT Art 60})
            \end{enumerate}
            \item A party cannot claim material breach if they themselves had committed the wrongful act -- \case{\textit{Gabčíkovo-Nagymaros Case} [1997] ICJ Rep 7 at [105]-[110]} (Page \pageref{case:[1997] ICJ Rep 7})
        \end{enumerate}
        \item Did the performance of the treaty become impossible? -- \convention{\textit{VCLT} Art 61} (Page \pageref{VCLT Art 61})
        \begin{enumerate}
            \item A state may terminate or withdraw from a treaty if its performance has become impossible because `an object indispensable for the execution of the treaty' has permanently disappeared or been destroyed -- \convention{\textit{VCLT} Art 61(1)} (Page \pageref{VCLT Art 61})
            \item However, impossibility of performance may not be invoked if the impossibility is the result of a breach by that party either of an obligation under that treaty or any other international obligations owed to any other party of the treaty -- \convention{\textit{VCLT} Art 61(2)} (Page \pageref{VCLT Art 61})
        \end{enumerate}
        \item Was there a fundamental change of circumstances that precluded the operation of the treaty? -- \convention{\textit{VCLT} Art 62} (Page \pageref{VCLT Art 62})
        \begin{enumerate}
            \item A fundamental change of circumstances entails: -- \case{\textit{Gabčíkovo-Nagymaros Case} [1997] ICJ Rep 7 at [105]-[110]} (Page \pageref{case:[1997] ICJ Rep 7})
            \begin{enumerate}
                \item The circumstances at the conclusion of the treaty must have been an essential basis of consent
                \item The change must not have been foreseen
                \item The change must radically transform the extent of the obligations still to be made performed                
            \end{enumerate}
            \item A fundamental change of circumstances may not be invoked as a grounds for termination/withdrawal unless: -- \convention{\textit{VCLT} Art 62(1)} (Page \pageref{VCLT Art 62})
            \begin{enumerate}
                \item The existence of those circumstances constituted an essential basis of the consent of the parties to be bound by the treaty; \textbf{and} -- \convention{\textit{VCLT} Art 62(1)(a)} (Page \pageref{VCLT Art 62})
                \item The effect of the change is radically to transform the extent of the obligations still to be performed under the treaty -- \convention{\textit{VCLT} Art 61(1)(b)} (Page \pageref{VCLT Art 62})
            \end{enumerate}
            \item A fundamental change of circumstances may not be invoked as a ground for terminating or withdrawing from a treaty: -- \convention{\textit{VCLT} Art 62(2)} (Page \pageref{VCLT Art 62})
            \begin{enumerate}
                \item If the treaty establishes a boundary -- \convention{\textit{VCLT} Art 62(2)(a)} (Page \pageref{VCLT Art 62})
                \item If the fundamental change is the result of a breach by the party invoking it -- \convention{\textit{VCLT} Art 62(2)(b)} (Page \pageref{VCLT Art 62})
            \end{enumerate}
            \item If a party invokes a fundamental change of circumstances as a ground for terminating or withdrawing from a treaty, it may also invoke the change as a ground for suspending the operation of the treaty -- \convention{\textit{VCLT} Art 62(3)} (Page \pageref{VCLT Art 62})
            \item International courts are very reluctant to find that impossibility and/or fundamental change of circumstances have been made out (i.e., they have a very high threshold and thus a very limited scope)
        \end{enumerate}
    \end{enumerate}
\end{enumerate}

\section{International Law and Australian Law}
\begin{enumerate}
    \item Was this a matter involving Australia, or a state which follows the Australian approach to adopting international law?
    \item How does international law influence Australian law?
    \begin{enumerate}
        \item International law applies between states, but may also recognise institutions of domestic law that have an extensive/important role in international law (e.g., corporations) -- \case{\textit{Barcelona Traction (Belgium v Spain)} [1970] ICJ Rep 3} [where corporations could be recognised within international law]
        \item Absent/inconsistent domestic law is not excuse for failing to meet international obligations -- \case{\textit{Alabama Claims Arbitration (US/Britain)} (1872)}; \case{\textit{Sandline Arbitration} (1998)}
        \item Expert evidence cannot be adduced to prove or explain statements of international law -- \case{\textit{ACCC v PT Garuda (No 9)} [2013] FCA 23 at [31] (Perram J)} (Page \pageref{case:ACCC v Garuda})
        \item Common law can be developed with regard to international law, where it is not inconsistent with domestic law -- \case{\textit{Chow Hung Ching v R} (1949) 77 CLR 449} (Page \pageref{case:Chow Hung Ching})
        \begin{enumerate}
            \item However, international law is not incorporated as part of Australian law, but rather is a source -- Latham CJ at page 462; Dixon J at page 477
        \end{enumerate}
        \item International law cannot automatically be imported/included in Australian law, but remains a ``legitimate and important influence on the development of the law'' -- \case{\textit{Mabo v Queensland (No 2)} (1992) 175 CLR 1, 42 (Brennan J)} (Page \pageref{case: Mabo})
        \begin{enumerate}
            \item This is especially the case when referring to aspects of international law that touch on universal human values (e.g., human rights)
        \end{enumerate}
        \item However, in cases such as a civil claim for torture (or any other serious crime forbidden under international law), the common law of state doctrine should reflect universal norms -- \case{\textit{Habib v Commonwealth} (2010) 183 CJR 62 (Black CJ)} (Page \pageref{case: Habib v Commonwealth})
    \end{enumerate}
    \item Has customary international law been implemented in Australian law? \\ \textit{Always explore monism and dualism for a nuanced discussion, and describe how this results in Australia's hard transformation approach.}
    \begin{enumerate}
        \item Whilst there is no clear authority, the automatic incorporation of customary international law in Australia has been rejected
        \item Australia has rejected the monism approach (where states and international law form one entity), and has adopted a hard transformation approach that tends towards dualism (where states and international law form two separate entities) (Section \ref{sec:Monism and Dualism} on Page \pageref{sec:Monism and Dualism})
        \begin{enumerate}
            \item Under the \textbf{hard transformation approach}, only legislation may implement the provisions of international law into domestic law; otherwise, international law does not apply (this is the approach favoured in Australia, following \case{\textit{Chow Hung Ching v R} (1949) 77 CLR 449 (Latham CJ at 462; Dixon J at 477)} (Page \pageref{case:Chow Hung Ching}))
            \item The \textbf{soft transformation approach} holds that legislation or court decisions may implement the provisions of international law (discussed by Latham CJ in \case{\textit{Chow Hung Ching v R} (1949) 77 CLR 449} (Page \pageref{case:Chow Hung Ching})), but so far has been rejected -- \case{\textit{Dietrich v R} [1992] HCA 37} (Page \pageref{case: Dietrich v R})
            \begin{enumerate}
                \item ``International law is not as such part of the law of Australia, but a universally recognized principle of international law would be applied by our courts'' -- \case{\textit{Chow Hung Ching v R} (1949) 77 CLR 449, 462 (Latham CJ)} (Page \pageref{case:Chow Hung Ching})
            \end{enumerate}
        \end{enumerate}
        \item If the country is instead following the UK's approach, see Section \ref{sec:Customary International Law in Australian Law} on Page \pageref{sec:Customary International Law in Australian Law}
    \end{enumerate}
    \item Was this a matter involving international criminal law?
    \begin{enumerate}
        \item Customary/international criminal law can never be a part of the Australian common law/transformed or implemented by the courts (it can only be imported by statute) -- \case{\textit{Nulyarimanna v Thompson} (1999) 165 ALR 621} (Page \pageref{case: Nulyarimanna v Thompson})
        \begin{enumerate}
            \item At [20], Wilcox J held that if domestic criminal law could be influenced by customary/international criminal law, it would lead to the position where international obligations have greater obligations than domestic consequences, sidelining domestic law and thus a state's independence to make its own criminal laws
            \item This is moreover a position adopted in the UK -- \case{\textit{R v Jones} [2006] 2 All ER 741 (Lords Bingham, Mance and Hoffman)} (Page \pageref{case: R v Jones})
        \end{enumerate}
        \item \textit{Jus cogens} principles of international law are not automatically part of Australian common law, and criminal offences must be created by statute, not by the courts -- \case{\textit{Nulyarimanna v Thompson} (1999) 165 ALR 621 at [17], [20], [32], [57] (Wilcox and Whitlam JJ)} (Page \pageref{case: Nulyarimanna v Thompson})
        \begin{enumerate}
            \item For example, the \textit{jus cogens} prohibition of genocide was not automatically part of Australian common law, and had to be created by statute (see the above paragraphs for context) 
            \item Merkel J dissented, and held that the approach that should be taken was the `common law adoption approach', which is that a rule of international law is to be adopted by a court so long as it is not inconsistent with legislation or public policy
        \end{enumerate}
    \end{enumerate}
    \item Does the matter involve a treaty being implemented into Australian law?
    \begin{enumerate}
        \item \label{treaty entry executive power} The power to enter into treaties is exclusively an Executive prerogative power -- \statute{\textit{Constitution} s 61} (Page \pageref{Constitution s 61})
        \begin{enumerate}
            \item In interpreting the external affairs power, the HCA has held that they only need to look at whether the law applies geographically externally to Australia, not whether the international law was void by virtue of the underlying treaty being void -- \case{\textit{Horta v Commonwealth} (1994) 181 CLR 183}
            \item ``The federal executive, through the Crown's representative, possessed exclusive and unfettered treaty-making power'' -- \case{\textit{Koowarta v Bjelke-Petersen} (1982) 153 CLR 168 at [215] (Stephen J)} (Page \pageref{case:Koowarta v Bjelke-Petersen})
        \end{enumerate}
        \item \label{treaty legislature powers} The power to implement the provisions of a treaty is a legislative power -- \statute{\textit{Constitution} s 51(xxix)} [the `external affairs' power]
        \begin{enumerate}
            \item Treaty provisions do not form a part of Australia law until they have been implemented by statute -- \case{\textit{Dietrich v R} [1992] HCA 37 (Brennan CJ, Mason and McHugh JJ)} (Page \pageref{case: Dietrich v R})
            \item Resolutions of international organisations (e.g., the UN Security Council) do not form a part of Australian law until they have been implemented by statute -- \case{\textit{Bradley v Commonwealth} (1973) 128 CLR 557} (Page \pageref{case:Bradley v Commonwealth})
            \begin{enumerate}
                \item If Parliament `approves' a treaty, it is not binding
                \item The mere approval of Parliament does not give a treaty the force of law (this practice has since lapsed)
            \end{enumerate}
            \item Under \statute{\textit{Constitution} s 51(xxix)}, the law must carry into effect treaty obligations, and be reasonably considered to be appropriate and adapted to achieving this objective -- \case{\textit{Commonwealth v Tasmania} (1983) 158 CLR 1 (Deane J)} (Page \pageref{case:Commonwealth v Tasmania})
            \item Ratification only occurs after a treaty has been implemented into internal legislative provisions/given the force of law (the legislative approach is the preferred one, as it the most common and avoids uncertainty)
        \end{enumerate}
    \end{enumerate}
    \item Does the matter involve Australia entering into/making a treaty?
    \begin{enumerate}
        \item Australia can enter into two types of treaties:
        \begin{enumerate}
            \item Bilateral treaties, which enter into force for Australia after
            \begin{enumerate}
                \item Signature
                \item Subsequent exchange of notes stating that the constitutional process is completed
            \end{enumerate}
            \item Multilateral treaties, which enter into force for Australia after
            \begin{enumerate}
                \item Signature
                \item Subsequent ratification (or accession if there was no previous signature)
            \end{enumerate}
        \end{enumerate}
        \item \label{item:trick or treaty} Whilst there is no constitutional requirement for the Parliament to be involved in the treaty-making process, since 1996, Parliament has been consulted on the treaty-making process (without a veto) -- \article{\textit{Trick or Treaty?} (1995 Report of the Senate Legal and Constitutional References Committee)}
        \begin{enumerate}
            \item The present convention is that all proposed treaty conventions are tabled in Parliament at least 15 sitting days prior to any binding action being undertake (with exemptions for urgent or sensitive treaties)
            \item A National Interest Analysis (NIA) is also prepared, which is akin to an explanatory memorandum for a treaty
            \item the treaty should also be reviewed by the Joint Standing Committee on Treaties
        \end{enumerate}
    \end{enumerate}
    \item Should international law be used to interpret Australian statute?
    \begin{enumerate}
        \item International law can be used as extrinsic material when interpreting legislation which refers to a treaty -- \statute{\textit{Acts Interpretation Act 1901} (Cth) ss 15AB(1) and (2)(d)} (Page \pageref{Acts Interpretation Act s 15AB})
        \item International law can be used to interpret a legislative provision that incorporates a treaty provision; in this instance, the rules of treaty interpretation apply rather than statutory interpretation
        \item Is the \textit{Polites} principle enlivened?
        \begin{enumerate}
            \item The \textit{Polites} principle refers to the presumption that the Parliament intends to give effect to Australia's obligations under international law, in the absence of express words/intention to the contrary -- \case{\textit{Polites v Commonwealth} (1945) 70 CLR 60} (Page \pageref{case:Polites v Commonwealth})
            \begin{enumerate}
                \item It is a general rule of statutory interpretation that, in the absence of express words to the contrary, it is presumed that legislation is intended to be in conformity with the treaty-based rules of international law
            \end{enumerate}
            \item Once a treaty is ratified and implemented into domestic law, statutory interpretation requires courts to presume that legislation is intended to be in conformity with international law (Polites Principle) because Parliament, \textit{prima facie}, intends to give effect to Australia's international obligations
            \item The \textit{Polites} principle does not apply to constitutional interpretation, as it would violate the requirement for a referendum to modify the \textit{Constitution} under s 128 -- \case{\textit{Al-Kateb v Godwin} (2004) 208 ALR 124} (Page \pageref{case:Al-Kateb v Godwin})
            \begin{enumerate}
                \item At [66], McHugh J outlined that it was never the case that the \statute{\textit{Constitution}} should have been interpreted to conform with the rules of international law
                \item At [175], Kirby J (in dissent) held that the \statute{\textit{Constitution}} should be interpreted in a way that is generally harmonious with the basic principles of international law, including as that law states human rights and fundamental freedoms
            \end{enumerate}
        \end{enumerate}
    \end{enumerate}
    \item How is treaty law implemented into Australian law?
    \begin{enumerate}
        \item Signing of the treaty
        \begin{enumerate}
            \item The Commonwealth Executive has the exclusive power to sign treaties (see \ref{treaty entry executive power})
            \item Parliamentary approval is not necessary, but as a matter of convention, it has been sought (see \ref{treaty legislature powers})
        \end{enumerate}
        \item Implementation of the treaty
        \begin{enumerate}
            \item Parliament will generally be consulted about the treaty, but constitutionally, its approval is not required to sign the treaty (see \ref{treaty legislature powers})
            \item Following the \textit{Trick or Treaty} report (see \ref{item:trick or treaty}), the Executive will consult Parliament before signing treaties
            \item All proposed treaty actions are tabled in Parliament at least 15 sitting days prior to a binding action
            \item Parliament can provide recommendations and scrutinise the treaty, but ultimately the Executive has the final say
        \end{enumerate}
        \item Ratification of the treaty
        \begin{enumerate}
            \item The treaty is ratified by the executive once domestic legislation is present to incorporate the terms of the treaty
            \item Under \convention{\textit{VCLT} Art 2(1)(b)} (Page \pageref{VCLT Art 2}), ratification is the intentional act whereby a state indicates its consent to be bound to a treaty if the parties intended to show their consent (generally, the depositary will collect the ratifications of all states)
        \end{enumerate}
    \end{enumerate}
\end{enumerate}