\begin{tcolorbox}[title = How to Answer a Problem Question]
    \flushleft
    \begin{enumerate}
        \item Read the question
        \begin{enumerate}
            \item Read the question in one go (no notes, no highlighter, just a straight reading)
            \item Read the question again, making notes of the potential issues in the margins and why specific words/provisions are being used
            \item Read the question a third time to ensure nothing has been missed, and whether any issues interact with each other
        \end{enumerate}
        \item Identify how the question will be answered (generally in sequential order)
        \item Make sure to cover IRAC:
        \begin{enumerate}
            \item Issue
            \item Rule
            \item Analysis
            \item Conclusion
        \end{enumerate}
        \item Unless the question is direct, use vague terminology and argue both sides (e.g., `it appears', `it might be the case that', etc.)
        \item It is good to refute arguments where possible and then point to a secondary argument, as this shows depth
        \item If there are facts that distinguish this case from precedent, make sure to engage in a brief but nuanced analysis
        \item Use subheadings to distinguish issues!
    \end{enumerate}
    Look at obligations both under a treaty and under general international law!
\end{tcolorbox}

\section{Development, Nature and Scope of Public International Law}

\section{Sources of Public International Law}
\begin{enumerate}
    \item Is the document a source of public international law?
    \begin{enumerate}
        \item The ICJ is the principal judicial organ of the UN
        \item Under Art 38(1) of the \convention{\textit{Statute of the International Court of Justice}} (Page \pageref{ICJ Statute Art 38}), the sources of public international law are:
        \begin{enumerate}
            \item Treaties
            \item Custom
            \item General principles of law
            \item Judicial decisions and the teachings of publicists
        \end{enumerate}
        \item Art 38(1) is `generally regarded as a complete statement of the sources of international law' -- \case{\textit{Ure v Commonwealth} (2016) 329 ALR 452 at [15] (Perram, Robertson and Moshinsky JJ)} (Page \pageref{case:Ure v Commonwealth})
        \item Did the state consent to be bound to the jurisdiction of the ICJ/to the terms of the source?
    \end{enumerate}
    \item Was the source a treaty?
    \begin{enumerate}
        \item Under Art 38(1)(a) of the \statute{\textit{ICJ Statute}}, treaties are a source of international law
        \item Was the document a bilateral and/or multilateral convention between two or more states?
        \item See Topic 3 scaffolds (\ref{scaffold:Topic 3} on Page \pageref{scaffold:Topic 3}) for a detailed analysis
        \item Note that if there are a number of instances of states contravening a treaty, it is not necessary for the states to provide consistently correct conduct and that some variations in practice are acceptable, and they do not form a new rule -- \case{\textit{Military and Paramilitary Activities in and against Nicaragua} [1986] ICJ Rep 14 [186]}
    \end{enumerate}
    \item Was there international custom involved?
    \begin{enumerate}
        \item Under Art 38(1)(b) of the \statute{\textit{ICJ Statute}}, custom is a source of international law, and requires two elements: -- \case{\textit{North Sea Continental Shelf Cases (Germany v Denmark; Germany v Netherlands)} [1969] ICJ Rep 3} (Page \pageref{case:North Sea Continental Shelf})
        \begin{enumerate}
            \item State practice - objective evidence that the custom is practiced widely
            \item \textit{Opinio juris} - the belief that the practice is legally required
        \end{enumerate}
        \item Unless they are persistent objectors, all states are bound to customary international law
        \item Certain customary norms are \textit{jus cogens}, which are peremptory norms from which no derogation is permitted
        \item Was there state practice?
        \begin{enumerate}
            \item State practice can be evidenced by materials that demonstrate the activities and views of states and state officials
            \item State practice can generate custom if the following requirements are met: -- \case{\textit{North Sea Continental Shelf Cases} (1969) ICJ Rep 3}
            \begin{enumerate}
                \item The practice was consistent over time (but not necessarily entirely uniform) -- \case{\textit{Military and Paramilitary Activities in and against Nicaragua} [1986] ICJ Rep 14} (Page \pageref{case:Military and paramilitary activities in Nicaragua})
                \item The practice was widespread
                \item The practice was representative of multiple states (especially those who are most likely affected by it)
                \item The practice was developed over a lengthy period of time (this is not a steadfast requirement; customary norms may still emerge rapidly if there is an overwhelming practice of it)
            \end{enumerate}
            \item State practice can be shown through government legal opinions, treaty provisions and conduct in connection with resolutions that the country makes -- \convention{UN \textit{Draft conclusions on identification of customary international law} Conclusion 10(2)}
            \item Under Conclusion 10(3), failure to react over time to a practice may serve as evidence of acceptance provided that the State was in a position to react and the circumstances called for some reaction
        \end{enumerate}
        \item Was there \textit{opinio juris}?
        \begin{enumerate}
            \item \textit{Opinio juris} refers to the belief that the practice is legally required
            \item \textit{Opinio juris} is hard to show, and generally can be shown through statements made by countries (\case{\textit{North Sea Continental Shelf Cases (Germany v Denmark; Germany v Netherlands)} [1969] ICJ Rep 3 at [77]}), although it can also be shown through an omission of a state, which evinces a belief that the said State is obligated by law to refrain from acting in a particular way (\case{\textit{The Lotus Case (France v Turkey)} (1927) PCIJ Series A No 10 at page 28})
            \item If there is extensive state practice, then \textit{opinio juris} tends to be less important, and vice-versa
        \end{enumerate}
        \item It is possible for treaty norms to become custom, and for treaty provisions to become customary international law -- \case{\textit{North Sea Continental Shelf Cases} (1969) ICJ Rep 3 at [72]}
        \begin{enumerate}
            \item However, the custom exists independently of the treaty -- \case{\textit{Military and Paramilitary Activities in and against Nicaragua} [1986] ICJ Rep 14 at [25]-[30] and [40]}
        \end{enumerate}
    \end{enumerate}
    \item Was there regional custom involved?
    \begin{enumerate}
        \item The ICJ has recognised that it is possible for regional custom to exist, but invoking it requires a higher standard than general international custom -- \case{\textit{Asylum Case (Colombia v Peru)} [1950] ICJ Rep 226}
        \begin{itemize}
            \item Regional custom must have a higher degree of stability and continuity to apply as international law in that area -- \case{\textit{Asylum Case (Colombia v Peru)} [1950] ICJ Rep 226}
            \item Such an example was made out in the English Court of Appeal in \case{\textit{R (app. Al-Saadoon v Sec. of Defence)} [2010] 1 All ER 271}, where rules of regional custom were found to exist, but had not met the high threshold to be invoked (Page \pageref{case:R (Al-Saadoon v Sec. of Defence)})
        \end{itemize}
    \end{enumerate}
    \item If there was custom involved, was the party a persistent objector?
    \begin{enumerate}
        \item The doctrine of a persistent objector is fairly narrow, and enunciates that states which consistently object to the emergence of a rule from its earliest point of gestation will not be bound by it -- \case{\textit{Anglo Norwegian Fisheries Case (UK v Norway)} [1951] ICJ Rep 116}
        \item A state cannot be a persistent objector to a \textit{jus cogens} principle -- \article{\textit{International Law Commission 2019 Report} Chapter V Conclusion 14} (Page \pageref{report:2019 ILC Conc. 14})
    \end{enumerate}
    \item Was the source a general principle of international law?
    \begin{enumerate}
        \item Under Art 38(1)(c) of the \statute{\textit{ICJ Statute}}, general principles of law recognised by civilised nations form a source of PIL, with the objective of avoiding the \textit{non liquet} (the situation where `it is not clear' by enabling the ICJ to look at different legal systems for inspiration)
        \item General principles of international law and municipal law are included in this provision
        \item General principles of law may be implicitly adopted in judicial decisions to enable a conclusion to be made -- \case{\textit{Bay of Bengal (Bangladesh/Myanmar)} [2012] ILTOS 12} (Page \pageref{case:Bangladesh v Myanmar})
        \item For example, various domestic legal systems were examined in relation to the issue of estoppel to aid the Tribunal in its decision -- \case{\textit{Chagos Marine Protected Area Arbitration (Mauritius v United Kingdom)} (2015) XXXI RIAA 359}
    \end{enumerate}
    \item Was there a judicial decision and/or the teachings of a publicist?
    \begin{enumerate}
        \item Whilst Art 38(1)(d) of the \statute{\textit{ICJ Statute}} enables judicial decisions and the work of publicists to be considered as sources of PIL, they are subsidiary means for the determinations of the rules of law, and are treated as having lesser significance than other sources
        \item Decisions taken by the ICJ do not constitute binding precedent in future decisions, but remain merely persuasive -- \statute{\textit{Statute of the International Court of Justice} Art 59}
        \begin{enumerate}
            \item These sources are ``resorted to by judicial tribunals not for the
            speculations of their authors concerning what the law ought to be, but for trustworthy
            evidence of what the law really is'' -- \case{\textit{The Paquete Habana} 175 US 677 (1900)}
        \end{enumerate}
        \item Was this a UN General Assembly Resolution?
        \begin{enumerate}
            \item The UN General Assembly (UNGA) is the plenary body of the UN, and as all UN members have a seat, it has become a great forum for state practice and \textit{opinio juris}
            \item Decisions of the UNGA are not binding, except in the key areas of (without these areas, the UN could not function):
            \begin{enumerate}
                \item Admission of member states
                \item Suspension of member states
                \item Matters related to the UN budget
            \end{enumerate}
            \item Resolutions of the UNGA can provide evidence for state practice
            \item UNGA resolutions can influence international law in three key ways
            \begin{enumerate}
                \item Interpreting the \textit{Charter of the United Nations}
                \item Affirming recognised customary norms (through passing a resolution)
                \item Influencing the creation of new customary norms
            \end{enumerate}
            \item UNGA resolutions, whilst normally not binding, may have normative value, and can provide ``evidence important for the establishing the existence of a rule or the emergence of a \textit{opinion juris}'' -- \case{\textit{Legality of the Threat or Use of Nuclear Weapons} [1996] ICJ Rep 254 at [70] - [73]} (Page \pageref{case:Legality of Nuclear Weapons [1996] ICJ Rep 254}); such evidence can include:
            \begin{enumerate}
                \item The voting records of the UNGA
                \item Transcripts of what was said on the floor of the UNGA
                \item Margins of the votes undertaken in the UNGA
            \end{enumerate}
        \end{enumerate}
        \item Was this a UN Security Council Resolution?
        \begin{enumerate}
            \item The UN Security Council (UNSC) has limited law-making capacity, but can adopt certain binding resolutions
            \item UNSC resolutions are binding only on the members of the UN -- \statute{\textit{Charter of the United Nations} Art 25} (Page \pageref{UN Charter Art 25})
        \end{enumerate}
    \end{enumerate}
    \item Was there a measure of soft law involved?
    \begin{itemize}
        \item Soft law refers to rules that are binding but vague, and/or `rules' that are clear but are not binding
        \item They can articulate standards/norms that will, over time, become binding, and can also be used to interpret other sources of international law
    \end{itemize}
\end{enumerate}

\section{The Law of Treaties}\label{scaffold:Topic 3}
\begin{enumerate}
    \item Was there a treaty involved?
    \begin{enumerate}
        \item ``Treaty'' means an international agreement concluded between States in written form and governed by international law, whether embodies in a single instrument or in two or more related instruments and whatever its particular designation -- \convention{\textit{VCLT} Art 2(1)(a)} (Page \pageref{VCLT Art 2})
        \item It has been accepted that a treaty may be written across multiple documents -- \case{\textit{Maritime Delimitation and Territorial Questions (Qatar v Bahrain)} (1994) ICJ Rep 112 at [23]} (Page \pageref{case:Qatar v Bahrain})
        \item `Agreed Minutes' (or another document evincing agreement between two or more states) can constitute a treaty --\case{\textit{Maritime Delimitation and Territorial Questions (Qatar v Bahrain)} (1994) ICJ Rep 112} (Page \pageref{case:Qatar v Bahrain}); \case{\textit{Bay of Bengal (Bangladesh/Myanmar)} [2012] ILTOS 12} (Page \pageref{case:Bangladesh v Myanmar})
        \item A unilateral declaration can be considered to have binding effect -- \case{\textit{Nuclear Test Cases (Australia v France)} (1974) ICJ Rep 253 at [43]}
        \begin{enumerate}
            \item ``An undertaking ... if given publicly with an intent to be bound, even though not made within the context of international negotiations, is binding''
            \item This case outlines four key characteristics for a unilateral declaration to be binding:
            \begin{enumerate}
                \item The undertaking is made publicly with an intention to be bound
                \item It must be clear and specific
                \item It can be oral or written
                \item It must be made by someone who is authorised by the state to make a binding conclusion
            \end{enumerate}
        \end{enumerate}
    \end{enumerate}
    \item Does the \convention{\textit{1969 Vienna Convention on the Law of Treaties} (\textit{VCLT})} apply?
    \begin{enumerate}
        \item Was the treaty between two or more states? -- \convention{\textit{VCLT Art 3}} (Page \pageref{VCLT Art 3})
        \begin{enumerate}
            \item Under Article 3, the VCLT does not influence agreements between states and other subjects or between other subjects of international law
        \end{enumerate}
        \item Was the treaty in writing? -- \convention{\textit{VCLT Art 3}} (Page \pageref{VCLT Art 3})
        \begin{enumerate}
            \item The VCLT applies to written treaties only, but as many of its provisions are now customary law, those provisions may still apply to non-written treaties (see Table \ref{tab:VCLT Articles that can apply as customary international law} on Page \pageref{tab:VCLT Articles that can apply as customary international law})
        \end{enumerate}
        \item Had the treaty commenced after 1980 (when the \convention{\textit{VCLT}} entered into force)?
        \begin{itemize}
            \item As the \convention{\textit{VCLT}} entered into force in 1980, it only applies to treaties concluded after 1980, but many of its provisions can apply to treaties concluded before 1980 as provisions of general international law (see Table \ref{tab:VCLT Articles that can apply as customary international law} on Page \pageref{tab:VCLT Articles that can apply as customary international law})
        \end{itemize}
    \end{enumerate}
    \item Was the treaty registered with the United Nations?
    \begin{itemize}
        \item A treaty must be registered with the UN in order to be used as a binding instrument in proceedings before the UN -- \convention{\textit{Charter of the United Nations} Art 102}; \convention{\textit{VCLT} Art 80} (Page \pageref{convention:UN Charter Art 102})
        \item This is not a requirement for a treaty to be binding in general, but is a requirement for the treaty to be recognised before the UN
        \item Registration only needs to be completed by one party
        \item ``Non-registration or late registration, on the other hand, does not have any consequence for the actual validity of the agreement, which remains no less binding upon the parties.'' -- \case{\textit{Maritime Delimitation and Territorial Questions (Qatar v Bahrain)} (1994) ICJ Rep 112 at [29]} (Page \pageref{case:Qatar v Bahrain})
    \end{itemize}
    \item Was the treaty signed by an appropriate authority/representative?
    \begin{enumerate}
        \item Was the party entering into the treaty a state, an international organisation or an international entity with capacity to enter into the treaty?
        \begin{enumerate}
            \item Every state possesses capacity to conclude treaties -- \convention{\textit{VCLT} Art 6} (Page \pageref{VCLT Art 6})
        \end{enumerate}
        \item Has the individual representing the party produced full powers evincing their authority to enter into the treaty? -- \convention{\textit{VCLT} Art 7(1)(a)} (Page \pageref{VCLT Art 2})
        \begin{enumerate}
            \item ``Full powers'' refers to a document emanating from the competent authority of a State designating a person or persons to represent the State for negotiating, adopting or authenticating the text of a treaty, for expressing the consent of the State to be bound by a treaty, or for accomplishing any other act with respect to a treaty -- \convention{\textit{VCLT} Art 2(1)(d)} (Page \pageref{VCLT Art 2})
            \item Heads of State, Heads of Government and Ministers of Foreign Affairs are taken to have the capacity to conclude treaties without producing full powers -- \convention{\textit{VCLT} Art 7(2)(a)} (Page \pageref{VCLT Art 2})
            \item Heads of diplomatic missions will likewise not need to produce full powers if they are accredited to adopt treaties in that area -- \convention{\textit{VCLT} Art 7(2)(b)} (Page \pageref{VCLT Art 2})
            \item A representative of a state will not need to produce full powers if they have been sent to a conference/organisation with the purpose of adopting the text of a treaty at that conference/organisation -- \convention{\textit{VCLT} Art 7(2)(c)} (Page \pageref{VCLT Art 2})
        \end{enumerate}
        \item If the individual has not produced full powers, is it evident from the practice of the States concerned or from other circumstances that the person is representing the State? -- \convention{\textit{VCLT} Art 7(1)(b)} (Page \pageref{VCLT Art 2})
    \end{enumerate}
    \item Did the state enter into the treaty?
    \begin{enumerate}
        \item Signing is a two step process, entailing signature, and either ratification or accession
        \item Upon \textbf{signing} a treaty, the state expresses a willingness to continue the treaty-making process, and agrees with the treaty in principle
        \begin{enumerate}
            \item However, the state is not bound by the treaty at this point
        \end{enumerate}
        \item If the treaty is a new one, was it \textbf{ratified} by the party?
        \begin{enumerate}
            \item Upon ratification, the party indicates that it has consented to be bound by the treaty once it enters into force
        \end{enumerate}
        \item If the treaty is an existing one, was it \textbf{accessioned} by the party?
        \begin{enumerate}
            \item This only applies if a state is becoming party to a treat that is already negotiated and signed by other states
            \item This has the same legal effect as ratification
        \end{enumerate}
    \end{enumerate}
    \item Was the treaty in force at the time of contention?
    \begin{enumerate}
        \item A treaty enters into force in accordance with the relevant provisions in the treaty -- \convention{\textit{VCLT} Art 24(1)} (Page \pageref{VCLT Art 24})
        \item If the treaty is silent on this point, it will enter into force when all parties have consented to being bound by it -- \convention{\textit{VCLT} Art 24(2)} (Page \pageref{VCLT Art 24})
        \item If a party signs a treaty after its formation, it will be binding upon that state on the day that consent to being bound is established -- \convention{\textit{VCLT} Art 24(3)} (Page \pageref{VCLT Art 24})
    \end{enumerate}
    \item Does the treaty apply to the present scenario?
    \begin{enumerate}
        \item The principle of \textit{pacta sunt servanda} requires that ``every treaty in force is binding upon the parties to it, and must be performed by them in good faith'' -- \convention{\textit{VCLT} Art 26} (Page \pageref{VCLT Art 26})
        \item A party may not invoke the provisions of its internal law as justification for failing to perform its obligations -- \convention{\textit{VCLT} Art 27} (Page \pageref{VCLT Art 27})
        \begin{enumerate}
            \item However, they may do so if the other party was aware of that law, and the law was not contrary to the treaty -- \convention{\textit{VCLT} Art 46} (Page \pageref{VCLT Art 46})
        \end{enumerate}
        \item If the treaty has been signed but not ratified/approved/accepted, a state is obliged to not undermine the spirit of the treaty, and moreover is required to refrain from acts that would defeat the object and purpose of the treaty -- \convention{\textit{VCLT} Art 18(a)} (Page \pageref{VCLT Art 18})
        \begin{itemize}
            \item The same principle also applies where a state has expressed its consent to be bound by the treaty, pending the entry into force of the treaty -- \convention{\textit{VCLT} Art 18(b)} (Page \pageref{VCLT Art 18})
        \end{itemize}
        \item Treaties do not impose obligations or create rights for third states in the absence of their consent (\textit{pacta tertiss nex nocent nec prosunt}) -- \convention{\textit{VCLT} Art 34} (Page \pageref{VCLT Art 34})
    \end{enumerate}
    \item Was there any reservation to the treaty?
    \begin{enumerate}
        \item Was there a reservation or an interpretive declaration?
        \begin{enumerate}
            \item A reservation is a unilateral statement, however phrased or named, made by a State, when signing, ratifying, accepting, approving or acceding to a treaty, whereby it purports to exclude or to modify the legal effect of certain provisions of the treaty in their application to that State -- \convention{\textit{VCLT} Art 2(1)(d)} (Page \pageref{VCLT Art 2})
            \item Interpretive declarations are statements made by a state to clarify its understanding of a treaty; it does not affect the legal effect of a treaty
        \end{enumerate}
        \item Was the reservation permissible?
        \begin{enumerate}
            \item By default, a reservation is permissible, unless: -- \convention{\textit{VCLT} Art 19} (Page \pageref{VCLT Art 19})
            \begin{enumerate}
                \item The reservation is prohibited by the treaty -- \convention{\textit{VCLT} Art 19(a)} (Page \pageref{VCLT Art 19})
                \item The treaty provides that only specified reservations may be made and the reservation in question is not in that list -- \convention{\textit{VCLT} Art 19(b)} (Page \pageref{VCLT Art 19})
                \item The reservation is otherwise incompatible with the object and purpose of the treaty -- \convention{\textit{VCLT} Art 19(c)} (Page \pageref{VCLT Art 19})
            \end{enumerate}
            \item Incompatibility hinges on whether it “affects an essential element of the treaty that is necessary to its general tenor, in such a way that the reservation impairs the \textit{raison d'être} [the most important reason] of the treaty” -- \article{\textit{ILC Guide to Practice on Reservations} Art 3.1.5}
            \item If a reservation is impermissible: \\ \textit{Briefly mention both points in discussion, and then note the first point is the predominant view.}
            \begin{enumerate}
                \item Traditionally, this \glspl{vitiate} the consent of the state to the treaty as a whole and results in the state not being a party to the treaty (this is the predominant view) -- \case{\textit{Reservations to Genocide Convention} [1951] ICJ Rep 15 at page 18}
                \item More recently, the offending reservation will be held void, with the state being bound without the protection of the reservation (i.e., it is cut out), unless consent is conditional on reservation, in which case they are not bound to the treaty at all
            \end{enumerate}
        \end{enumerate}
        \item Was the reservation accepted or objected to? -- \convention{\textit{VCLT} Art 20} (Page \pageref{VCLT Art 20})
        \begin{enumerate}
            \item If a treaty expressly allows for reservations, then no acceptance of a reservation is required by other parties -- \convention{\textit{VCLT} Art 20(1)} (Page \pageref{VCLT Art 20})
            \item If a treaty has a small number of parties and the application of the treaty in its entirety is an essential condition of signing, acceptance by all parties is required -- \convention{\textit{VCLT} Art 20(2)} (Page \pageref{VCLT Art 20})
            \item If a treaty is a constituent instrument of an international organisation, and unless it otherwise provides, a reservation requires the acceptance of the competent organ of that organisation -- \convention{\textit{VCLT} Art 20(3)} (Page \pageref{VCLT Art 20})
            \item Acceptance by the other contracting state(s) of the reservation results in the reserving state being bound by the treaty (with the reservation incorporated) -- \convention{\textit{VCLT} Art 20(4)(a)} (Page \pageref{VCLT Art 20})
            \item Objection to a reservation does not prevent entry into force of a treaty between the objecting state and the reserving state, unless the objecting state says otherwise -- \convention{\textit{VCLT} Art 20(4)(b)} (Page \pageref{VCLT Art 20})
            \item An act indicating consent to being bound by the treaty that contains a reservation is effective as soon as at least one other state has accepted the reservation -- \convention{\textit{VCLT} Art 20(4)(c)} (Page \pageref{VCLT Art 20})
            \item Unless the treaty provides otherwise, a reservation is considered to have been accepted if no objections are raised within 12 months of notification of the reservation, or by the date on which it consented to be bound to the treaty, whichever is the later -- \convention{\textit{VCLT} Art 20(5)} (Page \pageref{VCLT Art 20})
        \end{enumerate}
        \item What is the legal effect of the reservation?
        \begin{enumerate}
            \item If State A accepts State R's reservation, then the treaty is modified only between States A and R, to the the extent of the reservation -- \convention{\textit{VCLT} Art 21(1) and (2)} (Page \pageref{VCLT Art 21}); \case{\textit{Republic of India v CCDM Holdings, LLC} [2025] FCAFC 2 at [63]} (Page \pageref{case:India v CCDM})
            \begin{enumerate}
                \item However, other states will not be bound by this reservation; it acts as a side agreement between State A and State R -- \convention{\textit{VCLT} Art 21(2)} (Page \pageref{VCLT Art 21})
            \end{enumerate}
            \item If State B objects to State R's reservation, and says the treaty is not to apply, then there is no treaty between them at all -- \convention{\textit{VCLT} Art 20(4)(b)} (Page \pageref{VCLT Art 20})
            \item If State C objects to State R's reservation but does not say that the treaty is not to apply, then the treaty applies, but “the provisions to which the reservation applies do not apply ... to the extent of the reservation” -- \convention{\textit{VCLT} Art 21(3)} (Page \pageref{VCLT Art 21})
        \end{enumerate}
        \item Was the state a persistent objector?
        \begin{enumerate}
            \item States which consistently object to the emergence of a rule of custom from its earliest point of gestation will not be bound by it -- \case{\textit{Anglo Norwegian Fisheries Case (UK v Norway)} [1951] ICJ Rep 116} (Page \pageref{case:UK v Norway Fisheries})
            \item A state cannot be a persistent objector to a \textit{jus cogens} principle -- \article{\textit{International Law Commission 2019 Report} Chapter V Conclusion 14} (Page \pageref{report:2019 ILC Conc. 14})
        \end{enumerate}
    \end{enumerate}
    \item How was the treaty interpreted by the state?
    \begin{enumerate}
        \item There are a number of different approaches to treaty interpretation:
        \begin{enumerate}
            \item Formalist/Textual (formal adherence to the terms of the treaty)
            \item Restrictive (deference to state sovereignty)
            \item Teleological (to give effect to the object and purpose of the treaty)
            \item Effectiveness (to ensure the treaty regime remains as effective as possible)
            \item Originalist (to focus on the original purpose of the treaty)
        \end{enumerate}
        \item The Australian courts will apply the VCLT when interpreting a treaty that has been incorporated into Australian law -- \case{\textit{DHI22 v Qatar Airways} [2024] FCA 348 at [30] (Halley J)}, citing \case{\textit{Povey v Qantas Airways Ltd} (2005) 233 CLR 189 at [24] (Gleeson CJ, Gummow, Hayne and Heydon JJ)} (see Section \ref{sec:Interpretation of Treaties} on Page \pageref{sec:Interpretation of Treaties})
        \item The VCLT contains rules on how to interpret treaties -- \convention{\textit{VCLT} Art 31} (Page \pageref{VCLT Art 31})
        \item Under the VCLT, instruments used in treaty interpretation must have been adopted by all states -- \case{\textit{Whaling in the Antarctic Case} [2014] ICJ Rep 226 at [83]} (Page \pageref{case:Antarctic Whaling})
        \item As a last resort, supplementary means of interpretation can be used to interpret the provisions of a treaty under Art 31 -- \convention{\textit{VCLT} Art 32} (Page \pageref{VCLT Art 32})
    \end{enumerate}
    \item Is the treaty void or otherwise invalidated?
    \begin{enumerate}
        \item Is the treaty void?
        \begin{enumerate}
            \item If the State's representative had been coerced into entering the treaty, or there were acts or threats directed against the representative, a State's consent will not be made out and so the treaty will be void -- \convention{\textit{VCLT} Art 51} (Page \pageref{VCLT Art 51})
            \item If a State's consent was obtained through a threat or the use of force, it is void -- \convention{\textit{VCLT} Art 51} (Page \pageref{VCLT Art 52})
            \item If a treaty conflicts with a \textit{jus cogens} norm, it is void -- \convention{\textit{VCLT} Art 51} (Page \pageref{VCLT Art 53})
            \item If a new \textit{jus cogens} norm has emerged since the ratification of a treaty and the treaty conflicts with that \textit{jus cogens} norm, the treaty is void -- \convention{\textit{VCLT} Art 51} (Page \pageref{VCLT Art 64})
        \end{enumerate}
        \item Is the treaty invalid?
        \begin{enumerate}
            \item Did the state's consent to a treaty involve a violation of an internal law of fundamental importance? -- \convention{\textit{VCLT} Art 46(1)} (Page \pageref{VCLT Art 46})
            \begin{enumerate}
                \item A state may not invoke inconsistent internal law as a basis on which it could not sign a treaty, unless that rule is of manifest importance
            \end{enumerate}
            \item If a representative of a state had gone beyond what he was authorised to do so in signing the treaty, their omission to observe their limitations will not constitute an invalidation of the treaty, unless the restriction was notified to other states prior to the expression of consent -- \convention{\textit{VCLT} Art 47} (Page \pageref{VCLT Art 47})
        \end{enumerate}
        \item Was there an error of fact that formed the essential basis of consent? -- \convention{\textit{VCLT} Art 48} (Page \pageref{VCLT Art 48})
        \begin{enumerate}
            \item Consent may be validated by means of an error if the error relates to a fact or situation assumed by the state that existed at the time when the treaty was concluded, and forms an essential basis of its consent to be bound by the treaty -- \convention{\textit{VCLT} Art 48(1)} (Page \pageref{VCLT Art 48})
            \item An error of fact cannot be plead by a party if they contributed to it, could have avoided it, or were otherwise put on notice of a possible error -- \convention{\textit{VCLT} Art 48(2)} (Page \pageref{VCLT Art 48}); \case{\textit{Temple of Preah Vihear (Cambodia v Thailand)} [1962] ICJ Rep 6 at Page 17}
            \item If there is an error relating to only the wording of the treaty's text, its validity is not affected, and \convention{Art 79} is enlivened -- \convention{\textit{VCLT} Art 48(3)} (Page \pageref{VCLT Art 48})
        \end{enumerate}
        \item Had the state been induced to conclude the treaty by the fraudulent conduct of another state? -- \convention{\textit{VCLT} Art 49} (Page \pageref{VCLT Art 49})
    \end{enumerate}
    \item Are there grounds to terminate, withdraw or suspend the treaty? \\\vspace{8pt}
    \textit{The following constitute internal grounds of termination, withdrawal or suspension.}
    \begin{enumerate}
        \item Was the treaty terminated or withdrawn from under:
        \begin{enumerate}
            \item Its provisions? -- \convention{\textit{VCLT} Art 54(a)} (Page \pageref{VCLT Art 54})
            \item By consent of all of the parties after consultation with the other contracting states? -- \convention{\textit{VCLT} Art 54(b)} (Page \pageref{VCLT Art 54})
        \end{enumerate}
        \item Was the treaty suspended under:
        \begin{enumerate}
            \item Its provisions? -- \convention{\textit{VCLT} Art 57(a)} (Page \pageref{VCLT Art 57})
            \item By consent of all of the parties after consultation with the other contracting states? -- \convention{\textit{VCLT} Art 57(b)} (Page \pageref{VCLT Art 57})
        \end{enumerate}
    \end{enumerate}
    \textit{The following constitute external grounds of termination, withdrawal or suspension.}
    \begin{enumerate}[resume]
        \item Was there a denunciation or withdrawal from the treaty when there is no provision to do so? -- \convention{\textit{VCLT} Art 56} (Page \pageref{VCLT Art 56})
        \begin{enumerate}
            \item There is generally no right of denunciation, except where: -- \convention{\textit{VCLT} Art 56(1)} (Page \pageref{VCLT Art 56})
            \begin{enumerate}
                \item It is established that the parties intended to admit the possibility of denunciation or withdrawal -- \convention{\textit{VCLT} Art 56(1)(a)} (Page \pageref{VCLT Art 56})
                \item A right of denunciation or withdrawal may be implied by the nature of the treaty -- \convention{\textit{VCLT} Art 56(1)(b)} (Page \pageref{VCLT Art 56})
            \end{enumerate}
            \item Under this provision, a party must give at least 12 months' notice of its intention to denounce/withdrawal from the treaty -- \convention{\textit{VCLT} Art 56(2)} (Page \pageref{VCLT Art 56})
        \end{enumerate}
        \item Was there a material breach? -- \convention{\textit{VCLT} Art 60} (Page \pageref{VCLT Art 60})
        \begin{enumerate}
            \item A material breach involves a wrongful act being intentionally committed by a party -- \case{\textit{Gabčíkovo-Nagymaros Case} [1997] ICJ Rep 7 at [72]-[81]} (Page \pageref{case:[1997] ICJ Rep 7})
            \item In a bilateral treaty, this entitles the other party to terminate the treaty or suspend its operation, in whole or in part -- \convention{\textit{VCLT} Art 60(1)} (Page \pageref{VCLT Art 60})
            \item If there was a breach in a multilateral treaty: -- \convention{\textit{VCLT} Art 60(2)} (Page \pageref{VCLT Art 60})
            \begin{enumerate}
                \item The other parties can unanimously opt to suspend or terminate the treaty either (i) between themselves and the defaulting state, or (ii) between all parties -- \convention{\textit{VCLT} Art 60(2)(a)} (Page \pageref{VCLT Art 60})
                \item A party who has been especially affected has grounds to suspend the treaty in whole or in part between itself and the defaulting state -- \convention{\textit{VCLT} Art 60(2)(b)} (Page \pageref{VCLT Art 60})
                \item Any party other than the defaulting party may suspend the treaty in whole or in part if the breach is such that it radically changes the position of every party with respect to the further performance of its obligations under the treaty -- \convention{\textit{VCLT} Art 60(2)(c)} (Page \pageref{VCLT Art 60})
            \end{enumerate}
            \item Moreover, a material breach entails:
            \begin{enumerate}
                \item A repudiation of the treaty not sanctioned by the present Convention -- \convention{\textit{VCLT} Art 60(3)(a)} (Page \pageref{VCLT Art 60})
                \item The violation of a provision essential to the accomplishment of the object or purpose of this treaty -- \convention{\textit{VCLT} Art 60(3)(b)} (Page \pageref{VCLT Art 60})
            \end{enumerate}
            \item A party cannot claim material breach if they themselves had committed the wrongful act -- \case{\textit{Gabčíkovo-Nagymaros Case} [1997] ICJ Rep 7 at [92]-[94]} (Page \pageref{case:[1997] ICJ Rep 7})
        \end{enumerate}
        \item Did the performance of the treaty become impossible? -- \convention{\textit{VCLT} Art 61} (Page \pageref{VCLT Art 61})
        \begin{enumerate}
            \item A state may terminate or withdraw from a treaty if its performance has become impossible because `an object indispensable for the execution of the treaty' has permanently disappeared or been destroyed -- \convention{\textit{VCLT} Art 61(1)} (Page \pageref{VCLT Art 61})
            \item However, impossibility of performance may not be invoked if the impossibility is the result of a breach by that party either of an obligation under that treaty or any other international obligations owed to any other party of the treaty -- \convention{\textit{VCLT} Art 61(2)} (Page \pageref{VCLT Art 61})
        \end{enumerate}
        \item Was there a fundamental change of circumstances that precluded the operation of the treaty? -- \convention{\textit{VCLT} Art 62} (Page \pageref{VCLT Art 62})
        \begin{enumerate}
            \item Under the principle of \textit{pacta sunt servanda} (\article{VCLT Art 26} on Page \pageref{VCLT Art 26}), the party must have exhausted all possible avenues before claiming a fundamental change of circumstances
            \item A fundamental change of circumstances entails: -- \case{\textit{Gabčíkovo-Nagymaros Case} [1997] ICJ Rep 7 at [104]} (Page \pageref{case:[1997] ICJ Rep 7})
            \begin{enumerate}
                \item The circumstances at the conclusion of the treaty must have been an essential basis of consent
                \item The change must not have been foreseen
                \item The change must radically transform the extent of the obligations still to be made performed                
            \end{enumerate}
            \item A fundamental change of circumstances may not be invoked as a grounds for termination/withdrawal unless: -- \convention{\textit{VCLT} Art 62(1)} (Page \pageref{VCLT Art 62})
            \begin{enumerate}
                \item The existence of those circumstances constituted an essential basis of the consent of the parties to be bound by the treaty; \textbf{and} -- \convention{\textit{VCLT} Art 62(1)(a)} (Page \pageref{VCLT Art 62})
                \item The effect of the change is radically to transform the extent of the obligations still to be performed under the treaty -- \convention{\textit{VCLT} Art 61(1)(b)} (Page \pageref{VCLT Art 62})
            \end{enumerate}
            \item A fundamental change of circumstances may not be invoked as a ground for terminating or withdrawing from a treaty: -- \convention{\textit{VCLT} Art 62(2)} (Page \pageref{VCLT Art 62})
            \begin{enumerate}
                \item If the treaty establishes a boundary -- \convention{\textit{VCLT} Art 62(2)(a)} (Page \pageref{VCLT Art 62})
                \item If the fundamental change is the result of a breach by the party invoking it -- \convention{\textit{VCLT} Art 62(2)(b)} (Page \pageref{VCLT Art 62})
            \end{enumerate}
            \item If a party invokes a fundamental change of circumstances as a ground for terminating or withdrawing from a treaty, it may also invoke the change as a ground for suspending the operation of the treaty -- \convention{\textit{VCLT} Art 62(3)} (Page \pageref{VCLT Art 62})
            \item International courts are very reluctant to find that impossibility and/or fundamental change of circumstances have been made out (i.e., they have a very high threshold and thus a very limited scope, as suggested by the negative wording of the Articles) -- \case{{Gabčíkovo-Nagymaros Case} [1997] ICJ Rep 7 at [104]} (Page \pageref{case:[1997] ICJ Rep 7})
            \item \article{Art 62} has generally been accepted as a codification of the existing customary law on termination by fundamental change of circumstances -- \case{\textit{Fisheries Jurisdiction (United Kingdom v Iceland)} ICJ Reports 1973, pg. 63, para. 36}
        \end{enumerate}
    \end{enumerate}
\end{enumerate}

\section{International Law and Australian Law}
\begin{enumerate}
    \item Was this a matter involving Australia, or a state which follows the Australian approach to adopting international law?
    \item How does international law influence Australian law?
    \begin{enumerate}
        \item International law applies between states, but may also recognise institutions of domestic law that have an extensive/important role in international law (e.g., corporations) -- \case{\textit{Barcelona Traction (Belgium v Spain)} [1970] ICJ Rep 3 (Page 44)} [where corporations could be recognised within international law]
        \item Absent/inconsistent domestic law is not excuse for failing to meet international obligations -- \case{\textit{Alabama Claims Arbitration (US/Britain)} (1872)}; \case{\textit{Sandline Arbitration} (1998)}
        \item Expert evidence cannot be adduced to prove or explain statements of international law -- \case{\textit{ACCC v PT Garuda (No 9)} [2013] FCA 23 at [31] (Perram J)} (Page \pageref{case:ACCC v Garuda})
        \item Common law can be developed with regard to international law, where it is not inconsistent with domestic law -- \case{\textit{Chow Hung Ching v R} (1949) 77 CLR 449, 471 (Starke J)} (Page \pageref{case:Chow Hung Ching})
        \begin{enumerate}
            \item However, international law is not incorporated as part of Australian law, but rather is a \textbf{source} -- Latham CJ at page 462; Dixon J at page 477
        \end{enumerate}
        \item International law cannot automatically be imported/included in Australian law, but remains a ``\textbf{legitimate and important influence} on the development of the law'' -- \case{\textit{Mabo v Queensland (No 2)} (1992) 175 CLR 1, 42 (Brennan J)} (Page \pageref{case: Mabo})
        \begin{enumerate}
            \item This is especially the case when referring to aspects of international law that touch on universal human values (e.g., human rights)
        \end{enumerate}
        \item However, in cases such as a civil claim for torture (or any other serious crime forbidden under international law), the common law of state doctrine should reflect universal norms -- \case{\textit{Habib v Commonwealth} (2010) 183 CLR 62 at [7] (Black CJ)} (Page \pageref{case: Habib v Commonwealth})
    \end{enumerate}
    \item Has customary international law been implemented in Australian law? \\ \textit{Always explore monism and dualism for a nuanced discussion, and describe how this results in Australia's hard transformation approach.}
    \begin{enumerate}
        \item Whilst there is no clear authority, the automatic incorporation of customary international law in Australia has been rejected
        \item Australia has rejected the monism approach (where states and international law form one entity), and has adopted a hard transformation approach that tends towards dualism (where states and international law form two separate entities) (Section \ref{sec:Monism and Dualism} on Page \pageref{sec:Monism and Dualism})
        \begin{enumerate}
            \item Under the \textbf{hard transformation approach}, only legislation may implement the provisions of international law into domestic law; otherwise, international law does not apply (this is the approach favoured in Australia, following \case{\textit{Chow Hung Ching v R} (1949) 77 CLR 449 (Latham CJ at 462; Dixon J at 477)} (Page \pageref{case:Chow Hung Ching}))
            \item The \textbf{soft transformation approach} holds that legislation or court decisions may implement the provisions of international law (discussed by Latham CJ in \case{\textit{Chow Hung Ching v R} (1949) 77 CLR 449} (Page \pageref{case:Chow Hung Ching})), but so far has been rejected -- \case{\textit{Dietrich v R} [1992] HCA 57} (Page \pageref{case: Dietrich v R})
            \begin{enumerate}
                \item ``International law is not as such part of the law of Australia, but a universally recognized principle of international law would be applied by our courts'' -- \case{\textit{Chow Hung Ching v R} (1949) 77 CLR 449, 462 (Latham CJ)} (Page \pageref{case:Chow Hung Ching})
            \end{enumerate}
        \end{enumerate}
        \item If the country is instead following the UK's approach, see Section \ref{sec:Customary International Law in Australian Law} on Page \pageref{sec:Customary International Law in Australian Law}
    \end{enumerate}
    \item Was this a matter involving international criminal law?
    \begin{enumerate}
        \item Customary/international criminal law can never be a part of the Australian common law/transformed or implemented by the courts (it can only be imported by statute) -- \case{\textit{Nulyarimanna v Thompson} (1999) 165 ALR 621} (Page \pageref{case: Nulyarimanna v Thompson})
        \begin{enumerate}
            \item At [20], Wilcox J held that if domestic criminal law could be influenced by customary/international criminal law, it would lead to the position where international obligations have greater obligations than domestic consequences, sidelining domestic law and thus a state's independence to make its own criminal laws
            \item This is moreover a position adopted in the UK -- \case{\textit{R v Jones} [2006] 2 All ER 741 (Lords Bingham, Mance and Hoffman)} (Page \pageref{case: R v Jones})
        \end{enumerate}
        \item \textit{Jus cogens} principles of international law are not automatically part of Australian common law, and criminal offences must be created by statute, not by the courts -- \case{\textit{Nulyarimanna v Thompson} (1999) 165 ALR 621 at [17], [20], [32], [57] (Wilcox and Whitlam JJ)} (Page \pageref{case: Nulyarimanna v Thompson})
        \begin{enumerate}
            \item For example, the \textit{jus cogens} prohibition of genocide was not automatically part of Australian common law, and had to be created by statute (see the above paragraphs for context) 
            \item Merkel J dissented, and held that the approach that should be taken was the `common law adoption approach', which is that a rule of international law is to be adopted by a court so long as it is not inconsistent with legislation or public policy
        \end{enumerate}
    \end{enumerate}
    \item Does the matter involve a treaty being implemented into Australian law?
    \begin{enumerate}
        \item \label{treaty entry executive power} The power to enter into treaties is exclusively an Executive prerogative power -- \statute{\textit{Constitution} s 61} (Page \pageref{Constitution s 61})
        \begin{enumerate}
            \item In interpreting the external affairs power, the HCA has held that they only need to look at whether the law applies geographically externally to Australia, not whether the international law was void by virtue of the underlying treaty being void -- \case{\textit{Horta v Commonwealth} (1994) 181 CLR 183, 191}
            \item ``The federal executive, through the Crown's representative, possessed exclusive and unfettered treaty-making power'' -- \case{\textit{Koowarta v Bjelke-Petersen} (1982) 153 CLR 168 at [215] (Stephen J)} (Page \pageref{case:Koowarta v Bjelke-Petersen})
        \end{enumerate}
        \item \label{treaty legislature powers} The power to implement the provisions of a treaty is a legislative power -- \statute{\textit{Constitution} s 51(xxix)} [the `external affairs' power]
        \begin{enumerate}
            \item Treaty provisions do not form a part of Australia law until they have been implemented by statute -- \case{\textit{Dietrich v R} [1992] HCA 57 (Brennan CJ, Mason and McHugh JJ)} (Page \pageref{case: Dietrich v R})
            \item Resolutions of international organisations (e.g., the UN Security Council) do not form a part of Australian law until they have been implemented by statute -- \case{\textit{Bradley v Commonwealth} (1973) 128 CLR 557, 582 (Barwick CJ and Gibbs J)} (Page \pageref{case:Bradley v Commonwealth})
            \begin{enumerate}
                \item If Parliament `approves' a treaty, it is not binding
                \item The mere approval of Parliament does not give a treaty the force of law (this practice has since lapsed)
            \end{enumerate}
            \item Under \statute{\textit{Constitution} s 51(xxix)}, the law must carry into effect treaty obligations, and be reasonably considered to be appropriate and adapted to achieving this objective -- \case{\textit{Commonwealth v Tasmania} (1983) 158 CLR 1, 40 (Deane J)} (Page \pageref{case:Commonwealth v Tasmania})
            \item Ratification only occurs after a treaty has been implemented into internal legislative provisions/given the force of law (the legislative approach is the preferred one, as it the most common and avoids uncertainty)
        \end{enumerate}
    \end{enumerate}
    \item Does the matter involve Australia entering into/making a treaty?
    \begin{enumerate}
        \item Australia can enter into two types of treaties:
        \begin{enumerate}
            \item Bilateral treaties, which enter into force for Australia after
            \begin{enumerate}
                \item Signature
                \item Subsequent exchange of notes stating that the constitutional process is completed
            \end{enumerate}
            \item Multilateral treaties, which enter into force for Australia after
            \begin{enumerate}
                \item Signature
                \item Subsequent ratification (or accession if there was no previous signature)
            \end{enumerate}
        \end{enumerate}
        \item \label{item:trick or treaty} Whilst there is no constitutional requirement for the Parliament to be involved in the treaty-making process, since 1996, Parliament has been consulted on the treaty-making process (without a veto) -- \article{\textit{Trick or Treaty?} (1995 Report of the Senate Legal and Constitutional References Committee)}
        \begin{enumerate}
            \item The present convention is that all proposed treaty conventions are tabled in Parliament at least 15 sitting days prior to any binding action being undertake (with exemptions for urgent or sensitive treaties)
            \item A National Interest Analysis (NIA) is also prepared, which is akin to an explanatory memorandum for a treaty
            \item The treaty should also be reviewed by the Joint Standing Committee on Treaties
        \end{enumerate}
    \end{enumerate}
    \item Should international law be used to interpret Australian statute?
    \begin{enumerate}
        \item International law can be used as extrinsic material when interpreting legislation which refers to a treaty -- \statute{\textit{Acts Interpretation Act 1901} (Cth) ss 15AB(1) and (2)(d)} (Page \pageref{Acts Interpretation Act s 15AB})
        \item International law can be used to interpret a legislative provision that incorporates a treaty provision; in this instance, the rules of treaty interpretation apply rather than statutory interpretation
        \item Is the \textit{Polites} principle enlivened?
        \begin{enumerate}
            \item The \textit{Polites} principle refers to the presumption that the Parliament intends to give effect to Australia's obligations under international law, in the absence of express words/intention to the contrary -- \case{\textit{Polites v Commonwealth} (1945) 70 CLR 60, 77 (Dixon J)} (Page \pageref{case:Polites v Commonwealth})
            \begin{enumerate}
                \item It is a general rule of statutory interpretation that, in the absence of express words to the contrary, it is presumed that legislation is intended to be in conformity with the treaty-based rules of international law
            \end{enumerate}
            \item Once a treaty is ratified and implemented into domestic law, statutory interpretation requires courts to presume that legislation is intended to be in conformity with international law (\textit{Polites} principle) because Parliament, \textit{prima facie}, intends to give effect to Australia's international obligations
            \item The \textit{Polites} principle does not apply to constitutional interpretation, as it would violate the requirement for a referendum to modify the \textit{Constitution} under s 128 -- \case{\textit{Al-Kateb v Godwin} (2004) 208 ALR 124} (Page \pageref{case:Al-Kateb v Godwin})
            \begin{enumerate}
                \item At [66], McHugh J outlined that it was never the case that the \statute{\textit{Constitution}} should have been interpreted to conform with the rules of international law
                \item At [175], Kirby J (in dissent) held that the \statute{\textit{Constitution}} should be interpreted in a way that is generally harmonious with the basic principles of international law, including as that law states human rights and fundamental freedoms
            \end{enumerate}
        \end{enumerate}
    \end{enumerate}
    \item How is treaty law implemented into Australian law?
    \begin{enumerate}
        \item Signing of the treaty
        \begin{enumerate}
            \item The Commonwealth Executive has the exclusive power to sign treaties (see \ref{treaty entry executive power})
            \item Parliamentary approval is not necessary, but as a matter of convention, it has been sought (see \ref{treaty legislature powers})
        \end{enumerate}
        \item Implementation of the treaty
        \begin{enumerate}
            \item Parliament will generally be consulted about the treaty, but constitutionally, its approval is not required to sign the treaty (see \ref{treaty legislature powers})
            \item Following the \article{\textit{Trick or Treaty}} report (see \ref{item:trick or treaty}), the Executive will consult Parliament before signing treaties
            \item All proposed treaty actions are tabled in Parliament at least 15 sitting days prior to a binding action
            \item Parliament can provide recommendations and scrutinise the treaty, but ultimately the Executive has the final say
        \end{enumerate}
        \item Ratification of the treaty
        \begin{enumerate}
            \item The treaty is ratified by the executive once domestic legislation is present to incorporate the terms of the treaty
            \item Under \convention{\textit{VCLT} Art 2(1)(b)} (Page \pageref{VCLT Art 2}), ratification is the intentional act whereby a state indicates its consent to be bound to a treaty if the parties intended to show their consent (generally, the depositary will collect the ratifications of all states)
        \end{enumerate}
    \end{enumerate}
\end{enumerate}

\section{Personality, Statehood and Self-Determination}

\begin{enumerate}
    \item Does the entity have an international legal personality?
    \begin{enumerate}
        \item International legal personality gives an actor rights, duties and powers on the international plane, with international legal personality being a spectrum incurring different obligations and rights -- \case{\textit{Reparations for Injuries Suffered in the Service of the United Nations Advisory Opinion} [1949]}
        \begin{enumerate}
            \item With international legal personality, an entity can:
            \begin{enumerate}
                \item Make claims before international committees, courts and tribunals
                \item Be subject to some or all international legal obligations
                \item Be empowered to enter into treaties
                \item Enjoy the jurisdiction of some national courts
            \end{enumerate}
        \end{enumerate}
        \item States have the largest international legal personality, with the most rights and obligations
        \begin{enumerate}
            \item This is contingent on whether an entity has statehood
        \end{enumerate}
        \item A non-territorialised entity can still have international personality (e.g., the Holy See, which is the government of the Vatican City State, has international legal personality, but is not a state in the traditional sense)
        \item By default, a corporation does not possess international legal personality
        \begin{enumerate}
            \item Thus, corporations cannot be parties to a treaty
            \item Corporations may be parties to contracts governed by international law (`internationalised' contracts) -- \case{\textit{Texaco Overseas Petroleum Company v Libya} (1977) 53 ILR 389}
        \end{enumerate}
        \item Individuals have a limited degree of international legal personality
        \begin{enumerate}
            \item They cannot enter into treaties, and only have standing before international courts in limited circumstances
            \item Individuals are subject to international and criminal human rights laws -- \statute{\textit{ICC Statute} (1998) Art 25}; \case{\textit{Nuremberg Trial Judgement} (1947)}; \case{\textit{R v Bow Streets Magistrate, ex Parte Pinochet (No 3)} [1999] 2 All ER 97}
        \end{enumerate}
        \item International organisations do not possess general competence, but are instead governed by the principle of `specialty' (i.e., they can only act within the limits of their powers as defined by their constitutive treaties) -- \case{\textit{WHO Advisory Opinion} [1996] ICJ Rep 66}
        \begin{enumerate}
            \item International organisations may enter into treaties
            \item International organisations may be responsible for wrongful acts -- \convention{\textit{ILC Draft Articles on the Responsibility of International Organisations (2011)}}
            \item International organisations can seek compensation against states where their interests have been harmed -- \case{\textit{Reparation for Injuries Case} [1949] ICJ Rep 174}
        \end{enumerate}
        \item Corporations and individuals cannot go to international courts as only states can be parties to a dispute -- \statute{\textit{ICJ Statute} Art 34(1)}
        \begin{enumerate}
            \item Instead, other modes, such as arbitration (\statute{\textit{ICJ Statute} Art 33(1)}) are used to resolve the dispute peacefully -- \statute{\textit{ICJ Statute} Art 2(3)} (See Scaffold \ref{scaffold:Topic 13} on Page \pageref{scaffold:Topic 13})
        \end{enumerate}
    \end{enumerate}
    \textit{\textbf{Points 2-4 concern whether an entity is a state.}}
    \item Does the entity meet the criteria for statehood?
    \begin{enumerate}
        \item Does the entity have a permanent population? -- \convention{\textit{Montevideo Convention on the Rights and Duties of States} (1933) Art 1(a)}
        \begin{enumerate}
            \item There is no requirement as to the size of the population (e.g., Nauru has less than 10,000 people but is still a state), but the population must be permanent
        \end{enumerate}
        \item Does the entity have a defined territory? -- \convention{\textit{Montevideo Convention on the Rights and Duties of States} (1933) Art 1(b)}
        \begin{enumerate}
            \item Whilst the boundaries do not need to be fixed/undisputed, there must be a reasonably coherent territory that is effectively governed by the state
            \item E.g., the Vatican City is a microstate, and Nauru has an area of 21 km\textsuperscript{2}, but both are recognised as states
        \end{enumerate}
        \item Does the entity have a functioning government? -- \convention{\textit{Montevideo Convention on the Rights and Duties of States} (1933) Art 1(c)}
        \begin{enumerate}
            \item The form or quality of the government is irrelevant, so long as it can effectively control the area
        \end{enumerate}
        \item Does the entity have the capacity to enter into relations with other states? -- \convention{\textit{Montevideo Convention on the Rights and Duties of States} (1933) Art 1(d)}
        \begin{enumerate}
            \item A state cannot be subject to the control of another state
            \item So long as it is not placed under the legal authority of another state, it remains an independent state -- \case{\textit{Customs Union Between Germany and Austria} (1931) PCIJ Series A/B No 41, 57-8 (Judge Anzilotti)}
        \end{enumerate}
    \end{enumerate}
    \item Has the entity been recognised as a state by other states?
    \begin{enumerate}
        \item Recognition may be required to enjoy the benefits of statehood
        \begin{enumerate}
            \item If other states refuse to recognise to engage with the state, the state will have no partners to make treaties or exchange diplomatic representations and will not be welcomed into international organisations -- \article{Thomas Grant, Praeger 1999, 24}
        \end{enumerate}
        \item There are two theories on the effects of recognition:
        \begin{enumerate}
            \item Constitutive theory - recognition by other states is a precondition to statehood (i.e., a condition precedent)
            \item Declaratory theory (the predominant view) - recognition by other states does not create an entity's statehood, but merely acknowledges that the entity is a state (i.e., a condition subsequent) -- \case{\textit{Great Britain v Costa Rica} (1923)}
            \begin{enumerate}
                \item Under this theory, international law contains no prohibitions on declarations of independence (\case{\textit{Kosovo Advisory Opinion} [2010] ICJ Rep 403}), but this does not meant that other states would give recognition
            \end{enumerate}
        \end{enumerate}
        \item Recognition of a state will often be accompanied by a statement that addresses some of the Montevideo criteria
        \item International law contains no prohibition of declarations of independence, but other states should not recognise a unilateral seceding entity before it has acquired statehood -- \case{\textit{Kosovo Advisory Opinion} [2010] ICJ Rep 403}
        \item Recognition of a state may not be extended when a state is created as a result of unlawful use of force/aggression -- \textit{Stimson Doctrine}
        \begin{enumerate}
            \item Puppet states will not be recognised as states (e.g., Manchukuo, a Japanese puppet state in China from 1932-1945)
        \end{enumerate}
        \item If a state is created through the breach of peremptory norms of international law, it will not be recognised as a state -- \case{\textit{Kosovo Advisory Opinion} [2010] ICJ Rep 403}
    \end{enumerate}
    \item Has the entity's government or a foreign corporation been recognised as legitimate?
    \begin{enumerate}
        \item Does this concern a foreign government?
        \begin{enumerate}
            \item It is usually the case that a new government is given recognition, especially if the government has changed in the normal manner for that entity
            \item Australia will otherwise examine the constitutionality of the new government, the control of the government over the territory, the inter-governmental dealings, and the extent of international recognition, among other factors
        \end{enumerate}
        \item Does this concern a foreign corporation?
        \begin{enumerate}
            \item To see if a foreign body or person is legitimate, that body or person needs to be validly incorporated in a place outside Australia, by reference to the law of that place -- \statute{\textit{Foreign Corporations (Application of Laws) Act 1989} (Cth) s 7(2)}
            \item The application of (i) is not affected by the recognition or non-recognition at any time:
            \begin{enumerate}
                \item Of a foreign state or place -- \statute{\textit{Foreign Corporations (Application of Laws) Act 1989} (Cth) s 9(1)(a)}
                \item Of the government of a foreign place or state -- \statute{\textit{Foreign Corporations (Application of Laws) Act 1989} (Cth) s 9(1)(b)}
                \item That a place forms part of a foreign state -- \statute{\textit{Foreign Corporations (Application of Laws) Act 1989} (Cth) s 9(1)(c)}
                \item Of the entities created, organised or operating under the law applied by the people in a foreign state or place -- \statute{\textit{Foreign Corporations (Application of Laws) Act 1989} (Cth) s 9(1)(d)}
            \end{enumerate}
        \end{enumerate}
    \end{enumerate}
    \item Do the people of the entity have a right to self-determination?
    \begin{enumerate}
        \item The right to self-determination refers to the rights of all people to freely determine their political status and to pursue their economic, social and cultural development -- \convention{\textit{1966 ICCPR and 1966 ICECSR} Common Art 1}
        \begin{enumerate}
            \item Self-determination is ``the need to pay regard to the freely expressed will of peoples" -- \case{\textit{Western Sahara Advisory Opinion} [1975] ICJ Rep 12 at [59]}
            \item The right to self-determination is an \gls{erga omnes} customary rule of international law, with all states possessing an interest to protect that right -- \convention{\textit{UNGA Resolution 1514XV} (1960)}; \case{\textit{Chagos Islands Advisory Opinion} [2019] ICJ Rep 95}; \case{\textit{Construction of a Wall in Occupied Palestinian Territory} [2005] at [155]}
        \end{enumerate}
        \item The right to self-determination is subject to the principle of \textit{uti possidetis juris} (respect for existing frontiers) -- \case{\textit{Burkina Faso/Mali} [1986] ICJ Rep 554}
        \begin{enumerate}
            \item If a new state has emerged from an area granted independence, the new entity will possess the same borders/frontiers as it did prior to independence -- \case{\textit{Burkina Faso/Mali} [1986] ICJ Rep 554}
            \item This principle generally applies to colonies achieving independence
        \end{enumerate}
        \item Is there an external right of self-determination?
        \begin{enumerate}
            \item Generally, people cannot choose to become independent unless they are in one of three scenarios: -- \case{\textit{Reference Re Secession of Quebec} (1988) 2 SCR 217}; \convention{\textit{Declaration of Granting Independence to Colonial Countries and Peoples} (GA Res 1514 (XV) 1960)}
            \begin{enumerate}
                \item They are under colonial rule or are a non-self-governing territory
                \item They are subject to alien subjugation, domination or exploitation
                \item Possibly they are under oppression and are blocked from meaningful self-determination (`remedial secession')
            \end{enumerate}
            \item Where there is an external right of self-determination, the people have three choices (though there is no automatic or default choice): -- \case{\textit{Chagos Islands Advisory Opinion} [2019] ICJ Rep 95}
            \begin{enumerate}
                \item Emerge as a sovereign independent state
                \item Freely associate with an independent state (e.g., the Cook Islands' free association with New Zealand)
                \item Integrate with an independent state
            \end{enumerate}
            \item ``The recognised sources of international law established the right to self-determination of a people is normally fulfilled through internal self-determination" -- \case{\textit{Reference Re Secession of Quebec} (1998) 2 SCR 217 at [126]}
        \end{enumerate}
        \item Is there an internal right of self-determination?
        \begin{enumerate}
            \item The internal right to self-determination refers to the ability of the peoples of a state to choose independence for themselves
        \end{enumerate}
        \item Are there Indigenous people seeking self-determination?
        \begin{enumerate}
            \item This is an example of internal self-determination
            \item Indigenous people have the right to self-determination -- \convention{\textit{2007 UN Declaration on the Rights of Indigenous People} Art 3}
            \item Indigenous people have the right to autonomy or self-government in matters relating to their internal and local affairs -- \convention{\textit{2007 UN Declaration on the Rights of Indigenous People} Art 4}
        \end{enumerate} 
    \end{enumerate}
\end{enumerate}

\section{Title to Territory}
\begin{enumerate}
    \item Was there territory involved?
    \begin{enumerate}
        \item Once an entity has territory, it can attach statehood to 
        \item Only states can acquire territory, not individuals -- \case{\textit{Ure v Commonwealth} (1949) 79 CLR 1, 6 (Latham CJ)}
        \item Territory does not equate sovereignty, but sovereignty does imply ownership of territory
        \begin{enumerate}
            \item Sovereignty in relation to territory is `the right to exercise therein, to the exclusion of any other State, the functions of a State' -- \case{\textit{Island of Palmas} (1928)}
            \item Territorial sovereignty refers to a state's right to exercise exclusive jurisdiction within its territory
        \end{enumerate}
        \item Sovereignty encompasses a state's authority within territorial boundaries, jurisdiction refers to the specific legal powers and rights exercised by a state over certain activities under international law
    \end{enumerate}
    \item Was there the occupation of territory?
    \begin{enumerate}
        \item The territory must have been:
        \begin{enumerate}
            \item Unoccupied (not under the sovereignty of another state); or
            \item Terra nullius (abandoned, or otherwise no population) -- \case{\textit{Western Sahara Advisory Opinion} [1975] ICJ Rep 162}
            \begin{enumerate}
                \item Occupation is the exercise of sovereignty over territory that has been deemed as terra nullius -- \case{\textit{Western Sahara Advisory Opinion} ICJ Rep 162}; \case{\textit{Clipperton Island Arbitration (France v Mexico)} 1932}
                \item ``Territories inhabited by tribes or peoples having a social and political organization [are] not regarded as terrae nullius" -- \case{\textit{Western Sahara Advisory Opinion} [1975] ICJ Rep 162 at [80]}
                \item Territories inhabited by tribes or people having a social and political organisation are not regarded as terra nullius -- \case{\textit{Western Sahara Advisory Opinion} [1975] ICJ Rep 162}; \case{\textit{Mabo v Queensland (No. 2)} (1992) 175 CLR 1}
                \item International law has never regarded occupied territory as terra nullius - \case{\textit{Mabo v Queensland (No. 2)} (1992) 175 CLR 1 at [41], [47] (Brennan J)}
            \end{enumerate}
        \end{enumerate}
        \item Occupation is ``an original means of peaceably acquiring sovereignty over territory" - \case{\textit{Western Sahara Advisory Opinion} [1975] ICJ Rep 162}
        \item Whether territory was lawfully occupied or otherwise lawfully acquired is solely within the domain of international law, and cannot be questioned by Australian domestic law -- \case{\textit{Mabo v Queensland (No. 2)} (1992) 175 CLR 1 at [3] (Deane and Gaudron JJ)}
        \item Was there an intention to occupy?
        \begin{enumerate}
            \item Was there an expression of formal intent to claim possession of the land (\textit{animus occupandi}; e.g., the planting of a flag)?
            \item Was there a demonstration of effective control/occupation? -- \case{\textit{Clipperton Island Arbitration (France v Mexico)} 1932}
            \begin{enumerate}
                \item Was there a continuous and peaceful display of state authority?
                \item Was there a responsible authority that exercises governmental functions (\textit{effectivités})?
                \item Was the occupation's spatial extent, duration, continuity and peacefulness all of a sufficient degree?
            \end{enumerate}
            \item Did the claimant show that there was the exercise of a continuous and peaceful display of state authority over the territory?
            \begin{enumerate}
                \item This includes the presence of a responsible authority that exercises governmental functions
            \end{enumerate}
            \item Was the claimant a state actor or a private actor acting on behalf of a state organ?
            \begin{enumerate}
                \item Individuals may not acquire title in unoccupied land not claimed by a state -- \case{\textit{Ure v Commonwealth} [2016] FCAFC 8}
            \end{enumerate}
        \end{enumerate}
    \end{enumerate}
    \item Was there the prescription of territory?
    \begin{enumerate}
        \item Prescription refers to the acquisition of the title to territory formerly occupied by another state through a peaceful exercise of sovereignty for a period of time -- \case{\textit{Island of Palmas Case (Netherlands v US)} (1928)}; \case{\textit{Malaysia v Singapore} [2008] ICJ Rep 12}
        \item Discovery alone is insufficient; a positive intention to claim the territory must be shown -- \case{\textit{Island of Palmas Case (Netherlands v US)} (1928)}
        \item Are the criteria/elements for prescription met? -- \case{\textit{Kasikili/Sesdudu Island (Botswana v Namibia)} [1999] ICJ Rep 1045}
        \begin{enumerate}
            \item À titre de souverain (the possession has to be under state authority, and not by a private actor)
            \item Peaceful possession
            \begin{enumerate}
                \item The possession is not peaceful if it is challenged by another state, and hence no prescription can be found -- \case{\textit{US v Mexico} (1911)}
            \end{enumerate}
            \item The possession must be made public
            \item Uninterrupted
            \item Endure for a length of time
            \item Acquiescence by another state (meek protests by the original title holder or a failure to assert their time evinces an intention to surrender their title to the territory) -- \case{\textit{Malaysia v Singapore} [2008] ICJ Rep 12}
        \end{enumerate}
        \item Whether prescription has occurred takes into account two key factors:
        \begin{enumerate}
            \item Intention and will to act
            \begin{enumerate}
                \item An absence of reaction may well amount to acquiescence -- \case{\textit{Malaysia v Singapore} 2008} (e.g., no statement of rejection tends towards acceptance) 
            \end{enumerate}
            \item Actual exercise
            \begin{enumerate}
                \item Sovereignty can be shown through a display of state authority being open, public, continuous and peaceful (i.e., that the state has the intention to acquire that territory) -- \case{\textit{Island of Palmas Case (Netherlands v US)} (1928)}
                \item There is a high threshold -- \case{\textit{Land and Maritime Boundary between Cameroon and Nigeria (Cameroon v Nigeria)} [2002] ICJ Rep 303} (in this case, even though Cameroon (the initial territory holder)  had engaged in only occasional direct acts of administration, having limited material resource to devote to this distant area and hence could not `be viewed as an acquiescence in the loss' of the territory of Nigeria)
            \end{enumerate}
        \end{enumerate}
        \item To determine if prescription has occurred, the critical date must be decided, which is the the date `falling at the end of a period within which material facts a dispute is said to have occurred', and `after which the actions of the parties to a dispute can no longer affect the issue'
        \item \case{\textit{El Salvador v Honduras} [1992] ICJ Rep 351} provides some examples of the critical date:
        \begin{enumerate}
            \item When a new state emerged with boundaries determined by the \textit{uti possidetis} principle
            \item Arises from a tribunal decision
            \item Arises from a boundary treaty, from `acquiescence or recognition'
        \end{enumerate}
    \end{enumerate}
    \item Was there the cession, accretion, avulsion or conquest of the territory?
    \begin{enumerate}
        \item Cession is the voluntary transfer of sovereignty over territory from one state to another, whether by treaty or by consent
        \begin{enumerate}
            \item International law does not impose upon the parties any particular form of cession, but instead places an emphasis on the consent of the parties -- \case{\textit{Malaysia v Singapore} [2008]}
            \item Consent can either be through tacit agreement or through conduct -- \case{\textit{Malaysia v Singapore} [2008]}
            \item Consent by the state can be given through tacit or active recognition of a state's authority -- \case{\textit{Temple of Preah Vihear (Cambodia v Thailand)} [1962] ICJ Rep 6}
            \item There are two developments that restrict the availability of cession:
            \begin{enumerate}
                \item The VCLT now prevents the formation of treaties (including treaties of cession) that have been procured under force or by threat
                \item Cession may go against the principle of self-determination, as it may not reflect the will of the people in the territory (``any detachment by administering Power of Part of a non-self-governing territory, unless based on freely expressed and genuine will of people of the territory concerned is contrary to the right of self-determination'') -- \case{\textit{Legal Consequences of the Separation of the Chagos Archipelago from Mauritius in 1965} [2019] ICJ Rep 95}
            \end{enumerate}
        \end{enumerate}
        \item Accretion is the gain of physical territory through natural processes
        \item Avulsion is the loss of physical territory through natural processes
        \item Conquest is the forceful and unlawful acquisition of territory, and is illegal under international law -- \convention{\textit{UN Charter} Art 2(4)}
    \end{enumerate}
    \item Is there a dispute over territorial sovereignty?
    \begin{enumerate}
        \item Sovereignty is binary - a state either can or cannot have sovereignty over territory at a given point in time
        \item There is a presumption that the original claimer, by default, has a superior claim to title -- \case{\textit{Legal Status of Eastern Greenland (Norway v Denmark)} (1933) PCIJ}
        \item The ICJ will only take into account activities by the states that occurred before the dispute crystallised (the \textbf{critical date}) -- \case{\textit{Case Concerning Sovereignty over Pulau Ligitan and Pulau Spiadan (Indonesia v Malaysia)} [2002] ICJ Rep 625}; \case{\textit{Case Concerning Sovereignty Over Pedra Branca/Pulau Batu Puteh, Middle Rocks and South Ledge (Malaysia/Singapore)} [2008] ICJ Rep 12}
    \end{enumerate}
    \item Are maritime zones involved?
    \begin{enumerate}
        \item Maritime zones are governed by \convention{\textit{1982 UN Convention on the Law of the Seas (UNCLOS)}}
        \item \convention{\textit{UNCLOS}} prevails over any territorial sovereignty claims in respect of maritime zones which are inconsistent with its provisions to the extent of inconsistency -- \case{\textit{South China Sea Arbitration (Philippines v China)} [2016] PCA}
        \item Upon becoming a signatory to \convention{\textit{UNCLOS}}, any pre-existing claims are extinguished -- \case{\textit{South China Sea Arbitration (Philippines v China)} [2016] PCA}
        \item Is there a dispute over baselines?
        \begin{enumerate}
            \item Normal baselines follow the contours of the coast, and are the default baseline used -- \convention{\textit{UNCLOS} Art 5}
        \end{enumerate}
        \item Is there a dispute over internal waters?
        \begin{enumerate}
            \item A state has full sovereignty over its internal waters, and there is no right of innocent passage through internal waters -- \convention{\textit{UNCLOS} Art 8}
        \end{enumerate}
        \item Is there a dispute over the territorial sea?
        \begin{enumerate}
            \item The territorial sea extends 12 nautical miles from the baseline -- \convention{\textit{UNCLOS} Art 3}
            \item The state has territorial sovereignty in this area (and as such, it is inherently unclaimable), with very few exceptions (the most prevalent of which is the right of innocent passage through the territorial sea)
        \end{enumerate}
        \item Is there a dispute over rocks, islands or artificial islands?
        \begin{enumerate}
            \item \convention{\textit{UNCLOS}} is used to determine what constitutes as territory -- \case{\textit{South China Sea Arbitration (Philippines v China)} [2016] PCA}
            \item The normal baseline for measuring the breadth of the territorial sea is the low-water line along the coast -- \convention{\textit{UNCLOS} Art 5}
            \item In the case of islands situated on atolls or of islands that have fringing reefs, the baseline for measuring the breadth of the territorial sea is the seaward low-water line of the reef -- \convention{\textit{UNCLOS} Art 6}
            \item Rocks can be considered as islands if they can sustain human habitation or economic life of their own -- \convention{\textit{UNCLOS} Art 121}
            \item Rocks must remain unsubmerged at high tide to be considered as islands -- \convention{\textit{UNCLOS} Art 121}
        \end{enumerate}
        \item Is there a dispute over the contiguous zone?
        \begin{enumerate}
            \item The contiguous zone is the 12-24 nautical mile zone immediately following the territorial sea -- \convention{\textit{UNCLOS} Art 33(2)}
            \item In this zone, coastal states may only enforce the law against vessels in the matters of customs, tax, immigration and quarantine -- \convention{\textit{UNCLOS} Art 33(1)}
            \begin{enumerate}
                \item It is therefore not a zone of sovereignty, but rather a zone of enforcement jurisdiction
            \end{enumerate}
        \end{enumerate}
        \item Is there a dispute over the exclusive economic zone (EEZ)?
        \begin{enumerate}
            \item The EEZ extends up to 200 nautical miles from the baseline of a coastal state -- \convention{\textit{UNCLOS} Art 57}; \case{\textit{South China Sea Arbitration (Philippines v China)} [2016] PCA}
            \item A state can assert rights to all of the living and non-living resources (including everything in the soil, subsoil, water column, etc.), and everything on the surface (e.g., renewable energy resources) -- \convention{\textit{UNCLOS} Art 56(1)}
            \item A coastal state does not have sovereignty over the EEZ, but merely has rights over the resources and the ability to regulate them -- \convention{\textit{UNCLOS} Art 56}
            \begin{enumerate}
                \item Other states still have the rights of navigation; they just cannot use those resources -- \convention{\textit{UNCLOS} Art 58(1)}
            \end{enumerate}
        \end{enumerate}
        \item Is there a dispute over the continental shelf?
        \begin{enumerate}
            \item The continental shelf is an inherent seabed resources zone, with states having inherent ownership of everything on the seabed and the subsoil, as well a anything living on the seabed -- \convention{\textit{UNCLOS} Art 76(1)}
            \item The continental shelf extends to either 200 nautical miles from the baseline or 100 nautical miles from the 2500m isobath -- \convention{\textit{UNCLOS} Art 76(5)}
            \item The state must have applied for it, it is not automatically granted
            \item In the continental shelf, the sovereign state may explore and exploit its natural resources -- \convention{\textit{UNCLOS} Art 77(1)}
        \end{enumerate}
        \item Is there a dispute over the high seas?
        \begin{enumerate}
            \item The high seas are non-appropriable areas of the sea not subject ot the jurisdiction of any nation, and remain open -- \convention{\textit{UNCLOS} Art 87}; \convention{\textit{UNCLOS} Art 89}
            \item The high seas may only be used for peaceful purposes -- \convention{\textit{UNCLOS} Art 88}
            \item All states may sail their ships in the high seas, whether coastal or land-locked -- \convention{\textit{UNCLOS} Art 90}
        \end{enumerate}
        \item Is this a dispute over the archipelagic waters?
        \begin{enumerate}
            \item These refer to waters within an archipelago (a group of islands) -- \convention{\textit{UNCLOS} Art 46}
            \item Archipelagic baselines are drawn such that they encompass the outermost points of the outermost islands, up to a limit of 100 nautical miles -- \convention{\textit{UNCLOS} Art 47}
            \item These waters are held to be similar to the territorial sea, and states have sovereignty over the archipelagic waters -- \convention{\textit{UNCLOS}Art 49}
        \end{enumerate}
        \item Is there a dispute over the deep seabed (the Area)?
        \begin{enumerate}
            \item This cannot be appropriated by any state, and is managed by the International Seabed Authority (as established under \convention{\textit{UNCLOS} Art 156-157})
        \end{enumerate}
    \end{enumerate}
    \item Is Antarctica, airspace or outer space involved?
    \begin{enumerate}
        \item Is Antarctica involved?
        \begin{enumerate}
            \item No new claims to territory in Antarctica can be made, but existing claims are not renounced and are instead `frozen' -- \convention{\textit{1959 Antarctic Treaty} Art 4}
            \item There are 7 claimants to territory in Antarctica: Argentina, Australia, Chile, France, New Zealand, Norway, and the UK
        \end{enumerate}
        \item Is airspace involved?
        \begin{enumerate}
            \item A state has sovereignty over the airspace above its territory and territorial sea
            \item Other states have the freedom of overflight over the contiguous zone, the EEZ and the high seas
            \item The boundary between national airspace and outer space is not clearly defined
        \end{enumerate}
        \item Is outer space involved?
        \begin{enumerate}
            \item Outer space (including the moon and other celestial bodies) remains the province of mankind, and any exploration/use shall be carried out for the benefit of all countries -- \convention{\textit{1967 Outer Space Treaty} Art 1}
            \begin{enumerate}
                \item There are 113 parties to this treaty
            \end{enumerate}
            \item Outer space is not subject to national appropriation or claims of sovereignty by any means -- \convention{\textit{1967 Outer Space Treaty} Art 2}
        \end{enumerate}
        \item Is the moon or another celestial body involved?
        \begin{enumerate}
            \item The moon and other celestial bodies are the common heritage of mankind -- \convention{\textit{1979 Moon Agreement Art} 11(1)}
            \begin{enumerate}
                \item There are only 18 parties to this treaty, and aside from the \convention{\textit{1967 Outer Space Treaty}}, no other forms of state practice exist in this area
            \end{enumerate}
            \item No state can make a claim in sovereignty over the moon or a celestial body -- \convention{\textit{1979 Moon Agreement} Art 11(2)}
            \item States may explore and use the moon in a manner that doesn't prejudice the rights of other states -- \convention{\textit{1979 Moon Agreement} Art 11(4)}
        \end{enumerate}
    \end{enumerate}
\end{enumerate}

\section{State Jurisdiction}
\begin{enumerate}
    \item What type of state jurisdiction is being exercised?
    \begin{enumerate}
        \item Prescriptive jurisdiction, which is the power to enact laws/assert jurisdiction by legislation (legislative power)
        \item Adjudicative jurisdiction, which is the power to apply laws and decide disputes (judicial power)
        \item Enforcement jurisdiction, which is the ability to enforce laws within a state's territory (executive power)
    \end{enumerate}
    \item Does the state have a claim in civil jurisdiction?
    \begin{enumerate}
        \item This is the exercise by states of power (both prescriptive and adjudicative) over persons, matters or things in private disputes, with international law taking a very hands-off approach on this front
        \item An example of this is the Alien Torts Claims Act 1789 (US), which conferred jurisdiction on US federal courts in civil actions by non-nationals for violations of international law, although this is now interpreted in a more restrictive manner
        \begin{enumerate}
            \item The injury of Nigerian plaintiffs by Royal Dutch Petroleum -- \case{\textit{Kiobel v Royal Dutch Petroleum} (SCOTUS, 2013)}
            \item A Mexican national kidnapped by US agents in Mexico and then brought to the US for trial -- \case{\textit{Sosa v Alvare-Machain} (SCOTUS, 2014)}
        \end{enumerate}
    \end{enumerate}
    \textbf{For criminal jurisdiction, if there is an overlap in jurisdiction or jurisdiction is otherwise concurrent, the state with the most significant connection to the crime will have the first opportunity to prosecute (to prosecute, a state must have custody of the offender).}
    \item Does the state have a claim in criminal jurisdiction under the territorial principle?
    \begin{enumerate}
        \item The territorial principle holds that states may exercise criminal jurisdiction over nationals or non-nationals where the elements of a criminal offence take place within its territory
        \item Did the offence take place within a geographical nexus of the state?
        \begin{enumerate}
            \item If there is a collision in the high seas between two different nationalities, the victim vessel and thus state has no jurisdiction over the perpetrator vessel -- \convention{\textit{UNCLOS} Art 97}
            \item The grounds of an embassy remain the territory of the host state for the purposes of jurisdiction (but enforcement of that jurisdiction is a separate issue due to diplomatic inviolability) -- \case{\textit{R v Turnbull; ex parte Petroff} (ACTSC, 1971)}
            \item The territory of a state includes its territorial seas -- \case{\textit{R v Disun; R v Nardin} (WASC, 2003)}
        \end{enumerate}
        \item Did the offence take place within several states?
        \begin{enumerate}
            \item There are two approaches to this:
            \begin{enumerate}
                \item \textbf{Subjective territorial jurisdiction} - the exercise of prescriptive jurisdiction by the state in which the criminal offence originated, but was completed outside its territory (i.e., the start place of the offence)
                \item \textbf{Objective territorial jurisdiction} - the exercise of prescriptive jurisdiction by the state in which the criminal offence is completed, even if it was initiated outside its territory (i.e., the end place of the offence)
            \end{enumerate}
            \item States can exercise jurisdiction ``if one of the constituent elements of the offence, and more especially its effects, have taken place there. A state may, in certain circumstances, exercise prescriptive and adjudicative jurisdiction beyond their territory'' -- \case{\textit{SS Lotus Case} (1927) PCIJ at 18}
            \item Victoria follows the objective territorial jurisdiction approach -- \case{\textit{Ward v R} (1980) 142 CLR 308}
            \item NSW follows the objective territorial jurisdiction approach -- \statute{\textit{Crimes Act 1900} ss 10A, 10C(1)}
            \item A geographic nexus in NSW exists if:
            \begin{enumerate}
                \item The offence was committed wholly or partly in NSW (irrespective of whether there were any effects in NSW) -- \statute{\textit{Crimes Act 1900} s 10C(2)(a)}
                \item The offence is committed wholly outside NSW, but the offence has an effect in NSW -- \statute{\textit{Crimes Act 1900} s 10C(2)(b)}
            \end{enumerate}
            \item The objective territorial jurisdiction approach is also internationally:
            \begin{enumerate}
                \item \case{\textit{SS Lotus Case} (1927) PCIJ} held that because the ``effect" of the alleged crime was felt on the Turkish ship, and since the court had assimilated the Turkish ship to Turkish territory, the effects had been felt in Turkey's territory and hence Turkey had jurisdiction
                \item \case{\textit{US v Neil} (2002)}, where the victim of sexual assault was an American citizen, and even though the sexual misconduct had taken place in Mexican waters, the US had jurisdiction over the issue
            \end{enumerate}
        \end{enumerate}
        \item Does the legislation in question apply the `effects doctrine'?
        \begin{enumerate}
            \item This doctrine holds that the legislation is not intended to have an extra-territorial effect unless it says so expressly or by implication -- \article{Dennis Pearce, \textit{Statutory Interpretation in Australia}}
            \item It applies only if the provision in question expressly states that it applies extraterritorially (e.g., ``this legislation applies in the state and including any place outside of the state''); in Australia, \statute{\textit{Crimes Act 1900} (NSW) s 10C(2)} applies extra-territorially
        \end{enumerate}
        \item Does the legislation have extended objective territoriality?
        \begin{enumerate}
            \item Legislation can have a very broad/`extended' scope, where it can apply to acts that have been committed outside of the state, or have an effect or results on the state
            \item The effect can also be physical or intangible (e.g., a threat to the peace, order, or good government of the state)
        \end{enumerate}
        \item There are three categories/situations of offence that a state can exercise this power:
        \begin{enumerate}
            \item Offences committed wholly or partly on the territory of the prosecuting state
            \item Offences committed wholly outside the state, but which have an actual effect in the state
            \item Offences committed wholly outside the state, but which may have an effect on the state
        \end{enumerate}
    \end{enumerate}
    \item Does the state have a claim in criminal jurisdiction under the nationality principle?
    \begin{enumerate}
        \item The nationality principle holds that states may exercise criminal jurisdiction over their nationals, regardless of where the offence was committed -- \case{\textit{XYZ v Commonwealth} (2006) 227 CLR 532 at [130]}
        \item Determining the nationality of a person is left to municipal law
        \item Were there child sex offences committed outside Australia by an Australian national? -- \statute{\textit{Criminal Code 1995} (Cth) Div 272} (unsuccessfully challenged in \case{\textit{XYZ v Commonwealth} (2006) 227 CLR 532})
        \item It has been contended that the nationality principle also applies to citizens and residents of states -- \case{\textit{XYZ v Commonwealth} (2006) 227 CLR 532 (Gleeson CJ)}
        \item If an individual has renounced their nationality, such a renunciation must be accepted by the state and given effect before the nationality principle ceases to apply; otherwise, a unilateral renunciation of nationality is not effective and will not extinguish jurisdiction under the nationality principle
        \item In the case of terrorism, if the terrorists are `foreign fighters' (i.e., citizens of countries which are party to the \statute{\textit{Rome Statute}}), the nationality principle can be used by the ICJ to prosecute these individuals -- \article{\textit{Statement of the Prosecutor of the ICC on Alleged Crimes Committed by ISIS}}
        \item A state must have legislation enacting the ability to convict their nationals extra-territorially; e.g.,:
        \begin{enumerate}
            \item \statute{\textit{Crimes (Child Sex Tourism) Amendment Act 1994} (Cth)} holds that ``if the offence is committed outside Australia, a person can only be charged if at the time of the offence they were an Australian citizen or were ordinarily resident in Australia''
        \end{enumerate}
    \end{enumerate}
    \item Does the state have a claim in criminal jurisdiction under the protective (security) principle?
    \begin{enumerate}
        \item The protective (security) principle allows states to take actions against individuals that impaired a fundamental/vital state security interest, even if it occurred outside of the state by non-nationals (this is more concerned with the nature of the offence, rather than the nature of the offence)
        \item For a state to exercise criminal jurisdiction under the protective principle, a recognised linking point must be established between the state and the accused, and the state and the victim:
        \begin{enumerate}
            \item Crimes against a group of people itself are sufficient -- \case{\textit{A-G v Eichmann} (1961) 36 ILR 5}
        \end{enumerate}
        \item Was there a genocide of a people? -- \case{\textit{A-G v Eichmann} (Dist. Ct. of Jerusalem, 1961)}
        \item Was there the attempted murder of government agents? -- \case{\textit{US v Benitez} (US Ct. of Appeal for 11th Cir, 1984)}
        \item Was there an instance of propaganda against the state that fundamentally undermined the safety of the state? -- \case{\textit{Joyce v DPP} (HoL, 1946)}
        \begin{enumerate}
            \item Since Joyce had fraudulently acquired his British passport, this case raised questions over nationality, with Jallett LJ holding an alternate basis of jurisdiction being that no state should ignore a crime of treason committed against it
        \end{enumerate}
        \item Was there an instance of treason? -- \case{\textit{R v Casement} (1917) Eng Ct. of Crim. A}
        \item It is irrelevant as to how the defendant's custody was secured; a violation of international law in securing the defendant may not prevent the exercise of state criminal jurisdiction -- \case{\textit{A-G v Eichmann} (1961) 36 ILR 5}
        \item A state can exercise criminal jurisdiction over persons, including non-nationals, who have committed acts abroad which are prejudicial to the security of the state -- \case{\textit{A-G v Eichmann} (1961) 36 ILR 5}
    \end{enumerate}
    \item Does the state have a claim in criminal jurisdiction under the passive personality principle?
    \begin{enumerate}
        \item Under this principle, criminal jurisdiction can be asserted anywhere in the world to citizens of a state, enabling the national state of the victim to assert jurisdiction over the offender -- \case{\textit{US v Yunis} (US Ct. of A for 11th Cir., 1991)}
        \item The question of whether this is a valid basis for jurisdiction has been expressly reserved -- \case{\textit{SS Lotus Case} (1927) PCIJ}
        \begin{enumerate}
            \item This principle was controversial in the past, and was not universally accepted due to its concern about the ``bubble effect'' (where a foreign national is surrounded by a protective and invisible bubble from their own national laws)
            \item However, it is now more increasingly accepted, and is often now used to establish jurisdiction -- \convention{\textit{International Convention Against the Taking of Hostages} Art 5}
        \end{enumerate}
        \item This principle is especially recognised when applying it against serious and universally condemned crimes -- \case{\textit{US v Yunis} (US Ct. of A for 11th Cir., 1991)} (where the crime in question was for the taking of hostages after hijacking an aircraft and for piracy)
        \item Examples of this principle include:
        \begin{enumerate}
            \item Taking hostages during the hijacking of an aircraft -- \case{\textit{US v Yunis} (US Ct. of A for 11th Cir., 1991)}
            \item Harming Australians abroad - \statute{\textit{Criminal Code 1995} (Cth) Div 115}
            \begin{enumerate}
                \item Was there the murder of an Australian citizen or resident after 1 October 2002? -- \statute{Div 115.1}
                \item Was there the manslaughter of an Australian citizen or resident after 1 October 2002? -- \statute{Div 115.2}
                \item Was there serious harm to an Australian citizen or resident that was intentionally caused? -- \statute{Div 115.3}
                \item Was there serious harm to an Australian citizen or resident that was recklessly caused? -- \statute{Div 115.4}
            \end{enumerate}
        \end{enumerate}
    \end{enumerate}
    \item Does the state have a claim in criminal jurisdiction under the universality principle?
    \begin{enumerate}
        \item This is the exercise of jurisdiction due to their international seriousness (e.g., genocide), or since they might otherwise go unpunished
        \item There is no need to establish a link/nexus between the offender/offence and the prosecuting state, \textbf{other than custody}, as the main goal is to punish the offender for what they have done
        \begin{enumerate}
            \item A state can still exercise universal jurisdiction in absentia (i.e., no custody of the offender) for the most heinous crimes -- \case{\textit{Arrest Warrant} (2002) ICJ (Judges Higgings, Kooijmans and Buergenthal)}
            \begin{enumerate}
                \item Judge Guillame expressed a more restrictive view on this, highlighting the nuance of discussion on this front
            \end{enumerate}
            \item States are not mandated to exercise universal jurisdiction, even though it is permissive in nature -- \case{\textit{Criminal Complaint against Donald Rumsfeld} (2007) German Prosecutor General}
            \item Generally, the universality principle applies only when other states have not done anything to prosecute the offender, or when the offender is not in the custody of a state that has jurisdiction over them
        \end{enumerate}
        \item Was this a case of piracy (an attack on a vessel on the high seas for private ends)?
        \begin{enumerate}
            \item Piracy is any illegal acts of violence or detention committed for private ends in the high seas -- \convention{\textit{UNCLOS} Art 101(a)} (in Australia, it is prohibited in \statute{\textit{Crimes Act 1914} (Cth) Pt IV}, covering piracy by aircraft or by ship)
            \begin{enumerate}
                \item If it were to occur in the territorial sea, it would not be piracy and would instead likely be subject to the national law of the sea
                \item Attempted acts of piracy are sufficient (the \convention{\textit{UNCLOS}} definition is broad in this respect) -- \case{\textit{US v Dire} (Court of Appeal for the 4th Circuit, 2012)}
                \item Marine terrorism which aims to instil fear and achieve political objectives must be distinguished from piracy -- \case{\textit{ICR v Sea Shepherd}}
                \item Piracy requires that one vessel attack/harm another; piracy cannot apply to incidents that occur solely on board one vessel -- \case{\textit{Achille Lauro Incident} (1985)}
                \item Such incidents are classified under international criminal offences relating to the safety of maritime navigation -- \convention{\textit{1988 Convention for the Suppression of Unlawful Acts Against the Safety of Maritime Navigation}}
            \end{enumerate}
            \item Acts committed privately but on behalf of a sovereign state cannot constitute piracy -- \case{\textit{ICR v Sea Shepherd}}
            \item Marine terrorism is distinguished from piracy, as the former aims to instil fear and achieve political objectives; however, piracy cannot have any political objectives -- \case{\textit{ICR v Sea Shepherd}}; examples of marine terrorism include:
            \begin{enumerate}
                \item Seizure of a cruise liner on the high seas by a Portuguese dissident -- \case{\textit{Santa Maria} (1961)}
                \item The hijacking of an Italian cruise liner on the high seas with a political motive (note that there were attacks on board the cruise liner) -- \case{\textit{Achille Lauro Incidents} (1985)}
                \item Since the attacks were on board the ship, this indicates that there is not a true universal terrorism, as jurisdiction may only be exercised over crimes committed over their territory or over nationals of parties to the treaties -- \textit{contra} \case{\textit{US v Yunis}}
            \end{enumerate}
        \end{enumerate}
        \item Was this a case of slavery (the exercise of powers of ownership over persons)?
        \item Was this a case of genocide?
        \begin{enumerate}
            \item Genocide is any of the following acts committed with intent to destroy, in whole or in part, a national, ethnical, racial or religious group: -- \convention{\textit{Convention on the Prevention and Punishment of the Crime of Genocide} Art 2}
            \begin{enumerate}
                \item Killing members of the group
                \item Causing serious bodily or mental harm to members of the group;
                \item Deliberately inflicting on the group conditions of life calculated to bring about its physical destruction in whole or in part
                \item Imposing measures intended to prevent births within the group;
                \item Forcibly transferring children of the group to another group.
            \end{enumerate}
            \item Genocide requires an additional fault element (dolus specialis) - if there was an intention to destroy a group manifest through a single murder, that is sufficient, making it subject to universal jurisdiction
            \item Genocide is a basis for jurisdiction -- \case{\textit{A-G v Eichmann} (Dist Ct of Jerusalem, 1961)}; \case{\textit{Nulyarimma v Thompson} (1999) FCAFC}
            \begin{enumerate}
                \item Universal jurisdiction conferred by international law is a component of sovereignty, and it is up to each state to determine how it is fits into their constitutional arrangements -- \case{\textit{Nulyarimma v Thompson} (1999) FCAFC (Wilcox J)}
                \item The Holocaust constituted genocide, which is a ``grave offence against the law of nations itself" and that ``the jurisdiction to try such crimes under international law is universal" -- \case{\textit{A-G v Eichmann} (1961)}
                \item The fact that a state may exercise universal jurisdiction over a crime does not necessarily mean that a domestic court would automatically be able to try such offences in the absence of statutory authorisation -- \case{\textit{Nulyarimma v Thompson} (1999) FCAFC}
            \end{enumerate}
        \end{enumerate}
        \item Was this a case of war crimes (serious breaches of the law of wars), or crimes against humanity?
        \begin{enumerate}
            \item Crimes against humanity, crimes against peace and war crimes are defined in \convention{\textit{1945 Charter of the International Military Tribunal at Nuremberg}}
            \item Australia's retrospective legislation of crimes against humanity/war crimes is constitutionally valid, and that states have rights to exercise universal jurisdiction over war criminals -- \case{\textit{Polyukhovic v Commonwealth} (1991) HCA at [33] (Brennan J)}
            \begin{enumerate}
                \item Codification of such laws is relevant as it ensures that states have the domestic legal framework to prosecute such crimes, and that the courts have jurisdiction to hear such cases
            \end{enumerate}
        \end{enumerate}
        \item Was this a case of torture (state infliction of pain/suffering to obtain information)?
        \begin{enumerate}
            \item Torture includes any act by which severe pain or suffering (physical or mental) is inflicted intentionally on an individual -- \convention{\textit{1984 Convention Against Torture} Art 1(1)}
            \item State torture requires it to be done by a public official acting in an official capacity
            \item It is widely accepted that torture is subject to universal jurisdiction -- \case{\textit{Pinochet (No. 3)} [1999] 2 All ER 97}
            \begin{enumerate}
                \item It is an \gls{erga omnes} obligation: ``the jus cogens nature of the international crime of torture justifies states in taking universal jurisdiction over torture wherever committed. International law provides that offence jus cogens may be punished by any state'' -- \case{\textit{Pinochet (No 3)} (2000) (HoL)}
            \end{enumerate}
            \item Torture is criminalised within Australia -- \case{\textit{Criminal Code Act 1995} (Cth) Div 274}
        \end{enumerate}
    \end{enumerate}
    \item Does the state have a duty to prosecute or otherwise extradite?
    \begin{enumerate}
        \item The adoption of permissive jurisdiction means that if a crime can be identified, a state can assert jurisdiction over an offender
        \item If a crime is subject to universal jurisdiction, a state must exercise jurisdiction; if they have custody of such a defendant, they must either prosecute them or extradite them to a state where they will face prosecution -- \case{\textit{Belgium v Senegal} (2012) ICJ}
        \begin{enumerate}
            \item If there is a case of torture, this is arguably the case -- \convention{\textit{1984 Convention Against Torture} Art 7}
        \end{enumerate}
        \item For a duty to extradite to arise, there must be an extradition treaty in place between the two states, or the state must have domestic legislation that provides for extradition; such a treaty would enable the state to prosecute the offender in their own courts with respect to certain serious crimes, or to extradite an offender to a jurisdiction to face trial -- \case{\textit{Belgium v Senegal} (2012) ICJ}
        \item To extradite an offender, the crime for which an extradition is sought must be a crime in both states at the date of the commission of the offence -- \case{\textit{R v Bow Street Magistrates}}
        \item The \convention{\textit{Extradition Act 1988} (Cth)} governs extradition in Australia
        \begin{enumerate}
            \item A request for extradition is sent to Australia through diplomatic channels, and then goes to the Attorney-General
            \item The Attorney-General ``may, in his or her discretion'', inform the Magistrate that the request has been received -- \statute{\textit{Extradition Act 1988} (Cth) s 16}
            \item The Attorney-General must not pass on the request to the Magistrate unless they are of the opinion that the person is an ``extraditable person''
            \item The Magistrate will then conduct proceedings to determine whether the person is eligible for surrender; i.e., that: -- \statute{\textit{Extradition Act 1988} (Cth) s 19}
            \begin{enumerate}
                \item The double-criminality rule is satisfied
                \item There are no ``extradition objections'' 
                \item The guilt or innocence of the person is irrelevant
            \end{enumerate}
            \item If the Magistrate finds that the person is eligible for surrender, the Attorney-General then decides whether or not to extradite that person -- \statute{\textit{Extradition Act 1988} (Cth) s 22}
        \end{enumerate}
    \end{enumerate}
    \item Does the ICC have jurisdiction?
    \begin{enumerate}
        \item States need to be parties to the \statute{\textit{Rome Statute of the International Criminal Court}} (1998) for the ICC to have jurisdiction over them
        \item The ICC can only try cases that have occurred after 2002
        \item The subject matter jurisdiction of the ICC is limited to (a) genocide, (b)war crimes, (c)crimes against humanity and (d) the crime of aggression -- \statute{\textit{Rome Statute of the International Criminal Court} Art 5}
        \item Does the offender have personal jurisdiction? -- \statute{\textit{Rome Statute of the International Criminal Court} Art 12} (the ICC can try a case when:)
        \begin{enumerate}
            \item The offender is a national of a party to the \statute{\textit{Rome Statute}}
            \item The offender is in the territory of a state that is party to the \statute{\textit{Rome Statute}} (in this instance, the offender does not need to be a national to the party of the treaty)
        \end{enumerate}
        \item Is a state refusing to prosecute a crime it otherwise has a duty to do so?
        \begin{enumerate}
            \item If the state refuses to prosecute such a crime, the ICC has jurisdiction to do so -- \statute{\textit{Rome Statute of the International Criminal Court} Art 17}
            \item The ICC can only step in where a state is `unwilling or unable to genuinely carry out the investigation or prosecution' -- \statute{\textit{Rome Statute of the International Criminal Court} Art 17}
        \end{enumerate}
        \item Can an individual be held responsible and liable for the commission of crimes? -- \statute{\textit{Rome Statute of the International Criminal Court} Art 25}
        \item The official capacity of the person will not be a defence to the commission of a crime -- \statute{\textit{Rome Statute of the International Criminal Court} Art 27}
        \item Was custody of the offender illegally obtained?
        \begin{enumerate}
            \item In international law, if the charge is serious, this will be overlooked on the basis that justice is paramount -- \case{\textit{Prosecutor v Nikolić} (CITY Appeals Chamber, 2003)}
            \item In domestic law, this depends on the state
            \begin{enumerate}
                \item In Australia, custody generally needs to be legally obtained -- \case{\textit{Moti v R} (2011, HCA)}
                \item In South africa, custody generally needs to be legally obtained -- \case{\textit{State v Ebrahim} (1992) (SupCTSA)}
                \item Argentina waived the claim that Israel had illegally obtained custody of Eichmann, as the charge was so serious that it was not a defence to the charge -- \case{\textit{A-G v Eichmann} (1961) 36 ILR 5}
            \end{enumerate}
        \end{enumerate}
    \end{enumerate}
\end{enumerate}

\section{Immunity from Jurisdiction I (State Immunity)}
\begin{enumerate}
    \item Is this a situation involving the prosecution of a foreign state or an entity of a foreign state?
    \begin{enumerate}
        \item Subject to certain exceptions, foreign states and their officials enjoy procedural immunity from civil and criminal proceedings in the forum state -- \case{\textit{Pinochet (No 3)} [1999] 2 All ER 97 (Lord Brown-Wilkinson)}
        \begin{enumerate}
            \item This is derived from the idea that all sovereign states are equal, which is one of the fundamental principles of international law -- \case{\textit{Jurisdictional Immunities Case} [2012] ICJ Rep 99}
        \end{enumerate}
        \item Does the state follow the absolute or restrictive immunity principle?
        \begin{enumerate}
            \item The default/prevailing principle is the restrictive immunity principle
            \item The absolute immunity principle holds that states are completely immune from the jurisdiction of another state, and that this applies to all state acts, criminal and civil -- \case{\textit{Exchange v McFadden} (1812) (US Supreme Court)}
            \item The restrictive immunity principle holds that foreign states are afforded immunity only with respect to acts of a sovereign or governmental character, not for acts of a commercial character -- \case{\textit{Holland v Lampen-Wolfe} (2000) HoL}
            \begin{enumerate}
                \item This is partially codified in \convention{\textit{2004 Convention on Jurisdictional Immunities of States and Their Properties}} (however, it only had 24 signatories, short of the 30 required to bring it into force)
                \item Australia follows the restrictive immunity principle -- \statute{\textit{Foreign State Immunities} Act 1985 (Cth)}
            \end{enumerate}
        \end{enumerate}
    \end{enumerate}
    \item Is this a situation involving the Australian approach to foreign state immunity?
    \begin{enumerate}
        \item Is this a civil proceeding? - \convention{\textit{Foreign State Immunities Act 1985} (Cth) s 3(1)}
        \begin{enumerate}
            \item These provisions do not apply to criminal proceedings
            \item Proceedings can include a proceeding commenced by a regulator, not only the state (e.g., the ACCC seeking penalties for anti-competitive conduct) -- \case{\textit{PT Garuda v ACCC} [2012] HCA 33}
            \item Immunity is generally afforded to foreign states, except for the exceptions listed -- \convention{\textit{Foreign State Immunities Act} 1985 (Cth) s 9}
            \begin{enumerate}
                \item They apply to all foreign states, and also to `separate entities' of foreign states (i.e., organs of the government)
            \end{enumerate}
            \item Is an entity an organ/part of the state?
            \begin{enumerate}
                \item The most relevant factor in determining this is whether the entity was carrying out the foreign state's functions -- \case{\textit{PT Garuda v ACCC} [2011] FCFCA 52 (Lander and Greenwood JJ)}
                \item If an agency has governmental characteristics, it is likely to be granted immunity on this basis -- \case{\textit{PT Garuda v ACCC} [2012] HCA 33}
                \item Immunity is afforded to ``agencies or instrumentalities of the state or other entities, to the extent that they are entitled to perform and are actually performing acts in the exercise of sovereign authority of the state'' -- \convention{\textit{UN Convention} Art 2(1)(b)(iii)}
            \end{enumerate}
            \item Judgement can only be enforced/executed on the commercial property of a state, not their sovereign (diplomatic or military) property, unless the other state has provided a waiver -- \statute{\textit{Foreign State Immunities Act 1985} (Cth) ss 30-32}; \case{\textit{Firebird Global Master Fund II Ltd v Nauru} (2015) 326 ALR 396}
        \end{enumerate}
        \item Has the foreign state submitted to Australia's jurisdiction? -- \statute{\textit{Foreign State Immunities Act 1985} (Cth) s 10}
        \begin{enumerate}
            \item A waiver of immunity via an international agreement must be explicitly derived from its terms (either as an express provision or as a term implied by necessity) -- \case{\textit{Kingdom of Spain v Infrastructure Services Luxembourg S.à.r.l} [2023] HCA 11 at [25]}
            \item There is no implied exception to foreign state immunity, even on the basis of the jus cogens prohibition of torture -- \case{\textit{Zhang v Zemin} [2010] NSWCA 255}
            \item The foreign state may not withdraw their waiver once they have made it -- \statute{\textit{Foreign State Immunities Act 1985} (Cth) s 10(5)}
            \item The foreign state will have been taken to submit to jurisdiction by:
            \begin{enumerate}
                \item Instituting the proceeding -- \statute{\textit{Foreign State Immunities Act 1985} (Cth) s 10(6)(a)}
                \item Intervening in, or taking a step as a party to, the proceeding -- \statute{\textit{Foreign State Immunities Act 1985} (Cth) s 10(6)(b)}
            \end{enumerate}
            \item The foreign state will not have been taken to have submitted to jurisdiction if:
            \begin{enumerate}
                \item They have made an application for costs -- \statute{\textit{Foreign State Immunities Act 1985} (Cth) s 10(7)(a)}
                \item They have intervened for the purpose of asserting immunity -- \statute{\textit{Foreign State Immunities Act 1985} (Cth) s 10(7)(b)}
                \item If they are not a party to a proceeding, and have asserted an interest in property affected by that proceeding -- \statute{\textit{Foreign State Immunities Act 1985} (Cth) s 10(8)}
            \end{enumerate}
        \end{enumerate}
        \item A foreign state does not have immunity in a commercial transaction -- \statute{\textit{Foreign State Immunities Act 1985} (Cth) s 11}
        \begin{enumerate}
            \item Arrangements between a state and its entity of a ``commercial, trading and business character" are not afforded immunity -- \case{\textit{PT Garuda v ACCC} [2012] HCA 33 (French CJ, Gummow, Hayne and Crennan JJ)}
            \begin{enumerate}
                \item The exception applies if the underlying character of the case is commercial in nature
                \item In \case{\textit{PT Garuda v ACCC} [2012] HCA 33}, while the court accepted that PT Garuda was an instrumentality that had state immunity, the alleged act was commercial in nature and hence could not be afforded state immunity
            \end{enumerate}
            \item Payment of services (e.g., education) constitutes a commercial transaction -- \case{\textit{Australian International Islamic College Board Inc v Saudi Arabia} [2013] QCA 129}
            \item This exception is justified by neither threatening the dignity of the state nor interfering with its sovereign functions -- \case{\textit{Australian International Islamic College Board Inc v Saudi Arabia} [2013] QCA 129 at [24] (Holmes JA)}
            \item To distinguish between a governmental act and a non-governmental act, consider whether the act is one that any private citizen can perform -- \case{\textit{Kuwait Airways v Iraqi Airlines} (1955) Supreme Court of Canada}
            \item It is insufficient that the entity acted on the directions of the state; to attract immunity, it must be considered that the acts done by the separate entity possessed a governmental character -- \case{\textit{Kuwait Airways v Iraqi Airlines} (1955) Supreme Court of Canada}
        \end{enumerate}
        \item A foreign state does not have immunity where they are acting as an employer -- \statute{\textit{Foreign State Immunities Act 1985} (Cth) s 12}
        \item A foreign state does not have immunity if they caused personal injury or damage to property -- \statute{\textit{Foreign State Immunities Act 1985} (Cth) s 13}
        \begin{enumerate}
            \item Causing in injury to a person or property (e.g., by shooting) results in immunity not applying -- \case{\textit{Tokic v Government of Yugoslavia} (1999) NSWSC Unreported}
        \end{enumerate}
        \item A foreign state does not have immunity if they own or had interest in immovable property -- \statute{\textit{Foreign State Immunities Act 1985} (Cth) s 14}
        \item A foreign state does not have immunity where they are involved in a proceeding concerning intellectual property -- \statute{\textit{Foreign State Immunities Act 1985} (Cth) s 15}
        \item A foreign state does not have immunity where they are a member of a body corporate or unincorporated body/partnership established or operating in Australia -- \statute{\textit{Foreign State Immunities Act 1985} (Cth) s 16}
        \item A foreign state does not have immunity where they are involved in arbitration proceedings -- \statute{\textit{Foreign State Immunities Act 1985}(Cth) s 17}
        \item A foreign state does not have immunity where they are involved in proceedings in connection with a commercial ship -- \statute{\textit{Foreign State Immunities Act 1985} (Cth) s 18}
        \item Is the foreign state involved in a proceeding involving a bill of exchange? -- \statute{\textit{Foreign State Immunities Act 1985} (Cth) s 19}
        \item A foreign state is not immune in proceedings concerning its taxation obligations in Australia -- \statute{\textit{Foreign State Immunities Act 1985}(Cth) s 20}
    \end{enumerate}
    \item Is this a situation involving a foreign state official?
    \begin{enumerate}
        \item \Gls{jus cogens} norms do not override immunity -- \case{\textit{Jurisdictional Immunities of State Case} (ICJ)}
        \item The immunity of individuals as representatives of a foreign state is generally confined to the acts that they perform in their official capacity, carrying out governmental functions -- \case{\textit{Holland v Lampen-Wolfe} (2000) HoL}
        \begin{enumerate}
            \item Individuals cannot sue the state for personal injury, as the state can only act through servants and agents, with those official acts being acts of the state, and the state's immunity in respect of them is fundamental to the principle of state immunity -- \case{\textit{Jones v Saudi Arabia} [2000] HoL}; \case{\textit{Al-Adsani v UK} [2001] ECHR 657}
        \end{enumerate}
        \item Is the official an incumbent head of state facing civil proceedings in their representative capacity?
        \begin{enumerate}
            \item The ordinary principles of state immunity apply -- \statute{\textit{Foreign States Immunities Act 1985} (Cth) s 3(3)(b)}
        \end{enumerate}
        \item Is the official an incumbent head of state facing civil proceedings in their personal capacity?
        \begin{enumerate}
            \item A foreign head of state is entitled to the same immunity as a head of diplomatic mission, which grants them a wide immunity for civil proceedings -- \statute{\textit{Foreign States Immunities Act 1985} (Cth) s 36}; \case{\textit{Thor Shipping A/S v `Al Duhail'} (2008) 252 ALR 20}
            \item An incumbent foreign head of state is granted immunity from \gls{jus cogens} violations -- \case{\textit{Tatchell v Mugabe} (2004) 136 ILR 572}
        \end{enumerate}
        \item Is the official an incumbent head of state facing criminal proceedings in their personal capacity?
        \begin{enumerate}
            \item A foreign head of state is entitled to the same immunity as a head of diplomatic mission, which is a complete personal immunity for criminal proceedings -- \statute{\textit{Foreign States Immunities Act 1985} (Cth) s 36}; \case{\textit{Gaddafi} (2001) 125 ILR 490}
        \end{enumerate}
        \item Is the official a former head of state?
        \begin{enumerate}
            \item Former heads of state lose their personal immunity once they leave office, but retain functional immunity for acts performed in their official capacity whilst they were in office (reflected in \statute{\textit{Foreign States Immunities Act 1985} (Cth) s 36})
            \item A former head of state or the estate of a deceased head of state is not entitled to immunity in civil proceedings in respect of a private act done whilst in office as head of a foreign state -- \statute{\textit{Harb v Prince Abdul Aziz Bin Fahd Bin Abdul Aziz} [2013] EWCA Div 481}
        \end{enumerate}
        \item Is the official an incumbent foreign affairs minister?
        \begin{enumerate}
            \item An incumbent foreign affairs minister is immune from criminal proceedings -- \case{\textit{Arrest Warrant Case} [2002] ICJ Rep 3}
            \begin{enumerate}
                \item This immunity does not distinguish between acts performed in an official or personal capacity
                \item It is irrelevant if the Minister was in the arresting state on an official or private visit
                \item There are no exceptions for \gls{jus cogens} violations
                \item The Minister can still face prosecution in their home state if the home state chooses to waive immunity (although this is rare)
            \end{enumerate}
            \item Once the Minister leaves office, a much more limited residual immunity applies for the acts they performed in their official capacity - \case{\textit{Arrest Warrant Case} [2002] ICJ Rep 3}
            \item A foreign minister could still have individual criminal responsibility for acts committed before and after their time in office, and can be prosecuted as such -- \case{\textit{Arrest Warrant Case} [2002] ICJ Rep 3}
        \end{enumerate}
        \item Is the official otherwise part of the `troika', the minister for defence, or the minister for international trade?
        \begin{enumerate}
            \item The Troika consists of the Head of State, Head of Government and Minister for Foreign Affairs
            \item Prior to entering office, the troika is not afforded any immunity, and so they can be prosecuted for acts committed before they entered office
            \item The troika is afforded personal and material immunity by virtue of their position/status during their time in office for both civil and criminal proceedings -- \case{\textit{Arrest Warrant Case} [2002] ICJ Rep 3 at [51], [55]}
            \item The personal immunity ceases when they leave office and only a residual functional immunity applies for acts performed in their official capacity thereafter -- \case{\textit{Arrest Warrant Case} [2002] ICJ Rep 3}
            \item Personal immunity during their time in office is afforded to ensure the effective performance of their functions, and to enable them to travel freely to perform those functions -- \case{\textit{Arrest Warrant Case} [2002] ICJ Rep 3 at [53]}
            \begin{enumerate}
                \item No case may be brought against a serving member of the troika for acts done before entering office; a state must wait until their time in office has ended to pursue such a case -- \case{\textit{Arrest Warrant Case} [2002] ICJ Rep 3}
            \end{enumerate}
            \item This principle was codified in \convention{\textit{ILC Draft Article 3}}, and has found footing in \case{\textit{Arrest Warrant Case} [2002] ICJ Rep 3}
            \item Immunity may be extended to the Minister for Defence -- \case{\textit{Re Mofaz} (UK, 2004)}
            \item Immunity may be extended to the Minister for International Trade -- \case{\textit{Re Bo Xilai} (UK, 2005)}
            \item Immunity may be waived by the minister's own state, and the home state could prosecute the acts -- \case{\textit{Arrest Warrant Case} [2002] ICJ Rep 3}
            \item Immunity may otherwise be extended, depending on the nature of the diplomatic activities and whether there is a requirement of travel to fulfil those duties -- \case{\textit{Arrest Warrant Case} [2002] ICJ Rep 3 at [53]}
            \item However, the ILC has suggested to limit the scope of immunity only to the troika
        \end{enumerate}
    \end{enumerate}
    \item Is there a violation of a \gls{jus cogens} norm?
    \begin{enumerate}
        \item A state or state organ is not derived of functional immunity by reason of a serious violation of international human rights law or even a \gls{jus cogens} norm in the course of conducting a sovereign act -- \case{\textit{Jurisdictional Immunities (Germany v Italy; Greece Intervening)} [2012] ICJ Rep 99}
        \begin{enumerate}
            \item The only exception to this is the violation against the prohibition of torture -- \case{\textit{Pinochet (No 3)} [1992] 2 All ER 97}
            \item However, immunity is granted for civil claims, even if it is concerning torture -- \case{\textit{Jones v Ministry of Interior of the Kingdom of Saudi Arabia} [2006] UKHL 26}
            \begin{enumerate}
                \item E.g., Kuwait was entitled to state immunity from a civil suit from a person alleging state torture -- \case{\textit{Al-Adsani v UK} [2001] ECHR 657}
            \end{enumerate}
        \end{enumerate}
    \item Is this a situation concerning torture by a foreign state official?
    \begin{enumerate}
        \item Torture at the hands of a state are acts of an official character, and so heads of state are immune from prosecution for them -- \case{\textit{Jones v Saudi Arabia} [2007] AC 270}; \case{\textit{R v Bow Street Stipendary Magistrate; ex parte Pinochet (No 3)} [1992] 2 All ER 97}
        \begin{enumerate}
            \item However, such an immunity might not apply, even if the crime was serious, irrespective of whether the crime was committed whilst the head of state was in office or not -- \case{\textit{Pinochet} [1992] 2 All ER 97} (where immunity did not apply to torture)
        \end{enumerate}
        \item In Australia, torture by an official acting for a state is subject to immunity -- \convention{\textit{Zhang v Zemin} [2010] NSWCA 255}
        \begin{enumerate}
            \item This principle is reflected in the UK -- \case{\textit{Al Adsani v Kuwait} (1995) 103 ILR 420 (QBD and CA)}; \case{\textit{Jones v Saudi Arabia} [2007] AC 270}
        \end{enumerate}
    \end{enumerate}
    \end{enumerate}
\end{enumerate}

\begin{table}[H]
    \centering
    \begin{tabular}{|>{\raggedright\arraybackslash}p{0.2\textwidth}|>{\raggedright\arraybackslash}p{0.2\textwidth}|>{\raggedright\arraybackslash}p{0.2\textwidth}|>{\raggedright\arraybackslash}p{0.2\textwidth}|}
        \hline
        \textbf{Time} & \textbf{Before Office} & \textbf{During Office} & \textbf{After Office} \\\hline
        \textbf{Private \& Criminal} & No Immunity & Immune & No Immunity \\\hline
        \textbf{Private \& Civil} & No Immunity & Immune & No Immunity \\\hline
        \textbf{Official \& Criminal} & No Immunity & Immune & Residual Immunity (except for torture)\\\hline
        \textbf{Official \& Civil} & No Immunity & Immune & Residual Immunity\\\hline
    \end{tabular}
\end{table}

\section{Immunity From Jurisdiction II (Diplomatic Immunity)}
\begin{enumerate}
    \item Does a foreign state official have immunity from proceedings in international criminal courts?
    \begin{enumerate}
        \item The official capacity of an individual is irrelevant before an international criminal court -- \statute{\textit{Rome Statute of the International Criminal Court} Art 27}
        \item However, difficulties arise when states do not cooperate or are not part of the ICC -- \case{\textit{Al Bashir Case}} (in this case, Sudan is not a member of the ICC, and moreover refuses to cooperate with the ICC's warrant for al-Bashir)
    \end{enumerate}
    \item Does a foreign state official have immunity for \gls{jus cogens} violations?
    \begin{enumerate}
        \item The acts committed by the armed forces of a foreign state, regardless of where they are committed, are subject to immunity -- \case{\textit{Jurisdictional Immunities of the State (Germany v Italy)} [2012] ICJ Rep 99 at [65]}
        \item Immunity is not available for cases involving torture -- \case{\textit{R v Bow Street Stipendary Magistrate; ex parte Pinochet (No 3)} [1992] 2 All ER 97}
    \end{enumerate}
    \item Is this a case involving the foreign act of state doctrine?
    \begin{enumerate}
        \item The foreign act of state doctrine refers to the notion that ``courts will not adjudicate on the validity of acts and transactions of a foreign sovereign state within that sovereign's own territory" -- \case{\textit{Spycatcher Case} (1988) 165 CLR 30}
        \item This doctrine does not apply where the relevant conduct of a foreign state involves a breach of an established principle of international law -- \case{\textit{Hicks v Ruddock} (2007) 156 FCR 574}; \case{\textit{Habib v Commonwealth} (2010) 183 FCR 62}
    \end{enumerate}
    \item Is this a case involving diplomatic immunity?
    \begin{enumerate}
        \item The laws on diplomatic immunity are codified in the \convention{\textit{1961 Vienna Convention on Diplomatic Relations}}, which is given effect in Australia by the \statute{\textit{Diplomatic Privileges and Immunities Act 1967} (Cth)}
        \begin{enumerate}
            \item Article 1 defines a number of key terms in this area
            \item Article 3 sets out the functions of a diplomatic mission
            \item Articles 4-19 address several procedural matters
        \end{enumerate}
        \item A prerequisite for diplomatic immunity is that the two states must recognise each other as states to establish diplomatic relations
        \begin{enumerate}
            \item Whilst common, a physical presence, such as through an embassy, is not required to establish diplomatic relations -- \article{\textit{Kosovo Thanks You}}
        \end{enumerate}
        \item Diplomatic immunity is necessary to protect and facilitate the international diplomacy in so far as the state allows "not to benefit individuals but to ensure the efficient performance of the functions of diplomatic missions'' -- \convention{\textit{1961 Vienna Convention on Diplomatic Relations} Preamble}
        \item Was there a violation of a diplomat or a mission?
        \begin{enumerate}
            \item The premises of the diplomatic mission are inviolable and may not be entered without the consent of the head of the mission -- \convention{\textit{1961 Vienna Convention on Diplomatic Relations} Art 22(1)}
            \begin{enumerate}
                \item The private residence (Art 30(1)), papers and correspondence of a diplomatic agents are also inviolable, even if they have been used for illegal purposes (Art 30(2)) -- \convention{\textit{1961 Vienna Convention on Diplomatic Relations} Art 30}
                \item Serving legal proceedings on a diplomatic mission without the consent of the head of the mission breaches the mission's inviolability -- \case{\textit{Saudi Arabian Cultural Mission v Alramadi} [2024] FCA 1060}
            \end{enumerate}
            \item The receiving state is under a special duty to take all appropriate steps to protect the premises of the mission against any intrusion or damage and to prevent any disturbance of the peace of the mission or impairment of its dignity -- \convention{\textit{1961 Vienna Convention on Diplomatic Relations} Art 22(2)}; \case{\textit{Tehran Hostages} [1980] ICJ Rep 3}
            \begin{enumerate}
                \item This is a strict obligation; countries must protect the premises of diplomatic missions
                \item A mere peaceful protest does not impair the dignity of the mission or breach its peace -- \case{\textit{Minister for Foreign Affairs \& Trade v Magno} (1992) 37 FCR 298 (French J)}
                \item Such an incident would have to be assessed in the light of the surrounding circumstances and must take into account the particular circumstances of the domestic culture, with the application of this principle being that the sending State takes the receiving State as it finds it -- \case{\textit{Minister for Foreign Affairs \& Trade v Magno} (1992) 37 FCR 298, 236 at [30] (French J)}
            \end{enumerate}
            \item The person of a diplomatic agent is inviolable, and they cannot be arrested or detained - \convention{\textit{1961 Vienna Convention on Diplomatic Relations} Art 29}
            \begin{enumerate}
                \item A ``diplomatic agent" is the head of a mission or a member of the diplomatic staff of the mission - \convention{\textit{1961 Vienna Convention on Diplomatic Relations} Art 1(e)}
            \end{enumerate}
        \end{enumerate}
        \item Are there any limits to diplomatic immunity?
        \begin{enumerate}
            \item The sending state may not interfere with the internal affairs of the receiving state, nor use the mission premises in any manner incompatible with the functions of the mission or with international law generally (e.g., drug smuggling) -- \convention{\textit{1961 Vienna Convention on Diplomatic Relations} Art 41}
            \item If a diplomatic agent is declared \textit{persona non grata}, they must be recalled or their functions terminated -- \convention{\textit{1961 Vienna Convention on Diplomatic Relations} Art 9}
            \item The immunity of a diplomat could be waived by the sending state -- \convention{\textit{1961 Vienna Convention on Diplomatic Relations} Art 32}
        \end{enumerate}
        \item Is the individual a diplomatic agent?
        \begin{enumerate}
            \item Diplomatic agents enjoy administrative, civil and criminal immunity for their personal and official acts, except for: -- \convention{\textit{1961 Vienna Convention on Diplomatic Relations} Art 31(1)}
            \begin{enumerate}
                \item Real action relating to private immovable property in the territory of the receiving state -- \convention{\textit{1961 Vienna Convention on Diplomatic Relations} Art 31(1)(a)}
                \item Action relating to succession in which the diplomatic agent is involved in their private capacity -- \convention{\textit{1961 Vienna Convention on Diplomatic Relations} Art 31(1)(b)}
                \item Actions relating to any professional or commercial activity outside the scope of their duties (e.g., a side business; this does not apply to ordinary contracts that are incidental to the daily life of a diplomat) -- \convention{\textit{1961 Vienna Convention on Diplomatic Relations} Art 31(1)(c)}
            \end{enumerate}
            \item The exceptions under Art 31 do not extend to family law disputes -- \case{\textit{Diplomatic Immunity Case} (1973) Family Court of Australia}
            \item The exceptions under Art 31 do not extend to employment relationships that are a form of modern slavery -- \case{\textit{Basfar v Wong} [2022] UKSC 20}; \case{\textit{Reyes v Al-Malki} [2017] UKSC 61; [2019] AC 735}
            \begin{enumerate}
                \item This is a ``commercial activity" under \convention{\textit{1961 Vienna Convention on Diplomatic Relations} Art 31(1)(c)}
                \item \case{\textit{Basfar v Wong} [2022] UKSC 20} affirmed that the commercial activities exception (as established in the equivalent UK legislation) did not apply to ordinary contracts, such as employment contracts, that are incidental to the daily life of a diplomat, at \case{[37]}; however, it was held at \case{[56]} that the exception did not apply as this was not an ordinary contract since the diplomat had derived financial benefit by not paying the worker
            \end{enumerate}
            \item Other immunities available to diplomatic agents include:
            \begin{enumerate}
                \item They are not required to give evidence in court -- \convention{\textit{1961 Vienna Convention on Diplomatic Relations} Art 31(2)}
                \item They are exempt from social security provisions for themselves or for private servants in the mission -- \convention{\textit{1961 Vienna Convention on Diplomatic Relations} Art 33}
                \item They are immune from most taxes -- \convention{\textit{1961 Vienna Convention on Diplomatic Relations} Art 34}
                \item They are exempt from all personal service, from all public service obligations, and military obligations -- \convention{\textit{1961 Vienna Convention on Diplomatic Relations} Art 35}
                \item They are exempt from all customs duties and taxes and charges for articles for the official or personal use of the diplomatic agent, their family members, and the mission -- \convention{\textit{1961 Vienna Convention on Diplomatic Relations} Art 36(1)}
                \item However, if the articles are strongly suspected to violate such protections, then such a protection does not apply -- \convention{\textit{1961 Vienna Convention on Diplomatic Relations} Art 36(2)}
            \end{enumerate}
        \end{enumerate}
        \item A diplomatic agent's family members, if they are not nationals of the receiving state, enjoy the same privileges and immunities as the diplomatic agent -- \convention{\textit{1961 Vienna Convention on Diplomatic Relations} Art 37(1)}
        \item Administrative/technical staff of the mission, if they are not nationals or permanent residents of the receiving state, enjoy the same immunities as the diplomatic agent, except that the immunity from civil and administrative jurisdiction does not extend to acts performed outside the course of their duties -- \convention{\textit{1961 Vienna Convention on Diplomatic Relations Art} 37(2)}
        \begin{enumerate}
            \item The members of the administrative and technical staff are the members of the staff of the mission employed in the administrative and technical services of the mission -- \convention{\textit{1961 Vienna Convention on Diplomatic Relations} Art 1(f)}
        \end{enumerate}
        \item Service staff of the mission, if they are not nationals or permanent residents of the receiving state, enjoy immunities in respect of acts performed in the course of their duties, and exemption from dues and taxes on their salary/wages they receive from their employment -- \convention{\textit{1961 Vienna Convention on Diplomatic Relations} Art 37(3)}
        \begin{enumerate}
            \item This generally includes members like the kitchen staff, cleaners and other support staff, and encompasses all staff who are members of the staff of the mission in the domestic service of the mission -- \convention{\textit{1961 Vienna Convention on Diplomatic Relations} Art 1(g)}
            \item They are also eligible to avail the exception under \convention{\textit{1961 Vienna Convention on Diplomatic Relations} Art 33} exempting them from social security provisions
        \end{enumerate}
        \item Private servants in the mission, if they are not nationals or permanent residents of the receiving state, are exempt from dues and taxes on their wages -- \convention{\textit{1961 Vienna Convention on Diplomatic Relations} Art 37(4)}
        \begin{enumerate}
            \item Private servants are those who are in the domestic service of a member of the mission and who are not an employee of the sending state -- \convention{\textit{1961 Vienna Convention on Diplomatic Relations} Art 1(h)}
        \end{enumerate}
        \item Is there an abuse of diplomatic immunities?
        \begin{enumerate}
            \item If a diplomat abuses their immunities, the receiving state may declare them \textit{persona non grata}, without needing to provide a reason -- \convention{\textit{1961 Vienna Convention on Diplomatic Relations} Art 9(1)}
            \begin{enumerate}
                \item The sending state must then recall the diplomat or terminate their functions with the mission -- \convention{\textit{1961 Vienna Convention on Diplomatic Relations} Art 9(1)}
                \item If the sending state fails in their obligations, the receiving state can refuse to recognise them as a member of the mission -- \convention{\textit{1961 Vienna Convention on Diplomatic Relations} Art 9(2)}
            \end{enumerate}
            \item A sending state may waive the immunity of a diplomatic agent -- \convention{\textit{1961 Vienna Convention on Diplomatic Relations} Art 32(1)}
            \begin{enumerate}
                \item This waiver must be express -- \convention{\textit{1961 Vienna Convention on Diplomatic Relations} Art 32(2)}
                \item If a diplomatic agent initiates proceedings they otherwise had immunity for, they are precluded from invoking immunity from jurisdiction in respect of any directly-connected counterclaim -- \convention{\textit{1961 Vienna Convention on Diplomatic Relations} Art 32(3)}
                \item A waiver relating to civil and administrative jurisdiction does not imply a waiver of immunity in respect of the execution of the judgement, for which a separate waiver is required, and vice-versa - -1\convention{\textit{961 Vienna Convention on Diplomatic Relations} Art 32(4)}
            \end{enumerate}
            \item Waiver can sometimes be waived by the sending state with respect to car accidents
            \begin{enumerate}
                \item This applies to civil proceedings -- \case{\textit{Dickson v Del Solar} [1930] 1 KB 376}
                \item This applies to criminal proceedings -- \case{\textit{Georgian Diplomat in Washington D.C.} (1998)}
            \end{enumerate}
            \item Diplomatic agents enjoy privileges and immunities from the moment they enter the territory of the receiving state -- \convention{\textit{1961 Vienna Convention on Diplomatic Relations} Art 39(1)}
            \item When the function of a person enjoying diplomatic immunities ends, their personal immunity ceases the moment they leave the country, but a residual functional immunity covers acts performed during their official functions -- \convention{\textit{1961 Vienna Convention on Diplomatic Relations} Art 39(2)}
        \end{enumerate}
    \end{enumerate}
    \item At the international level, no diplomatic or state immunity can be afforded to any persons, and hence a diplomatic agent can be tried in an international criminal court -- \statute{\textit{Rome Statute of the International Criminal Court} Art 27}
    \begin{enumerate}
        \item However, this balances/respects the immunity provided at a diplomatic level (which is done for the convenience of the state to ensure its own diplomats are not targeted); ``The Court may not proceed with a request for surrender or assistance which would require the requested state to act inconsistently with its obligations under international law with respect to the state or diplomatic immunity of a person or property of a third state, unless the Court can first obtain the cooperation of that third state for the waiver of the immunity'' -- \statute{\textit{Rome Statute of the International Criminal Court} Art 98(1)}
        \item For states that are party to the \statute{\textit{Rome Statute}}, \statute{Art 27} will apply, but the situation gets more complex for states that are not party to the \statute{\textit{Rome Statute}}, as they may not recognise the ICC's jurisdiction
    \end{enumerate}
\end{enumerate}

\section{State Responsibility I (Wrongful Acts of State)}
\begin{enumerate}
    \item These rules do not apply in determining if there was a breach, but merely determine what happens following a breach -- \convention{\textit{Articles on State Responsibility for Internationally Wrongful Acts} (2001) Art 55}
    \begin{enumerate}
        \item The responsibility of a state may arise if:
        \begin{enumerate}
            \item An act or omission is attributable to a state
            \item The act or omission breaches international law
            \item The state is not able to raise any defence for the wrongful act or omission
            \item Another state invokes the responsibility of the state
        \end{enumerate}
    \end{enumerate}
    \item For these rules to apply, there must be a breach of a primary obligation or an internationally wrongful act -- \convention{\textit{2001 Articles on the Responsibility of States for Internationally Wrongful Acts} Art 2}
    \begin{enumerate}
        \item If such a breach has occurred, then injured state may invoke the responsibility of the other state for the injury caused by the internationally wrongful act -- \convention{\textit{2001 Articles on the Responsibility of States for Internationally Wrongful Acts} Art 1}
        \item If an act is considered unlawful under international law, it is irrelevant as to whether the act is lawful under domestic law -- \convention{\textit{2001 Articles on the Responsibility of States for Internationally Wrongful Acts} Art 3}
        \item If a state did not breach a provision but was not acting in conformity, there is still a breach of an international obligation and the rules around state responsibility are enlivened, irrespective of the origin or character of the act -- \convention{\textit{2001 Articles on the Responsibility of States for Internationally Wrongful Acts} Art 12}
        \item Was there a continuing breach from an initial first breach?
        \begin{enumerate}
            \item If the obligation is a continuing obligation, any further breaches from the first breach would also be considered as a continuing breach, so long as the state is not acting in conformity with the obligation -- \convention{\textit{2001 Articles on the Responsibility of States for Internationally Wrongful Acts} Art 14}
            \item The consequence of a continuing wrongful act will depend on the context as well as duration of the obligation breached, but will usually lead to a more onerous reparation (as the seriousness of the breach is merely increased if it is continued) -- \case{\textit{Rainbow Warrior Arbitration} (1990) XX RIAA 215}
            \begin{enumerate}
                \item ``It had practical consequences, since the seriousness of the breach and its prolongation in time cannot fail to have considerable bearing on the establishment of reparation which is adequate for a violation presenting these two features'' -- \case{\textit{Rainbow Warrior Arbitration} (1990) XX RIAA 215 at [101]}
            \end{enumerate}
        \end{enumerate}
        \item If the treaty/convention has its own primary rules in respect to the consequence of a breach, the rules of state responsibility are not enlivened, and the primary rules will apply instead -- \convention{\textit{2001 Articles on the Responsibility of States for Internationally Wrongful Acts} Art 55}
        \begin{enumerate}
            \item The majority of these provisions are considered to be customary international law, and are considered authoritative by the ICJ, despite not being a formal convention
        \end{enumerate}
        \item Breach of obligations under a treaty are still considered to be breaches of international law -- \case{\textit{Belgium v Senegal} [2012] ICJ Rep 422}
    \end{enumerate}
    \item Can a wrongful act be attributed to a state?
    \begin{enumerate}
        \item The conduct of any state organ (irrespective of which level of government) will be attributed to that state -- \convention{\textit{2001 Articles on the Responsibility of States for Internationally Wrongful Acts} Art 4(1)}
        \begin{enumerate}
            \item An organ includes any person or entity which held has that status in accordance with the state's internal law -- \convention{\textit{2001 Articles on the Responsibility of States for Internationally Wrongful Acts} Art 4(2)}
            \item Acts of a state government are attributable to the federal government of a state -- \convention{\textit{Kalgoorlie Riots Incident} (1934)}
            \item Acts of a court are attributable to the federal government of a state -- \convention{\textit{Immunity from Legal Process Advisory Opinions} [1999] ICJ Rep 62}
            \item Governments of states are internationally responsible for the actions of their courts -- \convention{\textit{Immunity from Legal Process Advisory Opinions} [1999] ICJ Rep 62}
        \end{enumerate}
        \item The rules of attribution arise when it is not clear whether a state had directly been responsible
        \begin{enumerate}
            \item A state is taken to be responsible if they failed to take certain actions
            \begin{enumerate}
                \item ``Every State's obligation not to allow knowingly its territory to be used for acts contrary to the rights of other States" (i.e., if a state is aware of the presence of danger, they must warn other states) - \case{\textit{Corfu Channel} [1949] ICJ Rep 4}, where Albania had a direct responsibility to not allow their territory to be used for unlawful purposes
            \end{enumerate}
            \item States have a responsibility to ensure that activities within their jurisdiction or control do not cause damage to the environment of other states or to areas beyond the limits of their jurisdiction -- \case{\textit{Trail Smelter} (1938/1941) III RIAA 1905}, where the failure to prevent pollution from a metals smelter in British Columbia causing damage to farms in Washington amounted to a failure of an obligation
        \end{enumerate}
        \item Did an entity act in excess of their authority from a state or otherwise contravened instructions?
        \begin{enumerate}
            \item If an organ of the state acts in excess of their authority or otherwise contravenes their instructions, states are still responsible/attributable for the actions of those organs -- \convention{\textit{2001 Articles on the Responsibility of States for Internationally Wrongful Acts} Art 7}; \case{\textit{Youmans v Mexico} (1926) IV RIAA 110}
            \begin{enumerate}
                \item This article has been interpreted as including organs acting with `apparent authority'; i.e., when the conduct appears to be state-authorised, even when it is not (e.g., wearing military uniforms, even for purely private acts) -- \convention{\textit{Caire Claim} (1929) 5 ADPIL Cases 146}
                \item The test for ascertaining apparent authority is whether the officials were acting using official means at their disposal; if they were, then irrespective of whether the conduct was authorised by the state or not, the conduct is attributable to the state -- \case{\textit{Southern Pacific Properties v Egypt} (1933)}
            \end{enumerate}
            \item If the conduct is purely private and there is no visible or apparent connection to the state, then there is no question of state responsibility, though courts have almost always erred towards attributing the conduct to the state except when it is glaringly obvious it is not
        \end{enumerate}
        \item The conduct of a non-organ can be attributed to the state if it is empowered by the law of the state to exercise elements of the state's governmental authority -- \convention{\textit{2001 Articles on the Responsibility of States for Internationally Wrongful Acts} Art 5}
        \begin{enumerate}
            \item There must be statute expressly authorising this
            \item The entity must be acting in that capacity at that moment
            \item The term `entities' enables parastatal entities that exercise governmental authority to serve the functions of a public character to be covered under the law of state responsibility -- \article{\textit{ILC Commentary on Article 5}}
        \end{enumerate}
        \item Conduct that is directed by or controlled by the state will be attributed to the state if they are acting on the instructions of, or under the direction or control of, the state that is carrying out the conduct -- \convention{\textit{2001 Articles on the Responsibility of States for Internationally Wrongful Acts} Art 8}
        \begin{enumerate}
            \item Conduct will be attributable to the state only if the state directed or controlled the specific operation in question and the conduct complained of was an integral part of that operation -- \article{\textit{ILC Commentary on Article 8}}
            \begin{enumerate}
                \item This follows the test of `effective control' (\case{\textit{Nicaragua v US} [1986] ICJ Rep 14}), as opposed to the `overall control' test (\case{\textit{`Tadić'} (1999) ICTY})
            \end{enumerate}
            \item For conduct to be directed by or controlled by the state, the state must have ``effective control of the operations in which the alleged violations were committed" - \case{\textit{Nicaragua v US} [1986] ICJ Rep 14 at [116]}; \case{\textit{Genocide Case} [2007] ICJ Rep 43} (the \case{\textit{Genocide Case}} affirms the `effective control' test)
            \begin{enumerate}
                \item There must be effective control ``in respect of each operation in which alleged violations incurred, not generally in respect of the overall actions taken by the persons or groups of persons having committed the violations'' -- \case{\textit{Genocide Case} [2007] ICJ Rep 43}
                \item Not having control over the choice of targets tends towards not having an effective control over the operations -- \case{\textit{Nicaragua v US} [1986] ICJ Rep 14}; \case{\textit{Genocide Case} [2007] ICJ Rep 43}
                \item Support given to actors, such as funding, training, etc., dp not amount to effective control if there is no conduct and state connection -- \case{\textit{Nicaragua v US} [1986] ICJ Rep 14}; \case{\textit{Genocide Case} [2007] ICJ Rep 43}
            \end{enumerate}
        \end{enumerate}
        \item The conduct of a person or a group of persons will be considered to be acts of the state if they are stepping in to exercise authority in the absence of official authorities - \convention{\textit{2001 Articles on the Responsibility of States for Internationally Wrongful Acts} Art 9}
        \begin{enumerate}
            \item E.g., in the 1978/79 Islamic revolution in Iran where the government was overthrown, Revolutionary Guards continued to perform official functions (e.g., customs duties at Tehran airport, with the actions of these Guards being attributed to the state)
        \end{enumerate}
        \item Was there an insurrectional/revolutionary movement?
        \begin{enumerate}
            \item The conduct of an insurrectional movement is not attributable to a state unless that insurrection is successful and they supplant the government of the state -- \convention{\textit{2001 Articles on the Responsibility of States for Internationally Wrongful Acts} Art 10(1)}
            \begin{enumerate}
                \item If the revolution is unsuccessful, then its actions can never be attributed to the state
                \item If the revolution is successful and forms a new government, then the actions of the insurrectional movement will be attributed to the state from the instance that the revolutionary movement began (not when the new government was formed) -- \case{\textit{Bolivar Railways Company Case} (1903) IX RIAA 445}
            \end{enumerate}
            \item If a revolutionary movement forms a new state from part of the territory of a pre-existing state or from territory under its administration, their conduct will be attributed to the new state -- \convention{\textit{2001 Articles on the Responsibility of States for Internationally Wrongful Acts} Art 10(2)}
            \item The acts of supporters of a revolution cannot be attributed to the new government following the success of the revolution, just as the acts of supporters of an existing government are not attributable to the government -- \case{\textit{Short v Iran} (1987) Iran-USTCR 76}
        \end{enumerate}
        \item Was the conduct acknowledged and adopted by the state?
        \begin{enumerate}
            \item Irrespective of whether conduct is attributable or not, if a state acknowledges and adopts the conduct as its own, then it is attributed to the state to the extent they have declared as such - \convention{\textit{2001 Articles on the Responsibility of States for Internationally Wrongful Acts} Art 11}
            \begin{enumerate}
                \item `Acknowledges and adopts' is a stronger threshold than mere `support and encouragement'
                \item Endorsing a situation and failing to prevent something amounts to ``adopting the conduct in question as its own'' --\case{\textit{US v Iran} (1980) 21 ILM 96}
                \item A state cannot be held responsible for the actions of its governments's supporters -- \case{\textit{Short v Iran} (1987) Iran-USTCR 76}
            \end{enumerate}
            \item Whilst the acts of private individuals or entities may not be attributable to a state, the state may nonetheless be directly responsible for failing to meet its obligations if there is injury to foreign nationals or foreign national interests -- \case{\textit{US Diplomatic and Consular Staff in Tehran} [1980] ICJ Rep 3}
            \begin{enumerate}
                \item If an insurrection happens, the state is not responsible for loss or damage sustained by foreign investors unless it can be shown that the government of that state failed to provide the standard of protection required, either by treaty or under customary law -- \case{\textit{Asian Agricultural Products} (1991) CSID Case No. ARB/87/3}
                \item Governments cannot be responsible for the acts of rebellious bodies where it is guilty of no breach of good faith or of no negligence in suppressing insurrection -- \case{\textit{Home Missionary Society Claim} (1921) VI RIAA 42}
            \end{enumerate}
        \end{enumerate}
    \end{enumerate}
    \item Were there any circumstances precluding wrongfulness (i.e., defences)?
    \begin{enumerate}
        \item These are all complete defences, not partial defences
        \begin{enumerate}
            \item The defences are hard to establish, and tend to apply in relatively extreme circumstances - \case{\textit{Rainbow Warrior Arbitration} (1990) XX RIAA 215}
        \end{enumerate}
        \item Act consented to by a state cannot be wrongful, even if they are otherwise illegal - \convention{\textit{2001 Articles on the Responsibility of States for Internationally Wrongful Acts} Art 20}
        \item If an act is taken in self-defence in conformity with the UN Charter, it is not wrongful - \convention{\textit{2001 Articles on the Responsibility of States for Internationally Wrongful Acts} Art 21}
        \item If a countermeasure meets certain thresholds, it will be regarded as a lawful measure of self-help and not a wrongful act -\convention{\textit{ 2001 Articles on the Responsibility of States for Internationally Wrongful Acts} Art 22}
        \begin{enumerate}
            \item This is founded on the notion that all states are equal under international law
            \item Domestic courts cannot adjudicate whether countermeasures were valid -- \case{\textit{Law Debenture Trust Corpn plc v Ukraine} [2023] UKSC 11}
            \item The countermeasure must be proportional to the injury suffered, and must not breach peremptory norms or otherwise amount to an unlawful use of force
            \item There are also a series of procedural requirements that must be met before undertaking a countermeasure:
            \begin{enumerate}
                \item Before a countermeasure is taken, the state must notify the breaching state to give the breaching state a chance to rectify its behaviour -- \convention{\textit{2001 Articles on the Responsibility of States for Internationally Wrongful Acts} Art 52} (countermeasures cannot be taken if the wrongful act has ceased per Art 52(3)(a), or the dispute is before a tribunal or court per Art 52(3)(b))
                \item A state should, as a first resort, only resort to reversible countermeasures with the aim of permitting the resumption of the performance of obligations -- \convention{\textit{2001 Articles on the Responsibility of States for Internationally Wrongful Acts} Art 49}
                \item Countermeasures can only last so long as the violation persists -- \convention{\textit{2001 Articles on the Responsibility of States for Internationally Wrongful Acts} Art 53}
            \end{enumerate}
        \end{enumerate}
        \item If an act occurs under force majeure (i.e., due to the occurrence of an irresistible force or of an unforeseen event, beyond the control of the State, making performance materially impossible), it will not be considered a wrongful act -\convention{\textit{ 2001 Articles on the Responsibility of States for Internationally Wrongful Acts} Art 23}
        \begin{enumerate}
            \item Force majeure does not apply where the situation is due, either alone or in combination of other factors, to the conduct of the wrongful state (the state raising the defence) -- \convention{\textit{2001 Articles on the Responsibility of States for Internationally Wrongful Acts} Art 23(2)(a)}
            \item Force majeure does not apply where the wrongful state (the state raising the defence) has assumed the risk of that situation occurring -- \convention{\textit{2001 Articles on the Responsibility of States for Internationally Wrongful Acts} Art 23(2)(b)}
            \item Force majeure does not apply where compliance is difficult or burdensome -- \case{\textit{Rainbow Warrior Arbitration} (1990) XX RIAA 215}
        \end{enumerate}
        \item If an act is taken in violation of an international obligation due to a situation of distress, and the state has no other way to save the lives of people or protect property in imminent peril, there is no wrongful act -\convention{\textit{ 2001 Articles on the Responsibility of States for Internationally Wrongful Acts} Art 24(1)} (here, the medical evacuation for treatment was precluded from wrongfulness under distress, but the failure to return him to the military facility was not precluded)
        \begin{enumerate}
            \item Distress entails ``a choice between a departure from an international obligation and a serious threat to the life or physical integrity of a State organ or of persons entrusted to its care" -- \case{\textit{Rainbow Warrior Arbitration} (1990) XX RIAA 215}
            \item Distress is not made out where it is, either alone or in party, caused by the wrongful state (the state invoking the defence) -- \convention{\textit{2001 Articles on the Responsibility of States for Internationally Wrongful Acts} Art 24(2)(a)}
            \item Distress is not made out where the act is likely to create a comparable or greater peril -- \convention{\textit{2001 Articles on the Responsibility of States for Internationally Wrongful Acts} Art 24(2)(b)}
        \end{enumerate}
        \item A wrongful state will not be responsible if they are not able to comply with an international obligation in order to safeguard an essential interest against a grave and imminent peril (i.e., a situation of necessity) -\convention{\textit{ 2001 Articles on the Responsibility of States for Internationally Wrongful Acts} Art 25(1)}
        \begin{enumerate}
            \item This must be the only way for the state to safeguard that essential interest -- \convention{\textit{2001 Articles on the Responsibility of States for Internationally Wrongful Acts} Art 25(1)(a)}
            \begin{enumerate}
                \item Necessity is ``concerned with departure from international obligations on the grounds of vital interests of the state" - \case{\textit{Rainbow Warrior Arbitration} (1990) XX RIAA 215}
                \item Necessity can only be invoked where the action taken is the `only way' to safeguard the essential interest of the state -- \case{\textit{Gabčíkovo-Nagymaros Project} [1997] ICJ Rep 7}
                \item Necessity may only preclude wrongfulness in circumstances where the departure from an international obligation is necessary to avoid a serious threat to the vital interest of the state
            \end{enumerate}
            \item The act performed in necessity must not seriously impair an essential interest of the state/s towards which the obligation exists, or the international community as a whole -- \convention{\textit{2001 Articles on the Responsibility of States for Internationally Wrongful Acts} Art 25(1)(b)}
            \item A state may not invoke necessity if the international obligation in question excludes the possibility of invoking necessity -- \convention{\textit{2001 Articles on the Responsibility of States for Internationally Wrongful Acts} Art 25(2)(a)}
            \item A state may not invoke necessity if it has contributed to the situation which gave rise to the situation of necessity -- \convention{\textit{2001 Articles on the Responsibility of States for Internationally Wrongful Acts} Art 25(2)(b)}
            \item Necessity seems to only be arguable in very strict circumstances, such as where the very existence of the State and its people is threatened -- \article{\textit{ILC Commentary on Article 25}}
        \end{enumerate}
        \item These defences do not apply to \gls{jus cogens} norms -- \convention{\textit{2001 Articles on the Responsibility of States for Internationally Wrongful Acts} Art 26}
        \item A reliance by a state on the circumstances precluding wrongfulness does not affect the underlying obligations, and the state may yet have a duty to compensate if a material loss is suffered by another state -- \convention{\textit{2001 Articles on the Responsibility of States for Internationally Wrongful Acts} Art 27}
    \end{enumerate}
    \item Can a state invoke responsibility against another state?
    \begin{enumerate}
        \item Was the obligation breached owed to the state, or a group of states including that state? -- \convention{\textit{2001 Articles on the Responsibility of States for Internationally Wrongful Acts} Art 42}
        \begin{enumerate}
            \item The obligation can be owed to that state specifically -- \convention{\textit{2001 Articles on the Responsibility of States for Internationally Wrongful Acts} Art 42(1)}
            \item The obligation can be owed to a group of states, or the international community at whole, such that the breach of the obligation: -- \convention{\textit{2001 Articles on the Responsibility of States for Internationally Wrongful Acts} Art 42(2)}
            \begin{enumerate}
                \item Specifically affects that state -- \convention{\textit{2001 Articles on the Responsibility of States for Internationally Wrongful Acts} Art 42(2)(a)}
                \item Is of a character to radically change the position of all other states in further performances of that obligation -- \convention{\textit{2001 Articles on the Responsibility of States for Internationally Wrongful Acts} Art 42(2)(b)}
            \end{enumerate}
        \end{enumerate}
        \item A state may be able to invoke responsibility even if they are not the injured state if: -- \convention{\textit{2001 Articles on the Responsibility of States for Internationally Wrongful Acts} Art 48}
        \begin{enumerate}
            \item The obligation breached is owed to a group of states including that state and is established for the protection of a collective interest of the group -- \convention{\textit{2001 Articles on the Responsibility of States for Internationally Wrongful Acts} Art 48(1)(a)}
            \begin{enumerate}
                \item Breaches of international conventions can fall within this category -- \case{\textit{Whaling in the Antarctic Case} [2014]}
            \end{enumerate}
            \item The obligation breached is owed to the international community as a whole -- \convention{\textit{2001 Articles on the Responsibility of States for Internationally Wrongful Acts} Art 48(1)(a)}
            \begin{enumerate}
                \item There is a common interest to ensure torture is not carried out by a state, and this is something that can be invoked as an obligation owing to the international community -- \case{\textit{Belgium v Senegal} [2012] ICJ Rep 422}
                \item This includes \gls{erga omnes} obligations, such as the right of self-determination -- \case{\textit{Legal Consequences of the Construction of a Wall in the Occupied Palestinian Territory} [2004] ICJ Rep 136 at [122], [155]}
                \item However, both states must consent to the ICJ's jurisdiction even in the case of \gls{erga omnes} obligations -- \case{\textit{DRC v Uganda} [2005] ICJ Rep 168}
            \end{enumerate}
            \item If responsibility is invoked under this, the invoking stay may claim from the responsible state:
            \begin{enumerate}
                \item Cessation of the wrongful act -- \convention{\textit{2001 Articles on the Responsibility of States for Internationally Wrongful Acts} Art 48(2)(a)}
                \item Performance of reparation to either the invoking state or the beneficiaries of the obligation breached -- \convention{\textit{2001 Articles on the Responsibility of States for Internationally Wrongful Acts} Art 48(2)(b)}
            \end{enumerate}
            \item Are there several states injured due to the same wrongful act?
            \begin{enumerate}
                \item Each injured state may separately invoke the responsibility of the state that has committed the wrongful act -- \convention{\textit{2001 Articles on the Responsibility of States for Internationally Wrongful Acts} Art 46}
            \end{enumerate}
            \item Are there several states responsible for the same internationally wrongful act?
            \begin{enumerate}
                \item The responsibility of each state may be invoked in relation to that act -- \convention{\textit{2001 Articles on the Responsibility of States for Internationally Wrongful Acts} Art 47(1)}
                \item An injured state may not recover, by way of compensation, more than the damage it has suffered -- \convention{\textit{2001 Articles on the Responsibility of States for Internationally Wrongful Acts} Art 47(2)(a)}
                \item Making a claim against one wrongful state does not preclude the rights of the injured state to take recourse against any other wrongful state -- \convention{\textit{2001 Articles on the Responsibility of States for Internationally Wrongful Acts} Art 47(2)(b)}
            \end{enumerate}
        \end{enumerate}
    \end{enumerate}
    \item What consequences/remedies will the state owe for their wrongful act?
    \begin{enumerate}
        \item A wrongful state is under an obligation to make reparation for their wrongful acts/injury -- \convention{\textit{2001 Articles on the Responsibility of States for Internationally Wrongful Acts} Art 31(1)}; \convention{\textit{Chorzow Factory Case} (1928) PCIJ (Ser A) No 17}
        \begin{enumerate}
            \item Injury includes moral and/or physical damage caused by the wrongful act of the state -- \convention{\textit{2001 Articles on the Responsibility of States for Internationally Wrongful Acts} Art 31(2)}
            \begin{enumerate}
                \item The overarching aim of reparation is that it ``must as far as possible, wipe out all the consequences of the illegal act and re-establish the situation which would not have existed if the act had not been committed" -- \case{\textit{Chorzow Factory Case} (1928) PCIJ (Ser A) No 17}; affirmed in \case{\textit{Eritea v Ethiopia Final Award} [2009]}
                \item In determining the remedies for stat responsibility,the financial ability of the state to pay compensation and the associated social and economic impacts upon the affected population are a relevant factor in assessing reparation -- \case{\textit{Eritea v Ethiopia Final Award} [2009]}
            \end{enumerate}
        \end{enumerate}
        \item Reparation can be in the form of restitution, compensation and/or satisfaction -- \convention{\textit{2001 Articles on the Responsibility of States for Internationally Wrongful Acts} Art 34}
        \item \textbf{Restitution} is the primary obligation of the state -- \convention{\textit{2001 Articles on the Responsibility of States for Internationally Wrongful Acts} Art 35}
        \begin{enumerate}
            \item The damaging state is under an obligation to re-establish the situation which existed before the wrongful Act, provided that:
            \begin{enumerate}
                \item Restitution is not materially impossible -- \convention{\textit{2001 Articles on the Responsibility of States for Internationally Wrongful Acts} Art 35(a)}
                \item Restitution does not involve a burden out of all proportion to the benefit deriving from restitution instead of compensation -- \convention{\textit{2001 Articles on the Responsibility of States for Internationally Wrongful Acts} Art 35(b)}
            \end{enumerate}
        \end{enumerate}
        \item If restitution cannot be achieved, \textbf{compensation} is the next available option -- \convention{\textit{2001 Articles on the Responsibility of States for Internationally Wrongful Acts} Art 36(1)}
        \begin{enumerate}
            \item This covers any financially assessable damage, including loss of profits -- \convention{\textit{2001 Articles on the Responsibility of States for Internationally Wrongful Acts} Art 36(2)}
        \end{enumerate}
        \item If restitution or compensation cannot be achieved, \textbf{satisfaction} is the next available option (and is the most common form of remedy) -- \convention{\textit{2001 Articles on the Responsibility of States for Internationally Wrongful Acts} Art 37(1)}
        \begin{enumerate}
            \item Satisfaction may include an acknowledgement of the breach, an expression of regret, a formal apology, or another appropriate modality -- \convention{\textit{2001 Articles on the Responsibility of States for Internationally Wrongful Acts} Art 37(2)}
            \item Satisfaction will not be out of proportion to the injury, and may not take a form humiliating to the responsible state -- \convention{\textit{2001 Articles on the Responsibility of States for Internationally Wrongful Acts} Art 37(3)}
        \end{enumerate}
    \end{enumerate}
\end{enumerate}

\section{State Responsibility II (Diplomatic Protection)} 
\begin{enumerate}
    \item If a national was harmed by a foreign state, their home state may elect to take up a claim if they wish to do so -- \convention{\textit{2006 Draft Articles on Diplomatic Protection} Art 1}
    \begin{enumerate}
        \item However, they are not obliged to exercise diplomatic protection, and it is instead at their discretion -- \case{\textit{Barcelona Traction} [1970] ICJ Rep 3}
        \item State may yet be compelled by their own domestic law (e.g., administrative law) to exercise diplomatic protection -- \case{\textit{Hicks v Ruddock} (2007) 156 FCR 574}; \case{\textit{Abbasi v Secretary of State} [2002] EWCA Civ 1598}
        \begin{enumerate}
            \item It was held that there this is the case in Australia at [69]: ``In Anglo-Australian law, the state is under no duty to exercise the right of diplomatic protection in respect of one of its nationals who has been mistreated by a foreign state.''
        \end{enumerate}
        \item States should exercise diplomatic protection when a significant event has occurred -- \convention{\textit{2006 Draft Articles on Diplomatic Protection} Art 19(a)}
        \begin{enumerate}
            \item They might also, if feasible, consider the views of injured persons with regard to resort to diplomatic protection and the reparation to be sought -- \convention{\textit{2006 Draft Articles on Diplomatic Protection} Art 19(b)}
        \end{enumerate}
        \item There is a shift in international law towards states being required to exercise diplomatic protection in certain circumstances:
        \begin{enumerate}
            \item Diplomatic protection might be required to be exercised when the national has been subjected to a serious violation of their human rights -- \article{\textit{ILC Commentary to 2006 Draft Articles on Diplomatic Protection Article 2} at [3]}
            \item There is state practice to suggest that ``although a state has discretion whether to exercise diplomatic protection or not, there is an obligation on that state, subject to judicial review, to do something to assist its nationals, which may include an obligation to give due consideration to the possibility of exercising diplomatic protection'' -- \article{\textit{ILC Commentary to 2006 Draft Articles on Diplomatic Protection}}
            \item States may have a responsibility to ensure that citizens are not placed in a worse position, even though they are not obliged to exercise diplomatic protection -- \case{\textit{Hicks v Ruddock} (2007) 156 FCR 574}
        \end{enumerate}
    \end{enumerate}
    \item Was there the mistreatment of a foreign national by a state?
    \begin{enumerate}
        \item What was the applicable standard of treatment?
        \begin{enumerate}
            \item The \textbf{national treatment} standard holds that there is no mistreatment of a foreign national if they are treated the same as a local citizen of that nation (favoured by developing nations)
            \item The \textbf{international minimum treatment} standard holds that irrespective of the laws of a nation, there is a certain minimum standard of treatment that foreign nationals can expect when in the territory of another state (favoured by developed nations) - \case{\textit{Neer v Mexico} (1926) 4 RIAA 60}; \case{\textit{Loewen Group v US} (2006) NAFTA Arbitration Tribunal}
            \begin{enumerate}
                \item Over time, the scope of diplomatic protection has now expanded beyond violations of the minimum standard for the treatment of aliens to encompass ``internationally guaranteed human rights" -- \case{\textit{Diallo (Preliminary Objections)} [2007] ICJ Rep 582 at [39]}
            \end{enumerate}
        \end{enumerate}
        \item A state must account for the safety and whereabouts of a foreign national - \convention{\textit{Quintanilla} (1926) 4 RIAA 60}
        \item Was there the expropriation of the property of foreign nationals?
        \begin{enumerate}
            \item Direct expropriation means that the foreign national has lost title of the property
            \item Indirect expropriation means the foreign national owns the property, but has lost the ability to use it, enjoy it, dispose of it, etc.
            \item The expropriation of property can be lawful
            \begin{enumerate}
                \item ``It is recognised in international law that measure stake by the state can interfere with property rights such an extent that these rights are rendered so useless that they must be deemed to have been expropriated" -- \case{\textit{Starrett Housing Corporation v Iran} (1987) 4 Iran-US CTR 122}
            \end{enumerate}
            \item However, expropriation must only be conducted where it is for a public purpose or non-discriminatory in nature, and compensation must be paid to the foreign national -- \case{\textit{Certain German Interest in Polish Upper Silesia}}
        \end{enumerate}
        \item Was there the breach of a treaty obligation to a foreign national?
        \begin{enumerate}
            \item An individual has a right to a fair trial -- \convention{\textit{International Covenant on Civil and Political Rights} Art 7}
            \item An individual has a right to not be tortured by a state -- \case{\textit{Convention Against Torture} Art 2}
            \item If there was a biased jury or appeals process against the national in local proceedings, or the local defendant was otherwise favoured, this amounts to mistreatment -- \case{\textit{Loewen Group v US} (2006) NAFTA Arbitration Tribunal}
        \end{enumerate}
        \item A state has an obligation to provide physical protection to foreign nationals who are faced with threats by third parties -- \case{\textit{Home Frontier and Foreign Missionary Society v Great Britain} (1921) VI RIAA 42}; \case{\textit{Asian Agricultural Products Ltd v Sri Lanka} (1991) CSID Case No. ARB/87/3}
        \item A state is responsible if an injury to an alien results from its failure to exercise due diligence to prevent or to punish the injury -- \convention{\textit{Draft Convention on the International Responsibility of States for Injuries to Aliens} Art 10(1)}
        \begin{enumerate}
            \item Injury can be direct -- \case{\textit{Neer v Mexico} (1926) 4 RIAA 60}
            \item Injury can be indirect, such as the poor treatment of a foreign national in prison -- \case{\textit{Roberts v Mexico} (1926) 4 RIAA 60}
            \item Under the international minimum standard, the test for direct injury should be ``in accordance with the ordinary standards of civilisation'', which is a stricter obligation than assessing merely whether the state had acted in a way -- \case{\textit{Roberts v Mexico} (1926) 4 RIAA 60}
        \end{enumerate}
        \item The neglect of a foreign national amounts to mistreatment (such treatment needs to amount to being cruel, harsh and unlawful) -- \case{\textit{Quintanilla} (1926) 4 RIAA 60 at [3]}
        \item It is irrelevant was to whether the state's laws empower the state to measure treatment against the international minimum standard or not -- \case{\textit{Neer v Mexico} (1926) 4 RIAA 60}
        \begin{enumerate}
            \item However, it is sufficient to investigate the relevant case, even if it is concluded within the state that different actions could be undertaken -- \case{\textit{Neer v Mexico} (1926) 4 RIAA 60}
        \end{enumerate}
    \end{enumerate}
    \item Is a claim admissible for a state to take responsibility for it?
    \begin{enumerate}
        \item The injured person must be a national of the claiming state -- \convention{\textit{2001 Articles on State Responsibility for Internationally Wrongful Acts} Art 44(a)}
        \item Before a state takes up a claim on behalf of a national, that national must have exhausted all of the local remedies available to them in the state that they were injured in -- \convention{\textit{2001 Articles on State Responsibility for Internationally Wrongful Acts} Art 44(b)}
    \end{enumerate}
    \item Is the nationality of claims requirement met?
    \begin{enumerate}
        \item States may only exercise a right of diplomatic protection in relation to their nationals (natural persons, or legal persons such as corporations) -- \case{\textit{Re (Al Rawi and Others) v Secretary of State for Foreign and Commonwealth Affairs} [2006] EWCA Civ 1279}
        \item It is a matter for states to grant nationality, whether it is by place of birth (jus soli), by descent (jus sanguinis), by naturalisation, or by succession -- \convention{\textit{2006 Draft Articles on Diplomatic Protection} Art 4}
        \begin{enumerate}
            \item A natural person is required to be a national of the state from the date of their injury to the date of the official presentation of the claim -- \convention{\textit{2006 Draft Articles on Diplomatic Protection} Art 5(1)}
            \begin{enumerate}
                \item However, if a person has lost their previous nationality for a reason unrelated to the claim, they are only required to be a national of the state at the date of the official presentation of the claim -- \convention{\textit{2006 Draft Articles on Diplomatic Protection} Art 5(2)}
            \end{enumerate}
            \item A state may exercise diplomatic protection in respect of a stateless person who, at the date of injury and at the date of presenting their claim, is lawfully and habitually resident in that State -- \convention{\textit{2006 Draft Articles on Diplomatic Protection} Art 8(1)}
            \item A state may exercise diplomatic protection in respect of a refugee who, at the date of injury and at the date of presenting their claim, is lawfully and habitually resident in that State -- \convention{\textit{2006 Draft Articles on Diplomatic Protection} Art 8(2)}
            \begin{enumerate}
                \item However, this does not apply in respect of an injury caused by an internationally wrongful act of the state of nationality of the refugee -- \convention{\textit{2006 Draft Articles on Diplomatic Protection} Art 8(3)}
            \end{enumerate}
            \item If a state is asserting nationality and making a claim against a state which arguably has closer ties to the individual, the claiming state does not need to prove an effective or genuine link but only needs to show a genuine link such that it can take up a claim against a state with which the national has very close ties -- \case{\textit{Nottebohm Case (Liechtenstein v Guatemala)} [1955] ICJ Rep 4}; \article{\textit{ILC Commentary to 2006 Draft Articles on Diplomatic Protection}}
        \end{enumerate}
        \item Does the individual hold multiple nationalities?
        \begin{enumerate}
            \item If the individual is a multiple national, all states can exercise diplomatic protection over them -- \convention{\textit{2006 Draft Articles on Diplomatic Protection} Art 6}
            \item If one of the states injures the individual, then the other states are entitled to exercise diplomatic protection on behalf of the injured individual -- \convention{\textit{2006 Draft Articles on Diplomatic Protection} Art 7}
            \begin{enumerate}
                \item However, an exception applies if the individual's predominant nationality is that of the injuring state at both the date of injury and the date of the official presentation of the claim -- \convention{\textit{2006 Draft Articles on Diplomatic Protection} Art 7}
            \end{enumerate}
            \item It is irrelevant at international law as to whether a state accepts a person as their national -- \case{\textit{Nottebohm Case (Liechtenstein v Guatemala)} [1955] ICJ Rep 4}
            \item To determine what the individual's predominant nationality is, look at factors such as their habitual residence, the state of their family ties, and the state of their economic interests -- \convention{\textit{ILC Commentay to 2006 Draft Articles on Diplomatic Protection Article 7} at [5]}
            \item Under the nationality of claims requirement, nationality was a legal bind which must show a genuine connection between a person and a state -- \case{\textit{Nottebohm Case (Liechtenstein v Guatemala)} [1955] ICJ Rep 4}
            \begin{enumerate}
                \item However, this is no longer used -- \article{\textit{ILC Commentary to 2006 Draft Articles on Diplomatic Protection Article 4} at [5]}
            \end{enumerate}
        \end{enumerate}
    \end{enumerate}
    \item Does this involve the diplomatic protection of a corporation?
    \begin{enumerate}
        \item The state of nationality of a corporation is the state under whose law the corporation was incorporated -- \convention{\textit{2006 Draft Articles on Diplomatic Protection} Art 9}
        \begin{enumerate}
            \item However, this does not apply where: -- \convention{\textit{2006 Draft Articles on Diplomatic Protection} Art 9}
            \begin{enumerate}
                \item The corporation is controlled by the nationals of another state
                \item The corporation has no substantial business activities in the state of incorporation
                \item The dealt of the management and financial control are located in another state
            \end{enumerate}
            In this case, the corporation's nationality shall be regarded as the state in which it predominantly carries out its business activities
            \item Other states will not be able to exercise protection of corporations not incorporated under their laws -- \case{\textit{Diallo Case (Guinea v DRC)} [2007] ICJ Rep 582}
            \item The diplomatic protection of corporations is available to the state of nationality of the corporation, and not the state of nationality of the shareholders of a corporation (this is a strict distinction) -- \convention{\textit{Barcelona Traction Case (Belgium v Spain)} [1970] ICJ Rep 3}
        \end{enumerate}
        \item States will not be entitled to exercise diplomatic protection in respect of shareholders unless: -- \convention{\textit{2006 Draft Articles on Diplomatic Protection} Art 11} (codifying \case{\textit{Barcelona Traction Case (Belgium v Spain)} [1970] ICJ Rep 3 at [65]-[68], [92]})
        \begin{enumerate}
            \item The corporation has ceased to exist for a reason unrelated to the injury -- \convention{\textit{2006 Draft Articles on Diplomatic Protection} Art 11(a)}
            \item The corporation had, at the date of injury, the nationality of the state alleged to be responsible for causing the injury, and incorporation in that state was required as a precondition for doing business there -- \convention{\textit{2006 Draft Articles on Diplomatic Protection} Art 11(b)}
            \item A state may only exercise diplomatic protection in respect of its national as a shareholder in a foreign company if losses were suffered from a breach of the direct rights of the national as a shareholder, the company ceases to exists, or the company's national state lacks the capacity to take action on its behalf -- \case{\textit{Republic of Guinea v Democratic Republic of Congo} [2007]}
        \end{enumerate}
        \item To the extent that an internationally wrongful act causes direct injury to the rights of shareholders (e.g., right to vote, right to dividend, etc.), the state of nationality of the shareholders is entitled to exercise diplomatic protections in respect of its individuals -- \convention{\textit{2006 Draft Articles on Diplomatic Protection} Art 12} (codifying \case{\textit{Barcelona Traction Case (Belgium v Spain)} [1970] ICJ Rep 3 at [47]})
        \begin{enumerate}
            \item It remains a fundamental rule in this area that an internationally wrongful act against shareholders in their capacity as shareholders is not to be regarded as an exception to the general legal regime of diplomatic protection -- \convention{\textit{Diallo Case (Guinea v DRC)} [2007] ICJ Rep 582}
            \item Shareholders have an action only when the act has been aimed at the direct rights of shareholders, not the corporation -- \case{\textit{Barcelona Traction Case (Belgium v Spain)} [1970] ICJ Rep 3}; affirmed in \case{\textit{Republic of Guinea v Democratic Republic of Congo} [2007] ICJ Rep 582}
        \end{enumerate}
    \end{enumerate}
    \item Have all local remedies been exhausted?
    \begin{enumerate}
        \item Before a state takes up a claim on behalf of a national, that national must have exhausted all of the local remedies available to them in the state that they were injured in -- \convention{\textit{2001 Articles on State Responsibility for Internationally Wrongful Acts} Art 44(b)}
        \begin{enumerate}
            \item This rule is based on the principle that ``the foreign state should, first of all, be given the opportunity of redressing the wrong alleged" -- \case{\textit{Finnish Shipowners Arbitration} (1934) 3 RIAA 1479}
        \end{enumerate}
        \item A state may not present a claim on behalf of their national until the national has exhausted all local remedies -- \convention{\textit{2006 Draft Articles on Diplomatic Protections} Art 14(1)}
        \begin{enumerate}
            \item ``Local remedies" means legal remedies which are open to an injured person before the judicial or administrative courts or bodies, whether ordinary or special, of the State alleged to be responsible for causing the injury \convention{\textit{2006 Draft Articles on Diplomatic Protections} Art 14(2)}
            \item The plaintiff must prove that there are no reasonably available effective remedies, and that they are exhausted all judicial and administrative remedies that are of a binding nature - \case{\textit{Norwegian Loans Case} [1957] ICJ Rep 9}
            \begin{enumerate}
                \item However, the balance of proof shifts to the defendant state when the legislation on its face does not have any remedy
                \item No such proof is needed if there is legislation which on its face deprives the claimants of a remedy and the defendant state must show that the existence of a remedy can be reasonably assumed
            \end{enumerate}
            \item All available judicial and administrative remedies, including avenues of appeal, must be exhausted (without the attempting of a settlement) -- \case{\textit{Loewen Group v US} (2006) NAFTA Arbitration Tribunal}
            \begin{enumerate}
                \item However, there is no requirement to appeal a decision if the appeal would fail to change the outcome of the case -- \case{\textit{Finnish Shipowners Arbitration (Finland v UK)} (1934) 3 RIAA 1479}
            \end{enumerate}
            \item ``It is sufficient if the essence of the claim has been brought before the competent tribunals and pursued as far as permitted by local law and procedures, and without success" - \case{\textit{Electronica Sicula SpA (ELSI) Case (US v Italy)} [1989] ICJ Rep 15}
            \item All of the evidence to be raised in international proceedings must have been raised in domestic proceedings - \case{\textit{Ambatielos Arbitration (Greece v UK)} (1956) 12 RIAA 83}
            \begin{enumerate}
                \item Failing to raise a piece of evidence (even if by accident) can be fatal to proceedings in diplomatic protection
                \item ``It is the whole system of legal protection, as provided by municipal law, which must have been put to the test before a State, as the protector of its nationals, can prosecute the claim on the international plane"
            \end{enumerate}
            \item Local remedies shall be exhausted where an international claim, or request for a declaratory judgement related to the claim, is brought preponderantly on the basis of an injury to a national or other person referred to in \convention{Draft Article 8} -- \convention{\textit{2006 Draft Articles on Diplomatic Protections} Art 14(3)}
        \end{enumerate}
        \item Local remedies do not need to be exhausted when: -- \convention{\textit{2006 Draft Articles on Diplomatic Protections} Art 15}
        \begin{enumerate}
            \item There are no reasonably available local remedies to provide effective redress, or the local remedies provide no reasonable possibility of such redress -- \convention{\textit{2006 Draft Articles on Diplomatic Protections} Art 15(a)}
            \item There is undue delay in the remedial process which is attributable to the State alleged to be responsible -- \convention{\textit{2006 Draft Articles on Diplomatic Protections} Art 15(b)}
            \item There was no relevant connection between the injured person and the State alleged to be responsible at the date of injury -- \convention{\textit{2006 Draft Articles on Diplomatic Protections} Art 15(c)}
            \item The injured person is manifestly precluded from pursuing local remedies -- \convention{\textit{2006 Draft Articles on Diplomatic Protections} Art 15(d)}
            \item The State alleged to be responsible has waived the requirement that local remedies be exhausted -- \convention{\textit{2006 Draft Articles on Diplomatic Protections} Art 15(e)}
        \end{enumerate}
    \end{enumerate}
\end{enumerate}

\section{Use of Force}
\begin{enumerate}
    \item Was there a use of force?
    \begin{enumerate}
        \item All member states of the UN must settle their disputes by peaceful means -- \convention{\textit{UN Charter} Art 2(3)}
        \begin{enumerate}
            \item This is customary international law, and so all states must abide by this provision -- \case{\textit{Nicaragua v US} [1986] ICJ Rep 14}
        \end{enumerate}
        \item The use of force is prohibited under international law -- \convention{\textit{UN Charter} Art 2(4)}
        \begin{enumerate}
            \item This has the purpose of ``saving succeeding generations from the scourge of war" - \convention{\textit{UN Charter} Preamble}
            \item This is a customary norm of international law, and so is binding on parties even if they are not part of the UN -- \case{\textit{Nicaragua Case} (1968)}
            \item This is moreover a `cornerstone of the United Nations Charter' -- \case{\textit{Armed Activities (DRC v Uganda)} (2005)}
            \item The prohibition includes the use of force by a state, or by paramilitary or irregular forces acting on behalf of a state -- \convention{\textit{Declaration on Friendly Relations 1970} (Res 2625) Principle 1}
            \begin{enumerate}
                \item This declaration reflects customary international law -- \case{\textit{Nicaragua Case} (1986) at [188], [191]}; \case{\textit{Armed Activities} (2005) at [162], [300]}
                \item The use of force includes the threat or use of force to violate the existing international boundaries of another state or as a means of solving international disputes, including territorial disputes and problems concerning frontiers of states 
                \item The use of force includes organising or encouraging the organisation of irregular forces or armed bands, including mercenaries, for incursion into the territory of another state
                \item The use of force includes organising, instigating, assisting or participating in acts of civil strife or terrorist acts in another state or acquiescing in organised activities within its territory directed towards the commission of such acts that involve the use of force 
            \end{enumerate}
            \item State-sanctioned terrorism is prohibited -- \convention{\textit{Declaration on Friendly Relations 1970} (Res 2625) Principle 1}
            \item Aggression, which is the use of armed force by a State against another state's sovereignty, territorial integrity or political independence, is prohibited -- \convention{\textit{Resolution on Definition of Aggression} 1974 Art 1}
            \begin{enumerate}
                \item A non-exhaustive list of acts that can constitute aggression is provided in \convention{\textit{Resolution on Definition of Aggression 1974} Art 3}, and includes the invasion, attack, occupation, annexation, bombardment, or blockade of another state by armed force; allowing territory to be used for perpetrating aggression against a third state; the sending by or behalf of a state of armed bands, groups, irregulars or mercenaries, which carry out acts of armed force against another state of such gravity listed above or its substantial involvement therein
                \item This Declaration reflects customary international law - \convention{\textit{Nicaragua} (1986) at [195]}; \convention{\textit{Armed Activities} (2005) at [146]}
            \end{enumerate}
        \end{enumerate}
        \item The application of the use of force is not limited to matters concerning the ``territorial integrity or political independence of any state'', as these are just specific examples that elaborate on the general prohibition of the use of force -- \textit{Brownlie, International Law and the Use of Force by States} (1963) at 366
        \begin{enumerate}
            \item However, it can be used as a limitation:
            \begin{enumerate}
                \item E.g., minesweeping operations in the waters of another state (1948, by the UK Navy within the Albanian Territorial Sea in the Corfu Channel)
                \item Forcible invasion by a country to rescue their citizens (e.g., an Israeli intervention in the 1976 Entebbe incident to rescue Israeli citizens on a hijacked airliner)
            \end{enumerate}
        \end{enumerate}
        \item Does one of the recognised exceptions to the use of force apply?
        \begin{enumerate}
            \item Self-defence -- \convention{\textit{UN Charter} Art 51} (this must be permitted by the UN Security Council at the first available opportunity)
            \item Collective security -- \convention{\textit{UN Charter} Chapter VII}
            \item Governmental consent to the use of force in their territory
        \end{enumerate}
        \item Even if the use of force cannot be established, there is at least an unlawful intervention in the sovereignty of another state -- \case{\textit{Nicaragua} (1986) at [224]}
    \end{enumerate}
    \item Was there an armed attack?
    \begin{enumerate}
        \item The unlawful use of force by one state against another constitutes armed force
        \item Was there a grave use of force? -- \case{\textit{Nicaragua Case} (1986) at [191]}; \case{\textit{Oil Platforms} (1984) at [51]}
        \item Were the scale and effects of the operation sufficient such as to constitute an armed attack? -- \case{\textit{Nicaragua} (1986) at [195]}
        \item Examples of armed attacks include:
        \begin{enumerate}
            \item The placing of mines on a single military vessel may amount to an armed attack -- \case{\textit{Oil Platforms} (2003) at [72]}
            \begin{enumerate}
                \item This is not the case for a consumer vessel, as military vessels are integral to the security of a country
            \end{enumerate}
            \item Capturing several towns near a country's border (given the massive scale and effects of this) -- \case{\textit{Armed Activities} (2005) at [110], [147]}
            \item Cybersecurity attacks, as cyber force can amount to non-kinetic uses of force in certain situations, provided they meet six criteria: -- \article{\textit{The Use of Cyber Force in International Law}, Michael Schmitt}
            \begin{enumerate}
                \item Severity
                \item Immediacy
                \item Directness
                \item Invasiveness
                \item Measurability of Effects
                \item State involvement
                \item Presumptive legality
            \end{enumerate}
        \end{enumerate}
        \item Armed attacks can be conducted by armed bands, irregulars, mercenaries, etc. acting on behalf of a state -- \case{\textit{Nicaragua} (1986) at [195]}; \convention{\textit{Resolution on Definition of Aggression 1974} Art 3(g)}
        \item Was there a use of direct armed force?
        \begin{enumerate}
            \item Invasion of a territory, missile attacks, the laying of mines, etc. all constitute a use of direct armed force -- \convention{\textit{Declaration on Friendly Relations 1970} (GA Res 2625)}
            \begin{enumerate}
                \item The laying of landmines constitutes direct armed force and an armed attack -- \case{\textit{Nicaragua} (1986) at [195]}
            \end{enumerate}
        \end{enumerate}
        \item Was there a use of indirect armed force?
        \begin{enumerate}
            \item Did the state have effective control of the irregular band?
            \begin{enumerate}
                \item The conduct of private individuals will be attributable to a state ``if the state directed or controlled the specific operation and conduct complained of was an integral part of the operation" -- \convention{\textit{ILC RSIWA Draft Article 8 Commentary} at [3]}; \case{\textit{Nicaragua} (1986) at [195]}
                \item Irregular bands includes mercenaries, private armies, third parties, etc.
            \end{enumerate}
            \item Sending `armed bands' (irregular forces) into another state's territory -- \case{\textit{Nicaragua} (1986) at [195], [247]}
            \item `Actively extending military, logistic, economic and financial support to irregular forces' (so long as it is directed to the war-fighting capacity of a state) -- \case{\textit{Armed Activities} (2005) at [161]-[165]}
            \item Providing weapons, logistical and/or other support to armed insurgents -- \case{\textit{Nicaragua} (1986) at [195], [205], [247], [251]}
            \end{enumerate}
            \item The `mere supply of funds' does not constitute a use of force -- \case{\textit{Nicaragua} (1986) at [228]}
            \item Armed attacks do not include:
            \begin{enumerate}
                \item Mere frontier incidents -- \case{\textit{Nicaragua} (1986) at [195]}
                \begin{enumerate}
                    \item These are minor disputes wherein the conflict only lasts for a few hours or days
                \end{enumerate}
                \item Armed incursions where the victim did not raise any complaints -- \case{\textit{Nicaragua} (1986) at [231]-[234]}
                \begin{enumerate}
                    \item Since the receiving state(s) did not complain about the incursion, it is presumed that they were not serious enough to amount to an armed attack
                \end{enumerate}
                \item Providing assistance to rebels or insurgents in the form of weapons, logistical or other support -- \case{\textit{Nicaragua} (1986) at [195], [230]}
            \end{enumerate}
            \item Armed attacks generally need to be directed at a state -- \case{\textit{Oil Platforms} (2003) at [64]}
            \begin{enumerate}
                \item Here, even if taken cumulatively, a missile attack, the laying of mines on a ship and then firing on a US ship by Iran did not amount to an armed attack, as there was no evidence that the attacks were aimed specifically at the US
            \end{enumerate}
            \item Localised border encounters between small infantry units, even those involving the loss of life, do not constitute armed attacks -- \case{\textit{Ethiopia v Eritrea} (2006) at [11]}
            \item An armed attack generally does not begin until a state's territory is affected, but it is possible for it to begin when an irreversible course of action (e.g., the launch of missiles) has begun
            \begin{enumerate}
                \item An exception may lie in the `accumulation of events' theory, where an armed attack arises from a number of armed attacks viewed as a whole; this theory has been implicitly accepted by the ICJ - \case{\textit{Nicaragua} (1986) at [231]}; \case{\textit{Oil Platforms} (2003) at [64]}; \case{\textit{Armed Activities} (2005) at [146]}
            \end{enumerate}
    \end{enumerate}
    \item Was there the threat of the use of force?
    \begin{enumerate}
        \item Since the use of armed force is prohibited under \convention{\textit{UN Charter} Art 2(4)}, the threat of the use of force is also prohibited -- \case{\textit{Nuclear Weapons Advisory Opinion} (1996) at [47]}
        \item The threat must be specifically directed at a state to constitute a threat of force
        \begin{enumerate}
            \item Military manoeuvres (e.g., positioning troops, aiming artillery/weapons) near a country's border do not necessarily constitute a threat of the use of force -- \case{\textit{Nicaragua} (1986) at [227]}
            \item An accumulation of forces can possibly constitute a threat, depending on its accompaniments (statement of intent, relevant exercises, etc.) -- \case{\textit{Nicaragua} (1986)}
        \end{enumerate}
    \end{enumerate}
    \item Can a state raise self-defence?
    \begin{enumerate}
        \item The right of self-defence is preserved only until the UN Security Council has taken the necessary measures to maintain international security and peace -- \convention{\textit{UN Charter} Art 51}
        \item Measures taken in self-defence must be reported to the UN Security Council at the earliest possible moment -- \convention{\textit{UN Charter} Art 51}
        \item Is the act of a state in response to a continuing attack or a past attack?
        \begin{enumerate}
            \item A state cannot use self-defence in response to a past attack or an unlawful retaliation
            \item A state cannot use self-defence if the attacks have stopped, as self-defence is only available to respond to an ongoing attack
        \end{enumerate}
        \item Was it anticipatory self-defence?
        \begin{enumerate}
            \item The ICJ has expressly reserved its opinion around anticipatory self-defence, and so there is not much jurisprudence to guide this notion -- \case{\textit{Nicaragua} (1986) at [194]}; \case{\textit{Armed Activities} (2005) at [143]}
            \item Anticipatory self-defence arises when self-defence is carried out on the basis that armed attacks are about to occur (e.g., a missile is coming)
            \item It is unclear as to whether states can practice self-defence against an imminent attack -- \case{\textit{Nicaragua} (1986) at [195]}
        \end{enumerate}
        \item Was it pre-emptive self-defence?
        \begin{enumerate}
            \item This notion has generally been denied in jurisprudence
            \item Pre-emptive self-defence operates on the basis that an armed attack will eventually occur, and so it is wise to reduce or disable the opposition's military strength
            \begin{enumerate}
                \item Arguments in favour of this approach include the changing nature of threats since 1945, that it is impractical to require states to wait until an armed attack has occurred, and that the UN Security Council is often unable to act in a timely manner, or act at all due to the veto of one of its five permanent members
                \item Arguments against this approach include that it is too vague and arbitrary, that it is unable to assess necessity and proportionality, and it may erode the prohibition on the use of force
            \end{enumerate}
        \end{enumerate}
        \item The attacked state must be a member of the UN -- \case{\textit{Ethiopia v Eritrea} (2006) at [11]}
        \item Was there an armed attack?
        \begin{enumerate}
            \item See the above; an armed attack is necessary to raise self-defence
            \item The `scale and effects of the operation' of the attacking state must be grave enough to constitute an armed attack -- \case{\textit{Nicaragua} (1986) at [195]} (e.g., a high degree of casualties, destruction of property, etc.)
            \item It is possible for an operation to be a use of force contrary to \convention{\textit{UN Charter Art} 2(4)}, but not an armed attack and hence not something that can be responded to by way of self-defence -- \case{\textit{Nicaragua} (1986) at [191]}; \convention{\textit{Oil Platforms} (2003) at [51], [61]}
            \item The right to self-defence is only enlivened by an armed attack, which is a most grave use of force assessed on the `scale and effects' of the operation -- \case{\textit{Nicaragua} (1986) at [195]}; \case{\textit{Armed Activities} (2005) at [146]}
        \end{enumerate}
        \item The response of the state must be ``necessary and proportionate to the armed attack and necessary to respond to it" -- \case{\textit{Nicaragua} (1986) at [176]}
        \begin{enumerate}
            \item Necessity means that there are no alternative means of repelling the attack
            \begin{enumerate}
                \item An act done in necessity is an almost in instant response and which has been done instinctively -- \case{\textit{Caroline Case} (1841/1842)}
            \end{enumerate}
            \item Proportionality means that the response must not be excessive in relation to the armed attack, and must only be for the purpose of repelling the attack (not counter-attacking)
            \begin{enumerate}
                \item Self-defence is only lawful if it is necessary and proportionate to the purpose of repelling an armed attack as a matter of customary international law -- \case{\textit{Legality of the Threat or Use of Nuclear Weapons Advisory Opinion} (1996) at [41]}; \textit{Nicaragua} (1986) at [194]
                \item The response must repel the attack, not seek to root out the source of the attack
            \end{enumerate}
        \end{enumerate}
        \item The capture of airports and towns of many hundreds of kilometres from the border is not proportionate or necessary, and cannot constitute self-defence -- \case{\textit{Armed Activities} (2005)}
        \item Attacking towns and cities near the border is not proportionate to a border attack and so cannot constitute self-defence -- \case{\textit{Armed Activities} (2005) at [147]}
        \item A missile attack on a civilian vessel does not meet the scales and effects for an armed attack, and so responding by attacking oil platforms (a government installation) is not self-defence, as it is neither necessary nor proportionate to the attack on the civilian vessel -- \case{\textit{Oil Platforms} (2003) at [64], [72]}
    \end{enumerate}
    \item Did the attacked state consent to the use of force?
    \begin{enumerate}
        \item Did the state positively consent to give the other state a right to intervene, without coercion? -- \case{\textit{Armed Activities} (2005)} (the DRC requested assistance from Uganda and Rwanda to help it quell rebel activity)
        \item Where there is an internal conflict that has reached the threshold of civil war, a state cannot consent to external intervention, and the attack will be unlawful
    \end{enumerate}
    \item Was there a terrorist attack?
    \begin{enumerate}
        \item Were the scales and effects of the terrorist attacks sufficient to constitute an armed attack? -- \case{\textit{Nicaragua} (1986) at [195]}
        \item Can the acts be attributed to a state?
        \begin{enumerate}
            \item It is the majority view that the attack by the terrorist must be attributable to the state -- \case{\textit{Israeli Wall} (2004)}
            \item The minority view is that no state attribution is possible, and self-defence against non-state actors might be permitted, especially when there is no real governmental authority in the state of the terrorist, and as such, no attribution is possible -- \case{\textit{Armed Activities} (2005) (Kooijmans and Simma JJ)}
            \begin{enumerate}
                \item Moreover, where a state is unwilling or unable to prevent its territory from being used to harbour terrorist groups or from being used to launch cross-border attacks against other states, this gives the victim state the right to use force within the territory of that unwilling/unstable state against the non-state group -- \article{\textit{UNSC Resolutions 1368 and 1373}} (passed post-9/11, but state practice on this area is largely undetermined) 
            \end{enumerate}
        \end{enumerate}
    \end{enumerate}
    \item Is there a need for collective security?
    \begin{enumerate}
        \item The UN Security Council has broad powers to take action necessary to respond to threats of the peace, breaches of the peace and acts of aggression
        \item The UN Security Council needs to believe that there is such a threat to or breach of global peace -- \convention{\textit{UN Charter} Art 39}
        \item The UNSC should first attempt measures that do not involve the used of armed force --  \convention{\textit{UN Charter} Art 41}
        \item If the previous measures have failed, then the UNSC can authorise armed force by member states if it believes it is necessary --  \convention{\textit{UN Charter} Art 42}
        \item The UNSC can take action so long as the veto is not exercised by at least one of its five permanent members
    \end{enumerate}
    \item Is there a need for humanitarian intervention?
    \begin{enumerate}
        \item This is an exception to the prohibition on the use of force, and allows another state to protect the nationals of that state from extreme cruelty or persecution
        \item Force should be used only as a very last resort, with there being `no better or more appropriate authority than the United Nations Security Council to authorise military intervention for humanitarian protection purposes' - \article{\textit{2001 International Commission on Intervention and State Responsibility}}
        \item The UK's legal position on the application of humanitarian intervention is that if an action in the UNSC is blocked, the UK would still be permitted to take exceptional measures in order to alleviate the scale of the overwhelming humanitarian catastrophe (e.g., in Syria, by deterring and disrupting further use of chemical weapons by the Syrian regime)
        \begin{enumerate}
            \item A legal basis for justifying humanitarian intervention is:
            \begin{enumerate}
                \item There is convincing evidence of extreme humanitarian distress on a large scale, requiring immediate and urgent relief
                \item It must be objectively clear that there is no practicable alternative to the use of force
                \item The proposed use of force must be necessary and proportionate to the aim of relief of humanitarian needs and strictly limited in time and scope of this aim
            \end{enumerate}
            \item Humanitarian intervention is a narrow and limited test, and there is not widespread state practice and support for this, so its persuasiveness is questionable
        \end{enumerate}
    \end{enumerate}
\end{enumerate}

\section{International Dispute Settlement}\label{scaffold:Topic 13}
\begin{enumerate}
    \item Is there a dispute?
    \begin{enumerate}
        \item A dispute is a disagreement on a point of law or fact; a conflict of legal views or interests between two persons -- \case{\textit{Movrommatis Palestine Concessions} [1924] at [11]}
        \begin{enumerate}
            \item Whether there exists an international dispute is a matter for objective determination
        \end{enumerate}
        \item The mere denial of a dispute does not prove non-existence of a dispute -- \case{\textit{Interpretation of Peace Treaties with Bulgaria, Hungary and Romania} [1950] at [74]}
    \end{enumerate}
    \item Is there a non-judicial means of dispute resolution available?
    \begin{enumerate}
        \item Is negotiation available?
        \begin{enumerate}
            \item Negotiation is the lowest-stakes method of international dispute settlement, and is the most common form attempted by states
            \item Negotiation may sometimes be a procedural precondition for the jurisdiction of an international court (i.e., negotiation must take place before judicial settlement occurs) -- \case{\textit{Ukraine v Russia} [2019] ICJ Rep 558}
            \begin{enumerate}
                \item Discussions at international organisations/fora are sufficient to satisfy this requirement - \case{\textit{South West Africa} [1962] ICJ 319}
            \end{enumerate}
        \end{enumerate}
        \item Is mediation available?
        \begin{enumerate}
            \item Mediation involves a third party (the mediator) who assists the parties in reaching a settlement
            \item The parties remain in control over the process, and set the scope for the mediator's involvement
        \end{enumerate}
        \item Is an inquiry available?
        \begin{enumerate}
            \item An inquiry involves an objective assessment of the evidence and the finding of facts -- \convention{\textit{1899 Hague Convention for the Pacific Settlement of International Disputes} Art 9}; \case{\textit{Dogger Bank Inquiry} (1904)}
            \item Inquiries are to be established by special agreement between the conflicting parties -- \convention{\textit{1899 Hague Convention for the Pacific Settlement of International Disputes} Art 10}
            \item Inquiries are limited to statements of fact, and are to be provided with all resources as required -- \convention{\textit{1899 Hague Convention for the Pacific Settlement of International Disputes} Art 11-14}
        \end{enumerate}
        \item Is conciliation available?
        \begin{enumerate}
            \item Conciliation involves a third party who considers legal and non-legal factors to recommend terms of settlement, and is an option for dispute resolution -- \convention{\textit{1948 Pact of Botogá} Articles XV-XXX}
            \item For parties to the \convention{\textit{General Act for the Pacific Settlement of International Disputes}}, there must be compulsory conciliation if the dispute is not resolved by diplomacy -- \convention{\textit{1928 General Act for the Pacific Settlement of International Disputes} Art 1-15}
            \item If this is a matter involving \convention{\textit{UNCLOS}}, provisions for conciliation exist -- \case{\textit{Australia-Timor Leste Conciliation} (2016)}
        \end{enumerate}
        \item Is arbitration available?
        \begin{enumerate}
            \item Arbitrations involve the application of international law to resolve deadlocks by creating binding awards
            \item Arbitration can either be:
            \begin{enumerate}
                \item Ad-hoc inter-state (i.e., established only when there is conflict between states) -- \case{\textit{South China Sea Arbitration} (2016)}
                \item Institutional inter-state (i.e., it is a standing, go-to method of dispute resolution)
                \item Individual/corporation vs state (this is generally under bilateral investment treaties, where arbitration allows for disputes between corporations and states to be resolved -- \case{\textit{Philip Morris v Australia} (2015)} (under the Hong-Kong Australia BIT))
            \end{enumerate}
        \end{enumerate}
    \end{enumerate}
    \item Have the preconditions to the ICJ's jurisdiction been satisfied?
    \begin{enumerate}
        \item States must use other methods of dispute settlement to attempt to resolve the dispute in a peaceful manner before raising the dispute to a judicial body -- \convention{\textit{UN Charter} Art 2(3)}; \convention{\textit{UN Charter} Art 33(1)}
        \begin{enumerate}
            \item States must have refrained from the threat or use of force -- \convention{\textit{UN Charter} Art 2(4)}
            \item The UNSC will call upon the parties to settle their disputes by such means whenever necessary -- \convention{\textit{UN Charter} Art 33(2)}
        \end{enumerate}
        \item If negotiation has been attempted, is there a bar to jurisdiction?
        \begin{itemize}
            \item The precondition of negotiation for the ICJ's jurisdiction is met only when there has been a failure of negotiations, or when negotiations have become futile or resulted in a deadlock -- \case{\textit{Russia v Ukraine} [2019] ICJ Rep 558}
            \begin{enumerate}
                \item Discussions at international organisations/fora are sufficient to satisfy this requirement -- \case{\textit{South West Africa} [1962] ICJ 319}
            \end{enumerate}
        \end{itemize}
    \end{enumerate}
    \item Does the ICJ have jurisdiction?
    \begin{enumerate}
        \item Only member states of the UN can be in cases heard before the ICJ -- \statute{\textit{ICJ Statute} Art 34(1)}
        \begin{enumerate}
            \item All member states of the UN are automatically parties to the ICJ Statute, and thus can be parties to cases before the ICJ -- \statute{\textit{ICJ Statute} Art 83(1)}
        \end{enumerate}
        \item Judgements of the ICJ are binding only on the parties to the dispute -- \statute{\textit{ICJ Statute} Art 59}
        \begin{enumerate}
            \item However, judgements may influence the development of international law -- \statute{\textit{ICJ Statute} Art 38(1)(d)}
        \end{enumerate}
        \item There is a fundamental difference between a violation of international law and the jurisdiction of the ICJ; just because thee is a breach does not mean that the ICJ has jurisdiction over it -- \case{\textit{Democratic Republic of the Congo v Uganda} [2005] ICJ Rep 168}
        \item Contentious jurisdiction of the ICJ is present either by \textit{compromis}, compromissory clause, or compulsory jurisdiction, demonstrating state consent to be bound to the ICJ's jurisdiction, which is essential for its jurisdiction
        \begin{enumerate}
            \item \textit{Compromis} jurisdiction is an agreement between the parties to submit a dispute to the ICJ for resolution -- \statute{\textit{ICJ Statute} Art 36(1)}
            \begin{enumerate}
                \item The states will identify the issues they wish for the ICJ to decide upon
                \item The states can specify the rules/parameters that the ICJ should apply in deciding the dispute
            \end{enumerate}
            \item A compromissory clause is a clause in a treaty that provides for the ICJ to have jurisdiction over disputes arising from the treaty, with this being very clear from the text of the treaty -- \statute{\textit{ICJ Statute} Art 36(1)}
            \item The ICJ's compulsory jurisdiction arises on the basis of a state having accepted the ICJ's jurisdiction in advance; hence, it is known as jurisdiction under the optional clause (given that there is no requirement to submit to it) -- \statute{\textit{ICJ Statute} Art 36(2)}
            \begin{enumerate}
                \item States may give their consent through a declaration accepting the ICJ's jurisdiction
                \item Declarations can be made with reservations/conditions -- \statute{\textit{ICJ Statute} Art 36(3)}; \case{\textit{Nicaragua v US} [1984] ICJ Rep 392 at [59]}
                \item A party may rely on the reservation of the other party to dispute the ICJ's jurisdiction -- \case{\textit{Whaling in the Antarctic} [2014] ICJ Rep 226}
                \item A declaration of compulsory jurisdiction is only effective if both states have made such a declaration and both states accept the ICJ's jurisdiction over the present dispute
                \item Reservations made by a state are reciprocal, and can be used against them, with the ICJ taking only the narrowest approach to jurisdiction (e.g., if a state accepts the ICJ's jurisdiction over X + Y, and another state accepts it over Y and Z, the ICJ's jurisdiction in a dispute between the parties would apply to Y ony) -- \case{\textit{Nicaragua v United States} [1984] ICJ Rep 392}; \case{\textit{Norwegian Loans Case} [1957] ICJ Rep 9}; \case{\textit{Interhandel Case} [1959] ICJ Rep 6}
            \end{enumerate}
            \item Jurisdiction of the ICJ may be established under \textit{forum prorogatum}, which is the agreement of the parties to submit a dispute to the ICJ after the dispute has arisen, and thus is a form of \textit{compromis} jurisdiction -- \statute{\textit{ICJ Statute} Art 38(5)}
            \item The PCIJ's jurisdiction can be transferred to the ICJ -- \statute{\textit{ICJ Statute} Art 36(5)}
        \end{enumerate}
        \item Even if a state does not consent to the ICJ's jurisdiction, the parties still need to resolve their dispute by peaceful means -- \statute{\textit{ICJ Statute} Art 33}; \case{\textit{Federal Republic of Yugoslavia v NATO States} [1999] ICJ Rep 124}
        \item The states must agree that there is a dispute for the ICJ to have jurisdiction -- \case{\textit{Marshall Islands Case} [2016] ICJ Rep 883} (note Crawford J's dissent, raise that this theory may be flawed, especially since this case was tied and decided on the President's casting vote)
        \begin{enumerate}
            \item The test for whether a dispute exists is a matter for objective determination by the court
        \end{enumerate}
    \end{enumerate}
    \item Is the dispute admissible to the ICJ?
    \begin{enumerate}
        \item A dispute is not admissible if it is moot or hypothetical (e.g., the nuclear tests case was not admissible as nuclear testing had ceased by them) -- \case{\textit{Legality of the Threat or Use of Nuclear Weapons} [1996] ICJ Rep 254}
        \item A dispute is not admissible if it is not justiciable; it must contain some legal question and cannot be purely political -- \case{\textit{Legal Consequences of the Construction of a Wall in Occupied Palestinian Territory} [2004] ICJ 136}
        \item Does the applicant have standing?
        \begin{enumerate}
            \item Was there an injury?
            \item Is the nationality of claims rule satisfied? (Topic \ref{sec:Topic 11})
            \item Is the rule on the exhaustion of local remedies satisfied? (Topic \ref{sec:Topic 11})
            \item Is the dispute moot or hypothetical?
        \end{enumerate}
        \item Is there an indispensable third party that is not present? -- \case{\textit{Monetary Gold} [1945] ICJ Rep 19}; \case{\textit{East Timor} [1955] ICJ Rep 6}
        \begin{enumerate}
            \item An indispensable third party is one whose rights would be affected by the judgment of the ICJ
            \item The third party might be a vital subject matter to the case -- \case{\textit{Monetary Gold} [1945] ICJ Rep 19}
            \item A vital third party might have its rights and obligations significantly impacted by the subject matter and judgement of the dispute -- \case{\textit{East Timor} [1955] ICJ Rep 6}
            \item However, it is insufficient that there are legal implications on a third party; a third party's legal interests must be impacted, or they must otherwise be vital to the court's decision -- \case{\textit{Nauru v Australia} [1992]}
        \end{enumerate}
    \end{enumerate}
    \item Can another party intervene into the dispute?
    \begin{enumerate}
        \item A state can intervene into a case regarding their interest if: -- \case{\textit{El Salvador v Honduras (Nicaragua Intervening)} [1992] ICJ Rep 351 at [58]}
        \begin{enumerate}
            \item It has an interest of a legal nature that \textit{may} be affected by the decision in the case
            \item It can satisfy the burden of proof of proving their legal interest
            \item It can provide a clear identification that their legal interest will be affected by the decision in the case (an apprehension is not sufficient)
        \end{enumerate}
        \item A state that is party to a case cannot determine whether a third state can intervene into the dispute -- \case{\textit{El Salvador v Honduras (Nicaragua Intervening)} [1992] ICJ Rep 351 at [96]}
        \begin{enumerate}
            \item Only the ICJ can determine whether a third party can intervene into the dispute
            \item Power has already been given to the court by consent - if the existing parties have given the ICJ consent to hear the matter, the court then has the competence to permit an intervention even if it is opposed by one or both of the parties to the matter
        \end{enumerate}
    \end{enumerate}
    \item Can the ICJ issue provisional measures?
    \begin{enumerate}
        \item The competence of the ICJ to issue provisional measures is distinct from jurisdiction, and is not derived from the consent of the parties to the dispute
        \item The ICJ can issue provisional measures if there is the imminent risk of irreversible harm before a dispute can be heard -- \statute{\textit{ICJ Statute Article} 41(1)}; \case{\textit{Application of the Convention on the Prevention and Punishment of Crime of Genocide in the Gaza Strip} [2024] ICJ Rep 1}
        \item Provisional measures are legally binding, and give rise to an independent legal obligation -- \case{\textit{La Grand} [2001] ICJ Rep 466}
        \begin{enumerate}
            \item Failure to comply with provisional measures constitutes grounds for a claim by an affected state (i.e., it is a wrongful state act) -- \case{\textit{La Grand} [2001] ICJ Rep 466}
        \end{enumerate}
        \item It is currently unsettled whether the ICJ is empowered to judicially review decisions of the UNSC -- \case{\textit{Libya v US; Libya v UK} [1992] 44}
        \item Requirements for the invocation of provisional measures are:
        \begin{enumerate}
            \item \Gls{prima facie} jurisdiction, which indicates that the condition of jurisdiction needs not be fully satisfied, so long as it is apparent that the ICJ has jurisdiction over the dispute -- \case{\textit{South Africa v Israel} [2024] ICJ Rep 1}
            \item There must be a plausible claim of rights that needs to be protected -- \case{\textit{South Africa v Israel} [2024] ICJ Rep 1}
            \item The claim must be urgent -- \case{\textit{Georgia v Russia} [2011] at [143]}
            \item Failure to grant the provisional measure would cause irreparable harm to the rights of the applicant if provisional measures are not granted -- \case{\textit{South Africa v Israel} [2024] ICJ Rep 1}
        \end{enumerate}
    \end{enumerate}
    \item Is the ICJ operating in its advisory capacity?
    \begin{enumerate}
        \item The ICJ can provide an advisory opinion on any question at the request of authorised UN bodies
        \item Advisory opinions are not binding
        \item The court must have compelling reasons for refusing to give an advisory opinion, not merely because the legal question is difficult or unsettled -- \case{\textit{Kosovo Advisory Opinion} [2010] ICJ Rep 403 at [39]-[40]}; \case{\textit{Legal Consequences of the Construction of a Wall in the Occupied Palestinian Territory Advisory Opinion} [2004] ICJ Rep 136}
        \begin{enumerate}
            \item The fact that a legal question has political aspects is not a compelling reason to decline a request for an advisory opinion -- \case{\textit{Legal Consequences of the Construction of a Wall in the occupied Palestinian Territory Advisory Opinion} [2004] ICJ Rep 136}
            \item Whether a separate UN organ is a more appropriate body to request the advisory opinion is not a compelling reason to decline a request for an advisory opinion -- \case{\textit{Kosovo Advisory Opinion} [2010] ICJ Rep 403 at [39]-[40]}
        \end{enumerate}
        \item Subjects of advisory opinions include:
        \begin{enumerate}
            \item Questions of UN law -- \case{\textit{Certain Expenses of the United Nations} [1962] ICJ Rep 151}
            \item Questions around decolonisation -- \case{\textit{Chagos Islands} [2019] ICJ Rep 95}
            \item Questions around unlawful occupation -- \case{\textit{Legal Consequences of the Construction of a Wall in the occupied Palestinian Territory Advisory Opinion} [2004] ICJ Rep 136}
            \item Questions around climate change -- \case{\textit{Obligations of States in Respect of Climate Change}} [pending]
        \end{enumerate}
    \end{enumerate}
\end{enumerate}