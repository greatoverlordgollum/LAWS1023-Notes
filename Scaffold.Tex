\begin{tcolorbox}[title = How to Answer a Problem Question]
    \flushleft
    \begin{enumerate}
        \item Read the question
        \begin{enumerate}
            \item Read the question in one go (no notes, no highlighter, just a straight reading)
            \item Read the question again, making notes of the potential issues in the margins and why specific words/provisions are being used
            \item Read the question a third time to ensure nothing has been missed, and whether any issues interact with each other
        \end{enumerate}
        \item Identify how the question will be answered (generally in sequential order)
        \item Make sure to cover IRAC:
        \begin{enumerate}
            \item Issue
            \item Rule
            \item Analysis
            \item Conclusion
        \end{enumerate}
        \item Unless the question is direct, use vague terminology and argue both sides (e.g., `it appears', `it might be the case that', etc.)
        \item It is good to refute arguments where possible and then point to a secondary argument, as this shows depth
        \item If there are facts that distinguish this case from precedent, make sure to engage in a brief but nuanced analysis
        \item Use subheadings to distinguish issues!
    \end{enumerate}
    Look at obligations both under a treaty and under general international law!
\end{tcolorbox}

\section{Development, Nature and Scope of Public International Law}

\section{Sources of Public International Law}
\begin{enumerate}
    \item Is the document a source of public international law?
    \begin{enumerate}
        \item The ICJ is the principal judicial organ of the UN
        \item Under Art 38(1) of the \convention{\textit{Statute of the International Court of Justice}} (Page \pageref{ICJ Statute Art 38}), the sources of public international law are:
        \begin{enumerate}
            \item Treaties
            \item Custom
            \item General principles of law
            \item Judicial decisions and the teachings of publicists
        \end{enumerate}
        \item Art 38(1) is `generally regarded as a complete statement of the sources of international law' -- \case{\textit{Ure v Commonwealth} (2016) 329 ALR 452 at [15] (Perram, Robertson and Moshinsky JJ)} (Page \pageref{case:Ure v Commonwealth})
        \item Did the state consent to be bound to the jurisdiction of the ICJ/to the terms of the source?
    \end{enumerate}
    \item Was the source a treaty?
    \begin{enumerate}
        \item Under Art 38(1)(a) of the \statute{\textit{ICJ Statute}}, treaties are a source of international law
        \item Was the document a bilateral and/or multilateral convention between two or more states?
        \item See Topic 3 scaffolds (\ref{scaffold:Topic 3} on Page \pageref{scaffold:Topic 3}) for a detailed analysis
        \item Note that if there are a number of instances of states contravening a treaty, it is not necessary for the states to provide consistently correct conduct and that some variations in practice are acceptable, and they do not form a new rule -- \case{\textit{Military and Paramilitary Activities in and against Nicaragua} [1986] ICJ Rep 14 [186]}
    \end{enumerate}
    \item Was there international custom involved?
    \begin{enumerate}
        \item Under Art 38(1)(b) of the \statute{\textit{ICJ Statute}}, custom is a source of international law, and requires two elements: -- \case{\textit{North Sea Continental Shelf Cases (Germany v Denmark; Germany v Netherlands)} [1969] ICJ Rep 3} (Page \pageref{case:North Sea Continental Shelf})
        \begin{enumerate}
            \item State practice - objective evidence that the custom is practiced widely
            \item \textit{Opinio juris} - the belief that the practice is legally required
        \end{enumerate}
        \item Unless they are persistent objectors, all states are bound to customary international law
        \item Certain customary norms are \textit{jus cogens}, which are peremptory norms from which no derogation is permitted
        \item Was there state practice?
        \begin{enumerate}
            \item State practice can be evidenced by materials that demonstrate the activities and views of states and state officials
            \item State practice can generate custom if the following requirements are met: -- \case{\textit{North Sea Continental Shelf Cases} (1969) ICJ Rep 3}
            \begin{enumerate}
                \item The practice was consistent over time (but not necessarily entirely uniform) -- \case{\textit{Military and Paramilitary Activities in and against Nicaragua} [1986] ICJ Rep 14} (Page \pageref{case:Military and paramilitary activities in Nicaragua})
                \item The practice was widespread
                \item The practice was representative of multiple states (especially those who are most likely affected by it)
                \item The practice was developed over a lengthy period of time (this is not a steadfast requirement; customary norms may still emerge rapidly if there is an overwhelming practice of it)
            \end{enumerate}
            \item State practice can be shown through government legal opinions, treaty provisions and conduct in connection with resolutions that the country makes -- \convention{UN \textit{Draft conclusions on identification of customary international law} Conclusion 10(2)}
            \item Under Conclusion 10(3), failure to react over time to a practice may serve as evidence of acceptance provided that the State was in a position to react and the circumstances called for some reaction
        \end{enumerate}
        \item Was there \textit{opinio juris}?
        \begin{enumerate}
            \item \textit{Opinio juris} refers to the belief that the practice is legally required
            \item \textit{Opinio juris} is hard to show, and generally can be shown through statements made by countries (\case{\textit{North Sea Continental Shelf Cases (Germany v Denmark; Germany v Netherlands)} [1969] ICJ Rep 3 at [77]}), although it can also be shown through an omission of a state, which evinces a belief that the said State is obligated by law to refrain from acting in a particular way (\case{\textit{The Lotus Case (France v Turkey)} (1927) PCIJ Series A No 10 at page 28})
            \item If there is extensive state practice, then \textit{opinio juris} tends to be less important, and vice-versa
        \end{enumerate}
        \item It is possible for treaty norms to become custom, and for treaty provisions to become customary international law -- \case{\textit{North Sea Continental Shelf Cases} (1969) ICJ Rep 3 at [72]}
        \begin{enumerate}
            \item However, the custom exists independently of the treaty -- \case{\textit{Military and Paramilitary Activities in and against Nicaragua} [1986] ICJ Rep 14 at [25]-[30] and [40]}
        \end{enumerate}
    \end{enumerate}
    \item Was there regional custom involved?
    \begin{enumerate}
        \item The ICJ has recognised that it is possible for regional custom to exist, but invoking it requires a higher standard than general international custom -- \case{\textit{Asylum Case (Colombia v Peru)} [1950] ICJ Rep 226}
        \begin{itemize}
            \item Regional custom must have a higher degree of stability and continuity to apply as international law in that area -- \case{\textit{Asylum Case (Colombia v Peru)} [1950] ICJ Rep 226}
            \item Such an example was made out in the English Court of Appeal in \case{\textit{R (app. Al-Saadoon v Sec. of Defence)} [2010] 1 All ER 271}, where rules of regional custom were found to exist, but had not met the high threshold to be invoked (Page \pageref{case:R (Al-Saadoon v Sec. of Defence)})
        \end{itemize}
    \end{enumerate}
    \item If there was custom involved, was the party a persistent objector?
    \begin{enumerate}
        \item The doctrine of a persistent objector is fairly narrow, and enunciates that states which consistently object to the emergence of a rule from its earliest point of gestation will not be bound by it -- \case{\textit{Anglo Norwegian Fisheries Case (UK v Norway)} [1951] ICJ Rep 116}
        \item A state cannot be a persistent objector to a \textit{jus cogens} principle -- \article{\textit{International Law Commission 2019 Report} Chapter V Conclusion 14} (Page \pageref{report:2019 ILC Conc. 14})
    \end{enumerate}
    \item Was the source a general principle of international law?
    \begin{enumerate}
        \item Under Art 38(1)(c) of the \statute{\textit{ICJ Statute}}, general principles of law recognised by civilised nations form a source of PIL, with the objective of avoiding the \textit{non liquet} (the situation where `it is not clear' by enabling the ICJ to look at different legal systems for inspiration)
        \item General principles of international law and municipal law are included in this provision
        \item General principles of law may be implicitly adopted in judicial decisions to enable a conclusion to be made -- \case{\textit{Bay of Bengal (Bangladesh/Myanmar)} [2012] ILTOS 12} (Page \pageref{case:Bangladesh v Myanmar})
        \item For example, various domestic legal systems were examined in relation to the issue of estoppel to aid the Tribunal in its decision -- \case{\textit{Chagos Marine Protected Area Arbitration (Mauritius v United Kingdom)} (2015) XXXI RIAA 359}
    \end{enumerate}
    \item Was there a judicial decision and/or the teachings of a publicist?
    \begin{enumerate}
        \item Whilst Art 38(1)(d) of the \statute{\textit{ICJ Statute}} enables judicial decisions and the work of publicists to be considered as sources of PIL, they are subsidiary means for the determinations of the rules of law, and are treated as having lesser significance than other sources
        \item Decisions taken by the ICJ do not constitute binding precedent in future decisions, but remain merely persuasive -- \statute{\textit{Statute of the International Court of Justice} Art 59}
        \begin{enumerate}
            \item These sources are ``resorted to by judicial tribunals not for the
            speculations of their authors concerning what the law ought to be, but for trustworthy
            evidence of what the law really is'' -- \case{\textit{The Paquete Habana} 175 US 677 (1900)}
        \end{enumerate}
        \item Was this a UN General Assembly Resolution?
        \begin{enumerate}
            \item The UN General Assembly (UNGA) is the plenary body of the UN, and as all UN members have a seat, it has become a great forum for state practice and \textit{opinio juris}
            \item Decisions of the UNGA are not binding, except in the key areas of (without these areas, the UN could not function):
            \begin{enumerate}
                \item Admission of member states
                \item Suspension of member states
                \item Matters related to the UN budget
            \end{enumerate}
            \item Resolutions of the UNGA can provide evidence for state practice
            \item UNGA resolutions can influence international law in three key ways
            \begin{enumerate}
                \item Interpreting the \textit{Charter of the United Nations}
                \item Affirming recognised customary norms (through passing a resolution)
                \item Influencing the creation of new customary norms
            \end{enumerate}
            \item UNGA resolutions, whilst normally not binding, may have normative value, and can provide ``evidence important for the establishing the existence of a rule or the emergence of a \textit{opinion juris}'' -- \case{\textit{Legality of the Threat or Use of Nuclear Weapons} [1996] ICJ Rep 254 at [70] - [73]} (Page \pageref{case:Legality of Nuclear Weapons [1996] ICJ Rep 254}); such evidence can include:
            \begin{enumerate}
                \item The voting records of the UNGA
                \item Transcripts of what was said on the floor of the UNGA
                \item Margins of the votes undertaken in the UNGA
            \end{enumerate}
        \end{enumerate}
        \item Was this a UN Security Council Resolution?
        \begin{enumerate}
            \item The UN Security Council (UNSC) has limited law-making capacity, but can adopt certain binding resolutions
            \item UNSC resolutions are binding only on the members of the UN -- \statute{\textit{Charter of the United Nations} Art 25} (Page \pageref{UN Charter Art 25})
        \end{enumerate}
    \end{enumerate}
    \item Was there a measure of soft law involved?
    \begin{itemize}
        \item Soft law refers to rules that are binding but vague, and/or `rules' that are clear but are not binding
        \item They can articulate standards/norms that will, over time, become binding, and can also be used to interpret other sources of international law
    \end{itemize}
\end{enumerate}

\section{The Law of Treaties}\label{scaffold:Topic 3}
\begin{enumerate}
    \item Was there a treaty involved?
    \begin{enumerate}
        \item ``Treaty'' means an international agreement concluded between States in written form and governed by international law, whether embodies in a single instrument or in two or more related instruments and whatever its particular designation -- \convention{\textit{VCLT} Art 2(1)(a)} (Page \pageref{VCLT Art 2})
        \item It has been accepted that a treaty may be written across multiple documents -- \case{\textit{Maritime Delimitation and Territorial Questions (Qatar v Bahrain)} (1994) ICJ Rep 112 at [23]} (Page \pageref{case:Qatar v Bahrain})
        \item `Agreed Minutes' (or another document evincing agreement between two or more states) can constitute a treaty --\case{\textit{Maritime Delimitation and Territorial Questions (Qatar v Bahrain)} (1994) ICJ Rep 112} (Page \pageref{case:Qatar v Bahrain}); \case{\textit{Bay of Bengal (Bangladesh/Myanmar)} [2012] ILTOS 12} (Page \pageref{case:Bangladesh v Myanmar})
        \item A unilateral declaration can be considered to have binding effect -- \case{\textit{Nuclear Test Cases (Australia v France)} (1974) ICJ Rep 253 at [43]}
        \begin{enumerate}
            \item ``An undertaking ... if given publicly with an intent to be bound, even though not made within the context of international negotiations, is binding''
            \item This case outlines four key characteristics for a unilateral declaration to be binding:
            \begin{enumerate}
                \item The undertaking is made publicly with an intention to be bound
                \item It must be clear and specific
                \item It can be oral or written
                \item It must be made by someone who is authorised by the state to make a binding conclusion
            \end{enumerate}
        \end{enumerate}
    \end{enumerate}
    \item Does the \convention{\textit{1969 Vienna Convention on the Law of Treaties} (\textit{VCLT})} apply?
    \begin{enumerate}
        \item Was the treaty between two or more states? -- \convention{\textit{VCLT Art 3}} (Page \pageref{VCLT Art 3})
        \begin{enumerate}
            \item Under Article 3, the VCLT does not influence agreements between states and other subjects or between other subjects of international law
        \end{enumerate}
        \item Was the treaty in writing? -- \convention{\textit{VCLT Art 3}} (Page \pageref{VCLT Art 3})
        \begin{enumerate}
            \item The VCLT applies to written treaties only, but as many of its provisions are now customary law, those provisions may still apply to non-written treaties (see Table \ref{tab:VCLT Articles that can apply as customary international law} on Page \pageref{tab:VCLT Articles that can apply as customary international law})
        \end{enumerate}
        \item Had the treaty commenced after 1980 (when the \convention{\textit{VCLT}} entered into force)?
        \begin{itemize}
            \item As the \convention{\textit{VCLT}} entered into force in 1980, it only applies to treaties concluded after 1980, but many of its provisions can apply to treaties concluded before 1980 as provisions of general international law (see Table \ref{tab:VCLT Articles that can apply as customary international law} on Page \pageref{tab:VCLT Articles that can apply as customary international law})
        \end{itemize}
    \end{enumerate}
    \item Was the treaty registered with the United Nations?
    \begin{itemize}
        \item A treaty must be registered with the UN in order to be used as a binding instrument in proceedings before the UN -- \convention{\textit{Charter of the United Nations} Art 102}; \convention{\textit{VCLT} Art 80} (Page \pageref{convention:UN Charter Art 102})
        \item This is not a requirement for a treaty to be binding in general, but is a requirement for the treaty to be recognised before the UN
        \item Registration only needs to be completed by one party
        \item ``Non-registration or late registration, on the other hand, does not have any consequence for the actual validity of the agreement, which remains no less binding upon the parties.'' -- \case{\textit{Maritime Delimitation and Territorial Questions (Qatar v Bahrain)} (1994) ICJ Rep 112 at [29]} (Page \pageref{case:Qatar v Bahrain})
    \end{itemize}
    \item Was the treaty signed by an appropriate authority/representative?
    \begin{enumerate}
        \item Was the party entering into the treaty a state, an international organisation or an international entity with capacity to enter into the treaty?
        \begin{enumerate}
            \item Every state possesses capacity to conclude treaties -- \convention{\textit{VCLT} Art 6} (Page \pageref{VCLT Art 6})
        \end{enumerate}
        \item Has the individual representing the party produced full powers evincing their authority to enter into the treaty? -- \convention{\textit{VCLT} Art 7(1)(a)} (Page \pageref{VCLT Art 2})
        \begin{enumerate}
            \item ``Full powers'' refers to a document emanating from the competent authority of a State designating a person or persons to represent the State for negotiating, adopting or authenticating the text of a treaty, for expressing the consent of the State to be bound by a treaty, or for accomplishing any other act with respect to a treaty -- \convention{\textit{VCLT} Art 2(1)(d)} (Page \pageref{VCLT Art 2})
            \item Heads of State, Heads of Government and Ministers of Foreign Affairs are taken to have the capacity to conclude treaties without producing full powers -- \convention{\textit{VCLT} Art 7(2)(a)} (Page \pageref{VCLT Art 2})
            \item Heads of diplomatic missions will likewise not need to produce full powers if they are accredited to adopt treaties in that area -- \convention{\textit{VCLT} Art 7(2)(b)} (Page \pageref{VCLT Art 2})
            \item A representative of a state will not need to produce full powers if they have been sent to a conference/organisation with the purpose of adopting the text of a treaty at that conference/organisation -- \convention{\textit{VCLT} Art 7(2)(c)} (Page \pageref{VCLT Art 2})
        \end{enumerate}
        \item If the individual has not produced full powers, is it evident from the practice of the States concerned or from other circumstances that the person is representing the State? -- \convention{\textit{VCLT} Art 7(1)(b)} (Page \pageref{VCLT Art 2})
    \end{enumerate}
    \item Did the state enter into the treaty?
    \begin{enumerate}
        \item Signing is a two step process, entailing signature, and either ratification or accession
        \item Upon \textbf{signing} a treaty, the state expresses a willingness to continue the treaty-making process, and agrees with the treaty in principle
        \begin{enumerate}
            \item However, the state is not bound by the treaty at this point
        \end{enumerate}
        \item If the treaty is a new one, was it \textbf{ratified} by the party?
        \begin{enumerate}
            \item Upon ratification, the party indicates that it has consented to be bound by the treaty once it enters into force
        \end{enumerate}
        \item If the treaty is an existing one, was it \textbf{accessioned} by the party?
        \begin{enumerate}
            \item This only applies if a state is becoming party to a treat that is already negotiated and signed by other states
            \item This has the same legal effect as ratification
        \end{enumerate}
    \end{enumerate}
    \item Was the treaty in force at the time of contention?
    \begin{enumerate}
        \item A treaty enters into force in accordance with the relevant provisions in the treaty -- \convention{\textit{VCLT} Art 24(1)} (Page \pageref{VCLT Art 24})
        \item If the treaty is silent on this point, it will enter into force when all parties have consented to being bound by it -- \convention{\textit{VCLT} Art 24(2)} (Page \pageref{VCLT Art 24})
        \item If a party signs a treaty after its formation, it will be binding upon that state on the day that consent to being bound is established -- \convention{\textit{VCLT} Art 24(3)} (Page \pageref{VCLT Art 24})
    \end{enumerate}
    \item Does the treaty apply to the present scenario?
    \begin{enumerate}
        \item The principle of \textit{pacta sunt servanda} requires that ``every treaty in force is binding upon the parties to it, and must be performed by them in good faith'' -- \convention{\textit{VCLT} Art 26} (Page \pageref{VCLT Art 26})
        \item A party may not invoke the provisions of its internal law as justification for failing to perform its obligations -- \convention{\textit{VCLT} Art 27} (Page \pageref{VCLT Art 27})
        \begin{enumerate}
            \item However, they may do so if the other party was aware of that law, and the law was not contrary to the treaty -- \convention{\textit{VCLT} Art 46} (Page \pageref{VCLT Art 46})
        \end{enumerate}
        \item If the treaty has been signed but not ratified/approved/accepted, a state is obliged to not undermine the spirit of the treaty, and moreover is required to refrain from acts that would defeat the object and purpose of the treaty -- \convention{\textit{VCLT} Art 18(a)} (Page \pageref{VCLT Art 18})
        \begin{itemize}
            \item The same principle also applies where a state has expressed its consent to be bound by the treaty, pending the entry into force of the treaty -- \convention{\textit{VCLT} Art 18(b)} (Page \pageref{VCLT Art 18})
        \end{itemize}
        \item Treaties do not impose obligations or create rights for third states in the absence of their consent (\textit{pacta tertiss nex nocent nec prosunt}) -- \convention{\textit{VCLT} Art 34} (Page \pageref{VCLT Art 34})
    \end{enumerate}
    \item Was there any reservation to the treaty?
    \begin{enumerate}
        \item Was there a reservation or an interpretive declaration?
        \begin{enumerate}
            \item A reservation is a unilateral statement, however phrased or named, made by a State, when signing, ratifying, accepting, approving or acceding to a treaty, whereby it purports to exclude or to modify the legal effect of certain provisions of the treaty in their application to that State -- \convention{\textit{VCLT} Art 2(1)(d)} (Page \pageref{VCLT Art 2})
            \item Interpretive declarations are statements made by a state to clarify its understanding of a treaty; it does not affect the legal effect of a treaty
        \end{enumerate}
        \item Was the reservation permissible?
        \begin{enumerate}
            \item By default, a reservation is permissible, unless: -- \convention{\textit{VCLT} Art 19} (Page \pageref{VCLT Art 19})
            \begin{enumerate}
                \item The reservation is prohibited by the treaty -- \convention{\textit{VCLT} Art 19(a)} (Page \pageref{VCLT Art 19})
                \item The treaty provides that only specified reservations may be made and the reservation in question is not in that list -- \convention{\textit{VCLT} Art 19(b)} (Page \pageref{VCLT Art 19})
                \item The reservation is otherwise incompatible with the object and purpose of the treaty -- \convention{\textit{VCLT} Art 19(c)} (Page \pageref{VCLT Art 19})
            \end{enumerate}
            \item Incompatibility hinges on whether it “affects an essential element of the treaty that is necessary to its general tenor, in such a way that the reservation impairs the \textit{raison d'être} [the most important reason] of the treaty” -- \article{\textit{ILC Guide to Practice on Reservations} Art 3.1.5}
            \item If a reservation is impermissible: \\ \textit{Briefly mention both points in discussion, and then note the first point is the predominant view.}
            \begin{enumerate}
                \item Traditionally, this \glspl{vitiate} the consent of the state to the treaty as a whole and results in the state not being a party to the treaty (this is the predominant view) -- \case{\textit{Reservations to Genocide Convention} [1951] ICJ Rep 15 at page 18}
                \item More recently, the offending reservation will be held void, with the state being bound without the protection of the reservation (i.e., it is cut out), unless consent is conditional on reservation, in which case they are not bound to the treaty at all
            \end{enumerate}
        \end{enumerate}
        \item Was the reservation accepted or objected to? -- \convention{\textit{VCLT} Art 20} (Page \pageref{VCLT Art 20})
        \begin{enumerate}
            \item If a treaty expressly allows for reservations, then no acceptance of a reservation is required by other parties -- \convention{\textit{VCLT} Art 20(1)} (Page \pageref{VCLT Art 20})
            \item If a treaty has a small number of parties and the application of the treaty in its entirety is an essential condition of signing, acceptance by all parties is required -- \convention{\textit{VCLT} Art 20(2)} (Page \pageref{VCLT Art 20})
            \item If a treaty is a constituent instrument of an international organisation, and unless it otherwise provides, a reservation requires the acceptance of the competent organ of that organisation -- \convention{\textit{VCLT} Art 20(3)} (Page \pageref{VCLT Art 20})
            \item Acceptance by the other contracting state(s) of the reservation results in the reserving state being bound by the treaty (with the reservation incorporated) -- \convention{\textit{VCLT} Art 20(4)(a)} (Page \pageref{VCLT Art 20})
            \item Objection to a reservation does not prevent entry into force of a treaty between the objecting state and the reserving state, unless the objecting state says otherwise -- \convention{\textit{VCLT} Art 20(4)(b)} (Page \pageref{VCLT Art 20})
            \item An act indicating consent to being bound by the treaty that contains a reservation is effective as soon as at least one other state has accepted the reservation -- \convention{\textit{VCLT} Art 20(4)(c)} (Page \pageref{VCLT Art 20})
            \item Unless the treaty provides otherwise, a reservation is considered to have been accepted if no objections are raised within 12 months of notification of the reservation, or by the date on which it consented to be bound to the treaty, whichever is the later -- \convention{\textit{VCLT} Art 20(5)} (Page \pageref{VCLT Art 20})
        \end{enumerate}
        \item What is the legal effect of the reservation?
        \begin{enumerate}
            \item If State A accepts State R's reservation, then the treaty is modified only between States A and R, to the the extent of the reservation -- \convention{\textit{VCLT} Art 21(1) and (2)} (Page \pageref{VCLT Art 21}); \case{\textit{Republic of India v CCDM Holdings, LLC} [2025] FCAFC 2 at [63]} (Page \pageref{case:India v CCDM})
            \begin{enumerate}
                \item However, other states will not be bound by this reservation; it acts as a side agreement between State A and State R -- \convention{\textit{VCLT} Art 21(2)} (Page \pageref{VCLT Art 21})
            \end{enumerate}
            \item If State B objects to State R's reservation, and says the treaty is not to apply, then there is no treaty between them at all -- \convention{\textit{VCLT} Art 20(4)(b)} (Page \pageref{VCLT Art 20})
            \item If State C objects to State R's reservation but does not say that the treaty is not to apply, then the treaty applies, but “the provisions to which the reservation applies do not apply ... to the extent of the reservation” -- \convention{\textit{VCLT} Art 21(3)} (Page \pageref{VCLT Art 21})
        \end{enumerate}
        \item Was the state a persistent objector?
        \begin{enumerate}
            \item States which consistently object to the emergence of a rule of custom from its earliest point of gestation will not be bound by it -- \case{\textit{Anglo Norwegian Fisheries Case (UK v Norway)} [1951] ICJ Rep 116} (Page \pageref{case:UK v Norway Fisheries})
            \item A state cannot be a persistent objector to a \textit{jus cogens} principle -- \article{\textit{International Law Commission 2019 Report} Chapter V Conclusion 14} (Page \pageref{report:2019 ILC Conc. 14})
        \end{enumerate}
    \end{enumerate}
    \item How was the treaty interpreted by the state?
    \begin{enumerate}
        \item There are a number of different approaches to treaty interpretation:
        \begin{enumerate}
            \item Formalist/Textual (formal adherence to the terms of the treaty)
            \item Restrictive (deference to state sovereignty)
            \item Teleological (to give effect to the object and purpose of the treaty)
            \item Effectiveness (to ensure the treaty regime remains as effective as possible)
            \item Originalist (to focus on the original purpose of the treaty)
        \end{enumerate}
        \item The Australian courts will apply the VCLT when interpreting a treaty that has been incorporated into Australian law -- \case{\textit{DHI22 v Qatar Airways} [2024] FCA 348 at [30] (Halley J)}, citing \case{\textit{Povey v Qantas Airways Ltd} (2005) 233 CLR 189 at [24] (Gleeson CJ, Gummow, Hayne and Heydon JJ)} (see Section \ref{sec:Interpretation of Treaties} on Page \pageref{sec:Interpretation of Treaties})
        \item The VCLT contains rules on how to interpret treaties -- \convention{\textit{VCLT} Art 31} (Page \pageref{VCLT Art 31})
        \item Under the VCLT, instruments used in treaty interpretation must have been adopted by all states -- \case{\textit{Whaling in the Antarctic Case} [2014] ICJ Rep 226 at [83]} (Page \pageref{case:Antarctic Whaling})
        \item As a last resort, supplementary means of interpretation can be used to interpret the provisions of a treaty under Art 31 -- \convention{\textit{VCLT} Art 32} (Page \pageref{VCLT Art 32})
    \end{enumerate}
    \item Is the treaty void or otherwise invalidated?
    \begin{enumerate}
        \item Is the treaty void?
        \begin{enumerate}
            \item If the State's representative had been coerced into entering the treaty, or there were acts or threats directed against the representative, a State's consent will not be made out and so the treaty will be void -- \convention{\textit{VCLT} Art 51} (Page \pageref{VCLT Art 51})
            \item If a State's consent was obtained through a threat or the use of force, it is void -- \convention{\textit{VCLT} Art 51} (Page \pageref{VCLT Art 52})
            \item If a treaty conflicts with a \textit{jus cogens} norm, it is void -- \convention{\textit{VCLT} Art 51} (Page \pageref{VCLT Art 53})
            \item If a new \textit{jus cogens} norm has emerged since the ratification of a treaty and the treaty conflicts with that \textit{jus cogens} norm, the treaty is void -- \convention{\textit{VCLT} Art 51} (Page \pageref{VCLT Art 64})
        \end{enumerate}
        \item Is the treaty invalid?
        \begin{enumerate}
            \item Did the state's consent to a treaty involve a violation of an internal law of fundamental importance? -- \convention{\textit{VCLT} Art 46(1)} (Page \pageref{VCLT Art 46})
            \begin{enumerate}
                \item A state may not invoke inconsistent internal law as a basis on which it could not sign a treaty, unless that rule is of manifest importance
            \end{enumerate}
            \item If a representative of a state had gone beyond what he was authorised to do so in signing the treaty, their omission to observe their limitations will not constitute an invalidation of the treaty, unless the restriction was notified to other states prior to the expression of consent -- \convention{\textit{VCLT} Art 47} (Page \pageref{VCLT Art 47})
        \end{enumerate}
        \item Was there an error of fact that formed the essential basis of consent? -- \convention{\textit{VCLT} Art 48} (Page \pageref{VCLT Art 48})
        \begin{enumerate}
            \item Consent may be validated by means of an error if the error relates to a fact or situation assumed by the state that existed at the time when the treaty was concluded, and forms an essential basis of its consent to be bound by the treaty -- \convention{\textit{VCLT} Art 48(1)} (Page \pageref{VCLT Art 48})
            \item An error of fact cannot be plead by a party if they contributed to it, could have avoided it, or were otherwise put on notice of a possible error -- \convention{\textit{VCLT} Art 48(2)} (Page \pageref{VCLT Art 48}); \case{\textit{Temple of Preah Vihear (Cambodia v Thailand)} [1962] ICJ Rep 6 at Page 17}
            \item If there is an error relating to only the wording of the treaty's text, its validity is not affected, and \convention{Art 79} is enlivened -- \convention{\textit{VCLT} Art 48(3)} (Page \pageref{VCLT Art 48})
        \end{enumerate}
        \item Had the state been induced to conclude the treaty by the fraudulent conduct of another state? -- \convention{\textit{VCLT} Art 49} (Page \pageref{VCLT Art 49})
    \end{enumerate}
    \item Are there grounds to terminate, withdraw or suspend the treaty? \\\vspace{8pt}
    \textit{The following constitute internal grounds of termination, withdrawal or suspension.}
    \begin{enumerate}
        \item Was the treaty terminated or withdrawn from under:
        \begin{enumerate}
            \item Its provisions? -- \convention{\textit{VCLT} Art 54(a)} (Page \pageref{VCLT Art 54})
            \item By consent of all of the parties after consultation with the other contracting states? -- \convention{\textit{VCLT} Art 54(b)} (Page \pageref{VCLT Art 54})
        \end{enumerate}
        \item Was the treaty suspended under:
        \begin{enumerate}
            \item Its provisions? -- \convention{\textit{VCLT} Art 57(a)} (Page \pageref{VCLT Art 57})
            \item By consent of all of the parties after consultation with the other contracting states? -- \convention{\textit{VCLT} Art 57(b)} (Page \pageref{VCLT Art 57})
        \end{enumerate}
    \end{enumerate}
    \textit{The following constitute external grounds of termination, withdrawal or suspension.}
    \begin{enumerate}[resume]
        \item Was there a denunciation or withdrawal from the treaty when there is no provision to do so? -- \convention{\textit{VCLT} Art 56} (Page \pageref{VCLT Art 56})
        \begin{enumerate}
            \item There is generally no right of denunciation, except where: -- \convention{\textit{VCLT} Art 56(1)} (Page \pageref{VCLT Art 56})
            \begin{enumerate}
                \item It is established that the parties intended to admit the possibility of denunciation or withdrawal -- \convention{\textit{VCLT} Art 56(1)(a)} (Page \pageref{VCLT Art 56})
                \item A right of denunciation or withdrawal may be implied by the nature of the treaty -- \convention{\textit{VCLT} Art 56(1)(b)} (Page \pageref{VCLT Art 56})
            \end{enumerate}
            \item Under this provision, a party must give at least 12 months' notice of its intention to denounce/withdrawal from the treaty -- \convention{\textit{VCLT} Art 56(2)} (Page \pageref{VCLT Art 56})
        \end{enumerate}
        \item Was there a material breach? -- \convention{\textit{VCLT} Art 60} (Page \pageref{VCLT Art 60})
        \begin{enumerate}
            \item A material breach involves a wrongful act being intentionally committed by a party -- \case{\textit{Gabčíkovo-Nagymaros Case} [1997] ICJ Rep 7 at [72]-[81]} (Page \pageref{case:[1997] ICJ Rep 7})
            \item In a bilateral treaty, this entitles the other party to terminate the treaty or suspend its operation, in whole or in part -- \convention{\textit{VCLT} Art 60(1)} (Page \pageref{VCLT Art 60})
            \item If there was a breach in a multilateral treaty: -- \convention{\textit{VCLT} Art 60(2)} (Page \pageref{VCLT Art 60})
            \begin{enumerate}
                \item The other parties can unanimously opt to suspend or terminate the treaty either (i) between themselves and the defaulting state, or (ii) between all parties -- \convention{\textit{VCLT} Art 60(2)(a)} (Page \pageref{VCLT Art 60})
                \item A party who has been especially affected has grounds to suspend the treaty in whole or in part between itself and the defaulting state -- \convention{\textit{VCLT} Art 60(2)(b)} (Page \pageref{VCLT Art 60})
                \item Any party other than the defaulting party may suspend the treaty in whole or in part if the breach is such that it radically changes the position of every party with respect to the further performance of its obligations under the treaty -- \convention{\textit{VCLT} Art 60(2)(c)} (Page \pageref{VCLT Art 60})
            \end{enumerate}
            \item Moreover, a material breach entails:
            \begin{enumerate}
                \item A repudiation of the treaty not sanctioned by the present Convention -- \convention{\textit{VCLT} Art 60(3)(a)} (Page \pageref{VCLT Art 60})
                \item The violation of a provision essential to the accomplishment of the object or purpose of this treaty -- \convention{\textit{VCLT} Art 60(3)(b)} (Page \pageref{VCLT Art 60})
            \end{enumerate}
            \item A party cannot claim material breach if they themselves had committed the wrongful act -- \case{\textit{Gabčíkovo-Nagymaros Case} [1997] ICJ Rep 7 at [92]-[94]} (Page \pageref{case:[1997] ICJ Rep 7})
        \end{enumerate}
        \item Did the performance of the treaty become impossible? -- \convention{\textit{VCLT} Art 61} (Page \pageref{VCLT Art 61})
        \begin{enumerate}
            \item A state may terminate or withdraw from a treaty if its performance has become impossible because `an object indispensable for the execution of the treaty' has permanently disappeared or been destroyed -- \convention{\textit{VCLT} Art 61(1)} (Page \pageref{VCLT Art 61})
            \item However, impossibility of performance may not be invoked if the impossibility is the result of a breach by that party either of an obligation under that treaty or any other international obligations owed to any other party of the treaty -- \convention{\textit{VCLT} Art 61(2)} (Page \pageref{VCLT Art 61})
        \end{enumerate}
        \item Was there a fundamental change of circumstances that precluded the operation of the treaty? -- \convention{\textit{VCLT} Art 62} (Page \pageref{VCLT Art 62})
        \begin{enumerate}
            \item Under the principle of \textit{pacta sunt servanda} (\article{VCLT Art 26} on Page \pageref{VCLT Art 26}), the party must have exhausted all possible avenues before claiming a fundamental change of circumstances
            \item A fundamental change of circumstances entails: -- \case{\textit{Gabčíkovo-Nagymaros Case} [1997] ICJ Rep 7 at [104]} (Page \pageref{case:[1997] ICJ Rep 7})
            \begin{enumerate}
                \item The circumstances at the conclusion of the treaty must have been an essential basis of consent
                \item The change must not have been foreseen
                \item The change must radically transform the extent of the obligations still to be made performed                
            \end{enumerate}
            \item A fundamental change of circumstances may not be invoked as a grounds for termination/withdrawal unless: -- \convention{\textit{VCLT} Art 62(1)} (Page \pageref{VCLT Art 62})
            \begin{enumerate}
                \item The existence of those circumstances constituted an essential basis of the consent of the parties to be bound by the treaty; \textbf{and} -- \convention{\textit{VCLT} Art 62(1)(a)} (Page \pageref{VCLT Art 62})
                \item The effect of the change is radically to transform the extent of the obligations still to be performed under the treaty -- \convention{\textit{VCLT} Art 61(1)(b)} (Page \pageref{VCLT Art 62})
            \end{enumerate}
            \item A fundamental change of circumstances may not be invoked as a ground for terminating or withdrawing from a treaty: -- \convention{\textit{VCLT} Art 62(2)} (Page \pageref{VCLT Art 62})
            \begin{enumerate}
                \item If the treaty establishes a boundary -- \convention{\textit{VCLT} Art 62(2)(a)} (Page \pageref{VCLT Art 62})
                \item If the fundamental change is the result of a breach by the party invoking it -- \convention{\textit{VCLT} Art 62(2)(b)} (Page \pageref{VCLT Art 62})
            \end{enumerate}
            \item If a party invokes a fundamental change of circumstances as a ground for terminating or withdrawing from a treaty, it may also invoke the change as a ground for suspending the operation of the treaty -- \convention{\textit{VCLT} Art 62(3)} (Page \pageref{VCLT Art 62})
            \item International courts are very reluctant to find that impossibility and/or fundamental change of circumstances have been made out (i.e., they have a very high threshold and thus a very limited scope, as suggested by the negative wording of the Articles) -- \case{{Gabčíkovo-Nagymaros Case} [1997] ICJ Rep 7 at [104]} (Page \pageref{case:[1997] ICJ Rep 7})
            \item \article{Art 62} has generally been accepted as a codification of the existing customary law on termination by fundamental change of circumstances -- \case{\textit{Fisheries Jurisdiction (United Kingdom v Iceland)} ICJ Reports 1973, pg. 63, para. 36}
        \end{enumerate}
    \end{enumerate}
\end{enumerate}

\section{International Law and Australian Law}
\begin{enumerate}
    \item Was this a matter involving Australia, or a state which follows the Australian approach to adopting international law?
    \item How does international law influence Australian law?
    \begin{enumerate}
        \item International law applies between states, but may also recognise institutions of domestic law that have an extensive/important role in international law (e.g., corporations) -- \case{\textit{Barcelona Traction (Belgium v Spain)} [1970] ICJ Rep 3 (Page 44)} [where corporations could be recognised within international law]
        \item Absent/inconsistent domestic law is not excuse for failing to meet international obligations -- \case{\textit{Alabama Claims Arbitration (US/Britain)} (1872)}; \case{\textit{Sandline Arbitration} (1998)}
        \item Expert evidence cannot be adduced to prove or explain statements of international law -- \case{\textit{ACCC v PT Garuda (No 9)} [2013] FCA 23 at [31] (Perram J)} (Page \pageref{case:ACCC v Garuda})
        \item Common law can be developed with regard to international law, where it is not inconsistent with domestic law -- \case{\textit{Chow Hung Ching v R} (1949) 77 CLR 449, 471 (Starke J)} (Page \pageref{case:Chow Hung Ching})
        \begin{enumerate}
            \item However, international law is not incorporated as part of Australian law, but rather is a \textbf{source} -- Latham CJ at page 462; Dixon J at page 477
        \end{enumerate}
        \item International law cannot automatically be imported/included in Australian law, but remains a ``\textbf{legitimate and important influence} on the development of the law'' -- \case{\textit{Mabo v Queensland (No 2)} (1992) 175 CLR 1, 42 (Brennan J)} (Page \pageref{case: Mabo})
        \begin{enumerate}
            \item This is especially the case when referring to aspects of international law that touch on universal human values (e.g., human rights)
        \end{enumerate}
        \item However, in cases such as a civil claim for torture (or any other serious crime forbidden under international law), the common law of state doctrine should reflect universal norms -- \case{\textit{Habib v Commonwealth} (2010) 183 CLR 62 at [7] (Black CJ)} (Page \pageref{case: Habib v Commonwealth})
    \end{enumerate}
    \item Has customary international law been implemented in Australian law? \\ \textit{Always explore monism and dualism for a nuanced discussion, and describe how this results in Australia's hard transformation approach.}
    \begin{enumerate}
        \item Whilst there is no clear authority, the automatic incorporation of customary international law in Australia has been rejected
        \item Australia has rejected the monism approach (where states and international law form one entity), and has adopted a hard transformation approach that tends towards dualism (where states and international law form two separate entities) (Section \ref{sec:Monism and Dualism} on Page \pageref{sec:Monism and Dualism})
        \begin{enumerate}
            \item Under the \textbf{hard transformation approach}, only legislation may implement the provisions of international law into domestic law; otherwise, international law does not apply (this is the approach favoured in Australia, following \case{\textit{Chow Hung Ching v R} (1949) 77 CLR 449 (Latham CJ at 462; Dixon J at 477)} (Page \pageref{case:Chow Hung Ching}))
            \item The \textbf{soft transformation approach} holds that legislation or court decisions may implement the provisions of international law (discussed by Latham CJ in \case{\textit{Chow Hung Ching v R} (1949) 77 CLR 449} (Page \pageref{case:Chow Hung Ching})), but so far has been rejected -- \case{\textit{Dietrich v R} [1992] HCA 57} (Page \pageref{case: Dietrich v R})
            \begin{enumerate}
                \item ``International law is not as such part of the law of Australia, but a universally recognized principle of international law would be applied by our courts'' -- \case{\textit{Chow Hung Ching v R} (1949) 77 CLR 449, 462 (Latham CJ)} (Page \pageref{case:Chow Hung Ching})
            \end{enumerate}
        \end{enumerate}
        \item If the country is instead following the UK's approach, see Section \ref{sec:Customary International Law in Australian Law} on Page \pageref{sec:Customary International Law in Australian Law}
    \end{enumerate}
    \item Was this a matter involving international criminal law?
    \begin{enumerate}
        \item Customary/international criminal law can never be a part of the Australian common law/transformed or implemented by the courts (it can only be imported by statute) -- \case{\textit{Nulyarimanna v Thompson} (1999) 165 ALR 621} (Page \pageref{case: Nulyarimanna v Thompson})
        \begin{enumerate}
            \item At [20], Wilcox J held that if domestic criminal law could be influenced by customary/international criminal law, it would lead to the position where international obligations have greater obligations than domestic consequences, sidelining domestic law and thus a state's independence to make its own criminal laws
            \item This is moreover a position adopted in the UK -- \case{\textit{R v Jones} [2006] 2 All ER 741 (Lords Bingham, Mance and Hoffman)} (Page \pageref{case: R v Jones})
        \end{enumerate}
        \item \textit{Jus cogens} principles of international law are not automatically part of Australian common law, and criminal offences must be created by statute, not by the courts -- \case{\textit{Nulyarimanna v Thompson} (1999) 165 ALR 621 at [17], [20], [32], [57] (Wilcox and Whitlam JJ)} (Page \pageref{case: Nulyarimanna v Thompson})
        \begin{enumerate}
            \item For example, the \textit{jus cogens} prohibition of genocide was not automatically part of Australian common law, and had to be created by statute (see the above paragraphs for context) 
            \item Merkel J dissented, and held that the approach that should be taken was the `common law adoption approach', which is that a rule of international law is to be adopted by a court so long as it is not inconsistent with legislation or public policy
        \end{enumerate}
    \end{enumerate}
    \item Does the matter involve a treaty being implemented into Australian law?
    \begin{enumerate}
        \item \label{treaty entry executive power} The power to enter into treaties is exclusively an Executive prerogative power -- \statute{\textit{Constitution} s 61} (Page \pageref{Constitution s 61})
        \begin{enumerate}
            \item In interpreting the external affairs power, the HCA has held that they only need to look at whether the law applies geographically externally to Australia, not whether the international law was void by virtue of the underlying treaty being void -- \case{\textit{Horta v Commonwealth} (1994) 181 CLR 183, 191}
            \item ``The federal executive, through the Crown's representative, possessed exclusive and unfettered treaty-making power'' -- \case{\textit{Koowarta v Bjelke-Petersen} (1982) 153 CLR 168 at [215] (Stephen J)} (Page \pageref{case:Koowarta v Bjelke-Petersen})
        \end{enumerate}
        \item \label{treaty legislature powers} The power to implement the provisions of a treaty is a legislative power -- \statute{\textit{Constitution} s 51(xxix)} [the `external affairs' power]
        \begin{enumerate}
            \item Treaty provisions do not form a part of Australia law until they have been implemented by statute -- \case{\textit{Dietrich v R} [1992] HCA 57 (Brennan CJ, Mason and McHugh JJ)} (Page \pageref{case: Dietrich v R})
            \item Resolutions of international organisations (e.g., the UN Security Council) do not form a part of Australian law until they have been implemented by statute -- \case{\textit{Bradley v Commonwealth} (1973) 128 CLR 557, 582 (Barwick CJ and Gibbs J)} (Page \pageref{case:Bradley v Commonwealth})
            \begin{enumerate}
                \item If Parliament `approves' a treaty, it is not binding
                \item The mere approval of Parliament does not give a treaty the force of law (this practice has since lapsed)
            \end{enumerate}
            \item Under \statute{\textit{Constitution} s 51(xxix)}, the law must carry into effect treaty obligations, and be reasonably considered to be appropriate and adapted to achieving this objective -- \case{\textit{Commonwealth v Tasmania} (1983) 158 CLR 1, 40 (Deane J)} (Page \pageref{case:Commonwealth v Tasmania})
            \item Ratification only occurs after a treaty has been implemented into internal legislative provisions/given the force of law (the legislative approach is the preferred one, as it the most common and avoids uncertainty)
        \end{enumerate}
    \end{enumerate}
    \item Does the matter involve Australia entering into/making a treaty?
    \begin{enumerate}
        \item Australia can enter into two types of treaties:
        \begin{enumerate}
            \item Bilateral treaties, which enter into force for Australia after
            \begin{enumerate}
                \item Signature
                \item Subsequent exchange of notes stating that the constitutional process is completed
            \end{enumerate}
            \item Multilateral treaties, which enter into force for Australia after
            \begin{enumerate}
                \item Signature
                \item Subsequent ratification (or accession if there was no previous signature)
            \end{enumerate}
        \end{enumerate}
        \item \label{item:trick or treaty} Whilst there is no constitutional requirement for the Parliament to be involved in the treaty-making process, since 1996, Parliament has been consulted on the treaty-making process (without a veto) -- \article{\textit{Trick or Treaty?} (1995 Report of the Senate Legal and Constitutional References Committee)}
        \begin{enumerate}
            \item The present convention is that all proposed treaty conventions are tabled in Parliament at least 15 sitting days prior to any binding action being undertake (with exemptions for urgent or sensitive treaties)
            \item A National Interest Analysis (NIA) is also prepared, which is akin to an explanatory memorandum for a treaty
            \item The treaty should also be reviewed by the Joint Standing Committee on Treaties
        \end{enumerate}
    \end{enumerate}
    \item Should international law be used to interpret Australian statute?
    \begin{enumerate}
        \item International law can be used as extrinsic material when interpreting legislation which refers to a treaty -- \statute{\textit{Acts Interpretation Act 1901} (Cth) ss 15AB(1) and (2)(d)} (Page \pageref{Acts Interpretation Act s 15AB})
        \item International law can be used to interpret a legislative provision that incorporates a treaty provision; in this instance, the rules of treaty interpretation apply rather than statutory interpretation
        \item Is the \textit{Polites} principle enlivened?
        \begin{enumerate}
            \item The \textit{Polites} principle refers to the presumption that the Parliament intends to give effect to Australia's obligations under international law, in the absence of express words/intention to the contrary -- \case{\textit{Polites v Commonwealth} (1945) 70 CLR 60, 77 (Dixon J)} (Page \pageref{case:Polites v Commonwealth})
            \begin{enumerate}
                \item It is a general rule of statutory interpretation that, in the absence of express words to the contrary, it is presumed that legislation is intended to be in conformity with the treaty-based rules of international law
            \end{enumerate}
            \item Once a treaty is ratified and implemented into domestic law, statutory interpretation requires courts to presume that legislation is intended to be in conformity with international law (\textit{Polites} principle) because Parliament, \textit{prima facie}, intends to give effect to Australia's international obligations
            \item The \textit{Polites} principle does not apply to constitutional interpretation, as it would violate the requirement for a referendum to modify the \textit{Constitution} under s 128 -- \case{\textit{Al-Kateb v Godwin} (2004) 208 ALR 124} (Page \pageref{case:Al-Kateb v Godwin})
            \begin{enumerate}
                \item At [66], McHugh J outlined that it was never the case that the \statute{\textit{Constitution}} should have been interpreted to conform with the rules of international law
                \item At [175], Kirby J (in dissent) held that the \statute{\textit{Constitution}} should be interpreted in a way that is generally harmonious with the basic principles of international law, including as that law states human rights and fundamental freedoms
            \end{enumerate}
        \end{enumerate}
    \end{enumerate}
    \item How is treaty law implemented into Australian law?
    \begin{enumerate}
        \item Signing of the treaty
        \begin{enumerate}
            \item The Commonwealth Executive has the exclusive power to sign treaties (see \ref{treaty entry executive power})
            \item Parliamentary approval is not necessary, but as a matter of convention, it has been sought (see \ref{treaty legislature powers})
        \end{enumerate}
        \item Implementation of the treaty
        \begin{enumerate}
            \item Parliament will generally be consulted about the treaty, but constitutionally, its approval is not required to sign the treaty (see \ref{treaty legislature powers})
            \item Following the \article{\textit{Trick or Treaty}} report (see \ref{item:trick or treaty}), the Executive will consult Parliament before signing treaties
            \item All proposed treaty actions are tabled in Parliament at least 15 sitting days prior to a binding action
            \item Parliament can provide recommendations and scrutinise the treaty, but ultimately the Executive has the final say
        \end{enumerate}
        \item Ratification of the treaty
        \begin{enumerate}
            \item The treaty is ratified by the executive once domestic legislation is present to incorporate the terms of the treaty
            \item Under \convention{\textit{VCLT} Art 2(1)(b)} (Page \pageref{VCLT Art 2}), ratification is the intentional act whereby a state indicates its consent to be bound to a treaty if the parties intended to show their consent (generally, the depositary will collect the ratifications of all states)
        \end{enumerate}
    \end{enumerate}
\end{enumerate}

\section{Personality, Statehood and Self-Determination}

\begin{enumerate}
    \item Does the entity have an international legal personality?
    \begin{enumerate}
        \item International legal personality gives an actor rights, duties and powers on the international plane, with international legal personality being a spectrum incurring different obligations and rights -- \case{\textit{Reparations for Injuries Suffered in the Service of the United Nations Advisory Opinion} [1949]}
        \begin{enumerate}
            \item With international legal personality, an entity can:
            \begin{enumerate}
                \item Make claims before international committees, courts and tribunals
                \item Be subject to some or all international legal obligations
                \item Be empowered to enter into treaties
                \item Enjoy the jurisdiction of some national courts
            \end{enumerate}
        \end{enumerate}
        \item States have the largest international legal personality, with the most rights and obligations
        \begin{enumerate}
            \item This is contingent on whether an entity has statehood
        \end{enumerate}
        \item A non-territorialised entity can still have international personality (e.g., the Holy See, which is the government of the Vatican City State, has international legal personality, but is not a state in the traditional sense)
        \item By default, a corporation does not possess international legal personality
        \begin{enumerate}
            \item Thus, corporations cannot be parties to a treaty
            \item Corporations may be parties to contracts governed by international law (`internationalised' contracts) -- \case{\textit{Texaco Overseas Petroleum Company v Libya} (1977) 53 ILR 389}
        \end{enumerate}
        \item Individuals have a limited degree of international legal personality
        \begin{enumerate}
            \item They cannot enter into treaties, and only have standing before international courts in limited circumstances
            \item Individuals are subject to international and criminal human rights laws -- \statute{\textit{ICC Statute} (1998) Art 25}; \case{\textit{Nuremberg Trial Judgement} (1947)}; \case{\textit{R v Bow Streets Magistrate, ex Parte Pinochet (No 3)} [1999] 2 All ER 97}
        \end{enumerate}
        \item International organisations do not possess general competence, but are instead governed by the principle of `specialty' (i.e., they can only act within the limits of their powers as defined by their constitutive treaties) -- \case{\textit{WHO Advisory Opinion} [1996] ICJ Rep 66}
        \begin{enumerate}
            \item International organisations may enter into treaties
            \item International organisations may be responsible for wrongful acts -- \convention{\textit{ILC Draft Articles on the Responsibility of International Organisations (2011)}}
            \item International organisations can seek compensation against states where their interests have been harmed -- \case{\textit{Reparation for Injuries Case} [1949] ICJ Rep 174}
        \end{enumerate}
        \item Corporations and individuals cannot go to international courts as only states can be parties to a dispute -- \statute{\textit{ICJ Statute} Art 34(1)}
        \begin{enumerate}
            \item Instead, other modes, such as arbitration (\statute{\textit{ICJ Statute} Art 33(1)}) are used to resolve the dispute peacefully -- \statute{\textit{ICJ Statute} Art 2(3)} (See Scaffold \ref{scaffold:Topic 13} on Page \pageref{scaffold:Topic 13})
        \end{enumerate}
    \end{enumerate}
    \textit{\textbf{Points 2-4 concern whether an entity is a state.}}
    \item Does the entity meet the criteria for statehood?
    \begin{enumerate}
        \item Does the entity have a permanent population? -- \convention{\textit{Montevideo Convention on the Rights and Duties of States} (1933) Art 1(a)}
        \begin{enumerate}
            \item There is no requirement as to the size of the population (e.g., Nauru has less than 10,000 people but is still a state), but the population must be permanent
        \end{enumerate}
        \item Does the entity have a defined territory? -- \convention{\textit{Montevideo Convention on the Rights and Duties of States} (1933) Art 1(b)}
        \begin{enumerate}
            \item Whilst the boundaries do not need to be fixed/undisputed, there must be a reasonably coherent territory that is effectively governed by the state
            \item E.g., the Vatican City is a microstate, and Nauru has an area of 21 km\textsuperscript{2}, but both are recognised as states
        \end{enumerate}
        \item Does the entity have a functioning government? -- \convention{\textit{Montevideo Convention on the Rights and Duties of States} (1933) Art 1(c)}
        \begin{enumerate}
            \item The form or quality of the government is irrelevant, so long as it can effectively control the area
        \end{enumerate}
        \item Does the entity have the capacity to enter into relations with other states? -- \convention{\textit{Montevideo Convention on the Rights and Duties of States} (1933) Art 1(d)}
        \begin{enumerate}
            \item A state cannot be subject to the control of another state
            \item So long as it is not placed under the legal authority of another state, it remains an independent state -- \case{\textit{Customs Union Between Germany and Austria} (1931) PCIJ Series A/B No 41, 57-8 (Judge Anzilotti)}
        \end{enumerate}
    \end{enumerate}
    \item Has the entity been recognised as a state by other states?
    \begin{enumerate}
        \item Recognition may be required to enjoy the benefits of statehood
        \begin{enumerate}
            \item If other states refuse to recognise to engage with the state, the state will have no partners to make treaties or exchange diplomatic representations and will not be welcomed into international organisations -- \article{Thomas Grant, Praeger 1999, 24}
        \end{enumerate}
        \item There are two theories on the effects of recognition:
        \begin{enumerate}
            \item Constitutive theory - recognition by other states is a precondition to statehood (i.e., a condition precedent)
            \item Declaratory theory (the predominant view) - recognition by other states does not create an entity's statehood, but merely acknowledges that the entity is a state (i.e., a condition subsequent) -- \case{\textit{Great Britain v Costa Rica} (1923)}
            \begin{enumerate}
                \item Under this theory, international law contains no prohibitions on declarations of independence (\case{\textit{Kosovo Advisory Opinion} [2010] ICJ Rep 403}), but this does not meant that other states would give recognition
            \end{enumerate}
        \end{enumerate}
        \item Recognition of a state will often be accompanied by a statement that addresses some of the Montevideo criteria
        \item International law contains no prohibition of declarations of independence, but other states should not recognise a unilateral seceding entity before it has acquired statehood -- \case{\textit{Kosovo Advisory Opinion} [2010] ICJ Rep 403}
        \item Recognition of a state may not be extended when a state is created as a result of unlawful use of force/aggression -- \textit{Stimson Doctrine}
        \begin{enumerate}
            \item Puppet states will not be recognised as states (e.g., Manchukuo, a Japanese puppet state in China from 1932-1945)
        \end{enumerate}
        \item If a state is created through the breach of peremptory norms of international law, it will not be recognised as a state -- \case{\textit{Kosovo Advisory Opinion} [2010] ICJ Rep 403}
    \end{enumerate}
    \item Has the entity's government or a foreign corporation been recognised as legitimate?
    \begin{enumerate}
        \item Does this concern a foreign government?
        \begin{enumerate}
            \item It is usually the case that a new government is given recognition, especially if the government has changed in the normal manner for that entity
            \item Australia will otherwise examine the constitutionality of the new government, the control of the government over the territory, the inter-governmental dealings, and the extent of international recognition, among other factors
        \end{enumerate}
        \item Does this concern a foreign corporation?
        \begin{enumerate}
            \item To see if a foreign body or person is legitimate, that body or person needs to be validly incorporated in a place outside Australia, by reference to the law of that place -- \statute{\textit{Foreign Corporations (Application of Laws) Act 1989} (Cth) s 7(2)}
            \item The application of (i) is not affected by the recognition or non-recognition at any time:
            \begin{enumerate}
                \item Of a foreign state or place -- \statute{\textit{Foreign Corporations (Application of Laws) Act 1989} (Cth) s 9(1)(a)}
                \item Of the government of a foreign place or state -- \statute{\textit{Foreign Corporations (Application of Laws) Act 1989} (Cth) s 9(1)(b)}
                \item That a place forms part of a foreign state -- \statute{\textit{Foreign Corporations (Application of Laws) Act 1989} (Cth) s 9(1)(c)}
                \item Of the entities created, organised or operating under the law applied by the people in a foreign state or place -- \statute{\textit{Foreign Corporations (Application of Laws) Act 1989} (Cth) s 9(1)(d)}
            \end{enumerate}
        \end{enumerate}
    \end{enumerate}
    \item Do the people of the entity have a right to self-determination?
    \begin{enumerate}
        \item The right to self-determination refers to the rights of all people to freely determine their political status and to pursue their economic, social and cultural development -- \convention{\textit{1966 ICCPR and 1966 ICECSR} Common Art 1}
        \begin{enumerate}
            \item Self-determination is ``the need to pay regard to the freely expressed will of peoples" -- \case{\textit{Western Sahara Advisory Opinion} [1975] ICJ Rep 12 at [59]}
            \item The right to self-determination is an \gls{erga omnes} customary rule of international law, with all states possessing an interest to protect that right -- \convention{\textit{UNGA Resolution 1514XV} (1960)}; \case{\textit{Chagos Islands Advisory Opinion} [2019] ICJ Rep 95}; \case{\textit{Construction of a Wall in Occupied Palestinian Territory} [2005] at [155]}
        \end{enumerate}
        \item The right to self-determination is subject to the principle of \textit{uti possidetis juris} (respect for existing frontiers) -- \case{\textit{Burkina Faso/Mali} [1986] ICJ Rep 554}
        \begin{enumerate}
            \item If a new state has emerged from an area granted independence, the new entity will possess the same borders/frontiers as it did prior to independence -- \case{\textit{Burkina Faso/Mali} [1986] ICJ Rep 554}
            \item This principle generally applies to colonies achieving independence
        \end{enumerate}
        \item Is there an external right of self-determination?
        \begin{enumerate}
            \item Generally, people cannot choose to become independent unless they are in one of three scenarios: -- \case{\textit{Reference Re Secession of Quebec} (1988) 2 SCR 217}; \convention{\textit{Declaration of Granting Independence to Colonial Countries and Peoples} (GA Res 1514 (XV) 1960)}
            \begin{enumerate}
                \item They are under colonial rule or are a non-self-governing territory
                \item They are subject to alien subjugation, domination or exploitation
                \item Possibly they are under oppression and are blocked from meaningful self-determination (`remedial secession')
            \end{enumerate}
            \item Where there is an external right of self-determination, the people have three choices (though there is no automatic or default choice): -- \case{\textit{Chagos Islands Advisory Opinion} [2019] ICJ Rep 95}
            \begin{enumerate}
                \item Emerge as a sovereign independent state
                \item Freely associate with an independent state (e.g., the Cook Islands' free association with New Zealand)
                \item Integrate with an independent state
            \end{enumerate}
            \item ``The recognised sources of international law established the right to self-determination of a people is normally fulfilled through internal self-determination" -- \case{\textit{Reference Re Secession of Quebec} (1998) 2 SCR 217 at [126]}
        \end{enumerate}
        \item Is there an internal right of self-determination?
        \begin{enumerate}
            \item The internal right to self-determination refers to the ability of the peoples of a state to choose independence for themselves
        \end{enumerate}
        \item Are there Indigenous people seeking self-determination?
        \begin{enumerate}
            \item This is an example of internal self-determination
            \item Indigenous people have the right to self-determination -- \convention{\textit{2007 UN Declaration on the Rights of Indigenous People} Art 3}
            \item Indigenous people have the right to autonomy or self-government in matters relating to their internal and local affairs -- \convention{\textit{2007 UN Declaration on the Rights of Indigenous People} Art 4}
        \end{enumerate} 
    \end{enumerate}
\end{enumerate}

\section{Title to Territory}
\begin{enumerate}
    \item Was there territory involved?
    \begin{enumerate}
        \item Once an entity has territory, it can attach statehood to 
        \item Only states can acquire territory, not individuals -- \case{\textit{Ure v Commonwealth} (1949) 79 CLR 1, 6 (Latham CJ)}
        \item Territory does not equate sovereignty, but sovereignty does imply ownership of territory
        \begin{enumerate}
            \item Sovereignty in relation to territory is `the right to exercise therein, to the exclusion of any other State, the functions of a State' -- \case{\textit{Island of Palmas} (1928)}
            \item Territorial sovereignty refers to a state's right to exercise exclusive jurisdiction within its territory
        \end{enumerate}
        \item Sovereignty encompasses a state's authority within territorial boundaries, jurisdiction refers to the specific legal powers and rights exercised by a state over certain activities under international law
    \end{enumerate}
    \item Was there the occupation of territory?
    \begin{enumerate}
        \item The territory must have been:
        \begin{enumerate}
            \item Unoccupied (not under the sovereignty of another state); or
            \item Terra nullius (abandoned, or otherwise no population) -- \case{\textit{Western Sahara Advisory Opinion} [1975] ICJ Rep 162}
            \begin{enumerate}
                \item Occupation is the exercise of sovereignty over territory that has been deemed as terra nullius -- \case{\textit{Western Sahara Advisory Opinion} ICJ Rep 162}; \case{\textit{Clipperton Island Arbitration (France v Mexico)} 1932}
                \item ``Territories inhabited by tribes or peoples having a social and political organization [are] not regarded as terrae nullius" -- \case{\textit{Western Sahara Advisory Opinion} [1975] ICJ Rep 162 at [80]}
                \item Territories inhabited by tribes or people having a social and political organisation are not regarded as terra nullius -- \case{\textit{Western Sahara Advisory Opinion} [1975] ICJ Rep 162}; \case{\textit{Mabo v Queensland (No. 2)} (1992) 175 CLR 1}
                \item International law has never regarded occupied territory as terra nullius - \case{\textit{Mabo v Queensland (No. 2)} (1992) 175 CLR 1 at [41], [47] (Brennan J)}
            \end{enumerate}
        \end{enumerate}
        \item Occupation is ``an original means of peaceably acquiring sovereignty over territory" - \case{\textit{Western Sahara Advisory Opinion} [1975] ICJ Rep 162}
        \item Whether territory was lawfully occupied or otherwise lawfully acquired is solely within the domain of international law, and cannot be questioned by Australian domestic law -- \case{\textit{Mabo v Queensland (No. 2)} (1992) 175 CLR 1 at [3] (Deane and Gaudron JJ)}
        \item Was there an intention to occupy?
        \begin{enumerate}
            \item Was there an expression of formal intent to claim possession of the land (\textit{animus occupandi}; e.g., the planting of a flag)?
            \item Was there a demonstration of effective control/occupation? -- \case{\textit{Clipperton Island Arbitration (France v Mexico)} 1932}
            \begin{enumerate}
                \item Was there a continuous and peaceful display of state authority?
                \item Was there a responsible authority that exercises governmental functions (\textit{effectivités})?
                \item Was the occupation's spatial extent, duration, continuity and peacefulness all of a sufficient degree?
            \end{enumerate}
            \item Did the claimant show that there was the exercise of a continuous and peaceful display of state authority over the territory?
            \begin{enumerate}
                \item This includes the presence of a responsible authority that exercises governmental functions
            \end{enumerate}
            \item Was the claimant a state actor or a private actor acting on behalf of a state organ?
            \begin{enumerate}
                \item Individuals may not acquire title in unoccupied land not claimed by a state -- \case{\textit{Ure v Commonwealth} [2016] FCAFC 8}
            \end{enumerate}
        \end{enumerate}
    \end{enumerate}
    \item Was there the prescription of territory?
    \begin{enumerate}
        \item Prescription refers to the acquisition of the title to territory formerly occupied by another state through a peaceful exercise of sovereignty for a period of time -- \case{\textit{Island of Palmas Case (Netherlands v US)} (1928)}; \case{\textit{Malaysia v Singapore} [2008] ICJ Rep 12}
        \item Discovery alone is insufficient; a positive intention to claim the territory must be shown -- \case{\textit{Island of Palmas Case (Netherlands v US)} (1928)}
        \item Are the criteria/elements for prescription met? -- \case{\textit{Kasikili/Sesdudu Island (Botswana v Namibia)} [1999] ICJ Rep 1045}
        \begin{enumerate}
            \item À titre de souverain (the possession has to be under state authority, and not by a private actor)
            \item Peaceful possession
            \begin{enumerate}
                \item The possession is not peaceful if it is challenged by another state, and hence no prescription can be found -- \case{\textit{US v Mexico} (1911)}
            \end{enumerate}
            \item The possession must be made public
            \item Uninterrupted
            \item Endure for a length of time
            \item Acquiescence by another state (meek protests by the original title holder or a failure to assert their time evinces an intention to surrender their title to the territory) -- \case{\textit{Malaysia v Singapore} [2008] ICJ Rep 12}
        \end{enumerate}
        \item Whether prescription has occurred takes into account two key factors:
        \begin{enumerate}
            \item Intention and will to act
            \begin{enumerate}
                \item An absence of reaction may well amount to acquiescence -- \case{\textit{Malaysia v Singapore} 2008} (e.g., no statement of rejection tends towards acceptance) 
            \end{enumerate}
            \item Actual exercise
            \begin{enumerate}
                \item Sovereignty can be shown through a display of state authority being open, public, continuous and peaceful (i.e., that the state has the intention to acquire that territory) -- \case{\textit{Island of Palmas Case (Netherlands v US)} (1928)}
                \item There is a high threshold -- \case{\textit{Land and Maritime Boundary between Cameroon and Nigeria (Cameroon v Nigeria)} [2002] ICJ Rep 303} (in this case, even though Cameroon (the initial territory holder)  had engaged in only occasional direct acts of administration, having limited material resource to devote to this distant area and hence could not `be viewed as an acquiescence in the loss' of the territory of Nigeria)
            \end{enumerate}
        \end{enumerate}
        \item To determine if prescription has occurred, the critical date must be decided, which is the the date `falling at the end of a period within which material facts a dispute is said to have occurred', and `after which the actions of the parties to a dispute can no longer affect the issue'
        \item \case{\textit{El Salvador v Honduras} [1992] ICJ Rep 351} provides some examples of the critical date:
        \begin{enumerate}
            \item When a new state emerged with boundaries determined by the \textit{uti possidetis} principle
            \item Arises from a tribunal decision
            \item Arises from a boundary treaty, from `acquiescence or recognition'
        \end{enumerate}
    \end{enumerate}
    \item Was there the cession, accretion, avulsion or conquest of the territory?
    \begin{enumerate}
        \item Cession is the voluntary transfer of sovereignty over territory from one state to another, whether by treaty or by consent
        \begin{enumerate}
            \item International law does not impose upon the parties any particular form of cession, but instead places an emphasis on the consent of the parties -- \case{\textit{Malaysia v Singapore} [2008]}
            \item Consent can either be through tacit agreement or through conduct -- \case{\textit{Malaysia v Singapore} [2008]}
            \item Consent by the state can be given through tacit or active recognition of a state's authority -- \case{\textit{Temple of Preah Vihear (Cambodia v Thailand)} [1962] ICJ Rep 6}
            \item There are two developments that restrict the availability of cession:
            \begin{enumerate}
                \item The VCLT now prevents the formation of treaties (including treaties of cession) that have been procured under force or by threat
                \item Cession may go against the principle of self-determination, as it may not reflect the will of the people in the territory (``any detachment by administering Power of Part of a non-self-governing territory, unless based on freely expressed and genuine will of people of the territory concerned is contrary to the right of self-determination'') -- \case{\textit{Legal Consequences of the Separation of the Chagos Archipelago from Mauritius in 1965} [2019] ICJ Rep 95}
            \end{enumerate}
        \end{enumerate}
        \item Accretion is the gain of physical territory through natural processes
        \item Avulsion is the loss of physical territory through natural processes
        \item Conquest is the forceful and unlawful acquisition of territory, and is illegal under international law -- \convention{\textit{UN Charter} Art 2(4)}
    \end{enumerate}
    \item Is there a dispute over territorial sovereignty?
    \begin{enumerate}
        \item Sovereignty is binary - a state either can or cannot have sovereignty over territory at a given point in time
        \item There is a presumption that the original claimer, by default, has a superior claim to title -- \case{\textit{Legal Status of Eastern Greenland (Norway v Denmark)} (1933) PCIJ}
        \item The ICJ will only take into account activities by the states that occurred before the dispute crystallised (the \textbf{critical date}) -- \case{\textit{Case Concerning Sovereignty over Pulau Ligitan and Pulau Spiadan (Indonesia v Malaysia)} [2002] ICJ Rep 625}; \case{\textit{Case Concerning Sovereignty Over Pedra Branca/Pulau Batu Puteh, Middle Rocks and South Ledge (Malaysia/Singapore)} [2008] ICJ Rep 12}
    \end{enumerate}
    \item Are maritime zones involved?
    \begin{enumerate}
        \item Maritime zones are governed by \convention{\textit{1982 UN Convention on the Law of the Seas (UNCLOS)}}
        \item \convention{\textit{UNCLOS}} prevails over any territorial sovereignty claims in respect of maritime zones which are inconsistent with its provisions to the extent of inconsistency -- \case{\textit{South China Sea Arbitration (Philippines v China)} [2016] PCA}
        \item Upon becoming a signatory to \convention{\textit{UNCLOS}}, any pre-existing claims are extinguished -- \case{\textit{South China Sea Arbitration (Philippines v China)} [2016] PCA}
        \item Is there a dispute over baselines?
        \begin{enumerate}
            \item Normal baselines follow the contours of the coast, and are the default baseline used -- \convention{\textit{UNCLOS} Art 5}
        \end{enumerate}
        \item Is there a dispute over internal waters?
        \begin{enumerate}
            \item A state has full sovereignty over its internal waters, and there is no right of innocent passage through internal waters -- \convention{\textit{UNCLOS} Art 8}
        \end{enumerate}
        \item Is there a dispute over the territorial sea?
        \begin{enumerate}
            \item The territorial sea extends 12 nautical miles from the baseline -- \convention{\textit{UNCLOS} Art 3}
            \item The state has territorial sovereignty in this area (and as such, it is inherently unclaimable), with very few exceptions (the most prevalent of which is the right of innocent passage through the territorial sea)
        \end{enumerate}
        \item Is there a dispute over rocks, islands or artificial islands?
        \begin{enumerate}
            \item \convention{\textit{UNCLOS}} is used to determine what constitutes as territory -- \case{\textit{South China Sea Arbitration (Philippines v China)} [2016] PCA}
            \item The normal baseline for measuring the breadth of the territorial sea is the low-water line along the coast -- \convention{\textit{UNCLOS} Art 5}
            \item In the case of islands situated on atolls or of islands that have fringing reefs, the baseline for measuring the breadth of the territorial sea is the seaward low-water line of the reef -- \convention{\textit{UNCLOS} Art 6}
            \item Rocks can be considered as islands if they can sustain human habitation or economic life of their own -- \convention{\textit{UNCLOS} Art 121}
            \item Rocks must remain unsubmerged at high tide to be considered as islands -- \convention{\textit{UNCLOS} Art 121}
        \end{enumerate}
        \item Is there a dispute over the contiguous zone?
        \begin{enumerate}
            \item The contiguous zone is the 12-24 nautical mile zone immediately following the territorial sea -- \convention{\textit{UNCLOS} Art 33(2)}
            \item In this zone, coastal states may only enforce the law against vessels in the matters of customs, tax, immigration and quarantine -- \convention{\textit{UNCLOS} Art 33(1)}
            \begin{enumerate}
                \item It is therefore not a zone of sovereignty, but rather a zone of enforcement jurisdiction
            \end{enumerate}
        \end{enumerate}
        \item Is there a dispute over the exclusive economic zone (EEZ)?
        \begin{enumerate}
            \item The EEZ extends up to 200 nautical miles from the baseline of a coastal state -- \convention{\textit{UNCLOS} Art 57}; \case{\textit{South China Sea Arbitration (Philippines v China)} [2016] PCA}
            \item A state can assert rights to all of the living and non-living resources (including everything in the soil, subsoil, water column, etc.), and everything on the surface (e.g., renewable energy resources) -- \convention{\textit{UNCLOS} Art 56(1)}
            \item A coastal state does not have sovereignty over the EEZ, but merely has rights over the resources and the ability to regulate them -- \convention{\textit{UNCLOS} Art 56}
            \begin{enumerate}
                \item Other states still have the rights of navigation; they just cannot use those resources -- \convention{\textit{UNCLOS} Art 58(1)}
            \end{enumerate}
        \end{enumerate}
        \item Is there a dispute over the continental shelf?
        \begin{enumerate}
            \item The continental shelf is an inherent seabed resources zone, with states having inherent ownership of everything on the seabed and the subsoil, as well a anything living on the seabed -- \convention{\textit{UNCLOS} Art 76(1)}
            \item The continental shelf extends to either 200 nautical miles from the baseline or 100 nautical miles from the 2500m isobath -- \convention{\textit{UNCLOS} Art 76(5)}
            \item The state must have applied for it, it is not automatically granted
            \item In the continental shelf, the sovereign state may explore and exploit its natural resources -- \convention{\textit{UNCLOS} Art 77(1)}
        \end{enumerate}
        \item Is there a dispute over the high seas?
        \begin{enumerate}
            \item The high seas are non-appropriable areas of the sea not subject ot the jurisdiction of any nation, and remain open -- \convention{\textit{UNCLOS} Art 87}; \convention{\textit{UNCLOS} Art 89}
            \item The high seas may only be used for peaceful purposes -- \convention{\textit{UNCLOS} Art 88}
            \item All states may sail their ships in the high seas, whether coastal or land-locked -- \convention{\textit{UNCLOS} Art 90}
        \end{enumerate}
        \item Is this a dispute over the archipelagic waters?
        \begin{enumerate}
            \item These refer to waters within an archipelago (a group of islands) -- \convention{\textit{UNCLOS} Art 46}
            \item Archipelagic baselines are drawn such that they encompass the outermost points of the outermost islands, up to a limit of 100 nautical miles -- \convention{\textit{UNCLOS} Art 47}
            \item These waters are held to be similar to the territorial sea, and states have sovereignty over the archipelagic waters -- \convention{\textit{UNCLOS}Art 49}
        \end{enumerate}
        \item Is there a dispute over the deep seabed (the Area)?
        \begin{enumerate}
            \item This cannot be appropriated by any state, and is managed by the International Seabed Authority (as established under \convention{\textit{UNCLOS} Art 156-157})
        \end{enumerate}
    \end{enumerate}
    \item Is Antarctica, airspace or outer space involved?
    \begin{enumerate}
        \item Is Antarctica involved?
        \begin{enumerate}
            \item No new claims to territory in Antarctica can be made, but existing claims are not renounced and are instead `frozen' -- \convention{\textit{1959 Antarctic Treaty} Art 4}
            \item There are 7 claimants to territory in Antarctica: Argentina, Australia, Chile, France, New Zealand, Norway, and the UK
        \end{enumerate}
        \item Is airspace involved?
        \begin{enumerate}
            \item A state has sovereignty over the airspace above its territory and territorial sea
            \item Other states have the freedom of overflight over the contiguous zone, the EEZ and the high seas
            \item The boundary between national airspace and outer space is not clearly defined
        \end{enumerate}
        \item Is outer space involved?
        \begin{enumerate}
            \item Outer space (including the moon and other celestial bodies) remains the province of mankind, and any exploration/use shall be carried out for the benefit of all countries -- \convention{\textit{1967 Outer Space Treaty} Art 1}
            \begin{enumerate}
                \item There are 113 parties to this treaty
            \end{enumerate}
            \item Outer space is not subject to national appropriation or claims of sovereignty by any means -- \convention{\textit{1967 Outer Space Treaty} Art 2}
        \end{enumerate}
        \item Is the moon or another celestial body involved?
        \begin{enumerate}
            \item The moon and other celestial bodies are the common heritage of mankind -- \convention{\textit{1979 Moon Agreement Art} 11(1)}
            \begin{enumerate}
                \item There are only 18 parties to this treaty, and aside from the \convention{\textit{1967 Outer Space Treaty}}, no other forms of state practice exist in this area
            \end{enumerate}
            \item No state can make a claim in sovereignty over the moon or a celestial body -- \convention{\textit{1979 Moon Agreement} Art 11(2)}
            \item States may explore and use the moon in a manner that doesn't prejudice the rights of other states -- \convention{\textit{1979 Moon Agreement} Art 11(4)}
        \end{enumerate}
    \end{enumerate}
\end{enumerate}

\section{State Jurisdiction}

\section{Immunity from Jurisdiction}
This includes the topics `Immunity from Jurisdiction I' and `Immunity from Jurisdiction II'.

\section{State Responsibility I (Wrongful Acts of State)}

\section{State Responsibility II (Diplomatic Protection)} 

\section{Use of Force}
\begin{enumerate}
    \item Was there a use of force?
    \begin{enumerate}
        \item All member states of the UN must settle their disputes by peaceful means -- \convention{\textit{UN Charter} Art 2(3)}
        \begin{enumerate}
            \item This is customary international law, and so all states must abide by this provision -- \case{\textit{Nicaragua v US} [1986] ICJ Rep 14}
        \end{enumerate}
        \item The use of force is prohibited under international law -- \convention{\textit{UN Charter} Art 2(4)}
        \begin{enumerate}
            \item This has the purpose of ``saving succeeding generations from the scourge of war" - \convention{\textit{UN Charter} Preamble}
            \item This is a customary norm of international law, and so is binding on parties even if they are not part of the UN -- \case{\textit{Nicaragua Case} (1968)}
            \item This is moreover a `cornerstone of the United Nations Charter' -- \case{\textit{Armed Activities (DRC v Uganda)} (2005)}
            \item The prohibition includes the use of force by a state, or by paramilitary or irregular forces acting on behalf of a state -- \convention{\textit{Declaration on Friendly Relations 1970} (Res 2625) Principle 1}
            \begin{enumerate}
                \item This declaration reflects customary international law -- \case{\textit{Nicaragua Case} (1986) at [188], [191]}; \case{\textit{Armed Activities} (2005) at [162], [300]}
                \item The use of force includes the threat or use of force to violate the existing international boundaries of another state or as a means of solving international disputes, including territorial disputes and problems concerning frontiers of states 
                \item The use of force includes organising or encouraging the organisation of irregular forces or armed bands, including mercenaries, for incursion into the territory of another state
                \item The use of force includes organising, instigating, assisting or participating in acts of civil strife or terrorist acts in another state or acquiescing in organised activities within its territory directed towards the commission of such acts that involve the use of force 
            \end{enumerate}
            \item State-sanctioned terrorism is prohibited -- \convention{\textit{Declaration on Friendly Relations 1970} (Res 2625) Principle 1}
            \item Aggression, which is the use of armed force by a State against another state's sovereignty, territorial integrity or political independence, is prohibited -- \convention{\textit{Resolution on Definition of Aggression} 1974 Art 1}
            \begin{enumerate}
                \item A non-exhaustive list of acts that can constitute aggression is provided in \convention{\textit{Resolution on Definition of Aggression 1974} Art 3}, and includes the invasion, attack, occupation, annexation, bombardment, or blockade of another state by armed force; allowing territory to be used for perpetrating aggression against a third state; the sending by or behalf of a state of armed bands, groups, irregulars or mercenaries, which carry out acts of armed force against another state of such gravity listed above or its substantial involvement therein
                \item This Declaration reflects customary international law - \convention{\textit{Nicaragua} (1986) at [195]}; \convention{\textit{Armed Activities} (2005) at [146]}
            \end{enumerate}
        \end{enumerate}
        \item The application of the use of force is not limited to matters concerning the ``territorial integrity or political independence of any state'', as these are just specific examples that elaborate on the general prohibition of the use of force -- \textit{Brownlie, International Law and the Use of Force by States} (1963) at 366
        \begin{enumerate}
            \item However, it can be used as a limitation:
            \begin{enumerate}
                \item E.g., minesweeping operations in the waters of another state (1948, by the UK Navy within the Albanian Territorial Sea in the Corfu Channel)
                \item Forcible invasion by a country to rescue their citizens (e.g., an Israeli intervention in the 1976 Entebbe incident to rescue Israeli citizens on a hijacked airliner)
            \end{enumerate}
        \end{enumerate}
        \item Does one of the recognised exceptions to the use of force apply?
        \begin{enumerate}
            \item Self-defence -- \convention{\textit{UN Charter} Art 51} (this must be permitted by the UN Security Council at the first available opportunity)
            \item Collective security -- \convention{\textit{UN Charter} Chapter VII}
            \item Governmental consent to the use of force in their territory
        \end{enumerate}
        \item Even if the use of force cannot be established, there is at least an unlawful intervention in the sovereignty of another state -- \case{\textit{Nicaragua} (1986) at [224]}
    \end{enumerate}
    \item Was there an armed attack?
    \begin{enumerate}
        \item The unlawful use of force by one state against another constitutes armed force
        \item Was there a grave use of force? -- \case{\textit{Nicaragua Case} (1986) at [191]}; \case{\textit{Oil Platforms} (1984) at [51]}
        \item Were the scale and effects of the operation sufficient such as to constitute an armed attack? -- \case{\textit{Nicaragua} (1986) at [195]}
        \item Examples of armed attacks include:
        \begin{enumerate}
            \item The placing of mines on a single military vessel may amount to an armed attack -- \case{\textit{Oil Platforms} (2003) at [72]}
            \begin{enumerate}
                \item This is not the case for a consumer vessel, as military vessels are integral to the security of a country
            \end{enumerate}
            \item Capturing several towns near a country's border (given the massive scale and effects of this) -- \case{\textit{Armed Activities} (2005) at [110], [147]}
            \item Cybersecurity attacks, as cyber force can amount to non-kinetic uses of force in certain situations, provided they meet six criteria: -- \article{\textit{The Use of Cyber Force in International Law}, Michael Schmitt}
            \begin{enumerate}
                \item Severity
                \item Immediacy
                \item Directness
                \item Invasiveness
                \item Measurability of Effects
                \item State involvement
                \item Presumptive legality
            \end{enumerate}
        \end{enumerate}
        \item Armed attacks can be conducted by armed bands, irregulars, mercenaries, etc. acting on behalf of a state -- \case{\textit{Nicaragua} (1986) at [195]}; \convention{\textit{Resolution on Definition of Aggression 1974} Art 3(g)}
        \item Was there a use of direct armed force?
        \begin{enumerate}
            \item Invasion of a territory, missile attacks, the laying of mines, etc. all constitute a use of direct armed force -- \convention{\textit{Declaration on Friendly Relations 1970} (GA Res 2625)}
            \begin{enumerate}
                \item The laying of landmines constitutes direct armed force and an armed attack -- \case{\textit{Nicaragua} (1986) at [195]}
            \end{enumerate}
        \end{enumerate}
        \item Was there a use of indirect armed force?
        \begin{enumerate}
            \item Did the state have effective control of the irregular band?
            \begin{enumerate}
                \item The conduct of private individuals will be attributable to a state ``if the state directed or controlled the specific operation and conduct complained of was an integral part of the operation" -- \convention{\textit{ILC RSIWA Draft Article 8 Commentary} at [3]}; \case{\textit{Nicaragua} (1986) at [195]}
                \item Irregular bands includes mercenaries, private armies, third parties, etc.
            \end{enumerate}
            \item Sending `armed bands' (irregular forces) into another state's territory -- \case{\textit{Nicaragua} (1986) at [195], [247]}
            \item `Actively extending military, logistic, economic and financial support to irregular forces' (so long as it is directed to the war-fighting capacity of a state) -- \case{\textit{Armed Activities} (2005) at [161]-[165]}
            \item Providing weapons, logistical and/or other support to armed insurgents -- \case{\textit{Nicaragua} (1986) at [195], [205], [247], [251]}
            \end{enumerate}
            \item The `mere supply of funds' does not constitute a use of force -- \case{\textit{Nicaragua} (1986) at [228]}
            \item Armed attacks do not include:
            \begin{enumerate}
                \item Mere frontier incidents -- \case{\textit{Nicaragua} (1986) at [195]}
                \begin{enumerate}
                    \item These are minor disputes wherein the conflict only lasts for a few hours or days
                \end{enumerate}
                \item Armed incursions where the victim did not raise any complaints -- \case{\textit{Nicaragua} (1986) at [231]-[234]}
                \begin{enumerate}
                    \item Since the receiving state(s) did not complain about the incursion, it is presumed that they were not serious enough to amount to an armed attack
                \end{enumerate}
                \item Providing assistance to rebels or insurgents in the form of weapons, logistical or other support -- \case{\textit{Nicaragua} (1986) at [195], [230]}
            \end{enumerate}
            \item Armed attacks generally need to be directed at a state -- \case{\textit{Oil Platforms} (2003) at [64]}
            \begin{enumerate}
                \item Here, even if taken cumulatively, a missile attack, the laying of mines on a ship and then firing on a US ship by Iran did not amount to an armed attack, as there was no evidence that the attacks were aimed specifically at the US
            \end{enumerate}
            \item Localised border encounters between small infantry units, even those involving the loss of life, do not constitute armed attacks -- \case{\textit{Ethiopia v Eritrea} (2006) at [11]}
            \item An armed attack generally does not begin until a state's territory is affected, but it is possible for it to begin when an irreversible course of action (e.g., the launch of missiles) has begun
            \begin{enumerate}
                \item An exception may lie in the `accumulation of events' theory, where an armed attack arises from a number of armed attacks viewed as a whole; this theory has been implicitly accepted by the ICJ - \case{\textit{Nicaragua} (1986) at [231]}; \case{\textit{Oil Platforms} (2003) at [64]}; \case{\textit{Armed Activities} (2005) at [146]}
            \end{enumerate}
    \end{enumerate}
    \item Was there the threat of the use of force?
    \begin{enumerate}
        \item Since the use of armed force is prohibited under \convention{\textit{UN Charter} Art 2(4)}, the threat of the use of force is also prohibited -- \case{\textit{Nuclear Weapons Advisory Opinion} (1996) at [47]}
        \item The threat must be specifically directed at a state to constitute a threat of force
        \begin{enumerate}
            \item Military manoeuvres (e.g., positioning troops, aiming artillery/weapons) near a country's border do not necessarily constitute a threat of the use of force -- \case{\textit{Nicaragua} (1986) at [227]}
            \item An accumulation of forces can possibly constitute a threat, depending on its accompaniments (statement of intent, relevant exercises, etc.) -- \case{\textit{Nicaragua} (1986)}
        \end{enumerate}
    \end{enumerate}
    \item Can a state raise self-defence?
    \begin{enumerate}
        \item The right of self-defence is preserved only until the UN Security Council has taken the necessary measures to maintain international security and peace -- \convention{\textit{UN Charter} Art 51}
        \item Measures taken in self-defence must be reported to the UN Security Council at the earliest possible moment -- \convention{\textit{UN Charter} Art 51}
        \item Is the act of a state in response to a continuing attack or a past attack?
        \begin{enumerate}
            \item A state cannot use self-defence in response to a past attack or an unlawful retaliation
            \item A state cannot use self-defence if the attacks have stopped, as self-defence is only available to respond to an ongoing attack
        \end{enumerate}
        \item Was it anticipatory self-defence?
        \begin{enumerate}
            \item The ICJ has expressly reserved its opinion around anticipatory self-defence, and so there is not much jurisprudence to guide this notion -- \case{\textit{Nicaragua} (1986) at [194]}; \case{\textit{Armed Activities} (2005) at [143]}
            \item Anticipatory self-defence arises when self-defence is carried out on the basis that armed attacks are about to occur (e.g., a missile is coming)
            \item It is unclear as to whether states can practice self-defence against an imminent attack -- \case{\textit{Nicaragua} (1986) at [195]}
        \end{enumerate}
        \item Was it pre-emptive self-defence?
        \begin{enumerate}
            \item This notion has generally been denied in jurisprudence
            \item Pre-emptive self-defence operates on the basis that an armed attack will eventually occur, and so it is wise to reduce or disable the opposition's military strength
            \begin{enumerate}
                \item Arguments in favour of this approach include the changing nature of threats since 1945, that it is impractical to require states to wait until an armed attack has occurred, and that the UN Security Council is often unable to act in a timely manner, or act at all due to the veto of one of its five permanent members
                \item Arguments against this approach include that it is too vague and arbitrary, that it is unable to assess necessity and proportionality, and it may erode the prohibition on the use of force
            \end{enumerate}
        \end{enumerate}
        \item The attacked state must be a member of the UN -- \case{\textit{Ethiopia v Eritrea} (2006) at [11]}
        \item Was there an armed attack?
        \begin{enumerate}
            \item See the above; an armed attack is necessary to raise self-defence
            \item The `scale and effects of the operation' of the attacking state must be grave enough to constitute an armed attack -- \case{\textit{Nicaragua} (1986) at [195]} (e.g., a high degree of casualties, destruction of property, etc.)
            \item It is possible for an operation to be a use of force contrary to \convention{\textit{UN Charter Art} 2(4)}, but not an armed attack and hence not something that can be responded to by way of self-defence -- \case{\textit{Nicaragua} (1986) at [191]}; \convention{\textit{Oil Platforms} (2003) at [51], [61]}
            \item The right to self-defence is only enlivened by an armed attack, which is a most grave use of force assessed on the `scale and effects' of the operation -- \case{\textit{Nicaragua} (1986) at [195]}; \case{\textit{Armed Activities} (2005) at [146]}
        \end{enumerate}
        \item The response of the state must be ``necessary and proportionate to the armed attack and necessary to respond to it" -- \case{\textit{Nicaragua} (1986) at [176]}
        \begin{enumerate}
            \item Necessity means that there are no alternative means of repelling the attack
            \begin{enumerate}
                \item An act done in necessity is an almost in instant response and which has been done instinctively -- \case{\textit{Caroline Case} (1841/1842)}
            \end{enumerate}
            \item Proportionality means that the response must not be excessive in relation to the armed attack, and must only be for the purpose of repelling the attack (not counter-attacking)
            \begin{enumerate}
                \item Self-defence is only lawful if it is necessary and proportionate to the purpose of repelling an armed attack as a matter of customary international law -- \case{\textit{Legality of the Threat or Use of Nuclear Weapons Advisory Opinion} (1996) at [41]}; \textit{Nicaragua} (1986) at [194]
                \item The response must repel the attack, not seek to root out the source of the attack
            \end{enumerate}
        \end{enumerate}
        \item The capture of airports and towns of many hundreds of kilometres from the border is not proportionate or necessary, and cannot constitute self-defence -- \case{\textit{Armed Activities} (2005)}
        \item Attacking towns and cities near the border is not proportionate to a border attack and so cannot constitute self-defence -- \case{\textit{Armed Activities} (2005) at [147]}
        \item A missile attack on a civilian vessel does not meet the scales and effects for an armed attack, and so responding by attacking oil platforms (a government installation) is not self-defence, as it is neither necessary nor proportionate to the attack on the civilian vessel -- \case{\textit{Oil Platforms} (2003) at [64], [72]}
    \end{enumerate}
    \item Did the attacked state consent to the use of force?
    \begin{enumerate}
        \item Did the state positively consent to give the other state a right to intervene, without coercion? -- \case{\textit{Armed Activities} (2005)} (the DRC requested assistance from Uganda and Rwanda to help it quell rebel activity)
        \item Where there is an internal conflict that has reached the threshold of civil war, a state cannot consent to external intervention, and the attack will be unlawful
    \end{enumerate}
    \item Was there a terrorist attack?
    \begin{enumerate}
        \item Were the scales and effects of the terrorist attacks sufficient to constitute an armed attack? -- \case{\textit{Nicaragua} (1986) at [195]}
        \item Can the acts be attributed to a state?
        \begin{enumerate}
            \item It is the majority view that the attack by the terrorist must be attributable to the state -- \case{\textit{Israeli Wall} (2004)}
            \item The minority view is that no state attribution is possible, and self-defence against non-state actors might be permitted, especially when there is no real governmental authority in the state of the terrorist, and as such, no attribution is possible -- \case{\textit{Armed Activities} (2005) (Kooijmans and Simma JJ)}
            \begin{enumerate}
                \item Moreover, where a state is unwilling or unable to prevent its territory from being used to harbour terrorist groups or from being used to launch cross-border attacks against other states, this gives the victim state the right to use force within the territory of that unwilling/unstable state against the non-state group -- \article{\textit{UNSC Resolutions 1368 and 1373}} (passed post-9/11, but state practice on this area is largely undetermined) 
            \end{enumerate}
        \end{enumerate}
    \end{enumerate}
    \item Is there a need for collective security?
    \begin{enumerate}
        \item The UN Security Council has broad powers to take action necessary to respond to threats of the peace, breaches of the peace and acts of aggression
        \item The UN Security Council needs to believe that there is such a threat to or breach of global peace -- \convention{\textit{UN Charter} Art 39}
        \item The UNSC should first attempt measures that do not involve the used of armed force --  \convention{\textit{UN Charter} Art 41}
        \item If the previous measures have failed, then the UNSC can authorise armed force by member states if it believes it is necessary --  \convention{\textit{UN Charter} Art 42}
        \item The UNSC can take action so long as the veto is not exercised by at least one of its five permanent members
    \end{enumerate}
    \item Is there a need for humanitarian intervention?
    \begin{enumerate}
        \item This is an exception to the prohibition on the use of force, and allows another state to protect the nationals of that state from extreme cruelty or persecution
        \item Force should be used only as a very last resort, with there being `no better or more appropriate authority than the United Nations Security Council to authorise military intervention for humanitarian protection purposes' - \article{\textit{2001 International Commission on Intervention and State Responsibility}}
        \item The UK's legal position on the application of humanitarian intervention is that if an action in the UNSC is blocked, the UK would still be permitted to take exceptional measures in order to alleviate the scale of the overwhelming humanitarian catastrophe (e.g., in Syria, by deterring and disrupting further use of chemical weapons by the Syrian regime)
        \begin{enumerate}
            \item A legal basis for justifying humanitarian intervention is:
            \begin{enumerate}
                \item There is convincing evidence of extreme humanitarian distress on a large scale, requiring immediate and urgent relief
                \item It must be objectively clear that there is no practicable alternative to the use of force
                \item The proposed use of force must be necessary and proportionate to the aim of relief of humanitarian needs and strictly limited in time and scope of this aim
            \end{enumerate}
            \item Humanitarian intervention is a narrow and limited test, and there is not widespread state practice and support for this, so its persuasiveness is questionable
        \end{enumerate}
    \end{enumerate}
\end{enumerate}

\section{International Dispute Settlement}\label{scaffold:Topic 13}
\begin{enumerate}
    \item Is there a dispute?
    \begin{enumerate}
        \item A dispute is a disagreement on a point of law or fact; a conflict of legal views or interests between two persons -- \case{\textit{Movrommatis Palestine Concessions} [1924] at [11]}
        \begin{enumerate}
            \item Whether there exists an international dispute is a matter for objective determination
        \end{enumerate}
        \item The mere denial of a dispute does not prove non-existence of a dispute -- \case{\textit{Interpretation of Peace Treaties with Bulgaria, Hungary and Romania} [1950] at [74]}
    \end{enumerate}
    \item Is there a non-judicial means of dispute resolution available?
    \begin{enumerate}
        \item Is negotiation available?
        \begin{enumerate}
            \item Negotiation is the lowest-stakes method of international dispute settlement, and is the most common form attempted by states
            \item Negotiation may sometimes be a procedural precondition for the jurisdiction of an international court (i.e., negotiation must take place before judicial settlement occurs) -- \case{\textit{Ukraine v Russia} [2019] ICJ Rep 558}
            \begin{enumerate}
                \item Discussions at international organisations/fora are sufficient to satisfy this requirement - \case{\textit{South West Africa} [1962] ICJ 319}
            \end{enumerate}
        \end{enumerate}
        \item Is mediation available?
        \begin{enumerate}
            \item Mediation involves a third party (the mediator) who assists the parties in reaching a settlement
            \item The parties remain in control over the process, and set the scope for the mediator's involvement
        \end{enumerate}
        \item Is an inquiry available?
        \begin{enumerate}
            \item An inquiry involves an objective assessment of the evidence and the finding of facts -- \convention{\textit{1899 Hague Convention for the Pacific Settlement of International Disputes} Art 9}; \case{\textit{Dogger Bank Inquiry} (1904)}
            \item Inquiries are to be established by special agreement between the conflicting parties -- \convention{\textit{1899 Hague Convention for the Pacific Settlement of International Disputes} Art 10}
            \item Inquiries are limited to statements of fact, and are to be provided with all resources as required -- \convention{\textit{1899 Hague Convention for the Pacific Settlement of International Disputes} Art 11-14}
        \end{enumerate}
        \item Is conciliation available?
        \begin{enumerate}
            \item Conciliation involves a third party who considers legal and non-legal factors to recommend terms of settlement, and is an option for dispute resolution -- \convention{\textit{1948 Pact of Botogá} Articles XV-XXX}
            \item For parties to the \convention{\textit{General Act for the Pacific Settlement of International Disputes}}, there must be compulsory conciliation if the dispute is not resolved by diplomacy -- \convention{\textit{1928 General Act for the Pacific Settlement of International Disputes} Art 1-15}
            \item If this is a matter involving \convention{\textit{UNCLOS}}, provisions for conciliation exist -- \case{\textit{Australia-Timor Leste Conciliation} (2016)}
        \end{enumerate}
        \item Is arbitration available?
        \begin{enumerate}
            \item Arbitrations involve the application of international law to resolve deadlocks by creating binding awards
            \item Arbitration can either be:
            \begin{enumerate}
                \item Ad-hoc inter-state (i.e., established only when there is conflict between states) -- \case{\textit{South China Sea Arbitration} (2016)}
                \item Institutional inter-state (i.e., it is a standing, go-to method of dispute resolution)
                \item Individual/corporation vs state (this is generally under bilateral investment treaties, where arbitration allows for disputes between corporations and states to be resolved -- \case{\textit{Philip Morris v Australia} (2015)} (under the Hong-Kong Australia BIT))
            \end{enumerate}
        \end{enumerate}
    \end{enumerate}
    \item Have the preconditions to the ICJ's jurisdiction been satisfied?
    \begin{enumerate}
        \item States must use other methods of dispute settlement to attempt to resolve the dispute in a peaceful manner before raising the dispute to a judicial body -- \convention{\textit{UN Charter} Art 2(3)}; \convention{\textit{UN Charter} Art 33(1)}
        \begin{enumerate}
            \item States must have refrained from the threat or use of force -- \convention{\textit{UN Charter} Art 2(4)}
            \item The UNSC will call upon the parties to settle their disputes by such means whenever necessary -- \convention{\textit{UN Charter} Art 33(2)}
        \end{enumerate}
        \item If negotiation has been attempted, is there a bar to jurisdiction?
        \begin{itemize}
            \item The precondition of negotiation for the ICJ's jurisdiction is met only when there has been a failure of negotiations, or when negotiations have become futile or resulted in a deadlock -- \case{\textit{Russia v Ukraine} [2019] ICJ Rep 558}
            \begin{enumerate}
                \item Discussions at international organisations/fora are sufficient to satisfy this requirement -- \case{\textit{South West Africa} [1962] ICJ 319}
            \end{enumerate}
        \end{itemize}
    \end{enumerate}
    \item Does the ICJ have jurisdiction?
    \begin{enumerate}
        \item Only member states of the UN can be in cases heard before the ICJ -- \statute{\textit{ICJ Statute} Art 34(1)}
        \begin{enumerate}
            \item All member states of the UN are automatically parties to the ICJ Statute, and thus can be parties to cases before the ICJ -- \statute{\textit{ICJ Statute} Art 83(1)}
        \end{enumerate}
        \item Judgements of the ICJ are binding only on the parties to the dispute -- \statute{\textit{ICJ Statute} Art 59}
        \begin{enumerate}
            \item However, judgements may influence the development of international law -- \statute{\textit{ICJ Statute} Art 38(1)(d)}
        \end{enumerate}
        \item There is a fundamental difference between a violation of international law and the jurisdiction of the ICJ; just because thee is a breach does not mean that the ICJ has jurisdiction over it -- \case{\textit{Democratic Republic of the Congo v Uganda} [2005] ICJ Rep 168}
        \item Contentious jurisdiction of the ICJ is present either by \textit{compromis}, compromissory clause, or compulsory jurisdiction, demonstrating state consent to be bound to the ICJ's jurisdiction, which is essential for its jurisdiction
        \begin{enumerate}
            \item \textit{Compromis} jurisdiction is an agreement between the parties to submit a dispute to the ICJ for resolution -- \statute{\textit{ICJ Statute} Art 36(1)}
            \begin{enumerate}
                \item The states will identify the issues they wish for the ICJ to decide upon
                \item The states can specify the rules/parameters that the ICJ should apply in deciding the dispute
            \end{enumerate}
            \item A compromissory clause is a clause in a treaty that provides for the ICJ to have jurisdiction over disputes arising from the treaty, with this being very clear from the text of the treaty -- \statute{\textit{ICJ Statute} Art 36(1)}
            \item The ICJ's compulsory jurisdiction arises on the basis of a state having accepted the ICJ's jurisdiction in advance; hence, it is known as jurisdiction under the optional clause (given that there is no requirement to submit to it) -- \statute{\textit{ICJ Statute} Art 36(2)}
            \begin{enumerate}
                \item States may give their consent through a declaration accepting the ICJ's jurisdiction
                \item Declarations can be made with reservations/conditions -- \statute{\textit{ICJ Statute} Art 36(3)}; \case{\textit{Nicaragua v US} [1984] ICJ Rep 392 at [59]}
                \item A party may rely on the reservation of the other party to dispute the ICJ's jurisdiction -- \case{\textit{Whaling in the Antarctic} [2014] ICJ Rep 226}
                \item A declaration of compulsory jurisdiction is only effective if both states have made such a declaration and both states accept the ICJ's jurisdiction over the present dispute
                \item Reservations made by a state are reciprocal, and can be used against them, with the ICJ taking only the narrowest approach to jurisdiction (e.g., if a state accepts the ICJ's jurisdiction over X + Y, and another state accepts it over Y and Z, the ICJ's jurisdiction in a dispute between the parties would apply to Y ony) -- \case{\textit{Nicaragua v United States} [1984] ICJ Rep 392}; \case{\textit{Norwegian Loans Case} [1957] ICJ Rep 9}; \case{\textit{Interhandel Case} [1959] ICJ Rep 6}
            \end{enumerate}
            \item Jurisdiction of the ICJ may be established under \textit{forum prorogatum}, which is the agreement of the parties to submit a dispute to the ICJ after the dispute has arisen, and thus is a form of \textit{compromis} jurisdiction -- \statute{\textit{ICJ Statute} Art 38(5)}
            \item The PCIJ's jurisdiction can be transferred to the ICJ -- \statute{\textit{ICJ Statute} Art 36(5)}
        \end{enumerate}
        \item Even if a state does not consent to the ICJ's jurisdiction, the parties still need to resolve their dispute by peaceful means -- \statute{\textit{ICJ Statute} Art 33}; \case{\textit{Federal Republic of Yugoslavia v NATO States} [1999] ICJ Rep 124}
        \item The states must agree that there is a dispute for the ICJ to have jurisdiction -- \case{\textit{Marshall Islands Case} [2016] ICJ Rep 883} (note Crawford J's dissent, raise that this theory may be flawed, especially since this case was tied and decided on the President's casting vote)
        \begin{enumerate}
            \item The test for whether a dispute exists is a matter for objective determination by the court
        \end{enumerate}
    \end{enumerate}
    \item Is the dispute admissible to the ICJ?
    \begin{enumerate}
        \item A dispute is not admissible if it is moot or hypothetical (e.g., the nuclear tests case was not admissible as nuclear testing had ceased by them) -- \case{\textit{Legality of the Threat or Use of Nuclear Weapons} [1996] ICJ Rep 254}
        \item A dispute is not admissible if it is not justiciable; it must contain some legal question and cannot be purely political -- \case{\textit{Legal Consequences of the Construction of a Wall in Occupied Palestinian Territory} [2004] ICJ 136}
        \item Does the applicant have standing?
        \begin{enumerate}
            \item Was there an injury?
            \item Is the nationality of claims rule satisfied? (Topic \ref{sec:Topic 11})
            \item Is the rule on the exhaustion of local remedies satisfied? (Topic \ref{sec:Topic 11})
            \item Is the dispute moot or hypothetical?
        \end{enumerate}
        \item Is there an indispensable third party that is not present? -- \case{\textit{Monetary Gold} [1945] ICJ Rep 19}; \case{\textit{East Timor} [1955] ICJ Rep 6}
        \begin{enumerate}
            \item An indispensable third party is one whose rights would be affected by the judgment of the ICJ
            \item The third party might be a vital subject matter to the case -- \case{\textit{Monetary Gold} [1945] ICJ Rep 19}
            \item A vital third party might have its rights and obligations significantly impacted by the subject matter and judgement of the dispute -- \case{\textit{East Timor} [1955] ICJ Rep 6}
            \item However, it is insufficient that there are legal implications on a third party; a third party's legal interests must be impacted, or they must otherwise be vital to the court's decision -- \case{\textit{Nauru v Australia} [1992]}
        \end{enumerate}
    \end{enumerate}
    \item Can another party intervene into the dispute?
    \begin{enumerate}
        \item A state can intervene into a case regarding their interest if: -- \case{\textit{El Salvador v Honduras (Nicaragua Intervening)} [1992] ICJ Rep 351 at [58]}
        \begin{enumerate}
            \item It has an interest of a legal nature that \textit{may} be affected by the decision in the case
            \item It can satisfy the burden of proof of proving their legal interest
            \item It can provide a clear identification that their legal interest will be affected by the decision in the case (an apprehension is not sufficient)
        \end{enumerate}
        \item A state that is party to a case cannot determine whether a third state can intervene into the dispute -- \case{\textit{El Salvador v Honduras (Nicaragua Intervening)} [1992] ICJ Rep 351 at [96]}
        \begin{enumerate}
            \item Only the ICJ can determine whether a third party can intervene into the dispute
            \item Power has already been given to the court by consent - if the existing parties have given the ICJ consent to hear the matter, the court then has the competence to permit an intervention even if it is opposed by one or both of the parties to the matter
        \end{enumerate}
    \end{enumerate}
    \item Can the ICJ issue provisional measures?
    \begin{enumerate}
        \item The competence of the ICJ to issue provisional measures is distinct from jurisdiction, and is not derived from the consent of the parties to the dispute
        \item The ICJ can issue provisional measures if there is the imminent risk of irreversible harm before a dispute can be heard -- \statute{\textit{ICJ Statute Article} 41(1)}; \case{\textit{Application of the Convention on the Prevention and Punishment of Crime of Genocide in the Gaza Strip} [2024] ICJ Rep 1}
        \item Provisional measures are legally binding, and give rise to an independent legal obligation -- \case{\textit{La Grand} [2001] ICJ Rep 466}
        \begin{enumerate}
            \item Failure to comply with provisional measures constitutes grounds for a claim by an affected state (i.e., it is a wrongful state act) -- \case{\textit{La Grand} [2001] ICJ Rep 466}
        \end{enumerate}
        \item It is currently unsettled whether the ICJ is empowered to judicially review decisions of the UNSC -- \case{\textit{Libya v US; Libya v UK} [1992] 44}
        \item Requirements for the invocation of provisional measures are:
        \begin{enumerate}
            \item \Gls{prima facie} jurisdiction, which indicates that the condition of jurisdiction needs not be fully satisfied, so long as it is apparent that the ICJ has jurisdiction over the dispute -- \case{\textit{South Africa v Israel} [2024] ICJ Rep 1}
            \item There must be a plausible claim of rights that needs to be protected -- \case{\textit{South Africa v Israel} [2024] ICJ Rep 1}
            \item The claim must be urgent -- \case{\textit{Georgia v Russia} [2011] at [143]}
            \item Failure to grant the provisional measure would cause irreparable harm to the rights of the applicant if provisional measures are not granted -- \case{\textit{South Africa v Israel} [2024] ICJ Rep 1}
        \end{enumerate}
    \end{enumerate}
    \item Is the ICJ operating in its advisory capacity?
    \begin{enumerate}
        \item The ICJ can provide an advisory opinion on any question at the request of authorised UN bodies
        \item Advisory opinions are not binding
        \item The court must have compelling reasons for refusing to give an advisory opinion, not merely because the legal question is difficult or unsettled -- \case{\textit{Kosovo Advisory Opinion} [2010] ICJ Rep 403 at [39]-[40]}; \case{\textit{Legal Consequences of the Construction of a Wall in the Occupied Palestinian Territory Advisory Opinion} [2004] ICJ Rep 136}
        \begin{enumerate}
            \item The fact that a legal question has political aspects is not a compelling reason to decline a request for an advisory opinion -- \case{\textit{Legal Consequences of the Construction of a Wall in the occupied Palestinian Territory Advisory Opinion} [2004] ICJ Rep 136}
            \item Whether a separate UN organ is a more appropriate body to request the advisory opinion is not a compelling reason to decline a request for an advisory opinion -- \case{\textit{Kosovo Advisory Opinion} [2010] ICJ Rep 403 at [39]-[40]}
        \end{enumerate}
        \item Subjects of advisory opinions include:
        \begin{enumerate}
            \item Questions of UN law -- \case{\textit{Certain Expenses of the United Nations} [1962] ICJ Rep 151}
            \item Questions around decolonisation -- \case{\textit{Chagos Islands} [2019] ICJ Rep 95}
            \item Questions around unlawful occupation -- \case{\textit{Legal Consequences of the Construction of a Wall in the occupied Palestinian Territory Advisory Opinion} [2004] ICJ Rep 136}
            \item Questions around climate change -- \case{\textit{Obligations of States in Respect of Climate Change}} [pending]
        \end{enumerate}
    \end{enumerate}
\end{enumerate}