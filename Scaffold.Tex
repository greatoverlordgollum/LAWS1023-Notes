\section{Development, Nature and Scope of Public International Law}

\section{Sources of Public International Law}
\begin{enumerate}
    \item Is the document a source of public international law?
    \begin{enumerate}
        \item The ICJ is the principal judicial organ of the UN
        \item Under Art 38(1) of the \statute{\textit{1969 Vienna Convention on the Law of Treaties}} (Page \pageref{ICJ Statute Art 38}), the sources of public international law are:
        \begin{enumerate}
            \item Treaties
            \item Custom
            \item General principles of law
            \item Judicial decisions and the teachings of publicists
        \end{enumerate}
        \item Art 38(1) is `generally regarded as a complete statement of the sources of international law' -- \case{\textit{Ure v Commonwealth} (2016) 329 ALR 452 at [15]} (Page \pageref{case:Ure v Commonwealth})
        \item Did the state consent to be bound to the jurisdiction of the ICJ/to the terms of the source?
    \end{enumerate}
    \item Was the source a treaty?
    \begin{enumerate}
        \item Under Art 38(1)(a) of the ICJ Statute, treaties are a source of international law
        \item Was the document a bilateral and/or multilateral convention between two or more states?
        \item See Topic 3 scaffolds (\ref{scaffold:Topic 3} on Page \pageref{scaffold:Topic 3}) for a detailed analysis
        \item Note that if there are a number of instances of states contravening a treaty, it is not necessary for the states to provide consistently correct conduct and that some variations in practice are acceptable, and they do not form a new rule -- \case{\textit{Military and Paramilitary Activities in and against Nicaragua} [1986] ICJ Rep 14}
    \end{enumerate}
    \item Was there international custom involved?
    \begin{enumerate}
        \item Under Art 38(1)(b) of the ICJ Statute, custom is a source of international law, and requires two elements:
        \begin{enumerate}
            \item State practice - objective evidence that the custom is practiced widely
            \item \textit{Opinio juris} - the belief that the practice is legally required
        \end{enumerate}
        \item Unless they are persistent objectors, all states are bound to customary international law
        \item Certain customary norms are \textit{jus cogens}, which are peremptory norms from which no derogation is permitted
        \item Was there state practice?
        \begin{enumerate}
            \item State practice can be evidenced by materials that demonstrate the activities and views of states and state officials
            \item State practice can generate custom if the following requirements are met: -- \case{\textit{North Sea Continental Shelf Cases} (1969) ICJ Rep 3}
            \begin{enumerate}
                \item The practice was consistent over time (but not necessarily entirely uniform)
                \item The practice was widespread
                \item The practice was representative of multiple states (especially those who are most likely affected by it)
                \item The practice was developed over a lengthy period of time (this is not a steadfast requriement; customary norms may still emerge rapidly if there is an overwhelming practice of it)
            \end{enumerate}
        \end{enumerate}
        \item Was there \textit{opinio juris}?
        \begin{enumerate}
            \item \textit{Opinio juris} refers to the belief that the practice is legally required
            \item If there is extensive state practice, then \textit{opinio juris} tends to be less important, and vice-versa
        \end{enumerate}
        \item It is possible for treaty norms to become custom, and for treaty provisions to become customary international law -- \case{\textit{North Sea Continental Shelf Cases} (1969) ICJ Rep 3}
        \begin{enumerate}
            \item However, the custom exists independently of the treaty -- \case{\textit{Military and Paramilitary Activities in and against Nicaragua} [1986] ICJ Rep 14}
        \end{enumerate}
    \end{enumerate}
    \item Was there regional custom involved?
    \begin{enumerate}
        \item The ICJ has recognised that it is possible for regional custom to exist, but invoking it requires a higher standard than general international custom -- \case{\textit{Asylum Case (Colombia v Peru)} [1950] ICJ Rep 226}
        \begin{itemize}
            \item Regional custom must have a higher degree of stability and continuity to apply as international law in that area -- \case{\textit{Asylum Case (Colombia v Peru)} [1950] ICJ Rep 226}
            \item Such an example was made out in the English Court of Appeal in \case{\textit{R (app. Al-Saadoon v Sec. of Defence)} [2010] 1 All ER 271}, where rules of regional custom were found to exist, but had not met the high threshold to be invoked (Page \pageref{case:R (Al-Saadoon v Sec. of Defence)})
        \end{itemize}
    \end{enumerate}
    \item If there was custom involved, was the party a persistent objector?
    \begin{enumerate}
        \item The doctrine of a persistent objector is fairly narrow, and enunciates that states which consistently object to the emergence of a rule from its earliest point of gestation will not be bound by it -- \case{\textit{Anglo Norwegian Fisheries Case (UK v Norway)} [1951] ICJ Rep 116}
        \item A state cannot be a persistent objector to a \textit{jus cogens} principle -- \textit{International Law Commission 2019 Report} Chapter V Conclusion 14 (Page \pageref{report:2019 ILC Conc. 14})
    \end{enumerate}
    \item Was the source a general principle of international law?
    \begin{enumerate}
        \item Under Art 38(1)(c) of the ICJ Statute, general principles of law recognised by civilised nations form a source of PIL, with the objective of avoiding the \textit{non liquet} (the situation where `it is not clear' by enabling the ICJ to look at different legal systems for inspiration)
        \item General principles of international law and municipal law are included in this provision
        \item General principles of law may be implicitly adopted in judicial decisions to enable a conclusion to be made -- \case{\textit{Bay of Bengal (Bangladesh/Myanmar)} [2012] ILTOS 12}
        \item For example, various domestic legal systems were examined in relation to the issue of estoppel to aid the Tribunal in its decision -- \case{\textit{Chagos Marine Protected Area Arbitration (Mauritius v United Kingdom)} (2015) XXXI RIAA 359}
    \end{enumerate}
    \item Was there a judicial decision and/or the teachings of a publicist?
    \begin{enumerate}
        \item Whilst Art 38(1)(d) of the \textit{ICJ Statute} enables judicial decisions and the work of publicists to be considered as sources of PIL, they are subsidiary means for the determinations of the rules of law, and are treated as having lesser significance than other sources
        \item Decisions taken by the ICJ do not constitute binding precedent in future decisions, but remain merely persuasive -- \statute{\textit{Statute of the International Court of Justice} Art 59}
        \begin{enumerate}
            \item These sources are ``resorted to by judicial tribunals not for the
            speculations of their authors concerning what the law ought to be, but for trustworthy
            evidence of what the law really is'' -- \case{\textit{The Paquete Habana} 175 US 677 (1900)}
        \end{enumerate}
        \item Was this a UN General Assembly Resolution?
        \begin{enumerate}
            \item The UN General Assembly (UNGA) is the plenary body of the UN, and as all UN members have a seat, it has become a great forum for state practice and \textit{opinio juris}
            \item Decisions of the UNGA are not binding, except in the key areas of (without these areas, the UN could not function):
            \begin{enumerate}
                \item Admission of member states
                \item Suspension of member states
                \item Matters related to the UN budget
            \end{enumerate}
            \item Resolutions of the UNGA can provide evidence for state practice
            \item UNGA resolutions can influence international law in three key ways
            \begin{enumerate}
                \item Interpreting the \textit{Charter of the United Nations}
                \item Affirming recognised customary norms (through passing a resolution)
                \item Influencing the creation of new customary norms
            \end{enumerate}
            \item UNGA resolutions, whilst normally not binding, may have normative value, and can provide ``evidence important for the establishing the existence of a rule or the emergence of a \textit{opinion juris}'' -- \case{\textit{Legality of the Threat or Use of Nuclear Weapons} [1996] ICJ Rep 254 at [70] - [73]} (Page \pageref{case:Legality of Nuclear Weapons [1996] ICJ Rep 254}); such evidence can include:
            \begin{enumerate}
                \item The voting records of the UNGA
                \item Transcripts of what was said on the floor of the UNGA
                \item Margins of the votes undertaken in the UNGA
            \end{enumerate}
        \end{enumerate}
        \item Was this a UN Security Council Resolution?
        \begin{enumerate}
            \item The UN Security Council (UNSC) has limited law-making capacity, but can adopt certain binding resolutions
            \item UNSC resolutions are binding only on the members of the UN -- \statute{\textit{Charter of the United Nations} Art 25}
        \end{enumerate}
    \end{enumerate}
    \item Was there a measure of soft law involved?
    \begin{itemize}
        \item Soft law refers to rules thare binding but vague, and/or `rules' that are clear but are not binding
        \item They can articulate standards/norms that will, over time, become binding, and can also be used to interpret other sources of international law
    \end{itemize}
\end{enumerate}

\section{The Law of Treaties}\label{scaffold:Topic 3}
\begin{enumerate}
    \item Was there a treaty involved?
    \begin{itemize}
        \item A treaty refers to ``an international agreement concluded between States in written form and governed by international law, whether embodied in a single instrument or in two or more related instruments and whatever its particular designation'' - Art 2(1)(a) of the \textit{1969 Vienna Convention on the Law of Treaties} (Page \pageref{VCLT Art 2})
    \end{itemize}
    \item Does the \textit{1969 Vienna Convention on the Law of Treaties} apply?
    \begin{enumerate}
        \item Was the treaty between two or more states? - VCLT Art 3 (Page \pageref{VCLT Art 3})
        \item Was the treaty in writing? - VCLT Art 3 (Page \pageref{VCLT Art 3})
    \end{enumerate}
    \item Was the treaty valid?
    \item Did the states involved enter into the treaty?
    \begin{enumerate}
        \item Was the party entering into the treaty a state, international organisation or an international entity with capacity to enter into a treaty?
        \item Has the individual representing the party produced full powers evincing their authority to enter into the treaty?
        \begin{enumerate}
            \item Heads of State, Heads of Government and Ministers of Foreign Affairs are taken to have the capacity to conclude treaties without producing full powers - VCLT Art 7 (Page \pageref{VCLT Art 7})
        \end{enumerate}
        \item Was the treaty signed by the party?
        \begin{itemize}
            \item Upon signing a treaty, the state expresses a willingness to continue the treaty-making process and agrees with the treaty in principle
            \item However, the signature does not bind the state to the treaty at this point in time
        \end{itemize}
        \item If the treaty is a new treaty, was it ratified by the party?
        \begin{itemize}
            \item Upon ratification, the state is bound by the terms of the treaty
        \end{itemize}
        \item If the treaty is an existing one, was it ascended to by the party?
        \begin{itemize}
            \item Ascension only arises when a state becomes a party to a treaty already in force (i.e., already negotiated and signed by other states)
            \item This has the same legal effect as ratification, and is binding upon the state
        \end{itemize}
    \end{enumerate}
    \item Was the treaty in force?
    \begin{itemize}
        \item A treaty enters into force when the relevant provisions in the treaty addressing this point are satisfied
        \item If the treaty is silent on this point, it will enter into force when all parties have consented to being bound by it -- VCLT Art 24(2) (Page \pageref{VCLT Art 24})
        \item If a party signs a treaty after its formation, it will be binding upon that state on the day that consent to being bound is established -- VCLT Art 24(3) (Page \pageref{VCLT Art 24})
    \end{itemize}
    \item Was there any reservation to the treaty?
    \item Are there grounds to terminate, withdraw or suspend the treaty?
\end{enumerate}

\section{International Law and Australian Law}