\section{Development, Nature and Scope of Public International Law}

\section{Sources of Public International Law}

\section{The Law of Treaties}
\begin{enumerate}
    \item Was there a treaty involved?
    \begin{itemize}
        \item A treaty refers to ``an international agreement concluded between States in written form and governed by international law, whether embodied in a single instrument or in two or more related instruments and whatever its particular designation'' - Art 2(1)(a) of the \textit{1969 Vienna Convention on the Law of Treaties} (Page \pageref{VCLT Art 2})
    \end{itemize}
    \item Does the \textit{1969 Vienna Convention on the Law of Treaties} apply?
    \begin{enumerate}
        \item Was the treaty between two or more states? - VCLT Art 3 (Page \pageref{VCLT Art 3})
        \item Was the treaty in writing? - VCLT Art 3 (Page \pageref{VCLT Art 3})
    \end{enumerate}
    \item Was the treaty valid?
    \item Did the states involved enter into the treaty?
    \begin{enumerate}
        \item Was the party entering into the treaty a state, international organisation or an international entity with capacity to enter into a treaty?
        \item Has the individual representing the party produced full powers evincing their authority to enter into the treaty?
        \begin{enumerate}
            \item Heads of State, Heads of Government and Ministers of Foreign Affairs are taken to have the capacity to conclude treaties without producing full powers - VCLT Art 7 (Page \pageref{VCLT Art 7})
        \end{enumerate}
        \item Was the treaty signed by the party?
        \begin{itemize}
            \item Upon signing a treaty, the state expresses a willingness to continue the treaty-making process and agrees with the treaty in principle
            \item However, the signature does not bind the state to the treaty at this point in time
        \end{itemize}
        \item If the treaty is a new treaty, was it ratified by the party?
        \begin{itemize}
            \item Upon ratification, the state is bound by the terms of the treaty
        \end{itemize}
        \item If the treaty is an existing one, was it ascended to by the party?
        \begin{itemize}
            \item Ascension only arises when a state becomes a party to a treaty already in force (i.e., already negotiated and signed by other states)
            \item This has the same legal effect as ratification, and is binding upon the state
        \end{itemize}
    \end{enumerate}
    \item Was the treaty in force?
    \begin{itemize}
        \item A treaty enters into force when the relevant provisions in the treaty addressing this point are satisfied
        \item If the treaty is silent on this point, it will enter into force when all parties have consented to being bound by it -- VCLT Art 24(2) (Page \pageref{VCLT Art 24})
        \item If a party signs a treaty after its formation, it will be binding upon that state on the day that consent to being bound is established -- VCLT Art 24(3) (Page \pageref{VCLT Art 24})
    \end{itemize}
    \item Was there any reservation to the treaty?
    \item Are there grounds to terminate, withdraw or suspend the treaty?
\end{enumerate}

\section{International Law and Australian Law}