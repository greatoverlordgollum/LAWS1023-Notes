\label{sec:Topic 3}
\section{Defining Treaties}
\begin{itemize}
    \item A treaty refers to a binding agreement between states (or international organisations) that is governed by international law
    \item They perform various functions, including:
    \begin{itemize}
        \item Transferring territory (like conveyance)
        \item Bargaining (like contracts)
        \item Setting out general international law (like legislation)
        \item Creating international organisation (like articles of association)
        \item Establishing new legal orders (like constitutions)
    \end{itemize}
    \item The primary treaty on treaties (but not the only one) is the \textit{1969 Vienna Convention on the Law of Treaties} (VCLT), which was based on the work of the International Law Commission\footnote{A body within the United Nations tasked with the codification and progressive development of public international law}, and is mostly declaratory of customary international law
    \begin{itemize}
        \item The VCLT was signed in 1969, but entered into force in 1980, and so only applies to treaties concluded after 1980; however, many of its provisions can apply to treaties concluded before 1980 as a matter of general international law
        \item Most provisions within the VCLT are customary, which can be helpful in resolving disputes as not all states are party to the VCLT, but its core rules nonetheless apply to them as a matter of custom
    \end{itemize}
    \item VCLT Art 2(1)(a) defines what a treaty is (an international agreement between States, in either a singular instrument or in multiple instruments, and in any form whatsoever, as long as it is written)
\end{itemize}
\begin{conventiondetails}{\textit{1969 Vienna Convention on the Law of Treaties} Article 2}\label{VCLT Art 2}
    \flushleft
    \textit{Use of Terms}

    \begin{enumerate}
        \item For the purposes of the present Convention:
        \begin{enumerate}[label=(\alph*)]
            \item ``treaty" means an international agreement concluded between States in written form and governed by international law, whether embodied in a single instrument or in two or more related instruments and whatever its particular designation; 
            \item ``ratification", ``acceptance", ``approval" and `accession" mean in each case the international act so named whereby a State establishes on the international plane its consent to be bound by a treaty; 
            \item ``full powers" means a document emanating from the competent authority of a State designating a person or persons to represent the State for negotiating, adopting or authenticating the text of a treaty, for expressing the consent of the State to be bound by a treaty, or for accomplishing any other act with respect to a treaty; 
            \item ``reservation" means a unilateral statement, however phrased or named, made by a State, when signing, ratifying, accepting, approving or acceding to a treaty, whereby it purports to exclude or to modify the legal effect of certain provisions of the treaty in their application to that State; 
            \item ``negotiating State" means a State which took part in the drawing up and adoption of the text of the treaty; 
            \item ``contracting State" means a State which has consented to be bound by the treaty, whether or not the treaty has entered into force; 
            \item ``party" means a State which has consented to be bound by the treaty and for which the treaty is in force; 
            \item ``third State" means a State not a party to the treaty; 
            \item ``international organization" means an intergovernmental organization. 
        \end{enumerate}
        \item  The provisions of paragraph 1 regarding the use of terms in the present Convention are without 
        prejudice to the use of those terms or to the meanings which may be given to them in the internal law of 
        any State. 
    \end{enumerate}
\end{conventiondetails}

\begin{itemize}
    \item Under Art 3 of the VCLT, the definition given in Art 2(1)(a) does not affect agreements between states and other subjects of international law, or between those other subjects (i.e., it only affects agreements between states)
    \begin{itemize}
        \item However, equivalent norms of custom, or another treaty, may apply to the treaties not covered within the scope of the VCLT
    \end{itemize}
    \item The VCLT additionally does not apply to non-written treaties on the text of Art 3
    \begin{itemize}
        \item However, it does not foreclose the possibility of an oral agreement/treaty, and as much of the VCLT is custom, the rules set out in it will still apply to oral agreements/treaties, but which of those rules falls into that scope is vague
    \end{itemize}
\end{itemize}

\begin{conventiondetails}{\textit{1969 Vienna Convention on the Law of Treaties} Article 3}\label{VCLT Art 3}
    \flushleft
    \textit{International agreements not within the scope of the present Convention}

    \vspace{\baselineskip}

    The fact that the present Convention does not apply to international agreements concluded between States and other subjects of international law or between such other subjects of international law, or to international agreements not in written form, shall not affect:
    \begin{enumerate}[label=(\alph*)]
        \item the legal force of such agreements; 
        \item the application to them of any of the rules set forth in the present Convention to which they would be subject under international law independently of the Convention; 
        \item the application of the Convention to the relations of States as between themselves under international agreements to which other subjects of international law are also parties.
    \end{enumerate}
\end{conventiondetails}

\begin{casedetails}{\textit{Legal Status of Eastern Greenland (Denmark v Norway)} (1993) PCIJ Series A/B, No 53}
    \flushleft
    In this case, the Permanent Court of International Justice held that Norway was bound by an oral undertaking given to Denmark that it would not oppose its claim to sovereignty over Greenland. The Court held that ``as a result of the undertaking [by the Norwegian Foreign Minister], Norway is under an obligation to refrain from contesting Danish sovereignty over Greenland as a whole, and \textit{a fortiori} to refrain from occupying a part of Greenland''.
\end{casedetails}

\begin{itemize}
    \item Treaties are very flexible, and may be embodied in one or several instruments (e.g., an exchange of notes (which is more than one instrument) can constitute a treaty), and there are no requirements as to the form of the treaty
    \begin{itemize}
        \item The key consideration is the objective intention of the parties (which can be discern from the text of the treaty)
    \end{itemize}
\end{itemize}

\begin{casedetails}{\textit{Maritime Delimitation and Territorial Questions (Qatar v Bahrain)} (1994) ICJ Rep 112}\label{case:Qatar v Bahrain}
    \flushleft
    In this case, the International Court of Justice held that an exchange of letters between the Emir of Qatar and the Ruler of Bahrain constituted a treaty, and that the exchange of letters was a valid means of concluding a treaty.
    
    \vspace{\baselineskip}

    ``The Minutes are not the simple record of a meeting ... They enumerate the commitments to which the Parties have consented ... They constitute an international agreement''.

\end{casedetails}

\begin{itemize}
    \item Unilateral declarations made by a party may have binding effect
\end{itemize}

\begin{casedetails}{\textit{Nuclear Test Cases (Australia v France)} (1974) ICJ Rep 253}
    \flushleft
    In this case, the International Court of Justice held that a unilateral declaration by France that it would not conduct nuclear tests in the atmosphere was binding on France, and that the declaration was a unilateral act having legal effect.
    
    \vspace{\baselineskip}

    ``An undertaking ... if given publicly with an intent to be bound, even though not made within the context of international negotiations, is binding''.
\end{casedetails}

\begin{itemize}
    \item There is no requirement that a treaty needs to involve `consideration' (i.e., a promise, price, detriment or forbearance given as a value for a promise); this results in the potential for treaties to be one-sided
\end{itemize}

\section{Entry into a Treaty}
\begin{itemize}
    \item Only states, international organisations and other international entities with capacity to enter into treaties (i.e., international persons) may be parties to a treaty
    \item Under Art 7 of the VCLT, Heads of State, Heads of Government and Ministers of Foreign Affairs have the capacity to conclude treaties without producing ``full powers''
    \begin{itemize}
        \item ``Full powers'' refers to a document or a set of documents evidencing authority for the bearing/undersigned individual to act on behalf of the state and thereby enter into treaties
        \item Since a state does not have any physical existence, it has to act through a representative (which can be one of the above individuals, or another individual who has produced full powers)
    \end{itemize}
\end{itemize}

\begin{conventiondetails}{\textit{1969 Vienna Convention on the Law of Treaties} Article 6}\label{VCLT Art 6}
    \textit{Capacity of States to conclude treaties}

    \vspace{\baselineskip}

    Every State possesses capacity to conclude treaties. 
\end{conventiondetails}

\begin{conventiondetails}{\textit{1969 Vienna Convention on the Law of Treaties} Article 7}\label{VCLT Art 7}
    \flushleft
    \textit{Full Powers}
    \begin{enumerate}
        \item A person is considered as representing a State for the purpose of adopting or authenticating the text of a treaty or for the purpose of expressing the consent of the State to be bound by a treaty if: 
        \begin{enumerate}[label=(\alph*)]
            \item he produces appropriate full powers; or 
            \item it appears from the practice of the States concerned or from other circumstances that their  intention was to consider that person as representing the State for such purposes and to dispense with full powers. 
        \end{enumerate}
        \item In virtue of their functions and without having to produce full powers, the following are  considered as representing their State:
        \begin{enumerate}[label=(\alph*)]
            \item Heads of State, Heads of Government and Ministers for Foreign Affairs, for the purpose of 
            performing all acts relating to the conclusion of a treaty; 
            \item heads of diplomatic missions, for the purpose of adopting the text of a treaty between the accrediting State and the State to which they are accredited; 
            \item representatives accredited by States to an international conference or to an international organization or one of its organs, for the purpose of adopting the text of a treaty in that conference, organization or organ. 
        \end{enumerate}
    \end{enumerate}
\end{conventiondetails}

\begin{itemize}
    \item Entry into a treaty is a two-step process, entailing:
    \begin{enumerate}
        \item Signature (which is when a state expresses a willingness to continue the treaty-making process, \textbf{but is not bound by the treaty at this point})
        \item Ratification (which indicates that the state consents to be bound by the treaty once it has been ratified)
        \item Acession (this only arises when a state becomes party to a treaty already negotiated and signed by other states, and has the same legal effect as ratification)
    \end{enumerate}
    \item There is a period of time between signature and ratification, which allows states to implement the provisions of the treaty into their domestic law, and for the state to prepare for the treaty to be given effect
    \item Signature is not sufficient for a state to be bound; they must have either ratified or acceded to the treaty
    \item A treaty enters into force when the relevant provisions in the treaty addressing this point have been satisfied
    \begin{itemize}
        \item If the treaty is silent on this point, it will enter into force when all the parties have consented to be bound by the treaty, following Art 24(2) of the \textit{1969 Vienna Convention on the Law of Treaties}
        \begin{itemize}
            \item However, a treaty will almost always include a provision on when it will enter into force
        \end{itemize}
        \item A treaty enters into force for a specific party when it has consented to be bound, and when the treaty has entered into force generally
    \end{itemize}
\end{itemize}

\begin{conventiondetails}{\textit{1969 Vienna Convention on the Law of Treaties} Article 24}\label{VCLT Art 24}
    \flushleft
    \textit{Entry into force}
    \begin{enumerate}
        \item A treaty enters into force in such manner and upon such date as it may provide or as the 
        negotiating States may agree. 
        \item Failing any such provision or agreement, a treaty enters into force as soon as consent to be bound by the treaty has been established for all the negotiating States. 
        \item When the consent of a State to be bound by a treaty is established on a date after the treaty has come into force, the treaty enters into force for that State on that date, unless the treaty otherwise provides. 
        \item The provisions of a treaty regulating the authentication of its text, the establishment of the consent of States to be bound by the treaty, the manner or date of its entry into force, reservations, the functions of the depositary and other matters arising necessarily before the entry into force of the treaty apply from the time of the adoption of its text.
    \end{enumerate}
\end{conventiondetails}

\section{Registration and Application of Treaties}
\begin{itemize}
    \item In order to be recognised as binding instruments before UN organisations, they must be registered with the UN, following Art 102 of the \textit{Charter of the United Nations}
    \begin{itemize}
        \item This is a position reinforced in Art 80 of the \textit{1969 Vienna Convention on the Law of Treaties}
    \end{itemize}
    \begin{itemize}
        \item This is not a requirement for the treaty to be binding, but is a requirement for the treaty to be recognised by the UN
    \end{itemize}
    \item The principle of \textit{pacta sunt servanda}, following \textit{1969 Vienna Convention on the Law of Treaties} Art 26, requires that ``every treaty in force is binding upon the parties to it, and must be performed by them in good faith"
    \item Under Art 27 of the VCLT, a party may not invoke the provisions of its internal law (i.e., domestic law) as justification for its failure to perform a treaty
    \begin{itemize}
        \item This is subject to Art 46 of the \textit{1969 Vienna Convention on the Law of Treaties}, which allows a party to invoke its internal law as a justification for its failure to perform a treaty if the other party was aware of that law, and the law is not contrary to the treaty
    \end{itemize}
\end{itemize}

\begin{conventiondetails}{\textit{Charter of the United Nations} Article 102}\label{convention:UN Charter Art 102}
    \flushleft
    \begin{enumerate}
        \item Every treaty and every international agreement entered into by any Member of the United Nations after the present Charter comes into force shall as soon as possible be registered with the Secretariat and published by it.
        \item No party to any such treaty or international agreement which has not been registered in accordance with the provisions of paragraph 1 of this Article may invoke that treaty or agreement before any organ of the United Nations.
    \end{enumerate}
\end{conventiondetails}

\begin{conventiondetails}{\textit{1969 Vienna Convention on the Law of Treaties} Article 80}
    \flushleft
    \textit{Registration and publication of treaties}

    \begin{enumerate}
        \item Treaties shall, after their entry into force, be transmitted to the Secretariat of the United Nations for registration or filing and recording, as the case may be, and for publication. 
        \item The designation of a depositary shall constitute authorization for it to perform the acts specified in the preceding paragraph.
    \end{enumerate}
\end{conventiondetails}

\begin{conventiondetails}{\textit{1969 Vienna Convention on the Law of Treaties} Article 26}\label{VCLT Art 26}
    \flushleft
    \textit{``Pacta sunt servanda"}
    \vspace{\baselineskip}

    Every treaty in force is binding upon the parties to it and must be performed by them in good faith.
\end{conventiondetails}

\begin{conventiondetails}{\textit{1969 Vienna Convention on the Law of Treaties} Article 27}\label{VCLT Art 27}
    \flushleft
    \textit{Internal law and observance of treaties}

    \vspace{\baselineskip}

    A party may not invoke the provisions of its internal law as justification for its failure to perform  a treaty. This rule is without prejudice to article 46.
\end{conventiondetails}

\begin{conventiondetails}{\textit{1969 Vienna Convention on the Law of Treaties} Article 46}\label{VCLT Art 46}
    \flushleft
    \textit{Provisions of internal law regarding competence to conclude treaties}

    \begin{enumerate}
        \item A State may not invoke the fact that its consent to be bound by a treaty has been expressed in violation of a provision of its internal law regarding competence to conclude treaties as invalidating its consent unless that violation was manifest and concerned a rule of its internal law of fundamental importance. 
        \item A violation is manifest if it would be objectively evident to any State conducting itself in the matter in accordance with normal practice and in good faith.
    \end{enumerate}
\end{conventiondetails}

\begin{itemize}
    \item Under Art 18 of the \textit{1969 Vienna Convention on the Law of Treaties}, a state is obliged to refrain from acts that would defeat the object and purpose of a treaty (e.g., if a treaty reuires objects to be returned, then Art 18 prohibits the state from destroying those objects during the transfer process)
    \item When states have signed a treaty that has not yet been ratified, they must not undermine the spirit of the treaty in this intermediary phase
    \item Under Art 34 of the \textit{1969 Vienna Convention on the Law of Treaties}, treaties do not impose obligations or create rights for third states in the absence of their consent (\textit{pacta tertiss nex nocent nec prosunt})
\end{itemize}

\begin{conventiondetails}{\textit{1969 Vienna Convention on the Law of Treaties} Article 18}\label{VCLT Art 18}
    \flushleft
    \textit{Obligation not to defeat the object and purpose of a treaty prior to its entry into force}

    \vspace{\baselineskip}

    A State is obliged to refrain from acts which would defeat the object and purpose of a treaty when:

    \begin{enumerate}[label=(\alph*)]
        \item it has signed the treaty or has exchanged instruments constituting the treaty subject to ratification, acceptance or approval, until it shall have made its intention clear not to become a party to the treaty; or 
        \item it has expressed its consent to be bound by the treaty, pending the entry into force of the treaty and provided that such entry into force is not unduly delayed.
    \end{enumerate}
\end{conventiondetails}

\begin{conventiondetails}{\textit{1969 Vienna Convention on the Law of Treaties} Article 34}\label{VCLT Art 34}
    \flushleft
    \textit{General rule regarding third States}

    \vspace{\baselineskip}

    A treaty does not create either obligations or rights for a third State without its consent.
\end{conventiondetails}

\section{Reservations to Treaties}
\begin{itemize}
    \item If a state agrees to the general principles of a treaty, but does not agree with a specific provision or a set of provisions, they can enact a reservation when signing the treaty
    \item This has the effect of the reserving state and the states with whom the reservation was made being bound to the extent which they agreed to, and the reserved provisions not applying between those states (but still applying between the other states)
    \item A reservation is defined in Art 2(1)(d) of the \textit{1969 Vienna Convention on the Law of Treaties} (on Page \pageref{VCLT Art 2})
    \item Reservations can be made by a state at any stage of the treaty-making procedure
    \item Reservations are different from an `interpretative declaration', which is a statement made by a state to clarify its understanding of a treaty, but does not affect the legal effect of the treaty
    \begin{itemize}
        \item States can use interpretative declarations to clarify their understanding of a treaty, but they cannot use them to change the legal effect of the treaty
    \end{itemize}
    \item The rules of reservation prescribed under the VCLT apply only to multilateral treaties (as a reservation to a bilateral treaty is effectively a counter-offer)
\end{itemize}

\subsection{Permissibility}
\begin{itemize}
    \item The default position taken under Art 19 of the VCLT is that reservations are permissible, unless they are explicitly prohibited by the treaty or the reservation is incompatible with the object and purpose of the treaty
\end{itemize}

\begin{conventiondetails}{\textit{1969 Vienna Convention on the Law of Treaties} Article 19}\label{VCLT Art 19}
    \flushleft
    \textit{Formulation of reservations}

    \vspace{\baselineskip}

    A State may, when signing, ratifying, accepting, approving or acceding to a treaty, formulate a  reservation unless:
    \begin{enumerate}[label=(\alph*)]
        \item the reservation is prohibited by the treaty; 
        \item the treaty provides that only specified reservations, which do not include the reservation in question, may be made; or 
        \item in cases not failing under subparagraphs (a) and (b), the reservation is incompatible with the object and purpose of the treaty. 
    \end{enumerate}
\end{conventiondetails}

\begin{itemize}
    \item Under the \textit{ILC Guide to Practice on Reservations}, the test for incompatibility of a reservation is whether ``a reservation is incompatible with the object and purpose of the treaty if it affects an essential element of the treaty that is necessary to its general tenor, in such a way that the reservation impairs the \textit{raison d'être} [the most important reason] of the treaty''
    \item If a reservation is impermissible, the traditional view is that the impermissible reservation vitiates the consent of the state to the treaty as a whole, and results in the state not being a party to the treaty, following \textit{Reservations to Genocide Convention} [1951] ICJ Rep 15
    \item The emerging view, especially for human rights treaties, is that the offending reservation is null and void, and may be severed, with the state bound by the treaty without the protection of the reservation (unless consent to be bound is conditional on the reservation)
    \begin{itemize}
        \item This will cut out/sever the reservation, and will bind a state without the protection of their reservation
    \end{itemize}
\end{itemize}

\subsection{Acceptance and Objection}
\begin{itemize}
    \item If a treaty expressly allows for reservations, then no acceptance of a reservation is required by the other parties
    \item Acceptance by all parties will be required if a treaty has a small number of parties, and the application of the treaty in its entirety is an essential condition of signing 
    \item In all other cases:
    \begin{itemize}
        \item Acceptance by the other contracting state(s) of the reservation results in the reserving state being bound by the treaty (with the reservation incorporated); and
        \item Objection to a reservation does not prevent entry into force of a treaty between the objecting state and the reserving state, unless the objecting state says otherwise
    \end{itemize}
\end{itemize}
\begin{conventiondetails}{\textit{1969 Vienna Convention on the Law of Treaties} Article 20}\label{VCLT Art 20}
    \flushleft
    \textit{Acceptance of and objection to reservations}

    \begin{enumerate}
        \item A reservation expressly authorized by a treaty does not require any subsequent acceptance by  the other contracting States unless the treaty so provides. 
        \item When it appears from the limited number of the negotiating States and the object and purpose of a treaty that the application of the treaty in its entirety between all the parties is an essential condition of the consent of each one to be bound by the treaty, a reservation requires acceptance by all the parties. 
        \item When a treaty is a constituent instrument of an international organization and unless it otherwise provides, a reservation requires the acceptance of the competent organ of that organization. 
        \item In cases not falling under the preceding paragraphs and unless the treaty otherwise provides:   
        \begin{enumerate}[label=(\alph*)]
            \item acceptance by another contracting State of a reservation constitutes the reserving State a party to the treaty in relation to that other State if or when the treaty is in force for those States; 
            \item an objection by another contracting State to a reservation does not preclude the entry into force of the treaty as between the objecting and reserving States unless a contrary intention is definitely expressed by the objecting State; 
            \item an act expressing a State's consent to be bound by the treaty and containing a reservation is effective as soon as at least one other contracting State has accepted the reservation.
        \end{enumerate}     
        \item For the purposes of paragraphs 2 and 4 and unless the treaty otherwise provides, a reservation is considered to have been accepted by a State if it shall have raised no objection to the reservation by the end of a period of twelve months after it was notified of the reservation or by the date on which it expressed its consent to be bound by the treaty, whichever is later. 
    \end{enumerate}
\end{conventiondetails}

\subsection{Legal Effect}
\begin{itemize}
    \item Three scenarios can arise when a permissible reservation is made:
    \begin{enumerate}
        \item If state A accepts state R's reservation, then the treaty is modified between A and R (but only between A and R) to the extent of the reservation (VCLT Art 21(1) and (2)) (Page \pageref{VCLT Art 21})
        \begin{itemize}
            \item Other parties will not be bound by this reservation; it acts like a side agreement with R along the lines of the reservation
        \end{itemize}
        \item If state B objects to state R's reservation and says the treaty is not to apply, then there is no treaty between them at all (VCLT Art 20(4)(b)) (Page \pageref{VCLT Art 20})
        \item If state C objects to state R's reservation but does not say that treaty is not to apply, then treaty applies but `the provisions to which the reservation relates do not apply ... to the extent of the reservation' (VCLT Art 21(3)) (Page \pageref{VCLT Art 21})
        
    \end{enumerate}
\end{itemize}
\begin{conventiondetails}{\textit{1969 Vienna Convention on the Law of Treaties} Article 21}\label{VCLT Art 21}
    \flushleft
    \textit{Legal effects of reservations and of objections to reservations}

    \begin{enumerate}
        \item  A reservation established with regard to another party in accordance with articles 19, 20 and 23:
        \begin{enumerate}[label=(\alph*)]
            \item modifies for the reserving State in its relations with that other party the provisions of the treaty to which the reservation relates to the extent of the reservation; and 
            \item modifies those provisions to the same extent for that other party in its relations with the reserving State.
        \end{enumerate}     
        \item The reservation does not modify the provisions of the treaty for the other parties to the treaty inter se. 
        \item When a State objecting to a reservation has not opposed the entry into force of the treaty between itself and the reserving State, the provisions to which the reservation relates do not apply as between the two States to the extent of the reservation.
    \end{enumerate}
\end{conventiondetails}

\begin{casedetails}{\textit{Republic of India v CCDM Holdings, LLC} [2025 FCAFC 2]}
    \flushleft
    This case ivolved India having entered a reservation to a Convention, with a question arising as to whether the reservation applied only in Indian proceedings, or whether it also applied to proceedings in Australia. Whilst it is rare for a reservation issue to come up in a domestic court, the Full Federal Court explained and applied the provisions of the VCLT on reservations in an enforcement of judgements case concerning foreign state immunity; the Court also referred to the \textit{ILC Guide to Practice on Reservations to Treaties} in its reasoning at [63] to [70]. At [63], the Court emphasised the reciprocal effect of reservations, holding that ``the effect of a reservation is that between the reserving and accepting state…the reservation modifies the provision of the treaty to the extent of the reservation for each party reciprocally (see Art 21(1)(a) and (b) of the Vienna Convention)''. This case reinforces the idea of reciprocity whereby if a reservation is made, it applies to both states (i.e., it removes the particular provision for both states, not just one state).
\end{casedetails}

\section{Interpretation of Treaties}
\begin{itemize}
    \item There are several conceptual approaches to treaty interpretation:
    \begin{itemize}
        \item Formalist/Textual (formal adherence to the terms of the treaty)
        \item Restrictive (deference to state sovereignty)
        \item Teleological (to give effect to the object and purpose of the treaty)
        \item Effectiveness (to ensure the treaty regime remains as effective as possible)
        \item Originalist (to focus on the original purpose of the treaty))
    \end{itemize}
    \item The Australian courts will apply the VCLT when interpreting a treaty that has been incorporated into Australian law
    \begin{itemize}
        \item In the example of \case{\textit{DHI22 v Qatar Airways} [2024] FCA 348}, the Court found that a claim in relation to invasive medical examinations was not addressed by the Montreal Convention (as these were not an `accident' within the meaning of the Convention)
        \item This case dealt with the liability of carriers for accidents that occur on board an aircraft
    \end{itemize}
    \item To ensure national uniformity in treaty interpretation for treaties incorporated into legislation, the courts do not apply the rules of statutory interpretation but instead apply the VCLT
\end{itemize}

\subsection{Key Rules of Treaty Interpretation}

\begin{conventiondetails}{\textit{1969 Vienna Convention on the Law of Treaties} Article 31}\label{VCLT Art 31}
    \flushleft
    \textit{General rule of interpretation}

    \begin{enumerate}
        \item A treaty shall be interpreted in good faith in accordance with the ordinary meaning to be given to the terms of the treaty in their context and in the light of its object and purpose. 
        \item The context for the purpose of the interpretation of a treaty shall comprise, in addition to the text, including its preamble and annexes:
        \begin{enumerate}[label=(\alph*)]
            \item any agreement relating to the treaty which was made between all the parties in connection with the conclusion of the treaty; 
            \item any instrument which was made by one or more parties in connection with the conclusion of the treaty and accepted by the other parties as an instrument related to the treaty. 
        \end{enumerate}
        \item There shall be taken into account, together with the context:
        \begin{enumerate}[label=(\alph*)]
            \item any subsequent agreement between the parties regarding the interpretation of the treaty or the application of its provisions; 
            \item any subsequent practice in the application of the treaty which establishes the agreement of the parties regarding its interpretation; 
            \item any relevant rules of international law applicable in the relations between the parties. 
        \end{enumerate}
        \item  A special meaning shall be given to a term if it is established that the parties so intended. 
    \end{enumerate}
\end{conventiondetails}

\begin{itemize}
    \item Good faith
    \begin{itemize}
        \item The requirement of good faith is enshrined in Art 31(1) of the VCLT (Page \pageref{VCLT Art 31})
    \end{itemize}
    \item Subsequent agreement/practice and applicable international law
    \begin{itemize}
        \item Under VCLT Art 31(3) (Page \pageref{VCLT Art 31}), any subsequent agreement or practice between the parties is to be taken into account when interpreting a treaty
        \item The resolutions of internal organisations can be taken into account as subsequent agreement/practice if this position is supported by all parties, following \textit{Whaling in the Antartic Case} [2014] ICJ Rep 226 at [83]
    \end{itemize}
\end{itemize}

\begin{casedetails}{\textit{Whaling in the Antarctic Case} [2014] ICJ Rep 226}
    \flushleft
    This case concerned Japan's whaling program for Minke whales around Antarctica, which Australia challenged. Australia was successful in getting the ICJ to hold that Japan's program amounted to commercial whaling, which is prohibited under the treaty for Antarctica.

    \tcblower

    \begin{longtable}{r|>{\raggedright\arraybackslash}p{0.95\textwidth}}
        [83] & Article VIII expressly contemplates the use of lethal methods, and the Court is of the view that Australia and New Zealand overstate the legal significance of the recommendatory resolutions and Guidelines on which they rely. First, many IWC resolutions were adopted without the support of all States parties to the Convention and, in particular, without the concurrence of Japan. Thus, such instruments cannot be regarded as subsequent agreement to an interpretation of Article VIII, nor as subsequent practice establishing an agreement of the parties regarding the interpretation of the treaty within the meaning of subparagraphs (a) and (b), respectively, of paragraph (3) of Article 31 of the Vienna Convention on the Law of Treaties. \\

         & Secondly, as a matter of substance, the relevant resolutions and Guidelines that have been approved by consensus call upon States parties to take into account whether research objectives can practically and scientifically be achieved by using non-lethal research methods, but they do not establish a requirement that lethal methods be used only when other methods are not available. \\

         & The Court however observes that the States parties to the ICRW have a duty to co-operate with the IWC and the Scientific Committee and thus should give due regard to recommendations calling for an assessment of the feasibility of non-lethal alternatives. The Court will return to this point when it considers the Parties' arguments regarding JARPA II (see paragraph 137).
    \end{longtable} 
\end{casedetails}

\begin{conventiondetails}{\textit{1969 Vienna Convention on the Law of Treaties} Article 32}
    \flushleft
    \textit{Supplementary means of interpretation}

    \vspace{\baselineskip}

    Recourse may be had to supplementary means of interpretation, including the preparatory work of the treaty and the circumstances of its conclusion, in order to confirm the meaning resulting from the application of article 31, or to determine the meaning when the interpretation according to article 31:
    \begin{enumerate}[label=(\alph*)]
        \item leaves the meaning ambiguous or obscure; or 
        \item leads to a result which is manifestly absurd or unreasonable.
    \end{enumerate}
\end{conventiondetails}

\begin{itemize}
    \item Supplementary means of interpretation
    \begin{itemize}
        \item This is governed by Art 32 of the VCLT
        \item Supplementary means of preparation include what is known as the preparatory works (\textit{travaux préparatoires}), which can include notes of discussions taken prior to signing/ratification of the treaty
        \item This is somewhat of a last-resort measure, and is generally used when the reader is scratching their head as to the meaning of the treaty
    \end{itemize}
\end{itemize}

\section{Invalidity of Treaties}
\subsection{Void}
\begin{conventiondetails}{\textit{1969 Vienna Convention on the Law of Treaties} Article 51}
    \flushleft
    \textit{Coercion of a representative of a State}

    \vspace{\baselineskip}

    The expression of a State's consent to be bound by a treaty which has been procured by the  coercion of its representative through acts or threats directed against him shall be without any legal  effect.
\end{conventiondetails}

\begin{conventiondetails}{\textit{1969 Vienna Convention on the Law of Treaties} Article 52}
    \flushleft
    \textit{Coercion of a State by the threat or use of force}

    \vspace{\baselineskip}

    A treaty is void if its conclusion has been procured by the threat or use of force in violation of the  principles of international law embodied in the Charter of the United Nations.
\end{conventiondetails}

\begin{conventiondetails}{\textit{1969 Vienna Convention on the Law of Treaties} Article 53}
    \flushleft
    \textit{Treaties conflicting with a peremptory norm of general international law (``jus cogens")}

    \vspace{\baselineskip}

    A treaty is void if, at the time of its conclusion, it conflicts with a peremptory norm of general international law. For the purposes of the present Convention, a peremptory norm of general international law is a norm accepted and recognized by the international community of States as a whole as a norm from which no derogation is permitted and which can be modified only by a subsequent norm of general international law having the same character.
\end{conventiondetails}

\begin{conventiondetails}{\textit{1969 Vienna Convention on the Law of Treaties} Article 64}
    \flushleft
    \textit{Emergence of a new peremptory norm of general international law (``jus cogens")}

    \vspace{\baselineskip}

    If a new peremptory norm of general international law emerges, any existing treaty which is in  conflict with that norm becomes void and terminates.
\end{conventiondetails}

\subsection{Invalid}

\begin{conventiondetails}{\textit{1969 Vienna Convention on the Law of Treaties} Article 46}
    \flushleft
    \textit{Provisions of internal law regarding competence to conclude treaties}

    \begin{enumerate}
        \item A State may not invoke the fact that its consent to be bound by a treaty has been expressed in violation of a provision of its internal law regarding competence to conclude treaties as invalidating its consent unless that violation was manifest and concerned a rule of its internal law of fundamental importance. 
        \item A violation is manifest if it would be objectively evident to any State conducting itself in the matter in accordance with normal practice and in good faith.
    \end{enumerate}
\end{conventiondetails}

\begin{conventiondetails}{\textit{1969 Vienna Convention on the Law of Treaties} Article 47}
    \flushleft
    \textit{Specific restrictions on authority to express the consent of a State}

    \vspace{\baselineskip}

    If the authority of a representative to express the consent of a State to be bound by a particular treaty has been made subject to a specific restriction, his omission to observe that restriction may not be invoked as invalidating the consent expressed by him unless the restriction was notified to the other negotiating States prior to his expressing such consent.
\end{conventiondetails}

\begin{conventiondetails}{\textit{1969 Vienna Convention on the Law of Treaties} Article 48}
    \flushleft
    \textit{Error}

    \begin{enumerate}
        \item A State may invoke an error in a treaty as invalidating its consent to be bound by the treaty if the error relates to a fact or situation which was assumed by that State to exist at the time when the treaty was concluded and formed an essential basis of its consent to be bound by the treaty. 
        \item Paragraph 1 shall not apply if the State in question contributed by its own conduct to the error or if the circumstances were such as to put that State on notice of a possible error. 
        \item An error relating only to the wording of the text of a treaty does not affect its validity; article 79 then applies.
    \end{enumerate}
\end{conventiondetails}

\begin{conventiondetails}{\textit{1969 Vienna Convention on the Law of Treaties} Article 49}
    \flushleft
    \textit{Fraud}

    \vspace{\baselineskip}

    If a State has been induced to conclude a treaty by the fraudulent conduct of another negotiating  State, the State may invoke the fraud as invalidating its consent to be bound by the treaty.
\end{conventiondetails}

\section{Termination, Withdrawal and Suspension}
\begin{itemize}
    \item Termination of a treaty refers to it ceasing to exist
    \item Denunciation/withdrawal refers to when a party withdraws from a treaty (if it is multilateral, it will continue to exist for other parties)
    \item Suspension refers to the treaty remaining on foot, but its performance has been suspended/stopped for some period of time
    \item Internal grounds for termination, withdrawal and suspension stem from the treaty itself or the will of the parties (the required steps are spelled out in the treaty itself)
    \item External grounds for termination, withdrawal and suspension stem from external factors (e.g., material breach)
\end{itemize}

\subsection{Express or Implied Agreement}
\begin{itemize}
    \item Under Arts 54 and 57 of the VCLT, a treaty can be terminated or suspended by agreement of the parties
    \item This is consistent with the consensual basis of international law
\end{itemize}

\begin{conventiondetails}{\textit{1969 Vienna Convention on the Law of Treaties} Article 54}
    \flushleft
    \textit{Termination of or withdrawal from a treaty under its provisions or by consent of the parties }

    \vspace{\baselineskip}

    The termination of a treaty or the withdrawal of a party may take place:
    \begin{enumerate}[label=(\alph*)]
        \item in conformity with the provisions of the treaty; or 
        \item at any time by consent of all the parties after consultation with the other contracting States.
    \end{enumerate}
\end{conventiondetails}

\begin{conventiondetails}{\textit{1969 Vienna Convention on the Law of Treaties} Article 57}
    \flushleft
    \textit{Suspension of the operation of a treaty under its provisions or by consent of the parties }

    \vspace{\baselineskip}

    The operation of a treaty in regard to all the parties or to a particular party may be suspended:
    \begin{enumerate}[label=(\alph*)]
        \item in conformity with the provisions of the treaty; or
        \item at any time by consent of all the parties after consultation with the other contracting States.
    \end{enumerate}
\end{conventiondetails}

\subsection{Denunciation/Withdrawal}
\begin{itemize}
    \item A party may denounce/withdrawal from a treaty if the treaty itself permits it, if all of the parties consent to the denouncement/withdrawal, or if the right to denounce/withdrawal can be implied from the nature of the treaty
    \item Under Art 56 of the VCLT, a party must give at least 12 months' notice of its intention to denounce/withdrawal from a treaty
\end{itemize}
\begin{conventiondetails}{\textit{1969 Vienna Convention on the Law of Treaties} Article 56}
    \flushleft
    \textit{Denunciation of or withdrawal from a treaty containing no provision regarding termination, denunciation or withdrawal}

    \begin{enumerate}
        \item A treaty which contains no provision regarding its termination and which does not provide for denunciation or withdrawal is not subject to denunciation or withdrawal unless:
        \begin{enumerate}[label=(\alph*)]
            \item it is established that the parties intended to admit the possibility of denunciation or withdrawal; or 
            \item a right of denunciation or withdrawal may be implied by the nature of the treaty.
        \end{enumerate}
        \item A party shall give not less than twelve months' notice of its intention to denounce or withdraw from a treaty under paragraph 1.
    \end{enumerate}
\end{conventiondetails}

\subsection{Material Breach}
\begin{itemize}
    \item If a state breaches a treaty, it commits an internationally wrongful act (see Topic 10)
    \item Serious breaches can have consequences under the law of treaties
    \item A material breach is an impermissible repudiation of the treaty or violation of a provision essential for achieving the object and purpose of the treaty, following VCLT Art 60
    \item If there has been a material breach, the other parties in the treaty can suspend or terminate the treaty, if they wish to do so
\end{itemize}
\begin{conventiondetails}{\textit{1969 Vienna Convention on the Law of Treaties} Article 60}
    \flushleft
    \textit{Termination or suspension of the operation of a treaty as a consequence of its breach}

    \begin{enumerate}
        \item A material breach of a bilateral treaty by one of the parties entitles the other to invoke the breach as a ground for terminating the treaty or suspending its operation in whole or in part. 
        \item A material breach of a multilateral treaty by one of the parties entitles:
        \begin{enumerate}[label=(\alph*)]
            \item the other parties by unanimous agreement to suspend the operation of the treaty in whole or in part or to terminate it either:
            \begin{enumerate}[label=(\roman*)]
                \item in the relations between themselves and the defaulting State; or 
                \item as between all the parties;
            \end{enumerate}
            \item a party specially affected by the breach to invoke it as a ground for suspending the operation of the treaty in whole or in part in the relations between itself and the defaulting State; 
            \item any party other than the defaulting State to invoke the breach as a ground for suspending the operation of the treaty in whole or in part with respect to itself if the treaty is of such a character that a material breach of its provisions by one party radically changes the position of every party with respect to the further performance of its obligations under the treaty.
        \end{enumerate}
        \item A material breach of a treaty, for the purposes of this article, consists in:
        \begin{enumerate}[label=(\alph*)]
            \item a repudiation of the treaty not sanctioned by the present Convention; or 
            \item the violation of a provision essential to the accomplishment of the object or purpose of the treaty.
        \end{enumerate}
        \item The foregoing paragraphs are without prejudice to any provision in the treaty applicable in the event of a breach. 
        \item Paragraphs 1 to 3 do not apply to provisions relating to the protection of the human person contained in treaties of a humanitarian character, in particular to provisions prohibiting any form of reprisals against persons protected by such treaties. 
    \end{enumerate}
\end{conventiondetails}

\subsection{Impossibility}
\begin{itemize}
    \item A state may terminate or withdraw from a treaty if its performance has become impossible because `an object indispensable for the secution of the treaty' has permanently disappeared or been destroyed, following VCLT Art 61
\end{itemize}
\begin{conventiondetails}{\textit{1969 Vienna Convention on the Law of Treaties} Article 61}
    \flushleft
    \textit{Supervening impossibility of performance}

    \begin{enumerate}
        \item A party may invoke the impossibility of performing a treaty as a ground for terminating or withdrawing from it if the impossibility results from the permanent disappearance or destruction of an object indispensable for the execution of the treaty. If the impossibility is temporary, it may be invoked only as a ground for suspending the operation of the treaty. 
        \item Impossibility of performance may not be invoked by a party as a ground for terminating, withdrawing from or suspending the operation of a treaty if the impossibility is the result of a breach by that party either of an obligation under the treaty or of any other international obligation owed to any other party to the treaty. 
    \end{enumerate}
\end{conventiondetails}

\subsection{Fundamental Change of Circumstances}
\begin{itemize}
    \item A state may suspend/terminate, or withdraw, from a treaty if there has been a fundamental change of circumstances since the treaty was concluded, following VCLT Art 62
    \item For this to happen, three requirements need to be satisfied:
    \begin{itemize}
        \item The circumstances at the conclusion of the treaty must have been an essential basis of consent
        \item The change must not have been foreseen
        \item The change must radically transform the extent of the obligations still to be performed
    \end{itemize}
    \item International courts are very reluctant to find that impossibility and/or fundamental change of circumstances have been made out (i.e., these have a very high threshold and consequently a very limited scope)
\end{itemize}
\begin{conventiondetails}{\textit{1969 Vienna Convention on the Law of Treaties} Article 62}
    \flushleft
    \textit{Fundamental change of circumstances}

    \begin{enumerate}
        \item A fundamental change of circumstances which has occurred with regard to those existing at the 
        time of the conclusion of a treaty, and which was not foreseen by the parties, may not be invoked as a ground for terminating or withdrawing from the treaty unless:
        \begin{enumerate}[label=(\alph*)]
            \item the existence of those circumstances constituted an essential basis of the consent of the parties to be bound by the treaty; and
            \item the effect of the change is radically to transform the extent of obligations still to be performed under the treaty.
        \end{enumerate}
        \item A fundamental change of circumstances may not be invoked as a ground for terminating or 
        withdrawing from a treaty:
        \begin{enumerate}[label=(\alph*)]
            \item if the treaty establishes a boundary; or 
            \item if the fundamental change is the result of a breach by the party invoking it either of an obligation under the treaty or of any other international obligation owed to any other party to the treaty.
        \end{enumerate}
        \item If, under the foregoing paragraphs, a party may invoke a fundamental change of circumstances as a ground for terminating or withdrawing from a treaty it may also invoke the change as a ground for suspending the operation of the treaty.
    \end{enumerate}
\end{conventiondetails}

\begin{casedetails}{\textit{Gabčíkovo-Nagymaros Case} [1997] ICJ Rep 7}
    \flushleft

    This case involved questions of treaty law, state responsibility, succession of states (new states succeed to their obligation of their parent states, e.g., Soviet Union $\rightarrow$ Russia), and international environmental law. It arose from disagreement over a joint project between Hungary and Czechoslovakia to construct a series of locks and dams along a shared stretch of the Danube (under a 1977 Treaty). Hungary suspended work on the project after environmental protests were conducted by civil society. Czechoslovakia investigated a unilateral alternative (`Variant C'), resulting in Hungary seeking to terminate the 1977 Treaty.

    \vspace{\baselineskip}
    
    The rules of the VCLT concerning the termination and suspension of treaties were considered by virtue of being part of customary international law (the VLCT itself was not applicable as the parties joined it after the 1977 Treaty; later treaties cannot be applied to earlier treaties). The 1977 Treaty contained no provision concerning termination, and therefore it could only be terminated according to limited grounds set out in VCLT. 
    
    \vspace{\baselineskip}
    
    Performance was not impossible (and in any event impossibility cannot be invoked by party which itself breaches treaty). Hungary's argument was that it could not comply with the terms of the treaty without severely damaging the surrounding environment. The plea of fundamental change of circumstances can only be applied in exceptional circumstances, and there were none here; the court refused to apply art 62, as there were no radical changes to the obligations of the parties. Hungary was not entitled to invoke Slovakia's breach of treaty for terminating, as at that time no breach had yet taken place. Slovakia adopted Variant C because of Hungary's breach; Hungary by its own conduct had prejudiced its right to terminate the treaty. Although both Hungary and Slovakia failed to comply with the treaty, this reciprocal conduct did not bring treaty to an end nor justify its termination.

    \vspace{\baselineskip}

	`The Court would set a precedent with disturbing implications for treaty relations and the integrity of the rule pacta sunt servanda if it were to conclude that a treaty in force between States, which the parties have implemented in considerable measure and at great cost over a period of years, might be unilaterally set aside on grounds of reciprocal non-compliance.'

    \vspace{\baselineskip}

	The Court is very reluctant to declare the treaty as ineffective, emphasising the centrality of \textit{pacta sunt servanda}. Thus, this case shows that treaties are very hard to get out of (when drafting treaties, it is wise to include provisions for breach and change of circumstance, as external measures are hard to invoke).
\end{casedetails}

\begin{itemize}
    \item The table below outlines which VCLT provisions reflect or may reflect customary international law
    % \item This table was compiled in April 2018 by Sydney Centre for International Law (SCIL) Interns, and was lightly edited by Dr Alison Pert
    \item In this table, the following authorities are used:
\end{itemize}

{\renewcommand{\arraystretch}{1.2}\begin{longtable}{>{\raggedright\arraybackslash}p{0.25\textwidth}>{\raggedright\arraybackslash}p{0.75\textwidth}}
    \textbf{Aust} & Aust, Anthony, \textit{Modern Treaty Law and Practice} (Cambridge University Press, 3rd ed, 2013) \\
    \textbf{Corten \& Klein} & Corten, Oliver, and Pierre Klein (eds), \textit{The Vienna Conventions on the Law of Treaties: A Commentary} (Oxford University Press, 2011) \\
    \textbf{Dörr \& Schmalenbach} & Dörr, Oliver and Kirsten Schmalenbach (eds), Vienna Convention on
    the Law of Treaties, A Commentary (Springer-Verlag, 2018) \\
    \textbf{Hollis} & Hollis, Duncan B (ed), \textit{The Oxford Guide to Treaties} (Oxford University Press: Oxford, 2012) \\
    \textbf{Villiger} & Villiger, Mark E, \textit{Commentary on the 1969 Vienna Convention on the Law of Treaties} (Martinus Nijhoff, 2009)
\end{longtable}}

{\renewcommand{\arraystretch}{1.2}
\begin{longtable}{L{1.7cm} L{4cm} L{9.3cm}}
    % First page header (appears only once)
    \caption{Customary International Law Status of VCLT Articles} 
    \label{tab:VCLT Articles that can apply as customary international law} \\ % Label immediately follows caption
    \toprule
    \textbf{Article} & \textbf{Customary International Law?} & \textbf{Authority} \\
    \hline
    \endfirsthead % End of first-page-only content
    
    % Repeating header for subsequent pages (no caption or label)
    \toprule
    \textbf{Article} & \textbf{Customary International Law?} & \textbf{Authority} \\
    \hline
    \endhead % End of repeating header
    
    % Footer for all pages except the last
    \multicolumn{3}{r}{\textit{Continued on next page}} \\
    \endfoot % End of repeating footer
    
    % Footer for the last page
    \bottomrule
    \endlastfoot % End of last-page footer
    
    1 & N/A & Schmalenbach in Dörr \& Schmalenbach, p. 94 \\
    \nopagebreak\hline
    2 & Yes & Maritime Delimitation and Territorial Questions between Qatar and Bahrain (Jurisdiction and Admissibility) [1994] ICJ Rep 112, para 23 \\
    \nopagebreak\hline
    2(1)(a) & Yes & Temple of Preah Vihear (Cambodia v Thailand) (Preliminary Objections) [1961] ICJ Rep 17, p. 31 \\
    \nopagebreak\hline
    \multirow{3}{*}{2(1)(d)} & \multirow{3}{4cm}{Yes} & Maritime Delimitation in the Indian Ocean (Somalia v. Kenya) (Judgement) [2017] ICJ Rep 3, para 42 \\ 
    & & Summary of Practice of the Secretary-General as Depository of Multilateral Treaties, prepared by the Treaty Section of the Office of Legal Affairs, UN, 1994, ST/LEG7/Rev.1, p. 49 para. 161 \\ 
    & & Land and Maritime Boundary between Cameroon and Nigeria (Cameroon v Nigeria: Equatorial Guinea intervening) (Judgment) ICJ Rep p.303, para. 263 \\
    \nopagebreak\hline
    3 & N/A & Schmalenbach in Dörr \& Schmalenbach, p. 56 \\
    \nopagebreak\hline
    \multirow{2}{*}{4} & \multirow{2}{4cm}{N/A} & Schmalenbach in Dörr \& Schmalenbach, p. 94 \\ 
    & & Dopagne in Corten \& Klein 2011, p. 80 \\
    \nopagebreak\hline
    \multirow{2}{*}{5} & \multirow{2}{4cm}{Unlikely} & Schmalenbach in Dörr \& Schmalenbach, p. 99 \\ 
    & & H. Anderson in Corten \& Klein, p. 103 \\
    \nopagebreak\hline
    \multirow{2}{*}{6} & \multirow{2}{4cm}{Yes} & Turp \& Roch in Corten \& Klein, p. 111 \\ 
    & & Schmalenbach in Dörr \& Schmalenbach, p. 115 \\
    \nopagebreak\hline
    \multirow{2}{*}{7} & \multirow{2}{4cm}{Partly} & Hoffmeister in Dörr \& Schmalenbach, p. 132 \\ 
    & & Maritime Delimitation in the Indian Ocean (Somalia v. Kenya) (Judgment) [2017] ICJ Rep, para 43 \\
    \nopagebreak\hline
    7(1) & Yes & Villiger, p. 146 \\
    \nopagebreak\hline
    7(2)(a) & Yes & Armed Activities on the Territory of the Congo (New Application: 2002) case (Democratic Republic of Congo v Rwanda), Jurisdiction and Admissibility, Judgment [2006] ICJ Reports 6, para. 46 \\
    \nopagebreak\hline
    7(2)(c) & ``Progressive development" & Hoffmeister in Dörr \& Schmalenbach, p. 133 \\
    \nopagebreak\hline
    \multirow{2}{*}{8} & \multirow{2}{4cm}{Maybe} & Angelet \& Leidgens in Corten \& Klein, p. 157 \\ 
    & & Hoffmeister in Dörr \& Schmalenbach, p. 146 \\
    \nopagebreak\hline
    \multirow{2}{*}{9(1)} & \multirow{2}{4cm}{Yes} & Villiger, p.163 \\ 
    & & Hoffmeister in Dörr \& Schmalenbach, p. 153 \\
    \nopagebreak\hline
    9(2) & Unclear & Aust, pp.79-80 \\
    \nopagebreak\hline
    \multirow{2}{*}{10} & \multirow{2}{4cm}{Yes} & Villiger, p. 171 \\ 
    & & Hoffmeister in Dörr \& Schmalenbach, p. 165 \\
    \nopagebreak\hline
    \multirow{2}{*}{11} & \multirow{2}{4cm}{``Some customary value"} & Szurek in Corten \& Klein, p. 192 \\ 
    & & Land and Maritime Boundary (Cameroon v Nigeria) (Judgment) ICJ Rep 303, para 264 \\
    \nopagebreak\hline
    12 & Yes in its entirety & Hoffmeister in Dörr \& Schmalenbach, p. 183 \\
    \nopagebreak\hline
    12(1)(a) & Yes & Maritime Delimitation in the Indian Ocean (Somalia v. Kenya) (Judgment) [2017] ICJ Rep 1, para 45 \\
    \nopagebreak\hline
    12(1)(b) & Yes & Aust, p. 91 \\
    \nopagebreak\hline
    12(1)(c) & No & Van Assche in Corten \& Klein, pp.216-217 \\
    \nopagebreak\hline
    12(2)(a) & Yes & Van Assche in Corten \& Klein, pp.216-217 \\
    \nopagebreak\hline
    \multirow{2}{*}{12(2)(b)} & \multirow{2}{4cm}{Maybe} & Van Assche in Corten \& Klein, pp.216-217 \\ 
    & & Aust p. 91 \\
    \nopagebreak\hline
    \multirow{2}{*}{13} & \multirow{2}{4cm}{Yes} & Van Assche in Corten \& Klein, p. 257 \\ 
    & & Hoffmeister in Dörr \& Schmalenbach, p. 195 \\
    \nopagebreak\hline
    14 & Yes, but its actual content uncertain & Hoffmeister in Dörr \& Schmalenbach, p. 203 \\
    \nopagebreak\hline
    \multirow{2}{*}{15} & \multirow{2}{4cm}{Yes} & Marchi in Corten \& Klein, p. 334 \\ 
    & & Hoffmeister in Dörr \& Schmalenbach 2018 pp. 219 \\
    \nopagebreak\hline
    \multirow{3}{*}{16} & \multirow{3}{4cm}{Yes} & Land and Maritime Boundary between Cameroon v Nigeria (Preliminary Objections), [1998], p.275, para. 31 \\ 
    & & Horchani, in Corten \& Klein, p. 337 \\ 
    & & Hoffmeister in Dörr \& Schmalenbach, p. 231 \\
    \nopagebreak\hline
    17 & Yes & Hilling, in Corten \& Klein, p. 364 \\
    \nopagebreak\hline
    \multirow{3}{*}{18} & \multirow{3}{4cm}{Yes} & Greece v Commission C-203/07 P [2008] ECR I-8161, para 64 \\ 
    & & Bradley in Hollis, pp. 212-213 \\ 
    & & Summary of Practice of the Secretary-General as Depository of Multilateral Treaties, prepared by the Treaty Section of the Office of Legal Affairs, UN, 1994, ST/LEG7/Rev.1, p.61, para. 204 \\
    \nopagebreak\hline
    \multirow{2}{*}{19} & \multirow{2}{4cm}{Yes} & European Commission on Human Rights, in Temelstasch case (1983) 5 EHRR 417, p. 432 \\ 
    & & Reservations to the Convention on Genocide (Advisory Opinion) ICJ Rep 15, p. 24 \\
    \nopagebreak\hline
    20 & Yes & Müller in Corten \& Klein, p. 495 \\
    \nopagebreak\hline
    20(4) & No & Walter in Dörr \& Schmalenbach, p. 311 \\
    \nopagebreak\hline
    20(5) & No & Aust, p. 128 \\
    \nopagebreak\hline
    21(1) & Yes & Muller in Corten \& Klein, p. 542 \\
    \nopagebreak\hline
    21(2) & Yes & Walter in Dörr \& Schmalenbach, pp. 349 \\
    \nopagebreak\hline
    21(3) & No & Swaine in Hollis, p. 294 \\
    \nopagebreak\hline
    \multirow{2}{*}{22} & \multirow{2}{4cm}{Yes} & Corten \& Klein 2011, p. 574 \\ 
    & & Armed Activities on the Territory of the Congo (Democratic Republic of the Congo v Rwanda), Jurisdiction and Admissibility, (Judgment) [2006] ICJ Rep 6, para 14 \\
    \nopagebreak\hline
    \multirow{2}{*}{22(1)} & \multirow{2}{4cm}{Yes} & Commentary on guideline 2.5.1, Report of the ILC to the General Assembly, 2003, A/58/10, p.199, para. 14 \\ 
    & & Walter in Dörr \& Schmalenbach, p. 361 \\
    \nopagebreak\hline
    22(3)(a) & Yes & Armed Activities on the Territory of the Congo (New Application: 2002) (Democratic Republic of the Congo v. Rwanda), Jurisdiction and Admissibility, Judgment, ICJ Reports 2006 p.6, para. 41 \\
    \nopagebreak\hline
    23 & Yes & Pellet and Schabs in Corten \& Klein, p. 596 \\
    \nopagebreak\hline
    \multirow{2}{*}{23(1) and (4)} & \multirow{2}{4cm}{Yes} & Commentary on guideline 2.5.1, Report of the ILC to the General Assembly, 2003, A/58/10, p. 199, para. 14 \\ 
    & & Walter in Dörr \& Schmalenbach, p. 380 \\
    \nopagebreak\hline
    23(2) and (3) & Yes & Walter in Dörr \& Schmalenbach, p. 380 \\
    \nopagebreak\hline
    \multirow{2}{*}{24} & \multirow{2}{4cm}{Yes} & Land and Maritime Boundary between Cameroon and Nigeria (Preliminary Objections) [1998] ICJ Rep 275, para. 31 \\ 
    & & Krieger in Dörr \& Schmalenbach, p. 425 \\
    \nopagebreak\hline
    25(1) & Yes & Villiger, p. 357 \\
    \nopagebreak\hline
    25(2) & Maybe & Mathy in Corten \& Klein, pp.640-641 \\
    \nopagebreak\hline
    \multirow{3}{*}{26} & \multirow{3}{4cm}{Binding as a general principle of IL} & Gabcikovo-Nagymaros Projects Case (Hungary v Slovakia) ICJ Reports 1997 p.7, para. 142 \\ 
    & & Pulp Mills on the River Uruguay (Argentina v. Uruguay) (Judgment) ICJ Rep 135, para 145 \\ 
    & & Boustany \& Didat in Corten \& Klein, p. 705 \\
    \nopagebreak\hline
    \multirow{4}{*}{27} & \multirow{4}{4cm}{Yes} & Certain Questions of Mutual Assistance in Criminal Matters (Djibouti v France) (Judgment) [2008] ICJ Rep 177, para. 124 \\ 
    & & ELSI (United States v Italy) (Judgment) [1989] ICJ Rep 15 \\ 
    & & Fisheries Case (UK v Norway) [1951] ICJ Rep 116, 132 \\ 
    & & Questions relating to the Obligation to prosecute or extradite (Belgium v Senegal) (Judgment) [2012] ICJ Rep 423, para 100 \\
    \nopagebreak\hline
    \multirow{2}{*}{28} & \multirow{2}{4cm}{Yes} & Application of the Convention on the Prevention and Punishment of the Crime of Genocide (Croatia v. Serbia) (Judgment) [2015] ICJ Rep 3, para 95 \\ 
    & & Case of Janowiec and Others v Russia [2013], ECHR 1003, para 121 \\
    \nopagebreak\hline
    \multirow{3}{*}{29} & \multirow{3}{4cm}{Likely yes} & Karagiannis in Corten \& Klein, p. 735 \\ 
    & & Council v Front Polisario [2013] ECJ C-104/16 P 973, para 95 \\ 
    & & von der Decken in Dörr \& Schmalenbach, p. 522 \\
    \nopagebreak\hline
    \multirow{3}{*}{30} & \multirow{3}{4cm}{Uncertain} & Aust, p. 228 \\ 
    & & von der Decken in Dörr \& Schmalenbach, p. 542 \\ 
    & & Orakhelashvili in Corten \& Klein, p. 774 \\
    \nopagebreak\hline
    \multirow{6}{*}{31} & \multirow{6}{4cm}{Yes} & Dispute regarding Navigational and Related Rights (Costa Rica v. Nicaragua) (Judgment) [2009] ICJ Rep 213, para. 47 \\ 
    & & Application of the Convention on the Prevention and Punishment of the Crime of Genocide (Bosnia and Herzegovina v. Serbia and Montenegro), (Judgment) [2007] ICJ Rep 43, para. 160 \\ 
    & & Arbitral Award (Guinea-Bissau v Senegal) [1991] ICJ Rep 53, para 48 \\ 
    & & Legal Consequences of the Construction of a Wall in the Occupied Palestinian Territory (Advisory Opinion) [2004] ICJ Rep 136, para. 94 \\ 
    & & Territorial Dispute (Libyan Arab Jamahiriya v Chad) (Judgment) ICJ Rep 83, para. 41 \\ 
    & & Sovereignty over Pulau Ligitan and Pulau Sipadan (Indonesia/Malaysia) (Judgment) [2002] ICJ Rep 645, para. 37 \\
    \nopagebreak\hline
    \multirow{2}{*}{32} & \multirow{2}{4cm}{Yes} & Same as Article 31 \\ 
    & & Le Bouthillier in Corten \& Klein, p. 846 \\
    \nopagebreak\hline
    \multirow{2}{*}{33} & \multirow{2}{4cm}{Yes, especially 33(4)} & LaGrand (Germany v. United States of America), (Judgment), ICJ Rep 466, para. 101 \\ 
    & & Dörr in Dörr \& Schmalenbach, p. 63 \\
    \nopagebreak\hline
    \multirow{3}{*}{34} & \multirow{3}{4cm}{Yes} & Harris and Sivakumaran, Cases and Materials on International Law (8th ed, 2015) p. 687 \\ 
    & & David in Corten \& Klein, pp.888-889 \\ 
    & & Proelss in Dörr \& Schmalenbach, p. 657 \\
    \nopagebreak\hline
    \multirow{2}{*}{35} & \multirow{2}{4cm}{Yes} & Laly-Chevalier in Corten \& Klein, p. 903 \\ 
    & & Question of the Delimitation of the Continental Shelf between Nicaragua and Colombia beyond 200 nautical miles from the Nicaraguan Coast (Nicaragua v. Colombia) (Preliminary Objections) [2016] ICJ Rep 100, (Separate opinion of Judge Owada) p. 154, para 34 \\
    \nopagebreak\hline
    \multirow{2}{*}{36(1)} & \multirow{2}{4cm}{Maybe} & D'Argent in Corten \& Klein, pp. 930-931 \\ 
    & & Proelss in Dörr \& Schmalenbach, p.732 \\
    \nopagebreak\hline
    36(2) & Probably & D'Argent in Corten \& Klein, p.944 \\
    \nopagebreak\hline
    37 & Maybe & Gaja in Corten \& Klein, p. 949 \\
    \nopagebreak\hline
    38 & Maybe & Proelss in Dörr \& Schmalenbach, p. 94 \\
    \nopagebreak\hline
    \multirow{2}{*}{39} & \multirow{2}{4cm}{Maybe} & Sands in Corten \& Klein, p. 968 \\ 
    & & Von der Decken in Dörr \& Schmalenbach, p. 761 \\
    \nopagebreak\hline
    \multirow{3}{*}{40} & \multirow{3}{4cm}{Yes, but diverging opinions} & Summary of Practice of the Secretary-General as Depository of Multilateral Treaties, prepared by the Treaty Section of the Office of Legal Affairs, UN, 1994, ST/LEG7/Rev.1, p.76, para. 252 \\ 
    & & von der Decken in Dörr \& Schmalenbach, p. 769 \\ 
    & & Ardault and Dormoy in Corten \& Klein, p. 980 \\
    \nopagebreak\hline
    \multirow{2}{*}{41} & \multirow{2}{4cm}{Likely} & Jadhav Case (India v. Pakistan) Provisional Measures, Order of 18 May 2017, [2017] ICJ Rep 231, Declaration of Judge Bhandari p. 5, para 12 \\ 
    & & Rignaux et al in Corten \& Klein, p. 994 \\
    \nopagebreak\hline
    \multirow{2}{*}{42} & \multirow{2}{4cm}{Yes} & Gabčikovo-Nagymaros (Hungary v Slovakia) [1997] ICJ Rep 92, para. 100 \\ 
    & & von der Decken in Dörr \& Schmalenbach, p. 794 \\
    \nopagebreak\hline
    \multirow{3}{*}{43} & \multirow{3}{4cm}{Yes} & Military and Paramilitary Activities in and against Nicaragua (Nicaragua v United States) (Merits) [1986] ICJ Rep 14, para 178 \\ 
    & & Tehran Hostages Case (United States v Iran) [1980] ICJ Rep 3, para 62 \\ 
    & & von der Decken in Dörr \& Schmalenbach, p. 816 \\
    \nopagebreak\hline
    44 & Yes & Falkowska in Corten \& Klein, p. 1053 \\
    \nopagebreak\hline
    \multirow{2}{*}{45} & \multirow{2}{4cm}{Yes} & Certain Norwegian Loans (France v Norway) Separate Opinion of Judge Lauterpacht [1956] ICJ, pp 56-7 \\ 
    & & Maritime Delimitation in the Indian Ocean (Somalia v. Kenya) (Judgment) [2017] ICJ Rep 161, para 49 \\
    \nopagebreak\hline
    46 & Yes & Von der Decken in Dörr \& Schmalenbach, p. 828 \\
    \nopagebreak\hline
    47 & Yes & Rensmann in Dörr \& Schmalenbach, p. 866 \\
    \nopagebreak\hline
    \multirow{3}{*}{48} & \multirow{3}{4cm}{Yes} & Rensmann in Dörr \& Schmalenbach, p. 878 \\ 
    & & Phillips Petroleum Co v Iran Case No 39, (1982) 70 ILR 483 \\ 
    & & Joe Verhoeven, 'Invalidity of Treaties: Anything New in/under the Vienna Conventions' in Enzo Cannizzaro (ed), The Law of Treaties Beyond the Vienna Convention, (Oxford Scholarship Online 2011) pp.302-303 \\
    \nopagebreak\hline
    49 & No & Rensmann in Dörr \& Schmalenbach, p. 913 \\
    \nopagebreak\hline
    \multirow{2}{*}{50} & \multirow{2}{4cm}{No} & Rensmann in Dörr \& Schmalenbach, p. 921 \\ 
    & & Cot in Corten \& Klein, p. 1173 \\
    \nopagebreak\hline
    \multirow{3}{*}{51} & \multirow{3}{4cm}{Yes} & Distefano in Corten \& Klein, p. 1185 \\ 
    & & Rensmann in Dörr \& Schmalenbach, p. 934 \\ 
    & & Dubai-Sharjah Border Arbitration (1981) 91 ILR 543, 569 \\
    \nopagebreak\hline
    \multirow{4}{*}{52} & \multirow{4}{4cm}{Yes} & Fisheries Jurisdiction (United Kingdom v. Iceland) (Jurisdiction) ICJ Rep 3, para.24 \\ 
    & & Genocide Case (Further Requests for the Indication of Provisional Measures) (separate opinion Lauterpacht) [1993] ICJ Rep 407, para 100 \\ 
    & & Schmalenbach in Dörr \& Schmalenbach, p. 959 \\ 
    & & Corten in Corten \& Klein, p. 1204 \\
    \nopagebreak\hline
    \multirow{5}{*}{53} & \multirow{5}{4cm}{Yes} & Lagerwall in Corten \& Klein, p. 1465 \\ 
    & & Armed Activities on the Territory of the Congo (Democratic Republic of the Congo v Rwanda) (Separate Opinion of Judge ad hoc Dugard) [2006] ICJ Rep 6, para. 8 \\ 
    & & Suy in Corten \& Klein, p. 1226 \\ 
    & & Schmalenbach in Dörr \& Schmalenbach, p. 966 \\ 
    & & Prosecutor v Morris Kallon and Brima Bazzy Kamara, (2004) case no. SCSL-2004-15 AR72 and no. SCSL-2004-16-AR72, paras 61-62 \\
    \nopagebreak\hline
    \multirow{2}{*}{54(a)} & \multirow{2}{4cm}{Yes} & Giegerich in Dörr \& Schmalenbach, p. 1018 \\ 
    & & Chapaux in Corten \& Klein, p. 1238 \\
    \nopagebreak\hline
    54(b) & Maybe & Chapaux in Corten \& Klein, p. 1240-41 \\
    \nopagebreak\hline
    \multirow{3}{*}{55} & \multirow{3}{4cm}{No} & Chapaux in Corten \& Klein, p.1247-48 \\ 
    & & Military and Paramilitary Activities in and against Nicaragua (Jurisdiction and Admissibility) [1984] ICJ Rep 392, para 60 \\ 
    & & Giegerich in Dörr \& Schmalenbach, p. 1058 \\
    \nopagebreak\hline
    56(1)(a) & Yes & Giegerich in Dörr \& Schmalenbach, p. 1063 \\
    \nopagebreak\hline
    56(2) & Yes but content uncertain & Giegerich in Dörr \& Schmalenbach, p. 1077 \\
    \nopagebreak\hline
    57 & No & Giegerich in Dörr \& Schmalenbach, p. 1082 \\
    \nopagebreak\hline
    58(1)(a) & Yes & Giegerich in Dörr \& Schmalenbach, p. 1084 \\
    \nopagebreak\hline
    58(1)(b) & Maybe & Giegerich in Dörr \& Schmalenbach, p. 1084 \\
    \nopagebreak\hline
    58(2) & Yes & European Communities-Measures Affecting the Importation of Certain Poultry Products, WT/DS69/R (12 March 1998) (Report of the Panel), para 206 \\
    \nopagebreak\hline
    59 & Yes & Giegerich in Dörr \& Schmalenbach, p. 1128 \\
    \nopagebreak\hline
    \multirow{3}{*}{60} & \multirow{3}{4cm}{Yes ``in many respects"} & Legal Consequences for the States of the Continued Presence of South Africa in Namibia (South West Africa) (Advisory Opinion) ICJ Rep 47, para 95 \\ 
    & & Gabčikovo-Nagymaros (Hungary v Slovakia) [1997] ICJ Rep 7, para 99, 109 \\ 
    & & Simma and Tams in Corten \& Klein, pp. 1356-1357 \\
    \nopagebreak\hline
    61 & Yes ``in many respects" & Gabčikovo-Nagymaros (Hungary v Slovakia) [1997] ICJ Rep 92, p.7, paras. 99 and 102 \\
    \nopagebreak\hline
    \multirow{3}{*}{62} & \multirow{3}{4cm}{Yes, ``in many respects"} & Fisheries Jurisdiction (United Kingdom v. Iceland) (Jurisdiction), ICJ Reports 1973 p.3, para. 36 \\ 
    & & Gabčikovo-Nagymaros (Hungary v Slovakia) [1997] ICJ Rep 92, para. 46 and 99 \\ 
    & & Shaw and Fournet in Corten \& Klein, p. 1416-19 \\
    \nopagebreak\hline
    \multirow{3}{*}{63} & \multirow{3}{4cm}{Yes} & Giegerich in Dörr \& Schmalenbach, p. 1197 \\ 
    & & United States Diplomatic and Consular Staff in Tehran (United States of America v Iran) (Merits) [1980] ICJ Rep 3, p.28, para. 54 \\ 
    & & Angelet in Corten \& Klein, p. 1440 \\
    \nopagebreak\hline
    \multirow{3}{*}{64} & \multirow{3}{4cm}{Yes} & Gabčikovo-Nagymaros (Hungary v Slovakia) [1997] ICJ Rep 92, para 112 \\ 
    & & Schmalenbach in Dörr \& Schmalenbach, p. 1206-7 \\ 
    & & Lagerwall in Corten \& Klein, p. 1459 and 1465 \\
    \nopagebreak\hline
    \multirow{4}{*}{65} & \multirow{4}{4cm}{Yes, ``generally reflect[s] customary international law"} & Fisheries Jurisdiction (United Kingdom v. Iceland) (Jurisdiction) ICJ Rep 3, para 44 \\ 
    & & Gabčikovo-Nagymaros (Hungary v Slovakia) [1997] ICJ Rep 92, para 109 \\ 
    & & Krieger in Dörr \& Schmalenbach, p. 1213 \\ 
    & & Prost in Corten \& Klein, p. 1490 and 1498 \\
    \nopagebreak\hline
    \multirow{3}{*}{66} & \multirow{3}{4cm}{Mixed views} & ``[G]enerally reflect[s]" CIL: Gabčikovo-Nagymaros Project (Hungary/Slovakia) [1997] ICJ Rep 7, para 109 \\ 
    & & Not CIL: Armed Activities Case (DRC v Rwanda) (Jurisdiction and Admissibility) [2006] ICJ Rep 6, para 125 \\ 
    & & Krieger in Dörr \& Schmalenbach, p. 1234 \\
    \nopagebreak\hline
    \multirow{2}{*}{67} & \multirow{2}{4cm}{Yes, ``generally reflect[s] customary international law"} & Gabčikovo-Nagymaros Project (Hungary/Slovakia) [1997] ICJ Rep 7, para 109 \\ 
    & & Tzanakopoulos in Corten \& Klein, p. 1548-9 \\
    \nopagebreak\hline
    \multirow{2}{*}{68} & \multirow{2}{4cm}{Yes, ``in many respects"} & Legal Consequences for the States of the Continued Presence of South Africa in Namibia (South West Africa) (Advisory Opinion) ICJ Rep 47, para. 94 \\ 
    & & Tzanakopoulos in Corten \& Klein, p. 1565 \\
    \nopagebreak\hline
    \multirow{2}{*}{70} & \multirow{2}{4cm}{Yes} & Appeal Relating to the Jurisdiction of the ICAO Council (India v Pakistan) [1972] ICJ Rep 46, 54, para 16 \\ 
    & & Wittich in Dörr \& Schmalenbach, p. 1299 \\
    \nopagebreak\hline
    \multirow{2}{*}{71} & \multirow{2}{4cm}{Unclear} & Crépeau, Côté and Rehaag in Corten \& Klein, p. 1615 \\ 
    & & Wittich in Dörr \& Schmalenbach, p. 1317 \\
    \nopagebreak\hline
    \multirow{2}{*}{72} & \multirow{2}{4cm}{Unclear} & Couveur and Espalie Berdud in Corten \& Klein, p. 1628 \\ 
    & & Wittich in Dörr \& Schmalenbach, p. 1326 \\
    \nopagebreak\hline
    \multirow{3}{*}{73-75} & \multirow{3}{4cm}{Generally not CIL} & Krieger in Dörr \& Schmalenbach, p. 94 \\ 
    & & Angelet and Clave in Corten \& Klein, p. 1678 \\ 
    & & Tomuschat in Corten \& Klein, p. 1689 \\
    \nopagebreak\hline
    \multirow{3}{*}{76 \& 77} & \multirow{3}{4cm}{Yes} & Tichy and Bittner in Dörr \& Schmalenbach, p. 1410, 1414, 1416 \\ 
    & & Caflisch in Corten \& Klein, p. 170 \\ 
    & & Ouguergouz, Villalpando \& Morgan-Foster in Corten \& Klein 2011, p. 1717 \\
    \nopagebreak\hline
    \multirow{2}{*}{78 \& 79} & \multirow{2}{4cm}{No progressive development} & Tichy \& Bittner in Dörr \& Schmalenbach, p. 1433, 1439 \\ 
    & & Kolb in Corten \& Klein , p. 1779, 1782 \\
    \nopagebreak\hline
    80 & Yes & Klein in Corten \& Klein , p. 1799 \\
    \nopagebreak\hline
    81-85 & No & Final Provisions \\
    \nopagebreak
\end{longtable}}