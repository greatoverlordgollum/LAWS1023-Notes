\label{sec:Topic 11}
\begin{tcolorbox}
    \textbf{\textit{This section of the State Responsibility topic is concerned with diplomatic protections.}}
\end{tcolorbox}

\begin{itemize}
    \item The theory of responsibility refers to the mistreatment of a foreign national being an injury to the national state, which is regarded as the protector of its citizens on the plane of public international law and thus may exercise diplomatic protection
    \item It was mentioned in \case{\textit{Panevezys-Saldutiskis Railway Case} (1938) PCIJ (ser A/B) No 76} that ``by resorting to diplomatic action or international judicial proceedings on his behalf, a State is in reality asserting its own right, the right to ensure in the person of its nationals respect for the rules of international law''
    \item International law does not regulate the treatment of foreign nationals in all areas (e.g., lawful discrimination in relation to voting, employment in the public service and ownership of property are not mistreatments of foreign nationals)
    \item This area tends to have primary relevance to the treatment of foreign nationals in connection with criminal prosecution and custody, admission and expulsion, and expropriation of property
    \item John Dugard held that ``Diplomatic protection provides a potential remedy for the protection of millions of aliens who have no access to remedies before international bodies and a more effective remedy to those who have access to the often ineffectual remedies contained in international human rights instruments."
    \item Ultimately, diplomatic protection is based on the theory that states are injured when their nationals are injured, and so the states are entitled to complain about this if they wish to
\end{itemize}

\section{Discretionary Nature of Diplomatic Protection}
\begin{itemize}
    \item Diplomatic protection refers to measures that resort to diplomatic action or another means of peaceful settlement by a state adopting in its own right the cause of its national in respect of an injury to that national caused by an intentionally wrongful act of another state
    \item This is distinct from the immunity granted to diplomats
    \item States can resort to any form of diplomatic action, which can range from a simple diplomatic communication to the commencement of proceedings in an international court or tribunal if there is jurisdiction and the invoking state has standing
    \item If a national is harmed by a foreign state, their home state may elect to take up a claim if they wish to do so, but they are not obliged to do so
    \item The laws on diplomatic protection are largely governed by the \convention{\textit{2006 Draft Articles on Diplomatic Protection}}
    \begin{itemize}
        \item The term `Draft' is used to indicate that the articles are complete but may not have the binding force of a treaty or another instrument of international law
    \end{itemize}
    \item A core principle of international law is that states are not obliged to exercise diplomatic protection, and they may choose to do so or not, following the approach taken in \case{\textit{Barcelona Traction} [1970] ICJ Rep 3}
\end{itemize}

\begin{casedetails}{\textit{Barcelona Traction} [1970] ICJ Rep 3}
    \flushleft
    The State must be viewed as \textbf{the sole judge to decide whether its protection will be granted, to what extent it is granted, and when it will cease}. It remains in this respect a discretionary power the exercise of which may be determined by considerations of a political or other nature, unrelated to the particular case. \textbf{Since the claim of the State is not identical with that of the individual or corporate person whose cause is espoused, the State enjoys complete freedom of action}.
\end{casedetails}

\begin{itemize}
    \item Whilst under international law states are not obliged to exercise diplomatic protection, they may be compelled to do so by their own domestic law, as seen in the cases of \case{\textit{Abbasi v Secretary of State} [2002] EWCA Civ 1598} and \case{\textit{Hicks v Ruddock} (2007) 156 FCR 574}
\end{itemize}

\begin{casedetails}{\textit{Abbasi v Secretary of State} [2002] EWCA Civ 1598}
    \flushleft
    This case concerned the detention of a British citizen, Mr. Abbasi, in Guantanamo Bay. The Court of Appeal held that whilst the UK government had no obligation under international law or domestic law to protect their national generally, they had a duty to take steps to secure his release under administrative law, as it was an expectation that the UK government would at least \textit{consider} making representations on his behalf to the US. By characterising this as an issue of administrative law, the court held that it was expected that the government would make those representations. In this case, it was determined on the evidence that the UK had satisfied the requirements under administrative law, and so Abbasi's claim was unsuccessful.
\end{casedetails}

\begin{casedetails}{\textit{Hicks v Ruddock} (2007) 156 FCR 574}
    \flushleft
    In this case, David Hicks sought an order of judicial review of a decision made by the Australian government not to request for his release and return by the US from Guantanamo Bay. Hicks sought for \gls{habeus corpus} and for a judicial review, which raise a similar question of administrative law as in \case{\textit{Abbasi v Secretary of State} [2002] EWCA Civ 1598}.

    \vspace{\baselineskip}

    Whilst these proceedings were commencing, a deal was reached, and Hicks was repatriated to Australia. In the early stages of the proceedings, the government submitted a motion seeking to dismiss the proceedings on the grounds that Hicks had no reasonable prospects of success, including by Tamberlin J holding that the Foreign Act of State doctrine (Section \ref{sec:Foreign Act of State Doctrine} on Page \pageref{sec:Foreign Act of State Doctrine}) should not apply in cases of a breach of a fundamental rule of international law, such as what occurred in this case.
    
    \vspace{\baselineskip}
    
    The court was not able to consider the judicial review point due to the deal made that saw Hicks released, and so the argument made in the UK in \case{\textit{Abbasi v Secretary of State} [2002] EWCA Civ 1598} was not able to be evaluated in the context of Australian law. As such, there is no conclusive determination in Australian law on whether a similar argument of an exception to a state's choice to exercise diplomatic protection exists under administrative law and its expectation of due consideration.
\end{casedetails}

\section{Mistreatment of Foreign Nationals}
\subsection{Standard of Treatment}
\begin{itemize}
    \item This section broadly considers how bad the treatment of a foreign national must be before it can be considered a mistreatment that is actionable by the home state
    \item There are two prevailing standards of treatment: national treatment (preferred by developing states) and international minimum standard (preferred by developed states)
    \begin{itemize}
        \item \textbf{National Treatment} refers to the notion that there is no mistreatment of a foreign national if they are treated the same as a local citizen of that nation
        \item \textbf{International Minimum Standard} refers to the notion that irrespective of the laws of a nation, there is a certain minimum standard of treatment that foreign nationals can expect when in the territory of another state
    \end{itemize}
    \item The international minimum standard is described in the following cases
\end{itemize}

\begin{casedetails}{\textit{Neer v Mexico} (1926) 4 RIAA 60}
    \flushleft
    This case concerned the killing of a US national, Paul Neer, by a group of armed men in Mexico, with attempts to prosecute the culprits being unsuccessful. The Commission held that:

    \begin{quote}
        [It] recognises the difficulty of devising a general formula for determining the boundary between an international delinquency of this type and an unsatisfactory use of power included in national sovereignty. It is ... possible to hold first that the propriety of governmental acts should be put to the test of international standards and second that the treatment of an alien in order to constitute an international delinquency should amount to an outrage to bad faith to wilful neglect of duty or to an insufficiency of governmental action so far short of international standards that every reasonable and impartial person would readily recognise its insufficiency.
    \end{quote}

    In this case, a question arose as to whether there was mistreatment on account of the failure of the Mexican authorities to take appropriate action to prosecute the culprits of the offence. Whilst this was difficult to enunciate, the Commission held that the answer to this question should be held to an international standard, which is a relatively undemanding standard.
\end{casedetails}

\begin{casedetails}{\textit{Quintanilla} (1926) 4 RIAA 60}
    \flushleft
    This case concerned a Mexican national, Francisco Quintanilla, who had been taken into custody by US authorities following an alleged assault and was later found dead beside the road. However, the US authorities had no records of what had happened to him.

    \begin{quote}
        The government can be held liable if it is proven that it has treated him [the foreign national] cruelly, harshly, unlawfully; so much more it is liable if it can say only that it took him into custody…and that it ignores what happened to him.
    \end{quote}

    From this case, it is seen that a state must account for the safety and whereabouts of a foreign national.
\end{casedetails}

\begin{casedetails}{\textit{Loewen Group v US} (2006) NAFTA Arbitration Tribunal}
    \flushleft
    This case surrounded a Canadian company who was standing trial in the US for claims made in tort, and for other claims. They were a company that purchased funeral homes in the US at a price that was far below market value.
    
    \vspace{\baselineskip}

    The legal basis of the US' argument was never clear (it was both contractual and tortious, for example), but Loewen Group nonetheless lost very badly in the US courts, and was subject to damages at \$500 million. Loewen Group then took the US to an arbitral tribunal under NAFTA, where there were protections for investments made by foreign nationals.

    \vspace{\baselineskip}

    The tribunal considered the international minimum standard of treatment, as stated in Art 1105 of the NAFTA treaty. It held that the international minimum standards had been violated, as the US trials were a mess and were clearly biased against the Loewen group, rendering the verdict improper and discreditable.
    
    \begin{quote}
        The trial court permitted the jury to be influenced by persistent appeals to local favouritism as against a foreign litigant…the whole trial and its resultant verdict were clearly improper and discreditable and cannot be squared with minimum standards of treatment and fair and equitable treatment.
    \end{quote}
\end{casedetails}

\section{Admissibility Requirements}
\begin{conventiondetails}{\textit{2001 Articles on State Responsibility for Internationally Wrongful Acts} Article 44}
    \flushleft
    \textit{Admissibility of claims}

    \vspace{\baselineskip}

    The responsibility of a State may not be invoked if:
    \begin{enumerate}[label=(\alph*)]
        \item the claim is not brought in accordance with any applicable rule relating to the nationality of claims; 
        \item the claim is one to which the rule of exhaustion of local remedies applies and any available and effective local remedy has not been exhausted. 
    \end{enumerate}
\end{conventiondetails}

\begin{itemize}
    \item \convention{\textit{ARSIWA} Art 44(a)} sets out the requirement that an injured person must be a national of the claiming state
    \item \convention{\textit{ARSIWA} Art 44(b)} holds that before a state takes up a claim on behalf of a national, that national must have exhausted all local remedies available to them in the state that they were injured in
\end{itemize}

\section{Nationality of Claims Requirement}
\begin{itemize}
    \item States may only exercise a right of diplomatic protection in relation to their nationals (natural persons, or legal persons such as corporations)
\end{itemize}

\begin{casedetails}{\textit{Re (Al Rawi and Others) Secretary of State} [2006] EWCA Civ 1279}
    \flushleft
    This case concerned an initial refusal of the UK to intervene with regard to British non-citizen residents detained by the US in Guantanamo Bay.

    \begin{quote}
        It is the long-standing policy of the UK Government not to offer consular or similar assistance to non-British Nationals…It should also be noted that the UK does not have the right to exercise diplomatic protection [in respect of non-nationals].
    \end{quote}
\end{casedetails}

\begin{itemize}
    \item Generally, it is a matter for states to determine the conditions for the grant of nationality, whether it's by place of birth (\textit{jus soli}), by descent (\textit{jus sanguinis}), or by naturalisation, with international law taking a hands-off approach to this
    \item Corporations generally have the nationality of the place of incorporation
\end{itemize}

\begin{conventiondetails}{\textit{2006 Draft Articles on Diplomatic Protection} Article 4}
    \flushleft
    \textit{State of nationality of a natural person}

    \vspace{\baselineskip}

    For the purposes of the diplomatic protection of a natural person, a State of nationality means a State whose nationality that person has acquired, in accordance with the law of that State, by birth, descent, naturalization, succession of States or in any other manner, not inconsistent with international law.
\end{conventiondetails}

\begin{conventiondetails}{\textit{2006 Draft Articles on Diplomatic Protection} Article 8}
    \flushleft
    \textit{Stateless persons and refugees}

    \begin{enumerate}
        \item A State may exercise diplomatic protection in respect of a stateless person who, at the date of injury and at the date of the official presentation of the claim, is lawfully and habitually resident in that State.
        \item A State may exercise diplomatic protection in respect of a person who is recognized as a refugee by that State, in accordance with internationally accepted standards, when that person, at the date of injury and at the date of the official presentation of the claim, is lawfully and habitually resident in that State.
        \item Paragraph 2 does not apply in respect of an injury caused by an internationally wrongful act of the State of nationality of the refugee.
    \end{enumerate}
\end{conventiondetails}

\begin{itemize}
    \item \convention{\textit{2006 Draft Articles on Diplomatic Protection} Art 4} holds that it is a matter for states to determine the conditions for the grant of nationality
    \begin{itemize}
        \item However, issues can arise as to the closest nationality if an individual has multiple nationalities, or if they are stateless or a refugee; otherwise, international law takes a hands-off approach
    \end{itemize}
    \item \convention{\textit{2006 Draft Articles on Diplomatic Protection} Art 8} enables the state of habitual residence of a stateless person or refugee to take up a claim on their behalf, thereby plugging the gap that otherwise arises with respect to stateless people, and ensuring they are not exempted from the law on diplomatic protection
\end{itemize}

\begin{casedetails}{\textit{Nottebohm Case (Liechtenstein v Guatemala)} [1955] ICJ Rep 4}
    \flushleft
    This case concerned a German national, Friedrich Nottebohm, who had been living in Guatemala for many years and had established a successful business, and had acquired Liechtenstein's citizenship, despite only having a brother who was a resident of Lichtenstein with no other links to the country. Nottebohm subsequently returned to Guatemala, and in 1943, his property was seized as part of war measures, and he was arrested and delivered to US authorities. At the end of World War II, Nottebohm was refused re-entry to Guatemala, and Liechtenstein took up his case under diplomatic protection, on the basis of mistreat.

    \vspace{\baselineskip}

    The ICJ held by 13:1 that the claim was inadmissible, as the nationality of claims requirement had not been satisfied. Whilst it was indeed for Liechtenstein to determine its rules relating to the acquisition of nationality, the ICJ held that it was not necessary to determine whether international law in general imposes any restrictions on states conferring nationality. It is international law which determines whether a state is entitled to exercise diplomatic protection, and to determine the jurisdiction of the court.

    \begin{quote}
       According to the practice of states, to arbitral and judicial decisions and to the opinion of writers, nationality is a legal bond having as its basis a social fact of attachment, a genuine connection of existence, interests and sentiments, together with the existence of reciprocal rights and duties. 
    \end{quote}

    Nottebohm's connection with Liechtenstein was tenuous at best, and the facts establish an absence of any bond of attachment with Liechtenstein, but on the contrary also demonstrate a long-standing and close connection with Guatemala. Generally, a genuine/effective link does not need to be established, but the law surrounding nationality is instead a relative one.
\end{casedetails}

\begin{itemize}
    \item In a postscript to \case{\textit{Nottebohm Case (Liechtenstein v Guatemala)} [1955] ICJ Rep 4}, the \article{\textit{ILC Commentary to 2006 Draft Articles on Diplomatic Protection}} held that
    \begin{quote}
        Article 4 does not require a State to prove an effective or genuine link between itself and its national, along the lines suggested in the Nottebohm case, as an additional factor for the exercise of diplomatic protection, even where the national possesses only one nationality ... the Court did not intend to expound a general rule applicable to all states but only a relative rule according to which a state in Liechtenstein's position was required to show a genuine link between itself and Mr Nottebohm in order to permit it to claim on his behalf against Guatemala with whom he had extremely close ties.
    \end{quote}
\end{itemize}

\subsection{Multiple Nationalities}
\begin{itemize}
    \item There are three situations that can arise with respect to individuals with multiple nationalities, as set out in \convention{\textit{2006 Draft Articles on Diplomatic Protection} Articles 6-7}:
    \begin{itemize}
        \item If they are a dual national, then both states can exercise diplomatic protection over an individual, following \convention{\textit{2006 Draft Articles on Diplomatic Protection} Article 6}
        \item If they are a dual national, if one of those states injures the individual then the other state can exercise diplomatic protection, following \convention{\textit{2006 Draft Articles on Diplomatic Protection} Article 7}
        \begin{itemize}
            \item However, an exception applies if a nationality is predominant both at the date of the injury and of the claim, with the state of predominant connection then being able to make the claim, following \convention{\textit{2006 Draft Articles on Diplomatic Protection} Article 7}
        \end{itemize}
    \end{itemize}
\end{itemize}

\begin{conventiondetails}{\textit{2006 Draft Articles on Diplomatic Protection} Article 6}
    \flushleft
    \textit{Multiple nationality and claim against a third State}

    \begin{enumerate}
        \item Any State of which a dual or multiple national is a national may exercise diplomatic protection in respect of that national against a State of which that person is not a national. 
        \item Two or more States of nationality may jointly exercise diplomatic protection in respect of a dual or multiple national.
    \end{enumerate}
\end{conventiondetails}

\begin{conventiondetails}{\textit{2006 Draft Articles on Diplomatic Protection} Article 7}
    \flushleft
    \textit{Multiple nationality and claim against a State of nationality}

    \vspace{\baselineskip}

    A State of nationality may not exercise diplomatic protection in respect of a person against a State of which that person is also a national unless the nationality of the former State is predominant, both at the date of injury and at the date of the official presentation of the claim.
\end{conventiondetails}

\section{Diplomatic Protection of Corporations}
\begin{conventiondetails}{\textit{2006 Draft Articles on Diplomatic Protection} Article 9}
    \flushleft
    \textit{State of nationality of a corporation}
    
    \vspace{\baselineskip}

    For the purposes of the diplomatic protection of a corporation, the State of nationality means the State under whose law the corporation was incorporated. However, when the corporation is controlled by nationals of another State or States and has no substantial business activities in the State of incorporation, and the seat of management and the financial control of the corporation are both located in another State, that State shall be regarded as the State of nationality.
\end{conventiondetails}

\begin{itemize}
    \item The test in \convention{\textit{2006 Draft Articles on Diplomatic Protection} Article 9} is a test of substance
\end{itemize}


\begin{casedetails}{\textit{Barcelona Traction Case (Belgium v Spain)} [1970] ICJ Rep 3}
    \flushleft
    Barcelona Traction was incorporated under Canadian law in 1911, whilst its main business was developing electricity supplies in Spain. The company was declared bankrupt by a Spanish court in 1948, resulting in financial losses for the company's shareholders, as well as damaging the company and entailing significant financial losses for the company. 88\% of the shareholders of Barcelona Traction were Belgian nations, and so Belgium sought to exercise diplomatic protection on behalf of the shareholders and the company, after Canada declined to take up such a claim. Spain objected that Belgium did not have standing, and that the claim was inadmissible because of a failure to meet the nationality of claims requirement. Belgium argued that they had standing as the majority of the shareholders were of Belgian nationality.

    \vspace{\baselineskip}

    When a state admits foreign investments or foreign nationals to its territory, whether natural or legal persons, it is bound to extend to them the protection of the law and assumes obligations concerning the treatment to be afforded to them. However, these obligations to the nationals of other states are not \gls{erga omnes} obligations.

    \vspace{\baselineskip}

    In this case, it is essential to establish whether the losses by the Belgian shareholders were suffered as a result of a breach of an obligation of which Belgium is a beneficiary. In deciding this, the court must consider the international legal significance of a domestic legal institution (i.e., the company). Here, the concept and structure of a limited liability company is based on the legal distinction between the company and the shareholder. Despite this separate legal personality, a wrong done to a company often causes damage to shareholders.

    \vspace{\baselineskip}

    However, the mere fact that damage is sustained by both the company and the shareholders does not mean that both are entitled to claim compensation; if a shareholder's interests are harmed by an injury to the company then the shareholder must look to the company to institute proceedings. However, the situation is different if the act complained of is aimed at the direct rights of the shareholders \gls{qua} shareholders, such as the right to a dividend, the right to attend company meetings, the right to a share of the assets upon liquidation, etc.

    \vspace{\baselineskip}

    The lifting of the corporate veil is an exceptional event in domestic and international law, but there are two situations in which the shareholder state could act:
    \begin{enumerate}
        \item Where the company ceases to exist (in which case shareholders can act) - but in this case, while the company has lost all of its assets in Spain, and has been placed into receivership in Canada, it is not been liquidated and does have the capacity to take corporate action
        \item The company's national state lacks capacity to take action; allocating corporate entities to states for purposes of diplomatic protection analogous with rules for nationality of individuals, however no absolute test of genuine connection has been found
        \begin{itemize}
            \item In this case, the company had been incorporated in Canada for over 50 years, and Canada did make numerous representations to Spain, remembering that diplomatic protection is entirely discretionary
            \item The codification of this second principle is harder to apply, and hasn't been codified in \convention{\textit{2006 Draft Articles on Diplomatic Protection} Arts 11-12}
        \end{itemize}
    \end{enumerate}
\end{casedetails}

\begin{itemize}
    \item It is worth noting that the adoption of the theory of diplomatic protection of shareholders, by opening the door of competing diplomatic claims, could create an atmosphere of confusion and insecurity in international economic relations
    \item The principles enunciated in \case{\textit{Barcelona Traction Case (Belgium v Spain)} [1970] ICJ Rep 3} have been codified in \convention{\textit{2006 Draft Articles on Diplomatic Protection} Articles 11-12}
\end{itemize}

\begin{conventiondetails}{\textit{2006 Draft Articles on Diplomatic Protection} Article 11}
    \flushleft
    \textit{Protection of shareholders}
    
    \vspace{\baselineskip}

    A State of nationality of shareholders in a corporation shall not be entitled to exercise diplomatic protection in respect of such shareholders in the case of an injury to the corporation unless: 
    \begin{enumerate}[label=(\alph*)]
        \item the corporation has ceased to exist according to the law of the State of incorporation for a reason unrelated to the injury; or 
        \item the corporation had, at the date of injury, the nationality of the State alleged to be responsible for causing the injury, and incorporation in that State was required by it as a precondition for doing business there. 
    \end{enumerate}
\end{conventiondetails}

\begin{conventiondetails}{\textit{2006 Draft Articles on Diplomatic Protection} Article 12}
    \flushleft
    \textit{Direct injury to shareholders}

    \vspace{\baselineskip}

    To the extent that an internationally wrongful act of a State causes direct injury to the rights of shareholders as such, as distinct from those of the corporation itself, the State of nationality of any such shareholders is entitled to exercise diplomatic protection in respect of its nationals.
\end{conventiondetails}

\begin{itemize}
    \item \convention{\textit{2006 Draft Articles on Diplomatic Protection} Article 12} (which applies with direct injury to shareholders) is different from \convention{\textit{2006 Draft Articles on Diplomatic Protection} Article 11} in that injuring a corporation will result in the law where the corporation was incorporated being applied, but direct injury to the shareholder (i.e., not by proxy, e.g., loss of value of share) will see the law where the shareholder is a national being applied (e.g., taking dividends, preventing a shareholder from participating in the governance of the company, etc.)
\end{itemize}

\begin{casedetails}{\textit{Diallo Case (Guinea v DRC)} [2007] ICJ Rep 582}
    \flushleft
    Diallo was a national of Guinea who had settled in the Democratic Republic of the Congo (DRC, which was then known as Zaire) in 1964 when he was 17 years old. he was a resident ever since, and had established two companies under DRC law. He was arrested in 1995 and was detained and then deported to Guinea on `public order' grounds. Guinea exercised diplomatic protection relating to the treatment of Diallo as an individual (he was still a Guinea national), as an \textit{associe} (shareholder) of the two companies, and in relation to the companies themselves.

    \vspace{\baselineskip}

    Because  of the development of international law in respect of the rights accorded to individuals, the scope of diplomatic protection, which was originally limited to violations of the minimum standard of treatment of aliens, has widened to include violations of internationally guaranteed human rights (i.e., the international minimum standard can be informed by international human rights law).

    \vspace{\baselineskip}

    It was not disputed that Diallo was a national of Guinea, so that the claim can be brought in relation to direct injury to him (his expulsion was a clear injury to Diallo as a person). In relation to Diallo's rights as an \textit{associe} (shareholder), \case{\textit{Barcelona Traction Case (Belgium v Spain)} [1970] ICJ Rep 3} was applied, which held that only the state of nationality of a corporation may exercise diplomatic protection on behalf of the corporation when its rights are injured. This ``remains the fundamental rule"; what amounts to an internationally wrongful act in the case of \textit{associes} is not to be regarded as an exception to the general legal regime of diplomatic protection. Guinea does have standing here in relation to allegedly unlawful acts directed at Diallo \textit{qua associe} (i.e., as a shareholder of the company because of the injury to those interests directly caused by the DRC's expulsion of him).

    \vspace{\baselineskip}

    In relation to Guinea's contention that it could exercise diplomatic protection of Diallo in `substitution' for the two companies, the ICJ held that state practice does not reveal an exception in customary international law allowing for protection by substitution, especially as shareholders are normally governed by bilateral or multilateral agreements for the protection of foreign affairs. The situation enunciated in the \convention{\textit{2006 Draft Articles on Diplomatic Protection} Article 11} is not present here, as the companies were not incorporated in Guinea, and so Guinea could not exercise diplomatic protection on behalf of the companies. Moreover, incorporation was not required by the DRC for the two companies to operate in the economic sectors that it did, and so \convention{Article 11} did not apply.
\end{casedetails}

\begin{itemize}
    \item There are a number of bases that must be systematically examined:
    \begin{itemize}
        \item State of nationality
        \item If there is an injury to the corporation
        \item If there is a direct injury to shareholders, go to the state of the shareholder
        \item It might then be possible for another state to exercise diplomatic protection
    \end{itemize}
\end{itemize}

\section{Exhaustion of Local Remedies}
\begin{itemize}
    \item For a claim in diplomatic protection to be admissible, the injured party must have exhausted all local remedies available to them in the state that they were injured in, as set out in \convention{\textit{ARSIWA} Article 44(b)}
    \item E.g., in Australia, a case must be taken all the way to the High Court before it is eligible to be considered for state protection
\end{itemize}

\begin{conventiondetails}{\textit{Articles on the Responsibility of States for Internationally Wrongful Acts} Article 44}
    \flushleft
    \textit{Admissibility of claims}

    \vspace{\baselineskip}

    The responsibility of a State may not be invoked if: 
    \begin{enumerate}[label=(\alph*)]
        \item the claim is not brought in accordance with any applicable rule relating to the nationality of claims;
        \item the claim is one to which the rule of exhaustion of local remedies applies and any available and effective local remedy has not been exhausted.
    \end{enumerate}    
\end{conventiondetails}

\begin{conventiondetails}{\textit{2006 Draft Articles on Diplomatic Protections} Article 14}
    \flushleft
    \textit{Exhaustion of local remedies}

    \begin{enumerate}
        \item A State may not present an international claim in respect of an injury to a national or other person referred to in draft article 8 before the injured person has, subject to draft article 15, exhausted all local remedies. 
        \item ``Local remedies” means legal remedies which are open to an injured person before the judicial or administrative courts or bodies, whether ordinary or special, of the State alleged to be responsible for causing the injury. 
        \item Local remedies shall be exhausted where an international claim, or request for a declaratory judgement related to the claim, is brought preponderantly on the basis of an injury to a national or other person referred to in draft article 8. 
    \end{enumerate}
\end{conventiondetails}

\begin{conventiondetails}{\textit{2006 Draft Articles on Diplomatic Protection} Article 15}
    \flushleft
    \textit{Exceptions to the local remedies rule }

    \vspace{\baselineskip}

    Local remedies do not need to be exhausted where:
    \begin{enumerate}[label=(\alph*)]
        \item there are no reasonably available local remedies to provide effective redress, or the local remedies provide no reasonable possibility of such redress; 
        \item there is undue delay in the remedial process which is attributable to the State alleged to be responsible; 
        \item there was no relevant connection between the injured person and the State alleged to be responsible at the date of injury; 
        \item the injured person is manifestly precluded from pursuing local remedies; or 
        \item the State alleged to be responsible has waived the requirement that local remedies be exhausted.
    \end{enumerate}
\end{conventiondetails}

\begin{itemize}
    \item This area contains a variety of case law that works in tandem with \convention{Draft Articles 14 and 15}
    \item This rule is based on the principle that `the foreign state should, first of all, be given the opportunity of redressing the wrong alleged', per \case{\textit{Finnish Shipowners Arbitration} (1934) 3 RIAA 1479}
    \item The local remedies rule does not apply when nationals are injured in situations of direct injury to the state (e.g., one state shoots down another state's aircraft, there is an infringement of diplomatic immunity or inviolability, etc.)
    \item It is for the plaintiff state to prove that there are no effective remedies (however, no such proof is needed if there is legislation which on its face deprives the claimants of a remedy), such as in the \case{\textit{Norwegian Loans Case} [1957] ICJ Rep 9}
    \item When seeking to establish that there are `no reasonably available remedies', it is not enough to show that the chances of success are low, or that further appeals are difficult or costly, but rather it must be shown that it is definite that no remedies are available
    \item A national must exhaust all available judicial remedies, including any avenues of appeal (and if there are no rights of appeal, but leave is required to gain access to those rights, then leave must be sought)
    \item A national must also exhaust all available administrative remedies entailing a binding decision, but there is no requirement to approach the executive for a discretionary `pardon', for their `clemency', etc.
    \item The basic arguments and evidence that were raised in domestic proceedings must also be raised in the international proceedings:
    \begin{itemize}
        \item ``It is sufficient if the essence of the claim has been brought before the competent tribunals and pursed as far as permitted by local law and procedures, and without success'', following \case{\textit{Electronica Sicula SpA (ELSI) Case} [1989] ICJ Rep 15} (which concerned an Italian government's requisition of property owned by ELSI, which was an Italian company owned by US companies)
        \item ``It is the whole system of legal protection, as provided by municipal law, which must have been put to the test before a State, as the protector of its nationals, can prosecute the claim on the international plane'', following \case{\textit{Ambatielos Arbitration (Greece v UK)} (1956) 12 RIAA 83}
        \begin{itemize}
            \item This case concerned the failure to call a witness in the UK proceedings
            \item These proceedings failed as the claim involved the failure of the injured person to call a relevant witness in the proceedings in the injuring state
            \item Such an oversight (or deliberate act) can be fatal to proceedings in diplomatic protection
        \end{itemize}
    \end{itemize}
    \item This rule examines the quality of the case that is being made; the case must first be taken to a local tribunal, and then pursued as far as it can be taken
\end{itemize}