\begin{itemize}
    \item The law of state responsibility is a system of rules in international rule that addresses the consequences of wrongdoing by states
    \item These are secondary rules of liability, to accompany primary rules of obligation
    \begin{itemize}
        \item Primary rules are those that determine whether a state has acted wrongfully
        \item Secondary rules are those that determine the consequences of wrongful acts
    \end{itemize}
    \item The laws of state responsibility have largely been codified in the \convention{\textit{2001 Articles on State Responsibility for Internationally Wrongful Acts} (ARSIWA/Articles on State Responsibility (ASR))}
    \begin{itemize}
        \item These articles are without prejudice to international criminal jurisdiction
        \item Whilst they are not set out in a convention like the \convention{\textit{Vienna Convention on the Law of Treaties}}, they are still mostly a codification, with some articles now being considered customary international law
        \item \convention{\textit{ARSIWA}} is also without prejudice to individual criminal responsibility, and the responsibility of international organisations
    \end{itemize}
\end{itemize}

\section{General Principles of State Responsibility}
\begin{itemize}
    \item State responsibility is a general regime of liability that applies to any breach of international law
    \begin{itemize}
        \item However, if a source of primary law (e.g., a treaty) provides for a specific regime of liability, that regime will apply instead with the effect of displacing the laws of state responsibility under \convention{\textit{ARSIWA} Art 55}
    \end{itemize}
    \item \convention{\textit{ARSIWA}} is agnostic as to the nature of the obligation breached or the context in which it is being applied, as it merely provides guidance on what happens \textit{after} a breach occurs
\end{itemize}

\begin{conventiondetails}{\textit{2001 Articles on the Responsibility of States for Internationally Wrongful Acts} Article 55}
    \flushleft
    \textit{Lex specialis}

    \vspace{\baselineskip}

    These articles do not apply where and to the extent that the conditions for the existence of an internationally wrongful act or the content or implementation of the international responsibility of a State are governed by special rules of international law.
\end{conventiondetails}

\begin{itemize}
    \item The responsibility of a state may arise if: (all of these elements must be established)
    \begin{itemize}
        \item An act or omission is attributed to a state (noting that states often conduct their affairs through organs)
        \begin{itemize}
            \item It must be determined whether the actions of an entity are attributable to the state
        \end{itemize}
        \item The act or omission breaches international law
        \begin{itemize}
            \item This refers to the breach of a primary obligation (e.g., custom, treaty, etc.; and the applicable standard of conduct)
            \item These obligations are additionally obligations of result (i.e., obligations that are made under a treaty)
        \end{itemize}
        \item The state is not able to raise any defence for the wrongful act/omission
        \begin{itemize}
            \item There is no excuse in `circumstances precluding wrongfulness' (e.g., self-defence, necessity, etc.)
        \end{itemize}
        \item Another state invokes the responsibility of the wrongful state
        \begin{itemize}
            \item Usually, only states that are injured by the acts of another state are able to invoke standing
        \end{itemize}
    \end{itemize}
\end{itemize}

\section{Attribution}
\begin{itemize}
    \item Since states conduct their affairs through individuals, rules of attribution are necessary to determine which actors and what conduct will give rise to state responsibility
    \item However, there are situations where states can be directly responsible (e.g., where a state fails to take certain actions); in these cases, the rules of attribution are not necessary and thus do not apply
\end{itemize}

\begin{casedetails}{\textit{Corfu Channel} [1949] ICJ Rep 4}
    \flushleft
    This case related to Syria's damage of UK naval vessels as they passed through the Corfu channel and the international strait in Albanian territorial waters. The ICJ held that Albania was responsible for failing to warn of the presence of the sea mines that the UK naval vessels had struck and were subsequently damaged by. Additionally, it was not necessary to consider why the mines had not been removed nor why they were there in the first place, as it was sufficient to note that they were there and had not been removed, and that no warning had been given.

    \vspace{\baselineskip}

    ``Every State's obligation not to allow knowingly its territory to be used for acts contrary to the rights of other States'' (i.e., if a state is aware of the presence of danger, they must warn other states).
\end{casedetails}

\begin{casedetails}{\textit{Trail Smelter} (1938/1941) III RIAA 1905}
    \flushleft
    This was a dispute between Canada and the US. Canada was held responsible for failing to prevent pollution from a metals smelter in British Columbia, causing damage to farms in Washington State (within the US). The smelter belched out sulphur dioxide, which reacted with water to form sulfuric acid, which fell as acid rain, thereby damaging farms. Canada and the US agreed to resolve the dispute through international arbitration, which found that Canada was responsible for the wrongful act, as they had failed to prevent the occurrence of that wrongful act.

    \vspace{\baselineskip}

    This case demonstrates that states have the responsibility to ensure that activities within their jurisdiction or control do not cause damage to the environment of other States or to areas beyond the limits of their natural jurisdiction.
\end{casedetails}

\begin{conventiondetails}{\textit{2001 Articles on the Responsibility of States for Internationally Wrongful Acts} Article 4}
    \flushleft
    \textit{Conduct of organs of a State}

    \begin{enumerate}
        \item The conduct of any State organ shall be considered an act of that State under international law, whether the organ exercises legislative, executive, judicial or any other functions, whatever position it holds in the organization of the State, and whatever its character as an organ of the central Government or of a territorial unit of the State.
        \item An organ includes any person or entity which has that status in accordance with the internal law of the State.
    \end{enumerate}    
\end{conventiondetails}

\begin{itemize}
    \item Under \convention{\textit{ARSIWA} Article 4}, the conduct can be measured at any level of government (e.g., local, state, federal, etc.), but will be attributable to the national government - if an entity is part of the state, its actions will constitute the actions of the state
\end{itemize}

\begin{casedetails}{\textit{Immunity from Legal Process Advisory Opinion} [1999] ICJ Rep 62}
    \flushleft
    In this case, the UN Special Rapporteur's immunity was not upheld by the Malaysian courts in relation to a statement made in the International Commercial Arbitration. The special rapporteur was a former Malaysian judge. ``According to a well-established rule of international law, the conduct of any organ of a state [including the courts of the state] must be regarded as an act of that State.'' The Court held that there was a breach of international law by Malaysia, and that the wrongful conduct was attributable to Malaysia as it was the conduct of their courts, which are organs of the Malaysian state.
\end{casedetails}

\begin{casedetails}{\textit{Kalgoorlie Riots Incident} (1934)}
    \flushleft
    This is an example of Australian state practice. There were attacks on foreign nationals on the Kalgoorlie goldfields in Western Australia. As the attacks were not prevented by the WA government, the Commonwealth government was held responsible by means of being responsible for the actions of the WA government. The WA government, as the executive of a state of Australia, was held to be an organ of the Australian State.
\end{casedetails}

\subsection{Excess of Authority or Contravention of Instructions}

\begin{conventiondetails}{\textit{2001 Articles on the Responsibility of States for Internationally Wrongful Acts} Article 7}
    \flushleft
    \textit{Excess of authority or contravention of instructions}

    \vspace{\baselineskip}

    The conduct of an organ of a State or of a person or entity empowered to exercise elements of the governmental authority shall be considered an act of the State under international law if the organ, person or entity acts in that capacity, even if it exceeds its authority or contravenes instructions.
\end{conventiondetails}

\begin{itemize}
    \item Under \convention{\textit{ARSIWA} Article 7}, states are still responsible/attributable for the actions of their organs if the organs exceed their authority or contravene instructions
    \item \convention{\textit{ARSIWA} Article 7} has been interpreted as considering whether an organ was acting with `apparent authority' (i.e., whether the conduct was unauthorised but still official, or whether it was private conduct removed from the scope of official functions)
    \item If a conduct is purely private, and the state is not involved at all, then the act cannot be attributed to the state (e.g., a uniformed police officer who is off duty and commits a crime will be attributable to a state as they can be considered as acting with apparent authority, even if they were not); the courts have erred towards almost always attributing the conduct to the state, except when it is glaringly not 
\end{itemize}

\begin{casedetails}{\textit{Youmans v Mexico} (1926) IV RIAA 110}
    \flushleft
    Youmans was a US national who was working in Mexico and helping to build a tunnel. A large number of Mexican nationals had been engaged to work on the project, and disputes arose over the wages paid to the workers, resulting in a riot breaking out. Over 1,000 people attacked the house Youmans was at, and caused his death when the local troops, who had been called to restore order, had opened fire on the rioters. The US claimed \$50,000 in damages from Mexico, as the perpetrators were not sufficiently punished.

    \vspace{\baselineskip}

    The Mexican soldiers had been acting in excess of their authority and their instructions, but the claims tribunal held this did not matter. ``Soldiers inflicting personal injuries or committing wanton destruction or looting always act in disobedience of some rules laid down by superior authority. There could be no liability whatever for such misdeeds if…acts committed by soldiers in contravention of instructions must always be considered as personal acts.'' As such, this case holds that it is sufficient to establish attribution even if entities of a state are acting beyond their authority.
\end{casedetails}

\begin{casedetails}{\textit{Caire Claim} (1929) 5 ADPIL Cases 146}
    \flushleft

    A French national was killed in an attempted extortion by two Mexican military officers who had been acting \gls{ultra vires}, as they had attempted to extort money from the French national. The \gls{ultra vires} actions of the officers was held to be irrelevant by the tribunal in determining attribution. ``The two officers, even if they are deemed to have acted outside their competence…and even if their superiors countermanded an order, have involved the responsibility of the State, since they acted under cover of their status as officers and used means placed at their disposal on account of that status.'' Here, the officers had been wearing their official uniforms, had taken Caire to an army barracks, and had used their military weapons, supporting the argument for attribution and the notion of apparent authority (i.e., it is irrelevant that they were acting beyond their authority, if it \textit{appeared} that they were acting within the bounds of their authority).
\end{casedetails}

\subsection{Actions of Non-Organs}

\begin{conventiondetails}{\textit{2001 Articles on the Responsibility of States for Internationally Wrongful Acts} Article 5}
    \flushleft
    \textit{Conduct of persons or entities exercising elements of governmental authority}

    \vspace{\baselineskip}

    The conduct of a person or entity which is not an organ of the State under article 4 but which is empowered by the law of that State to exercise elements of the governmental authority shall be considered an act of the State under international law, provided the person or entity is acting in that capacity in the particular instance.
\end{conventiondetails}

\begin{itemize}
    \item If a non-organ of a state has been empowered by the state to exercise government functions, attribution will be established (this reflects the notion that states are increasingly outsourcing governmental functions to non-state groups)
    \item This provision accounts for the para-statal (not really part of the state; quasi-state) entities that exercise governmental authority rather than state organs, such as statutory corporations and even some private companies entitled by law to exercise functions of a public character (e.g., corporations contracted to provide prison or immigration detention services that are empowered under law)
    \item However, there must be a statute expressly authorising this
    \item Usually, the conduct of private persons is not attributable to a state, but there are several situations in which this can occur (as individuals or companies that are not organs are not, under \convention{\textit{ARSIWA} Article 5} authorised to exercise elements of governmental authority)
\end{itemize}

\subsection{Conduct Directed or Controlled by a State}

\begin{conventiondetails}{\textit{2001 Articles on the Responsibility of States for Internationally Wrongful Acts} Article 8}
    \flushleft
    \textit{Conduct directed or controlled by a State}

    \vspace{\baselineskip}

    The conduct of a person or group of persons shall be considered an act of a State under international law if the person or group of persons is in fact acting on the instructions of, or under the direction or control of, that State in carrying out the conduct.
\end{conventiondetails}

\begin{itemize}
    \item \convention{\textit{ARSIWA} Article 8} holds that if a state directs or controls the conduct of a person or group of persons, then the conduct will be attributed to the state
    \item It generally addresses situations where individuals or groups are given instructions by a state to perform certain tasks (e.g., auxiliaries, mercenaries, private military contractors, etc.)
    \item Conduct will be attributable to the state only if the state directed or controlled the specific operation in question, and the conduct complained of was an integral part of that operation, as set out in the \article{\textit{ILC Commentary on Article 8}}
    \begin{itemize}
        \item This follows the test of `effective control' in \case{\textit{Nicaragua v US} [1986] ICJ Rep 14}, as opposed to the `overall control' test in \case{\textit{Tadić} (1999) ICTY}
    \end{itemize}
\end{itemize}

\begin{casedetails}{\textit{Military and Paramilitary Activities in and against Nicaragua} [1986] ICJ Rep 14}
    \flushleft
    See Page \pageref{case:Military and paramilitary activities in Nicaragua} for further details on the facts of this case. The US had provided planning, direction and support for certain operation carried out by Contra rebels against the Nicaraguan government. However, US financing, organising, training, supplying and equipping of the Contra rebels, selection of its targets, and the planning of the whole of its operation was not sufficient to attribute the acts of the Contras to the US, and as such, Contra could not be equated, for legal purposes, with an organ of the US government.

    \vspace{\baselineskip}

    There was ``no effective control of the operations in the course of which the alleged violations were committed''; as such, general dependence and support of a state was not sufficient. However, the US was still responsible for its own wrongful acts, e.g., violating the principle of non-intervention and the obligation not to use force.
\end{casedetails}

\begin{casedetails}{\textit{Genocide Case (Bosnia/Herzegovina v Serbia/Montenegro)} [2007] ICJ Rep 43}
    \flushleft
    This case affirms the test of effective control as set out in \case{\textit{Military and Paramilitary Activities in and against Nicaragua} [1986] ICJ Rep 14}. It concerned a series of massacres committed by the Army of Republika Srpska (VRS) in the Srebrenica area (Republika Srpska was a breakaway Serbian entity in Bosnia and Herzegovina). The key question in this case was whether Serbia and Montenegro (FRY) were internationally responsible for the acts of VRS? 
    
    \vspace{\baselineskip}
    
    The ICJ answered in the negative, as there was no direct FRY participation, VRS did not have the status of an organ of the government of FRY, nor was it equated with an organ of the state because of its dependence on the state, and as such, there was no `complete dependence'. 

    \vspace{\baselineskip}

    The words `no effective control' demonstrate that \case{\textit{Military and Paramilitary Activities in and against Nicaragua} [1986] ICJ Rep 14} was followed in considering \convention{\textit{ARSIWA} Article 8}, as as looking at the evidence, it could not be established that FRY was involved in each of the relevant operations which involved the violations of international law. It therefore isn't sufficient that there was a general relationship between FRY and VRS, and that the court must look specifically at each operation and see if the effective control test is satisfied. Control must be `in respect of each operation in which the alleged violations occurred, not generally in respect of the overall actions taken by the persons or groups of persons having committed the violations'.

\end{casedetails}

\subsection{Absence of Official Authorities}

\begin{conventiondetails}{\textit{2001 Articles on the Responsibility of States for Internationally Wrongful Acts} Article 9}
    \flushleft
    \textit{Conduct carried out in the absence or default of the official authorities}

    \vspace{\baselineskip}

    The conduct of a person or group of persons shall be considered an act of a State under international law if the person or group of persons is in fact exercising elements of the governmental authority in the absence or default of the official authorities and in circumstances such as to call for the exercise of those elements of authority.
\end{conventiondetails}

\begin{itemize}
    \item Under \convention{\textit{ARSIWA} Article 9}, an absence of official authorities, such as in a situation of unrest or revolution, the wrongful acts of people or groups who have stepped in to provide elements of governmental authority will be attributed to the state
    \item An example is the 1978/9 Islamic revolution in Iran, where the government was overthrown; Revolutionary Guards continued to perform official functions (e.g., migration, customs, immigration, etc. at Tehran airport), and the actions of these guards were attributed to the state
\end{itemize}

\subsection{Insurrectional Movements}

\begin{conventiondetails}{\textit{2001 Articles on the Responsibility of States for Internationally Wrongful Acts} Article 10}
    \flushleft
    \textit{Conduct of an insurrectional or other movement}

    \begin{enumerate}
        \item The conduct of an insurrectional movement which becomes the new Government of a State shall be considered an act of that State under international law.
        \item The conduct of a movement, insurrectional or other, which succeeds in establishing a new State in part of the territory of a pre-existing State or in a territory under its administration shall be considered an act of the new State under international law.
        \item This article is without prejudice to the attribution to a State of any conduct, however related to that of the movement concerned, which is to be considered an act of that State by virtue of articles 4 to 9.
    \end{enumerate}
\end{conventiondetails}

\begin{itemize}
    \item As a general principle under \convention{\textit{ARSIWA} Article 10}, the conduct of an insurrectional or other movement is not attributable to the state, unless the movement becomes the new government of a state
    \begin{itemize}
        \item If a revolutionary is unsuccessful, then its actions cannot ever be attributed to the state
        \item If the revolutionary is successful and forms a new government, then international law will attribute the actions of the revolutionary to the state from the instance that the revolutionary movement began
    \end{itemize}
\end{itemize}

\begin{casedetails}{\textit{Bolivar Railways Company Case} (1903) IX RIAA 445}
    \flushleft
    Venezuela was held responsible for services supplied by a company to a successful revolutionary regime. ``The nation is responsible for the obligations of a successful revolution from its beginning, because in theory, it represented ab initio a changing national will, crystallizing in the finally successful result ...''
\end{casedetails}

\begin{casedetails}{\textit{Short v Iran} (1987) Iran-USCTR 76 (Iran/US Claims Tribunal)}
    \flushleft
    Mr Short was a US national forced to leave Iran as a result of unrest in Iran following the Islamic revolution of 1978/9, contrary to international law. This case considered an issue of attribution; here, Mr Short was unable to identify any agent of the movement that had forced him to leave. ``The acts of supporters of a revolution cannot be attributed to the [new] government following the success of the revolution just as the acts of supporters of an existing government are not attributable to the government.''
\end{casedetails}

\subsection{Conduct Acknowledged and Adopted by a State}

\begin{conventiondetails}{\textit{2001 Articles on the Responsibility of States for Internationally Wrongful Acts} Article 11}
    \flushleft
    \textit{Conduct acknowledged and adopted by a State as its own}

    \vspace{\baselineskip}

    Conduct which is not attributable to a State under the preceding articles shall nevertheless be considered an act of that State under international law if and to the extent that the State acknowledges and adopts the conduct in question as its own.
\end{conventiondetails}

\begin{itemize}
    \item The text of \convention{\textit{ARSIWA} Article 11} uses `acknowledges and adopts', which is distinguished from mere `support and encouragement'
    \item Whilst the acts of private individuals or entities may not be attributable to a state, the state may nonetheless be directly responsible for failing to meet its obligations if there is injury to foreign nationals or foreign national interests
\end{itemize}

\begin{casedetails}{\textit{US Diplomatic and Consular Staff in Tehran} [1980] ICJ Rep 3}
    \flushleft
    This case involved the seizure of the US embassy in Tehran by Iranian students, who took the diplomatic staff as hostages. The Iranian government did not intervene to stop the seizure, nor did it condemn the actions of the students. The ICJ held that Iran was responsible for the wrongful acts of the students, as it had acknowledged and adopted their conduct as its own (had Iran not done so, it is likely that they would not have been considered as acts of the state).
    
    \vspace{\baselineskip}

    ``The approval given [to the occupation of the US Embassy by militants] by the Ayatollah Khomeini and other organs of the Iranian State, and the decision to perpetuate them, translated continuing occupation of the Embassy and detention of the hostages into acts of that State.''
\end{casedetails}

\begin{casedetails}{\textit{Home Missionary Society Claim} (1921) VI RIAA 42}
    \flushleft
    This case was a claim by the US against Britain on behalf of a religious group for losses in the British Colony of Sierra Leone in 1898 from a rebellion prompted by the imposition of a `hut tax'. A number of missionaries were the victims of attacks in Sierra Leone that flowed from the British government's imposition of a poll tax (which was highly regressionary).

    \vspace{\baselineskip}

    The tribunal held that a government can't be liable as the insurer of lives and property under these particular circumstances, and as such, it could not be found that Britain was in breach of any good faith, any negligence in suppressing the rebelling, and thus Britain was not therefore responsible itself for failing to meet the obligations.

    \vspace{\baselineskip}

    ``It is a well-established principle of international law that no government can be held responsible for the acts of rebellious bodies of men committed in violation of its authority, where it is guilty of no breach of good faith, or of no negligence in suppressing insurrection."

\end{casedetails}

\begin{casedetails}{\textit{Asian Agricultural Products} (1991) CSID Case No. ARB/87/3}
    \flushleft
    In this case, Sri Lanka was held responsible for damage to a seafood factor in Sri Lanka owned by a Hong Kong company. Whilst it was unclear whether the government or LTTE had attacked the factor, Sri Lanka had failed to exercise due diligence to protect the foreign owned property. The Sri Lankan military had conducted counterinsurgency measures against the Tamil Tigers (LTTE), damaging the factory. At the time, the factory was under the exclusive control of the Sri Lankan forces, and so they had failed to exercise due diligence (by failing to take best efforts to protect foreign owned property).
    
    \vspace{\baselineskip}
    
    ``It is a generally accepted rule of international law ... that ... [a] state on whose territory an insurrection occurs is not responsible for loss or damage sustained by foreign investors unless it can be shown that the government of that state failed to provide the standard of protection required, either by treaty, or under customary law" (or if it can be shown that the government itself was responsible).

\end{casedetails}

\section{Circumstances Precluding Wrongfulness (Defences)}
\begin{itemize}
    \item Under \convention{\textit{ARSIWA} Article 20}, a state  may consent (in advance) to an act that would otherwise be illegal
    \item Under \convention{\textit{ARSIWA} Article 21}, acts undertaken in self-defence are not wrongful if they are taken in conformity with the \convention{\textit{UN Charter}}
    \item Under \convention{\textit{ARSIWA} Article 22}, if they meet certain requirements, countermeasures will be regarded as a lawful exercise of self-help
\end{itemize}

\begin{conventiondetails}{\textit{2001 Articles on the Responsibility of States for Internationally Wrongful Acts} Article 20}
    \flushleft
    \textit{Consent}

    \vspace{\baselineskip}

    Valid consent by a State to the commission of a given act by another State precludes the wrongfulness of that act in relation to the former State to the extent that the act remains within the limits of that consent.
\end{conventiondetails}

\begin{conventiondetails}{\textit{2001 Articles on the Responsibility of States for Internationally Wrongful Acts} Article 21}
    \flushleft
    \textit{Self-defence}

    \vspace{\baselineskip}

    The wrongfulness of an act of a State is precluded if the act constitutes a lawful measure of self-defence taken in conformity with the Charter of the United Nations.
\end{conventiondetails}

\begin{conventiondetails}{\textit{2001 Articles on the Responsibility of States for Internationally Wrongful Acts} Article 22}
    \flushleft
    \textit{Countermeasures in respect of an internationally wrongful act}

    \vspace{\baselineskip}

    The wrongfulness of an act of a State not in conformity with an international obligation towards another State is precluded if and to the extent that the act constitutes a countermeasure taken against the latter State in accordance with chapter II of part three.
\end{conventiondetails}

\begin{itemize}
    \item Since all states are regarded as equal under public international law (there is no central authority), a proportionate countermeasure as allowed under \convention{\textit{ARSIWA} Article 22} is a means of bringing a wrongful act to an end
\end{itemize}

\begin{casedetails}{\textit{Law Debenture Trust Corpn plc v. Ukraine} [2023] UKSC 11}
    \flushleft
    ``If the availability of countermeasures at the level of international law were accepted as giving rise to a defence in domestic law, national courts would become the arbiter of inter-state disputes governed by international law which is not their function."

    \vspace{\baselineskip}

    This case involved a US\$3 bn. loan to Ukraine from Russia in 2013. Following the annexation of Crimea, Ukraine refused to pay the loan. The UK Supreme Court held that the defence of countermeasures was not available in domestic law, as it would be contrary to the principle of state immunity. They also held that courts cannot sit in judgement of the acts of sovereign states, as this would be contrary to the principle of state sovereignty.
\end{casedetails}

\begin{itemize}
    \item Under \convention{\textit{ARSIWA} Article 23}, force majeure holds that a state will not be responsible where it is not able to comply with an international obligation for reasons entirely beyond its control
    \item Under \convention{\textit{ARSIWA} Article 24}, distress holds that a state will not be responsible where it is not able to comply with an international obligation in order to save the lives of persons or to protect property in imminent peril
    \item Under \convention{\textit{ARSIWA} Article 25}, necessity holds that a state will not be responsible where it is not able to comply with an international obligation in order to safeguard an essential interest against a grave and imminent peril
    \item All of the above defences are complete defences, and are not partial defences
\end{itemize}

\begin{conventiondetails}{\textit{2001 Articles on the Responsibility of States for Internationally Wrongful Acts} Article 23}
    \flushleft
    \textit{Force majeure}

    \begin{enumerate}
        \item The wrongfulness of an act of a State not in conformity with an international obligation of that State is precluded if the act is due to force majeure, that is the occurrence of an irresistible force or of an unforeseen event, beyond the control of the State, making it materially impossible in the circumstances to perform the obligation.
        \item Paragraph 1 does not apply if:
        \begin{enumerate}[label=(\alph*)]
            \item the situation of force majeure is due, either alone or in combination with other factors, to the conduct of the State invoking it; or
            \item the State has  assumed the risk of that situation occurring.
        \end{enumerate}
    \end{enumerate}
\end{conventiondetails}

\begin{casedetails}{\textit{Rainbow Warrior Ruling} (1986) XIX RIAA 199}
    \flushleft
    This was a dispute between France and New Zealand over the sinking of the \textit{Rainbow Warrior} ship, which was a Greenpeace vessel, in Auckland as a result of explosive devices placed on board by French agents. A Dutch citizen was drowned in the attack. Two French agents (Major Mafart and Captain Prieur) were arrested, and plead guilty to manslaughter in the New Zealand courts, for which they were sentenced to 10 years' imprisonment each. This gave rise to a dispute between France and New Zealand; France insisted that the punishment was too harsh, noting that at the time Greenpeace was protesting against French nuclear testing in the Pacific Ocean.

    \vspace{\baselineskip}

    The dispute went to the UN Secretary General, who issued a ruling in his function as a conciliator (not as a judge). He held that France was to convey an apology to New Zealand for the attack that was contrary to international law, and that France was to pay compensation of US\$7m. Additionally, the French agents in New Zealand detention were to be transferred to an isolated island outside of Europe for a period of 3 years, before their subsequent release. Any differences that arose were to be referred to binding arbitration.
\end{casedetails}

\begin{casedetails}{\textit{Rainbow Warrior Arbitration} (1990) XX RIAA 215}
    \flushleft

    This is the sequel to the \case{\textit{Rainbow Warrior Ruling}}. Before the three years' detention was completed, Mafart was evacuated to France for medical treatment (without the consent of New Zealand or an examination by New Zealand doctors). Prieur was also repatriated to France as she was expecting her first child. New Zealand held that these were breaches of the ruling, but France held that it was not and invoked a range of defences.

    \vspace{\baselineskip}

    France cast force majeure here as a defence in absolute terms, but this was rejected by the tribunal as force majeure does not apply in circumstances such as this where compliance was more difficult or burdensome. The tribunal also held that distress had to be distinguished from necessity, stating that ``[w]hat was involved in distress was a choice between a departure from an international obligation and a serious threat to the life or physical integrity of a State organ or of persons entrusted to its care. Necessity, on the other hand, was concerned with departure from international obligations on the ground of vital interests of the State'' (France had raised this defence on the grounds that they had to protect the wellbeing of the individuals). The tribunal held that the removal of Mafart was not wrongful, as he required medical treatment that was not available on the island, but the failure to return him was wrongful. The tribunal also held that the removal of Prieur was wrongful, as no medical complications had arisen in her pregnancy. From these cases, the defences are hard to establish, and tend to apply in relatively extreme circumstances.
\end{casedetails}

\begin{conventiondetails}{\textit{2001 Articles on the Responsibility of States for Internationally Wrongful Acts} Article 24}
    \flushleft
    \textit{Distress}

    \begin{enumerate}
        \item The wrongfulness of an act of a State not in conformity with an international obligation of that State is precluded if the author of the act in question has no other reasonable way, in a situation of distress, of saving the author's life or the lives of other persons entrusted to the author's care.
        \item Paragraph 1 does not apply if:
        \begin{enumerate}[label=(\alph*)]
            \item the situation of distress is due, either alone or in combination with other factors, to the conduct of the State invoking it; or
            \item the act in question is likely to create a comparable or greater peril.
        \end{enumerate}
    \end{enumerate}
\end{conventiondetails}

\begin{conventiondetails}{\textit{2001 Articles on the Responsibility of States for Internationally Wrongful Acts} Article 25}
    \flushleft
    \textit{Necessity}

    \begin{enumerate}
        \item Necessity may not be invoked by a State as a ground for precluding the wrongfulness of an act not in conformity with an international obligation of that State unless the act:
        \begin{enumerate}[label=(\alph*)]
            \item is the only way for the State to safeguard an essential interest against a grave and imminent peril; and
            \item does not seriously impair an essential interest of the State or States towards which the obligation exists, or of the international community as a whole.
        \end{enumerate}
        \item In any case, necessity may not be invoked by a State as a ground for precluding wrongfulness if:
        \begin{enumerate}[label=(\alph*)]
            \item the international obligation in question excludes the possibility of invoking necessity; or
            \item the State has contributed to the situation of necessity.
        \end{enumerate}
    \end{enumerate}
\end{conventiondetails}

\section{Invocation, Standing and Implementation}
\begin{itemize}
    \item \convention{\textit{ARSIWA} Article 42} holds that a state may invoke the responsibility of another state if the obligation is owed to (a) that state individually (i.e., the breach of a bilateral treaty), or (b) a group of states including the state, or the international community as a whole, and the breach specially affects the invoking state or radically changes the position of all states with respect to the further performance of the obligation
\end{itemize}
\begin{conventiondetails}{\textit{2001 Articles on the Responsibility of States for Internationally Wrongful Acts} Article 42}
    \flushleft
    \textit{Invocation of responsibility by an injured State}

    \vspace{\baselineskip}

    A State is entitled as an injured State to invoke the responsibility of another State if the obligation breached is owed to:
    \begin{enumerate}[label=(\alph*)]
        \item that State individually; or
        \item a group of States including that State, or the international community as a whole, and the breach of the obligation:
        \begin{enumerate}[label=(\roman*)]
            \item specially affects that State; or
            \item is of such a character as radically to change the position of all the other States to which the obligation is owed with respect to the further performance of the obligation.
        \end{enumerate}
    \end{enumerate}
\end{conventiondetails}

\begin{itemize}
    \item \convention{\textit{ARSIWA} Article 48} outlines some exceptions to the rules of invocation set out in \convention{Article 42}
    \item A state other than an injured state is entitled to invoke the responsibility of another state if (a) the obligation is owed to a group of states, including that state, and is established for the protection of a collective interest or (b) the obligation is owed to the international community as a whole
\end{itemize}

\begin{conventiondetails}{\textit{2001 Articles on the Responsibility of States for Internationally Wrongful Acts} Article 48}
    \flushleft
    \textit{Invocation of responsibility by a State other than an injured State}

    \begin{enumerate}
        \item Any State other than an injured State is entitled to invoke the responsibility of another State in accordance with paragraph 2 if:
        \begin{enumerate}[label=(\alph*)]
            \item the obligation breached is owed to a group of States including that State, and is established for the protection of a collective interest of the group; or
            \item the obligation breached is owed to the international community as a whole.
        \end{enumerate}
        \item Any State entitled to invoke responsibility under paragraph 1 may claim from the responsible State:
        \begin{enumerate}[label=(\alph*)]
            \item cessation of the internationally wrongful act, and assurances and guarantees of non-repetition in
            accordance with article 30; and
            \item performance of the obligation of reparation in accordance with the preceding articles, in the interest of the injured State or of the beneficiaries of the obligation breached.
        \end{enumerate}
        \item The requirements for the invocation of responsibility by an injured State under articles 43, 44 and 45 apply to an invocation of responsibility by a State entitled to do so under paragraph 1.
    \end{enumerate}    
\end{conventiondetails}

\begin{casedetails}{\textit{Belgium v Senegal}}
    \flushleft
    ``The common interest in compliance with the relevant obligations under the Convention against Torture implies the entitlement of each State party to the Convention to make a claim concerning the cessation of an alleged breach by another State party.''

    \vspace{\baselineskip}

    Whilst Belgium had nothing to do with the alleged torture being carried out by Senegal, but was still able to invoke state responsibility on the grounds that Senegal owed an obligation to the international community as a whole.
\end{casedetails}

\begin{casedetails}{\textit{Whaling in the Antarctic Case} [2014]}
    \flushleft
    Australia was successful in arguing that Japan had breached international conventions that regulated whaling in the high seas. Australia mounted a case not on the basis that Japan's whaling had any effect on Australia, but instead on Japan's breach of the convention in general, which was sufficient to allow any party state to take action against Japan. Whilst the issue was not heard before the ICJ as the parties resolved the dispute, this case is not evidence supporting \convention{\textit{ARSIWA} Article 48}, and moreover demonstrates its wide acceptance as customary international law.
\end{casedetails}

\begin{itemize}
    \item \convention{\textit{ARSIWA} Article 46} holds that when there are several states injured by the same internationally wrongful act, each injured State may separately invoke the responsibility of the State which has committed the internationally wrongful act
    \item \convention{\textit{ARSIWA} Article 47} holds that when several states are responsible for the same internationally wrongful act, the responsibility of each State may be invoked in relation to that act
\end{itemize}

\begin{conventiondetails}{\textit{2001 Articles on the Responsibility of States for Internationally Wrongful Acts} Article 46}
    \flushleft
    \textit{Plurality of injured States}

    \vspace{\baselineskip}

    Where several States are injured by the same internationally wrongful act, each injured State may separately invoke the responsibility of the State which has committed the internationally wrongful act.
\end{conventiondetails}

\begin{conventiondetails}{\textit{2001 Articles on the Responsibility of States for Internationally Wrongful Acts} Article 47}
    \flushleft
    \textit{Plurality of responsible States}

    \begin{enumerate}
        \item Where several States are responsible for the same internationally wrongful act, the responsibility of each State may be invoked in relation to that act.
        \item Paragraph 1:
        \begin{enumerate}[label=(\alph*)]
            \item does not permit any injured State to recover, by way of compensation, more than the damage it has suffered;
            \item is without prejudice to any right of recourse against the other responsible States.
        \end{enumerate}
    \end{enumerate}
\end{conventiondetails}

\section{Consequences of Wrongful Acts (Remedies)}
\begin{casedetails}{\textit{Chorzow Factory Case} (1928) PCIJ (Ser. A) No. 17}
    \flushleft
    This case states that a wrongful state must make reparation.

    \vspace{\baselineskip}

    ``` ... reparation must, as far as possible, wipe out all the consequences of the illegal act and re-establish the situation which would, in all probability, have existed if that act had not been committed. Restitution in kind, or, if this is not possible, payment of a sum corresponding to the value which a restitution in kind would bear; the award, if need be, of damages for loss sustained which would not be covered by restitution in kind or payment in place of it — such are the principles which should serve to determine the amount of compensation due for an act contrary to international law'
\end{casedetails}

\begin{itemize}
    \item Reparation refers to the overriding concept to capture the various remedies available to a state that has suffered a wrongful act
    \item Restitution is the primary obligation of a state; if restitution is not possible, the state may be entitled to compensation
    \item The general principles of reparation have now been broken down in \convention{\textit{ARSIWA} Arts 31, 34-37}
    \item \convention{Article 31} holds that the responsible state is under an obligation to make full reparation for the injury caused by the internationally wrongful act
\end{itemize}

\begin{conventiondetails}{\textit{2001 Articles on the Responsibility of States for Internationally Wrongful Acts} Article 31}
    \flushleft
    \textit{Reparation}
    \begin{enumerate}
        \item The responsible State is under an obligation to make full reparation for the injury caused by the internationally wrongful act.
        \item Injury includes any damage, whether material or moral, caused by the internationally wrongful act of a State.
    \end{enumerate}
\end{conventiondetails}

\begin{itemize}
    \item \convention{Article 34} holds that full reparation for the injury caused by the internationally wrongful act shall take the form of restitution, compensation and satisfaction, either singly or in combination
\end{itemize}

\begin{conventiondetails}{\textit{2001 Articles on the Responsibility of States for Internationally Wrongful Acts} Article 34}
    \flushleft
    \textit{Forms of reparation}

    \vspace{\baselineskip}

    Full reparation for the injury caused by the internationally wrongful act shall take the form of restitution, compensation and satisfaction, either singly or in combination, in accordance with the provisions of this chapter.
\end{conventiondetails}

\begin{itemize}
    \item \convention{Article 35} holds that restitution is the primary obligation of a state, where a state responsible for an internationally wrongful act is under an obligation to make restitution; i.e., to re-establish the situation which existed before the wrongful act was committed provided that restitution is not materially impossible
\end{itemize}

\begin{conventiondetails}{\textit{2001 Articles on the Responsibility of States for Internationally Wrongful Acts} Article 35}
    \flushleft
    \textit{Restitution}

    \vspace{\baselineskip}

    A State responsible for an internationally wrongful act is under an obligation to make restitution, that is, to re-establish the situation which existed before the wrongful act was committed, provided and to the extent that restitution:

    \begin{enumerate}[label=(\alph*)]
        \item is not materially impossible;
        \item does not involve a burden out of all proportion to the benefit deriving from restitution instead of compensation
    \end{enumerate}
\end{conventiondetails}

\begin{itemize}
    \item \convention{Article 36} provides for compensation, for damage that is not made good by restitution
\end{itemize}

\begin{conventiondetails}{\textit{2001 Articles on the Responsibility of States for Internationally Wrongful Acts} Article 36}
    \flushleft
    \textit{Compensation}

    \begin{enumerate}
        \item The State responsible for an internationally wrongful act is under an obligation to compensate for the damage caused thereby, insofar as such damage is not made good by restitution.
        \item The compensation shall cover any financially assessable damage including loss of profits insofar as it is established.
    \end{enumerate}
\end{conventiondetails}

\begin{itemize}
    \item \convention{Article 37} provides for satisfaction, which is a form of reparation that is not restitution or compensation
    \begin{itemize}
        \item This may include an acknowledgement of the breach, an expression of regret, a formal apology or another appropriate modality
        \item Whilst it is more symbolic, it generally also entails a declaration of wrongdoing by a state
        \item Satisfaction shall not be out of proportion to the injury and may not take a form humiliating to the responsible state
        \item This is generally the most common form of remedy (it is rare that compensation is paid, and when it is paid, it is not large in terms of quantum)
    \end{itemize}
\end{itemize}

\begin{conventiondetails}{\textit{2001 Articles on the Responsibility of States for Internationally Wrongful Acts} Article 37}
    \flushleft
    \textit{Satisfaction}

    \begin{enumerate}
        \item The State responsible for an internationally wrongful act is under an obligation to give satisfaction for the injury caused by that act insofar as it cannot be made good by restitution or compensation.
        \item Satisfaction may consist in an acknowledgement of the breach, an expression of regret, a formal apology or another appropriate modality.
        \item Satisfaction shall not be out of proportion to the injury and may not take a form humiliating to the responsible State.
    \end{enumerate}
\end{conventiondetails}