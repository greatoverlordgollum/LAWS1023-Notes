% Defining the glossary for case summaries
\newglossary[slg]{summaries}{syi}{syg}{Case Summaries}

% Glossary entry for Al-Saadoon v Secretary of State for Defence
\newglossaryentry{summary:al-saadoon v sec. def}{
    type=summaries,
    name={\textit{Al-Saadoon v Secretary of State for Defence} [2010] EWCA Civ 776},
    sort={Al-Saadoon v Secretary of State for Defence},
    description={Al-Saadoon and others were detained in Iraq by British forces; the UK Court of Appeal held that the UK was bound by the Geneva Conventions, and had a duty to ensure detainees were treated in accordance with international law, including the prohibition on torture, demonstrating that regional customary norms can be applied as international law if they meet the high threshold of stability and continuity. British forces detained Al-Saadoon and others in Iraq, prompting a challenge on their treatment. The principle established is that regional customary norms, when sufficiently stable and continuous, can be enforced as international law, obligating states to uphold international humanitarian standards such as the prohibition on torture.}
}

% Glossary entry for Anglo-Norwegian Fisheries Case
\newglossaryentry{summary:anglo norwegian fisheries}{
    type=summaries,
    name={\textit{Anglo-Norwegian Fisheries Case} (1951) ICJ Rep 116},
    sort={Anglo-Norwegian Fisheries Case},
    description={The UK challenged Norway's method of drawing baselines, asserting a customary rule prohibiting it; the ICJ held that Norway was exempt from this rule due to its persistent objection from the rule's inception. The case involved a dispute where the UK contested Norway's baseline delineation for its fisheries zone. The principle confirmed is that a state persistently objecting to a customary international law rule from its formation is not bound by it.}
}

% Glossary entry for Asylum Case (Colombia v Peru)
\newglossaryentry{summary:asylum case}{
    type=summaries,
    name={\textit{Asylum Case (Colombia v Peru)} (1950) ICJ Rep 266},
    sort={Asylum Case},
    description={Colombia granted political asylum to de la Torre, which Peru challenged; Colombia invoked a regional customary norm, but the ICJ rejected this, holding that regional standards require a higher degree of stability and continuity to apply as international law. The case arose when Colombia granted asylum to a Peruvian political figure, contested by Peru. The principle established is that regional customary norms must exhibit significant stability and continuity to be recognized as binding international law.}
}

% Glossary entry for Bay of Bengal (Bangladesh v Myanmar)
\newglossaryentry{summary:bay of bengal (bangladesh v myanmar)}{
    type=summaries,
    name={\textit{Bay of Bengal (Bangladesh v Myanmar)} (2012) ITLOS 12},
    sort={Bay of Bengal (Bangladesh v Myanmar)},
    description={Bangladesh and Myanmar disputed the delimitation of their maritime boundary in the Bay of Bengal; the International Tribunal for the Law of the Sea resolved the dispute by implicitly adopting general principles of international law into its decision. The case involved a maritime boundary dispute between Bangladesh and Myanmar in the Bay of Bengal. The principle applied is that general principles of international law can be implicitly incorporated to resolve maritime delimitation disputes.}
}

% Glossary entry for Chagos Marine Protected Area Arbitration
\newglossaryentry{summary:chagos marine protected area}{
    type=summaries,
    name={\textit{Chagos Marine Protected Area Arbitration (Mauritius v UK)} (2015) XXXI RIAA 359},
    sort={Chagos Marine Protected Area Arbitration},
    description={Mauritius challenged the UK's establishment of a Marine Protected Area in the Chagos Archipelago; the Tribunal held that it could apply the doctrine of estoppel, derived from various domestic legal systems, to determine the position under international law. The dispute arose over the UK's creation of a marine protected area, contested by Mauritius. The principle established is that international tribunals may draw on domestic legal doctrines like estoppel to inform decisions under international law.}
}

% Glossary entry for Gabčíkovo-Nagymaros Case
\newglossaryentry{summary:gabcikovo-nagymaros case}{
    type=summaries,
    name={\textit{Gabčíkovo-Nagymaros Case} (1997) ICJ Rep 7},
    sort={Gabčíkovo-Nagymaros Case},
    description={Hungary and Czechoslovakia disagreed over a 1977 treaty for a joint dam project on the Danube; Hungary suspended work due to environmental concerns, and the ICJ held that neither impossibility nor fundamental change of circumstances justified termination, emphasizing the principle of \textit{pacta sunt servanda}. The case involved Hungary's attempt to terminate the treaty after suspending work, countered by Czechoslovakia's unilateral alternative. The principle reinforced is that treaties remain binding under \textit{pacta sunt servanda}, and termination based on impossibility or fundamental change of circumstances requires exceptional conditions.}
}

% Glossary entry for Legal Status of Eastern Greenland
\newglossaryentry{summary:legal status of eastern greenland}{
    type=summaries,
    name={\textit{Legal Status of Eastern Greenland (Denmark v Norway)} (1933) PCIJ Series A/B, No 53},
    sort={Legal Status of Eastern Greenland},
    description={The Permanent Court of International Justice held that Norway was bound by an oral undertaking given to Denmark not to oppose its claim to sovereignty over Greenland, obligating Norway to refrain from contesting Danish sovereignty. The case arose when Norway occupied Eastern Greenland in 1931, challenging Denmark's claim based on sovereignty from the 1700s, but the PCIJ found Denmark had displayed sufficient authority through a latent claim that had never been challenged until 1921. The principle established is that oral undertakings by states can create binding international obligations, and that very little exercise of sovereignty may suffice for territorial claims in thinly populated areas, provided no superior competing claim exists.}
}

% Glossary entry for Legality of the Threat or Use of Nuclear Weapons
\newglossaryentry{summary:nuclear weapons case}{
    type=summaries,
    name={\textit{Legality of the Threat or Use of Nuclear Weapons} (1996) ICJ Rep 226},
    sort={Legality of the Threat or Use of Nuclear Weapons},
    description={The ICJ held that the use of nuclear weapons is generally contrary to international law, but no specific rule prohibits their use in all circumstances, and UNGA resolutions, while not binding, can provide evidence of customary international law. The case addressed a request for an advisory opinion on the legality of nuclear weapons. The principle clarified is that UNGA resolutions may reflect customary international law, and nuclear weapon use is generally unlawful absent exceptional circumstances.}
}

% Glossary entry for Maritime Delimitation and Territorial Questions (Qatar v Bahrain)
\newglossaryentry{summary:qatar v bahrain}{
    type=summaries,
    name={\textit{Maritime Delimitation and Territorial Questions (Qatar v Bahrain)} (1994) ICJ Rep 112},
    sort={Maritime Delimitation and Territorial Questions (Qatar v Bahrain)},
    description={The ICJ held that the 1987 exchanges of letters and 1990 Doha Minutes between Qatar and Bahrain constituted binding international agreements, obligating them to submit their maritime and territorial disputes to the Court, emphasizing that exchanges of letters can form valid treaties. The case involved Qatar's unilateral referral of disputes over the Hawar Islands and maritime boundaries, contested by Bahrain. The principle established is that exchanges of letters and minutes can constitute binding treaties under international law.}
}

% Glossary entry for Nicaragua v Colombia
\newglossaryentry{summary:nicaragua v colombia}{
    type=summaries,
    name={\textit{Nicaragua v Colombia} (2012) ICJ Rep 624},
    sort={Nicaragua v Colombia},
    description={Nicaragua claimed Colombia violated its continental shelf and exclusive economic zone rights; the ICJ held that Nicaragua had rights to an exclusive economic zone, Colombia's actions were unlawful, and treaties reflecting \textit{opinio juris} may bind non-parties. The dispute arose from Nicaragua's challenge to Colombia's maritime actions. The principle confirmed is that treaties indicative of \textit{opinio juris} can bind non-parties, and states have enforceable exclusive economic zone rights.}
}

% Glossary entry for Nicaragua v United States
\newglossaryentry{summary:nicaragua case}{
    type=summaries,
    name={\textit{Nicaragua v United States} (1986) ICJ Rep 14},
    sort={Nicaragua v United States},
    description={Nicaragua sued the US for supporting Contra rebels; the ICJ held that the US violated international law by using force, that customary international law prohibits such force, and that state practice need only be consistent, not uniform, to form custom. The case stemmed from US support for Nicaraguan rebels, challenged by Nicaragua, and established that state sovereignty in customary international law extends to internal waters, territorial sea, and airspace above territory. The principle established is that customary international law bars the use of force, with consistent state practice sufficing to establish customary norms, and that states have complete territorial sovereignty subject to international law.}
}

% Glossary entry for North Sea Continental Shelf
\newglossaryentry{summary:north sea continental shelf}{
    type=summaries,
    name={\textit{North Sea Continental Shelf} (1969) ICJ Rep 3},
    sort={North Sea Continental Shelf},
    description={Germany, Denmark, and the Netherlands disputed continental shelf boundaries; the ICJ held that customary international law requires good-faith negotiation for delimitation, and equidistance is a method, not a binding rule. The case involved a dispute over North Sea continental shelf boundaries. The principle clarified is that states must negotiate in good faith to delimit continental shelves, with equidistance as a non-binding method.}
}

% Glossary entry for Nuclear Test Cases (Australia v France)
\newglossaryentry{summary:nuclear test cases}{
    type=summaries,
    name={\textit{Nuclear Test Cases (Australia v France)} (1974) ICJ Rep 253},
    sort={Nuclear Test Cases},
    description={The ICJ held that France's unilateral public declaration to cease atmospheric nuclear tests was binding, establishing that such declarations, made with intent to be bound, have legal effect. The case arose when Australia challenged France's nuclear testing in the Pacific. The principle confirmed is that unilateral declarations by states, publicly made with intent, create binding international obligations.}
}

% Glossary entry for Republic of India v CCDM Holdings, LLC
\newglossaryentry{summary:india v ccdm}{
    type=summaries,
    name={\textit{Republic of India v CCDM Holdings, LLC} [2025] FCAFC 2},
    sort={Republic of India v CCDM Holdings, LLC},
    description={The Full Federal Court of Australia applied VCLT provisions to determine that India's reservation to a convention modified treaty provisions reciprocally for both India and Australia in a foreign state immunity case, reinforcing the principle of reciprocity in treaty reservations. The case involved a dispute over whether India's reservation applied in Australian proceedings. The principle established is that treaty reservations have reciprocal effects, modifying obligations for both reserving and accepting states.}
}

% Glossary entry for The Paquete Habana
\newglossaryentry{summary:paquete habana}{
    type=summaries,
    name={\textit{The Paquete Habana} (1900) 175 US 677},
    sort={Paquete Habana},
    description={The US Supreme Court held that customary international law, which protects foreign fishing vessels in territorial waters, is part of US law, and judicial decisions provide trustworthy evidence of international law. The case involved the seizure of Cuban fishing vessels by the US during wartime. The principle confirmed is that customary international law is enforceable in domestic courts, with judicial decisions serving as compelling evidence.}
}

% Glossary entry for Ure v Commonwealth
\newglossaryentry{summary:ure v commonwealth}{
    type=summaries,
    name={\textit{Ure v Commonwealth} (2016) 329 ALR 452},
    sort={Ure v Commonwealth},
    description={Australian citizens attempted to claim sovereignty over unclaimed islands; the Federal Court held that only states, not individuals, can assert sovereignty under international law, affirming \statute{\textit{ICJ Statute} Art 38} as good law on international law sources. The case involved citizens' unauthorized sovereignty claims over islands and established that private individuals cannot acquire title in unoccupied land not claimed by a state, as territory must be claimed by state authority rather than private actors. The principle established is that only states have the capacity to assert territorial sovereignty under international law, and individuals may not acquire title in \textit{terra nullius}.}
}

% Glossary entry for Whaling in the Antarctic Case
\newglossaryentry{summary:whaling in the antarctic case}{
    type=summaries,
    name={\textit{Whaling in the Antarctic Case} (2014) ICJ Rep 226},
    sort={Whaling in the Antarctic Case},
    description={Australia challenged Japan's JARPA II whaling program, and the ICJ held that it constituted commercial whaling, violating the International Convention for the Regulation of Whaling, as it was not for scientific research, and resolutions, while not binding, inform treaty interpretation. The case involved Australia's claim that Japan's whaling was commercial, not scientific. The principle clarified is that treaty interpretation under the VCLT considers objective program design, and non-binding resolutions can guide interpretation.}
}

% Glossary entry for ACCC v PT Garuda (No 9)
\newglossaryentry{summary:accc v pt garuda}{
    type=summaries,
    name={\textit{ACCC v PT Garuda (No 9)} [2013] FCA 323},
    sort={ACCC v PT Garuda (No 9)},
    description={The Australian Competition and Consumer Commission (ACCC) alleged that PT Garuda engaged in price fixing for air cargo services, attempting to use expert evidence to prove public international law; the Federal Court rejected this, holding that public international law, like Australian law, cannot be proved by expert evidence, as the court itself determines the law. The case involved the ACCC's claim against PT Garuda for price fixing. The principle established is that public international law is treated as equivalent to Australian law in Australian courts and cannot be proved through expert evidence.}
}

% Glossary entry for Al-Kateb v Godwin
\newglossaryentry{summary:al-kateb v godwin}{
    type=summaries,
    name={\textit{Al-Kateb v Godwin} (2004) 219 CLR 562},
    sort={Al-Kateb v Godwin},
    description={The High Court considered whether a stateless Palestinian man could be indefinitely detained due to the absence of a state for deportation; the majority held that the Australian Constitution is not interpreted to conform with international law, as doing so would violate s 128, which requires a referendum for constitutional amendments. The case involved a challenge to the indefinite detention of a stateless person. The principle clarified is that the Polites principle, which presumes legislation aligns with international law, does not apply to constitutional interpretation.}
}

% Glossary entry for Alabama Claims Arbitration (US/Britain)
\newglossaryentry{summary:alabama claims arbitration}{
    type=summaries,
    name={\textit{Alabama Claims Arbitration (US/Britain)} (1872)},
    sort={Alabama Claims Arbitration (US/Britain)},
    description={The US sought compensation from Britain for failing to prevent the construction of Confederate ships during the US Civil War; the tribunal held that Britain could not justify its failure in due diligence by citing insufficient domestic legal authority, establishing that states cannot rely on absent or inconsistent domestic law to avoid international obligations. The case arose from Britain's failure to stop Confederate shipbuilding. The principle confirmed is that states are accountable for international obligations regardless of domestic law deficiencies.}
}

% Glossary entry for Barcelona Traction (Belgium v Spain)
\newglossaryentry{summary:barcelona traction}{
    type=summaries,
    name={\textit{Barcelona Traction (Belgium v Spain)} [1970] ICJ Rep 3},
    sort={Barcelona Traction (Belgium v Spain)},
    description={Belgium brought a claim against Spain for harm to a Canadian-incorporated company with Belgian shareholders; the ICJ recognized that corporations can have legal personality in international law, allowing their interests to be protected under certain conditions. The principle established is that international law may recognize domestic legal institutions like corporations for the purposes of legal protection.}
}

% Glossary entry for Bradley v Commonwealth
\newglossaryentry{summary:bradley v commonwealth}{
    type=summaries,
    name={\textit{Bradley v Commonwealth} (1973) 128 CLR 557},
    sort={Bradley v Commonwealth},
    description={The Australian government shut down communications to the Rhodesian Information Centre, an agent of the illegal Southern Rhodesian regime, pursuant to a UN Security Council resolution; the High Court held that the resolution had no legal effect in Australia without legislative implementation, as executive action alone cannot incorporate international obligations into domestic law. The case involved the government's attempt to enforce a UN resolution without statutory authority. The principle confirmed is that international resolutions, like treaties, require legislative implementation to have domestic legal effect in Australia.}
}

% Glossary entry for Burkina Faso/Mali
\newglossaryentry{summary:burkina faso mali}{
    type=summaries,
    name={\textit{Burkina Faso/Mali} [1986] ICJ Rep 554},
    sort={Burkina Faso/Mali},
    description={This case concerned the border between Burkina Faso and Mali, which was a colonial border; the ICJ held that the principle of \textit{uti possidetis juris} applies to the borders of states that have emerged from colonial rule. The principle established is that \textit{uti possidetis juris} prevents the independence and stability of new states being endangered by struggles challenging frontiers following the withdrawal of colonial power.}
}

% Glossary entry for Case Concerning Sovereignty Over Pedra Branca/Pulau Batu Puteh, Middle Rocks and South Ledge
\newglossaryentry{summary:pedra branca case}{
    type=summaries,
    name={\textit{Case Concerning Sovereignty Over Pedra Branca/Pulau Batu Puteh, Middle Rocks and South Ledge (Malaysia/Singapore)} [2008] ICJ Rep 12},
    sort={Case Concerning Sovereignty Over Pedra Branca/Pulau Batu Puteh, Middle Rocks and South Ledge},
    description={This case concerned a dispute over islets at the entrance to Singapore Strait, with Malaysia claiming original sovereignty through its predecessor state Johor, while Singapore claimed through occupation or transfer from the UK. The ICJ found that while Malaysia had original title, sovereignty had passed to Singapore by 1980 due to the UK and Singapore's continuous sovereign activities including operating a lighthouse, investigating marine incidents, and land reclamation, combined with Malaysia's failure to respond to these activities. The principle established is that territorial sovereignty can pass through state conduct demonstrating \textit{à titre de souverain} activities, particularly when the original sovereign fails to respond to such conduct.}
}

% Glossary entry for Case Concerning Sovereignty Over Pulau Ligitan and Pulau Sipadan
\newglossaryentry{summary:pulau ligitan sipadan case}{
    type=summaries,
    name={\textit{Case Concerning Sovereignty Over Pulau Ligitan and Pulau Sipadan (Indonesia v Malaysia)} [2002] ICJ Rep 625},
    sort={Case Concerning Sovereignty Over Pulau Ligitan and Pulau Sipadan},
    description={Indonesia and Malaysia disputed sovereignty over islands in the Celebes Sea, with the ICJ examining \textit{effectivities} (effective displays of state authority) to resolve the dispute. The Court found that Malaysia had engaged in diverse sovereign activities including legislative, administrative and quasi-judicial acts such as turtle egg collection regulations and lighthouse establishment, while Indonesia's activities were not of a governmental character. The principle established is that in disputes over small uninhabited islands, courts will examine \textit{effectivities} occurring before the dispute crystallized, with preference given to acts demonstrating genuine state authority \textit{à titre de souverain}.}
}

% Glossary entry for Chagos Islands Advisory Opinion
\newglossaryentry{summary:chagos islands advisory opinion}{
    type=summaries,
    name={\textit{Chagos Islands Advisory Opinion} [2019] ICJ Rep 95},
    sort={Chagos Islands Advisory Opinion},
    description={The ICJ was asked whether the UK's continued administration of the Chagos Archipelago after Mauritius's independence in 1968 was lawful; the Court held that the detachment of Chagos contrary to the will of the people breached their right to self-determination, making the UK's continued administration wrongful. The principle established is that peoples of non-self-governing territories are entitled to exercise self-determination over their territory as a whole, and respect for this right is an obligation \textit{erga omnes}.}
}

% Glossary entry for Chow Hung Ching v R
\newglossaryentry{summary:chow hung ching v r}{
    type=summaries,
    name={\textit{Chow Hung Ching v R} (1949) 77 CLR 449},
    sort={Chow Hung Ching v R},
    description={Chinese army laborers convicted of assault in Papua New Guinea, then under Australian UN mandate, claimed immunity as visiting armed forces; the High Court held that no immunity applied, as they were present as civilians, and that customary international law is a source, not a direct part, of Australian law, applicable only if consistent with domestic law. The case involved a claim of immunity by Chinese laborers for assault charges. The principle established is that customary international law influences Australian common law as a source, but only where it aligns with domestic law.}
}

% Glossary entry for Clipperton Island Arbitration
\newglossaryentry{summary:clipperton island arbitration}{
    type=summaries,
    name={\textit{Clipperton Island Arbitration (France v Mexico)} (1932)},
    sort={Clipperton Island Arbitration},
    description={France and Mexico disputed sovereignty over Clipperton Island, an uninhabited island 1200km off the Mexican coast, with Mexico claiming succession from Spanish discovery in 1836 and France claiming through occupation in 1858. The arbitrator held that Mexico had not exercised sovereignty before French arrival, making the island \textit{territorium nullius}, while France had clearly expressed its intention to consider the island French territory. The principle established is that occupation requires both intention to occupy (\textit{animus occupandi}) and some display of authority, though effective occupation requirements may be relaxed for remote uninhabited territories.}
}

% Glossary entry for Commonwealth v Tasmania
\newglossaryentry{summary:commonwealth v tasmania}{
    type=summaries,
    name={\textit{Commonwealth v Tasmania} (1983) 158 CLR 1},
    sort={Commonwealth v Tasmania},
    description={Tasmania challenged the validity of the World Heritage Conservation Act 1983 (Cth), which implemented the 1972 Convention Concerning the Protection of the World Cultural and Natural Heritage; the High Court held that the legislation was valid under the external affairs power, but the Commonwealth's power to implement treaties is not unlimited and must be proportionate to treaty objectives. The case involved a dispute over Commonwealth legislation protecting world heritage sites. The principle confirmed is that the external affairs power supports treaty implementation, but legislation must be reasonably appropriate and adapted to the treaty's objectives.}
}

% Glossary entry for Customs Union Between Germany and Austria
\newglossaryentry{summary:customs union germany austria}{
    type=summaries,
    name={\textit{Customs Union Between Germany and Austria} (1931) PCIJ Series A/B No 41},
    sort={Customs Union Between Germany and Austria},
    description={The PCIJ was asked whether Austria's entry into a customs union with Germany violated treaties preventing Austria's subsumption within the German Reich; the Court held Austria was in breach but remained a state as it was not under the complete legal authority of another state. The principle established is that a state maintains independence so long as it is not placed under the legal authority of another state, even if it surrenders some economic freedoms.}
}

% Glossary entry for Dietrich v R
\newglossaryentry{summary:dietrich v r}{
    type=summaries,
    name={\textit{Dietrich v R} [1992] HCA 57},
    sort={Dietrich v R},
    description={The accused argued for publicly-funded legal representation under the International Covenant on Civil and Political Rights; the High Court held that treaty provisions do not form part of Australian law unless implemented by statute, as ratification alone is an executive act without domestic legal effect. The case involved a claim for legal aid based on an unimplemented treaty. The principle established is that treaties require legislative incorporation to be binding in Australian law.}
}

% Glossary entry for Habib v Commonwealth
\newglossaryentry{summary:habib v commonwealth}{
    type=summaries,
    name={\textit{Habib v Commonwealth} (2010) 183 FCR 62},
    sort={Habib v Commonwealth},
    description={Habib, an Australian citizen, sued the Australian government for damages alleging torture overseas, claiming Australian authorities knew of his mistreatment; the Federal Court held that the foreign act of state doctrine yields to international prohibitions on torture, allowing the common law to reflect universal norms in serious international crimes. The case involved a civil claim for torture committed abroad. The principle established is that the common law should adapt to reflect universal international norms, such as the prohibition on torture, in civil claims involving serious crimes.}
}

% Glossary entry for Horta v Commonwealth
\newglossaryentry{summary:horta v commonwealth}{
    type=summaries,
    name={\textit{Horta v Commonwealth} (1994) 181 CLR 183},
    sort={Horta v Commonwealth},
    description={Plaintiffs challenged the validity of legislation implementing the 1989 Timor Gap Treaty, arguing it was void due to Indonesia's unlawful occupation of East Timor; the High Court held that the external affairs power supports laws applying to geographically external matters, regardless of the treaty's international legality. The case involved a challenge to a treaty recognizing Indonesia's control over East Timor. The principle confirmed is that the external affairs power validates legislation for external matters, even if the underlying treaty is questioned under international law.}
}

% Glossary entry for Island of Palmas Case
\newglossaryentry{summary:island of palmas case}{
    type=summaries,
    name={\textit{Island of Palmas Case (Netherlands v US)} (1928)},
    sort={Island of Palmas Case},
    description={The US claimed Palmas Island based on Spanish discovery and subsequent cession from Spain after the Spanish-American War, while the Netherlands claimed title through continuous exercise of state authority from 1677 onwards. Judge Huber held that discovery alone without subsequent governmental acts cannot prove sovereignty, and that the Netherlands had effectively occupied the island through continuous, peaceful, and public displays of state authority via the Dutch East India Company. The principle established is that territorial sovereignty requires both \textit{animus occupandi} and effective occupation, with the contemporaneous law determining the creation of rights while later law determines their continued existence.}
}

% Glossary entry for Koowarta v Bjelke-Petersen
\newglossaryentry{summary:koowarta v bjelke-petersen}{
    type=summaries,
    name={\textit{Koowarta v Bjelke-Petersen} (1982) 153 CLR 168},
    sort={Koowarta v Bjelke-Petersen},
    description={The Aboriginal Land Fund Commission sued Queensland for refusing to approve a land transfer to Aboriginal purchasers, relying on the Racial Discrimination Act 1975 (Cth); the High Court upheld the Act's validity as it implemented the 1969 International Convention on the Elimination of All Forms of Racial Discrimination, affirming that treaty implementation falls under the external affairs power. The case involved Queensland's refusal to allow Aboriginal land purchase. The principle established is that the external affairs power supports legislation implementing international treaties, including those addressing discrimination.}
}

% Glossary entry for Law Debenture Trust v Ukraine
\newglossaryentry{summary:law debenture trust v ukraine}{
    type=summaries,
    name={\textit{Law Debenture Trust v Ukraine} [2023] UKSC 11},
    sort={Law Debenture Trust v Ukraine},
    description={Ukraine ceased loan repayments to Russia, citing economic and military duress and countermeasures due to Russia's 2014 annexation of Crimea; the UK Supreme Court held that English law, not international law, governed the contract dispute, but customary international law can serve as a source for common law if consistent with domestic law. The case involved a loan dispute amid Russia-Ukraine conflict. The principle clarified is that customary international law is a source for English common law, but only where it aligns with domestic law.}
}

% Glossary entry for Mabo v Queensland (No 2)
\newglossaryentry{summary:mabo v queensland}{
    type=summaries,
    name={\textit{Mabo v Queensland (No 2)} (1992) 175 CLR 1},
    sort={Mabo v Queensland (No 2)},
    description={The High Court recognized native title for Indigenous Australians, rejecting automatic incorporation of international law into Australian law but acknowledging its influence, particularly for universal human values like human rights. The case involved a claim for Indigenous land rights and addressed whether British acquisition of sovereignty extinguished pre-existing Indigenous land rights, with the Court holding that the fiction of \textit{terra nullius} should be rejected and that acquisition of radical title did not necessarily extinguish native title. The principle established is that international law, while not automatically part of Australian law, is a legitimate influence on common law development, especially for universal norms, and that inhabited land cannot be classified as \textit{terra nullius}.}
}

% Glossary entry for Reference Re Secession of Quebec
\newglossaryentry{summary:reference re secession quebec}{
    type=summaries,
    name={\textit{Reference Re Secession of Quebec} (1988) 2 SCR 217},
    sort={Reference Re Secession of Quebec},
    description={The Supreme Court of Canada addressed whether Quebec had a right to unilateral secession under international law; the Court held that international law does not grant component parts of sovereign states the right to unilateral secession, and self-determination is ordinarily fulfilled through internal means. The principle established is that external self-determination through secession arises only in extreme cases of colonial domination, alien subjugation, or denial of internal self-determination.}
}

% Glossary entry for Reparations for Injuries Case
\newglossaryentry{summary:reparations for injuries case}{
    type=summaries,
    name={\textit{Reparations for Injuries Case} [1949] ICJ Rep 174},
    sort={Reparations for Injuries Case},
    description={The ICJ addressed whether the UN could pursue a claim against Israel for the death of a UN official in Jerusalem; the Court held the UN could seek compensation as it possessed international legal personality necessary to perform its functions. The principle established is that international organizations possess functional international personality different from but necessary for their operations, with the ``attribution of international personality'' being ``indispensable'' for discharging their functions.}
}

% Glossary entry for South China Sea Arbitration
\newglossaryentry{summary:south china sea arbitration}{
    type=summaries,
    name={\textit{South China Sea Arbitration (Philippines v China)} [2016] PCA},
    sort={South China Sea Arbitration},
    description={The Philippines challenged China's claims to the entirety of the South China Sea through the ``Nine-Dash Line'' and China's assertion of maritime zones from various features in the Sea. The Tribunal held that China's Nine-Dash Line was incompatible with UNCLOS as it extended well beyond the 200 nautical mile EEZ limit, that none of the disputed maritime features qualified as islands capable of generating an EEZ or continental shelf, and that China had violated the Philippines' EEZ rights. The principle established is that historical maritime claims are extinguished upon signing UNCLOS, and that maritime features must meet specific criteria under UNCLOS to generate extended maritime zones.}
}

% Glossary entry for Texaco Overseas Petroleum Company v Libya
\newglossaryentry{summary:texaco overseas petroleum v libya}{
    type=summaries,
    name={\textit{Texaco Overseas Petroleum Company v Libya} (1977) 53 ILR 389},
    sort={Texaco Overseas Petroleum Company v Libya},
    description={Texaco sought compensation after Libya nationalized the oil industry, and the arbitration was governed by international law rather than Libyan law; the tribunal held that while corporations lack international legal personality, they can be parties to contracts governed by international law. The principle established is that corporations, though not international persons, can be parties to ``internationalised'' contracts under public international law and avail themselves of its principles.}
}

% Glossary entry for Western Sahara Advisory Opinion
\newglossaryentry{summary:western sahara advisory opinion}{
    type=summaries,
    name={\textit{Western Sahara Advisory Opinion} [1975] ICJ Rep 12},
    sort={Western Sahara Advisory Opinion},
    description={The ICJ was asked about Western Sahara's legal status and Morocco's territorial claims; the Court defined self-determination as ``the need to pay regard to the freely expressed will of peoples'' and found no legal ties that would affect the application of decolonization resolutions. The case concerned competing claims by Morocco and Mauritania over the former Spanish colony, with the ICJ holding that territories inhabited by socially and politically organized peoples are not \textit{terra nullius}, even if sparsely populated by nomadic peoples, and that occupation is only valid if territory is truly uninhabited. The principle established is that self-determination requires respect for the freely expressed will of peoples, historical ties do not override this fundamental right, and occupation requires territory to be genuinely \textit{terra nullius}.}
}

% Glossary entry for WHO Advisory Opinion
\newglossaryentry{summary:who advisory opinion}{
    type=summaries,
    name={\textit{WHO Advisory Opinion} [1996] ICJ Rep 66},
    sort={WHO Advisory Opinion},
    description={The ICJ examined requests for advisory opinions on nuclear weapons legality from both the UN General Assembly and WHO; while accepting the UN's request, the Court rejected WHO's request as the question fell outside WHO's scope of activities. The principle established is that international organizations are governed by the principle of speciality and can only act within the limits of their powers as defined by their constitutive treaties.}
}