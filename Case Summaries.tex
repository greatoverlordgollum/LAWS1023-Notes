% Defining the glossary for case summaries
\newglossary[slg]{summaries}{syi}{syg}{Case Summaries}

% Glossary entry for Al-Saadoon v Secretary of State for Defence
\newglossaryentry{summary:al-saadoon v sec. def}{
    type=summaries,
    name={\textit{Al-Saadoon v Secretary of State for Defence} [2010] EWCA Civ 776},
    sort={Al-Saadoon v Secretary of State for Defence},
    description={Al-Saadoon and others were detained in Iraq by British forces; the UK Court of Appeal held that the UK was bound by the Geneva Conventions, and had a duty to ensure detainees were treated in accordance with international law, including the prohibition on torture, demonstrating that regional customary norms can be applied as international law if they meet the high threshold of stability and continuity. British forces detained Al-Saadoon and others in Iraq, prompting a challenge on their treatment. The principle established is that regional customary norms, when sufficiently stable and continuous, can be enforced as international law, obligating states to uphold international humanitarian standards such as the prohibition on torture.}
}

% Glossary entry for Anglo-Norwegian Fisheries Case
\newglossaryentry{summary:anglo norwegian fisheries}{
    type=summaries,
    name={\textit{Anglo-Norwegian Fisheries Case} (1951) ICJ Rep 116},
    sort={Anglo-Norwegian Fisheries Case},
    description={The UK challenged Norway's method of drawing baselines, asserting a customary rule prohibiting it; the ICJ held that Norway was exempt from this rule due to its persistent objection from the rule's inception. The case involved a dispute where the UK contested Norway's baseline delineation for its fisheries zone. The principle confirmed is that a state persistently objecting to a customary international law rule from its formation is not bound by it.}
}

% Glossary entry for Asylum Case (Colombia v Peru)
\newglossaryentry{summary:asylum case}{
    type=summaries,
    name={\textit{Asylum Case (Colombia v Peru)} (1950) ICJ Rep 266},
    sort={Asylum Case},
    description={Colombia granted political asylum to de la Torre, which Peru challenged; Colombia invoked a regional customary norm, but the ICJ rejected this, holding that regional standards require a higher degree of stability and continuity to apply as international law. The case arose when Colombia granted asylum to a Peruvian political figure, contested by Peru. The principle established is that regional customary norms must exhibit significant stability and continuity to be recognized as binding international law.}
}

% Glossary entry for Bay of Bengal (Bangladesh v Myanmar)
\newglossaryentry{summary:bay of bengal (bangladesh v myanmar)}{
    type=summaries,
    name={\textit{Bay of Bengal (Bangladesh v Myanmar)} (2012) ITLOS 12},
    sort={Bay of Bengal (Bangladesh v Myanmar)},
    description={Bangladesh and Myanmar disputed the delimitation of their maritime boundary in the Bay of Bengal; the International Tribunal for the Law of the Sea resolved the dispute by implicitly adopting general principles of international law into its decision. The case involved a maritime boundary dispute between Bangladesh and Myanmar in the Bay of Bengal. The principle applied is that general principles of international law can be implicitly incorporated to resolve maritime delimitation disputes.}
}

% Glossary entry for Chagos Marine Protected Area Arbitration
\newglossaryentry{summary:chagos marine protected area}{
    type=summaries,
    name={\textit{Chagos Marine Protected Area Arbitration (Mauritius v UK)} (2015) XXXI RIAA 359},
    sort={Chagos Marine Protected Area Arbitration},
    description={Mauritius challenged the UK's establishment of a Marine Protected Area in the Chagos Archipelago; the Tribunal held that it could apply the doctrine of estoppel, derived from various domestic legal systems, to determine the position under international law. The dispute arose over the UK's creation of a marine protected area, contested by Mauritius. The principle established is that international tribunals may draw on domestic legal doctrines like estoppel to inform decisions under international law.}
}

% Glossary entry for Gabčíkovo-Nagymaros Case
\newglossaryentry{summary:gabcikovo-nagymaros case}{
    type=summaries,
    name={\textit{Gabčíkovo-Nagymaros Case} (1997) ICJ Rep 7},
    sort={Gabčíkovo-Nagymaros Case},
    description={Hungary and Czechoslovakia disagreed over a 1977 treaty for a joint dam project on the Danube; Hungary suspended work due to environmental concerns, and the ICJ held that neither impossibility nor fundamental change of circumstances justified termination, emphasizing the principle of \textit{pacta sunt servanda}. The case involved Hungary's attempt to terminate the treaty after suspending work, countered by Czechoslovakia's unilateral alternative. The principle reinforced is that treaties remain binding under \textit{pacta sunt servanda}, and termination based on impossibility or fundamental change of circumstances requires exceptional conditions.}
}

% Glossary entry for Legal Status of Eastern Greenland
\newglossaryentry{summary:legal status of eastern greenland}{
    type=summaries,
    name={\textit{Legal Status of Eastern Greenland (Denmark v Norway)} (1933) PCIJ Series A/B, No 53},
    sort={Legal Status of Eastern Greenland},
    description={The Permanent Court of International Justice held that Norway was bound by an oral undertaking given to Denmark not to oppose its claim to sovereignty over Greenland, obligating Norway to refrain from contesting Danish sovereignty. The case arose when Norway occupied Eastern Greenland in 1931, challenging Denmark's claim based on sovereignty from the 1700s, but the PCIJ found Denmark had displayed sufficient authority through a latent claim that had never been challenged until 1921. The principle established is that oral undertakings by states can create binding international obligations, and that very little exercise of sovereignty may suffice for territorial claims in thinly populated areas, provided no superior competing claim exists.}
}

% Glossary entry for Legality of the Threat or Use of Nuclear Weapons
\newglossaryentry{summary:nuclear weapons case}{
    type=summaries,
    name={\textit{Legality of the Threat or Use of Nuclear Weapons} (1996) ICJ Rep 226},
    sort={Legality of the Threat or Use of Nuclear Weapons},
    description={The ICJ held that the use of nuclear weapons is generally contrary to international law, but no specific rule prohibits their use in all circumstances, and UNGA resolutions, while not binding, can provide evidence of customary international law. The case addressed a request for an advisory opinion on the legality of nuclear weapons. The principle clarified is that UNGA resolutions may reflect customary international law, and nuclear weapon use is generally unlawful absent exceptional circumstances.}
}

% Glossary entry for Maritime Delimitation and Territorial Questions (Qatar v Bahrain)
\newglossaryentry{summary:qatar v bahrain}{
    type=summaries,
    name={\textit{Maritime Delimitation and Territorial Questions (Qatar v Bahrain)} (1994) ICJ Rep 112},
    sort={Maritime Delimitation and Territorial Questions (Qatar v Bahrain)},
    description={The ICJ held that the 1987 exchanges of letters and 1990 Doha Minutes between Qatar and Bahrain constituted binding international agreements, obligating them to submit their maritime and territorial disputes to the Court, emphasizing that exchanges of letters can form valid treaties. The case involved Qatar's unilateral referral of disputes over the Hawar Islands and maritime boundaries, contested by Bahrain. The principle established is that exchanges of letters and minutes can constitute binding treaties under international law.}
}

% Glossary entry for Nicaragua v Colombia
\newglossaryentry{summary:nicaragua v colombia}{
    type=summaries,
    name={\textit{Nicaragua v Colombia} (2012) ICJ Rep 624},
    sort={Nicaragua v Colombia},
    description={Nicaragua claimed Colombia violated its continental shelf and exclusive economic zone rights; the ICJ held that Nicaragua had rights to an exclusive economic zone, Colombia's actions were unlawful, and treaties reflecting \textit{opinio juris} may bind non-parties. The dispute arose from Nicaragua's challenge to Colombia's maritime actions. The principle confirmed is that treaties indicative of \textit{opinio juris} can bind non-parties, and states have enforceable exclusive economic zone rights.}
}

% Glossary entry for Nicaragua v United States
\newglossaryentry{summary:nicaragua case}{
    type=summaries,
    name={\textit{Nicaragua v United States} (1986) ICJ Rep 14},
    sort={Nicaragua v United States},
    description={Nicaragua sued the US for supporting Contra rebels; the ICJ held that the US violated international law by using force, that customary international law prohibits such force, and that state practice need only be consistent, not uniform, to form custom. The case stemmed from US support for Nicaraguan rebels, challenged by Nicaragua, and established that state sovereignty in customary international law extends to internal waters, territorial sea, and airspace above territory. The principle established is that customary international law bars the use of force, with consistent state practice sufficing to establish customary norms, and that states have complete territorial sovereignty subject to international law.}
}

% Glossary entry for North Sea Continental Shelf
\newglossaryentry{summary:north sea continental shelf}{
    type=summaries,
    name={\textit{North Sea Continental Shelf} (1969) ICJ Rep 3},
    sort={North Sea Continental Shelf},
    description={Germany, Denmark, and the Netherlands disputed continental shelf boundaries; the ICJ held that customary international law requires good-faith negotiation for delimitation, and equidistance is a method, not a binding rule. The case involved a dispute over North Sea continental shelf boundaries. The principle clarified is that states must negotiate in good faith to delimit continental shelves, with equidistance as a non-binding method.}
}

% Glossary entry for Nuclear Test Cases (Australia v France)
\newglossaryentry{summary:nuclear test cases}{
    type=summaries,
    name={\textit{Nuclear Test Cases (Australia v France)} (1974) ICJ Rep 253},
    sort={Nuclear Test Cases},
    description={The ICJ held that France's unilateral public declaration to cease atmospheric nuclear tests was binding, establishing that such declarations, made with intent to be bound, have legal effect. The case arose when Australia challenged France's nuclear testing in the Pacific. The principle confirmed is that unilateral declarations by states, publicly made with intent, create binding international obligations.}
}

% Glossary entry for Republic of India v CCDM Holdings, LLC
\newglossaryentry{summary:india v ccdm}{
    type=summaries,
    name={\textit{Republic of India v CCDM Holdings, LLC} [2025] FCAFC 2},
    sort={Republic of India v CCDM Holdings, LLC},
    description={The Full Federal Court of Australia applied VCLT provisions to determine that India's reservation to a convention modified treaty provisions reciprocally for both India and Australia in a foreign state immunity case, reinforcing the principle of reciprocity in treaty reservations. The case involved a dispute over whether India's reservation applied in Australian proceedings. The principle established is that treaty reservations have reciprocal effects, modifying obligations for both reserving and accepting states.}
}

% Glossary entry for The Paquete Habana
\newglossaryentry{summary:paquete habana}{
    type=summaries,
    name={\textit{The Paquete Habana} (1900) 175 US 677},
    sort={Paquete Habana},
    description={The US Supreme Court held that customary international law, which protects foreign fishing vessels in territorial waters, is part of US law, and judicial decisions provide trustworthy evidence of international law. The case involved the seizure of Cuban fishing vessels by the US during wartime. The principle confirmed is that customary international law is enforceable in domestic courts, with judicial decisions serving as compelling evidence.}
}

% Glossary entry for Ure v Commonwealth
\newglossaryentry{summary:ure v commonwealth}{
    type=summaries,
    name={\textit{Ure v Commonwealth} (2016) 329 ALR 452},
    sort={Ure v Commonwealth},
    description={Australian citizens attempted to claim sovereignty over unclaimed islands; the Federal Court held that only states, not individuals, can assert sovereignty under international law, affirming \statute{\textit{ICJ Statute} Art 38} as good law on international law sources. The case involved citizens' unauthorized sovereignty claims over islands and established that private individuals cannot acquire title in unoccupied land not claimed by a state, as territory must be claimed by state authority rather than private actors. The principle established is that only states have the capacity to assert territorial sovereignty under international law, and individuals may not acquire title in \textit{terra nullius}.}
}

% Glossary entry for Whaling in the Antarctic Case
\newglossaryentry{summary:whaling in the antarctic case}{
    type=summaries,
    name={\textit{Whaling in the Antarctic Case} (2014) ICJ Rep 226},
    sort={Whaling in the Antarctic Case},
    description={Australia challenged Japan's JARPA II whaling program, and the ICJ held that it constituted commercial whaling, violating the International Convention for the Regulation of Whaling, as it was not for scientific research, and resolutions, while not binding, inform treaty interpretation. The case involved Australia's claim that Japan's whaling was commercial, not scientific. The principle clarified is that treaty interpretation under the VCLT considers objective program design, and non-binding resolutions can guide interpretation.}
}

% Glossary entry for ACCC v PT Garuda (No 9)
\newglossaryentry{summary:accc v pt garuda}{
    type=summaries,
    name={\textit{ACCC v PT Garuda (No 9)} [2013] FCA 323},
    sort={ACCC v PT Garuda (No 9)},
    description={The Australian Competition and Consumer Commission (ACCC) alleged that PT Garuda engaged in price fixing for air cargo services, attempting to use expert evidence to prove public international law; the Federal Court rejected this, holding that public international law, like Australian law, cannot be proved by expert evidence, as the court itself determines the law. The case involved the ACCC's claim against PT Garuda for price fixing. The principle established is that public international law is treated as equivalent to Australian law in Australian courts and cannot be proved through expert evidence.}
}

% Glossary entry for Al-Kateb v Godwin
\newglossaryentry{summary:al-kateb v godwin}{
    type=summaries,
    name={\textit{Al-Kateb v Godwin} (2004) 219 CLR 562},
    sort={Al-Kateb v Godwin},
    description={The High Court considered whether a stateless Palestinian man could be indefinitely detained due to the absence of a state for deportation; the majority held that the Australian Constitution is not interpreted to conform with international law, as doing so would violate s 128, which requires a referendum for constitutional amendments. The case involved a challenge to the indefinite detention of a stateless person. The principle clarified is that the Polites principle, which presumes legislation aligns with international law, does not apply to constitutional interpretation.}
}

% Glossary entry for Alabama Claims Arbitration (US/Britain)
\newglossaryentry{summary:alabama claims arbitration}{
    type=summaries,
    name={\textit{Alabama Claims Arbitration (US/Britain)} (1872)},
    sort={Alabama Claims Arbitration (US/Britain)},
    description={The US sought compensation from Britain for failing to prevent the construction of Confederate ships during the US Civil War; the tribunal held that Britain could not justify its failure in due diligence by citing insufficient domestic legal authority, establishing that states cannot rely on absent or inconsistent domestic law to avoid international obligations. The case arose from Britain's failure to stop Confederate shipbuilding. The principle confirmed is that states are accountable for international obligations regardless of domestic law deficiencies.}
}

% Glossary entry for Barcelona Traction (Belgium v Spain)
\newglossaryentry{summary:barcelona traction}{
    type=summaries,
    name={\textit{Barcelona Traction (Belgium v Spain)} [1970] ICJ Rep 3},
    sort={Barcelona Traction (Belgium v Spain)},
    description={Belgium brought a claim against Spain for harm to a Canadian-incorporated company with Belgian shareholders; the ICJ recognized that corporations can have legal personality in international law, allowing their interests to be protected under certain conditions. The principle established is that international law may recognize domestic legal institutions like corporations for the purposes of legal protection.}
}

% Glossary entry for Bradley v Commonwealth
\newglossaryentry{summary:bradley v commonwealth}{
    type=summaries,
    name={\textit{Bradley v Commonwealth} (1973) 128 CLR 557},
    sort={Bradley v Commonwealth},
    description={The Australian government shut down communications to the Rhodesian Information Centre, an agent of the illegal Southern Rhodesian regime, pursuant to a UN Security Council resolution; the High Court held that the resolution had no legal effect in Australia without legislative implementation, as executive action alone cannot incorporate international obligations into domestic law. The case involved the government's attempt to enforce a UN resolution without statutory authority. The principle confirmed is that international resolutions, like treaties, require legislative implementation to have domestic legal effect in Australia.}
}

% Glossary entry for Burkina Faso/Mali
\newglossaryentry{summary:burkina faso mali}{
    type=summaries,
    name={\textit{Burkina Faso/Mali} [1986] ICJ Rep 554},
    sort={Burkina Faso/Mali},
    description={This case concerned the border between Burkina Faso and Mali, which was a colonial border; the ICJ held that the principle of \textit{uti possidetis juris} applies to the borders of states that have emerged from colonial rule. The principle established is that \textit{uti possidetis juris} prevents the independence and stability of new states being endangered by struggles challenging frontiers following the withdrawal of colonial power.}
}

% Glossary entry for Case Concerning Sovereignty Over Pedra Branca/Pulau Batu Puteh, Middle Rocks and South Ledge
\newglossaryentry{summary:pedra branca case}{
    type=summaries,
    name={\textit{Case Concerning Sovereignty Over Pedra Branca/Pulau Batu Puteh, Middle Rocks and South Ledge (Malaysia/Singapore)} [2008] ICJ Rep 12},
    sort={Case Concerning Sovereignty Over Pedra Branca/Pulau Batu Puteh, Middle Rocks and South Ledge},
    description={This case concerned a dispute over islets at the entrance to Singapore Strait, with Malaysia claiming original sovereignty through its predecessor state Johor, while Singapore claimed through occupation or transfer from the UK. The ICJ found that while Malaysia had original title, sovereignty had passed to Singapore by 1980 due to the UK and Singapore's continuous sovereign activities including operating a lighthouse, investigating marine incidents, and land reclamation, combined with Malaysia's failure to respond to these activities. The principle established is that territorial sovereignty can pass through state conduct demonstrating \textit{à titre de souverain} activities, particularly when the original sovereign fails to respond to such conduct.}
}

% Glossary entry for Case Concerning Sovereignty Over Pulau Ligitan and Pulau Sipadan
\newglossaryentry{summary:pulau ligitan sipadan case}{
    type=summaries,
    name={\textit{Case Concerning Sovereignty Over Pulau Ligitan and Pulau Sipadan (Indonesia v Malaysia)} [2002] ICJ Rep 625},
    sort={Case Concerning Sovereignty Over Pulau Ligitan and Pulau Sipadan},
    description={Indonesia and Malaysia disputed sovereignty over islands in the Celebes Sea, with the ICJ examining \textit{effectivities} (effective displays of state authority) to resolve the dispute. The Court found that Malaysia had engaged in diverse sovereign activities including legislative, administrative and quasi-judicial acts such as turtle egg collection regulations and lighthouse establishment, while Indonesia's activities were not of a governmental character. The principle established is that in disputes over small uninhabited islands, courts will examine \textit{effectivities} occurring before the dispute crystallized, with preference given to acts demonstrating genuine state authority \textit{à titre de souverain}.}
}

% Glossary entry for Chagos Islands Advisory Opinion
\newglossaryentry{summary:chagos islands advisory opinion}{
    type=summaries,
    name={\textit{Chagos Islands Advisory Opinion} [2019] ICJ Rep 95},
    sort={Chagos Islands Advisory Opinion},
    description={The ICJ was asked whether the UK's continued administration of the Chagos Archipelago after Mauritius's independence in 1968 was lawful; the Court held that the detachment of Chagos contrary to the will of the people breached their right to self-determination, making the UK's continued administration wrongful. The principle established is that peoples of non-self-governing territories are entitled to exercise self-determination over their territory as a whole, and respect for this right is an obligation \textit{erga omnes}.}
}

% Glossary entry for Chow Hung Ching v R
\newglossaryentry{summary:chow hung ching v r}{
    type=summaries,
    name={\textit{Chow Hung Ching v R} (1949) 77 CLR 449},
    sort={Chow Hung Ching v R},
    description={Chinese army laborers convicted of assault in Papua New Guinea, then under Australian UN mandate, claimed immunity as visiting armed forces; the High Court held that no immunity applied, as they were present as civilians, and that customary international law is a source, not a direct part, of Australian law, applicable only if consistent with domestic law. The case involved a claim of immunity by Chinese laborers for assault charges. The principle established is that customary international law influences Australian common law as a source, but only where it aligns with domestic law.}
}

% Glossary entry for Clipperton Island Arbitration
\newglossaryentry{summary:clipperton island arbitration}{
    type=summaries,
    name={\textit{Clipperton Island Arbitration (France v Mexico)} (1932)},
    sort={Clipperton Island Arbitration},
    description={France and Mexico disputed sovereignty over Clipperton Island, an uninhabited island 1200km off the Mexican coast, with Mexico claiming succession from Spanish discovery in 1836 and France claiming through occupation in 1858. The arbitrator held that Mexico had not exercised sovereignty before French arrival, making the island \textit{territorium nullius}, while France had clearly expressed its intention to consider the island French territory. The principle established is that occupation requires both intention to occupy (\textit{animus occupandi}) and some display of authority, though effective occupation requirements may be relaxed for remote uninhabited territories.}
}

% Glossary entry for Commonwealth v Tasmania
\newglossaryentry{summary:commonwealth v tasmania}{
    type=summaries,
    name={\textit{Commonwealth v Tasmania} (1983) 158 CLR 1},
    sort={Commonwealth v Tasmania},
    description={Tasmania challenged the validity of the World Heritage Conservation Act 1983 (Cth), which implemented the 1972 Convention Concerning the Protection of the World Cultural and Natural Heritage; the High Court held that the legislation was valid under the external affairs power, but the Commonwealth's power to implement treaties is not unlimited and must be proportionate to treaty objectives. The case involved a dispute over Commonwealth legislation protecting world heritage sites. The principle confirmed is that the external affairs power supports treaty implementation, but legislation must be reasonably appropriate and adapted to the treaty's objectives.}
}

% Glossary entry for Customs Union Between Germany and Austria
\newglossaryentry{summary:customs union germany austria}{
    type=summaries,
    name={\textit{Customs Union Between Germany and Austria} (1931) PCIJ Series A/B No 41},
    sort={Customs Union Between Germany and Austria},
    description={The PCIJ was asked whether Austria's entry into a customs union with Germany violated treaties preventing Austria's subsumption within the German Reich; the Court held Austria was in breach but remained a state as it was not under the complete legal authority of another state. The principle established is that a state maintains independence so long as it is not placed under the legal authority of another state, even if it surrenders some economic freedoms.}
}

% Glossary entry for Dietrich v R
\newglossaryentry{summary:dietrich v r}{
    type=summaries,
    name={\textit{Dietrich v R} [1992] HCA 57},
    sort={Dietrich v R},
    description={The accused argued for publicly-funded legal representation under the International Covenant on Civil and Political Rights; the High Court held that treaty provisions do not form part of Australian law unless implemented by statute, as ratification alone is an executive act without domestic legal effect. The case involved a claim for legal aid based on an unimplemented treaty. The principle established is that treaties require legislative incorporation to be binding in Australian law.}
}

% Glossary entry for Habib v Commonwealth
\newglossaryentry{summary:habib v commonwealth}{
    type=summaries,
    name={\textit{Habib v Commonwealth} (2010) 183 FCR 62},
    sort={Habib v Commonwealth},
    description={Habib, an Australian citizen, sued the Australian government for damages alleging torture overseas, claiming Australian authorities knew of his mistreatment; the Federal Court held that the foreign act of state doctrine yields to international prohibitions on torture, allowing the common law to reflect universal norms in serious international crimes. The case involved a civil claim for torture committed abroad. The principle established is that the common law should adapt to reflect universal international norms, such as the prohibition on torture, in civil claims involving serious crimes.}
}

% Glossary entry for Horta v Commonwealth
\newglossaryentry{summary:horta v commonwealth}{
    type=summaries,
    name={\textit{Horta v Commonwealth} (1994) 181 CLR 183},
    sort={Horta v Commonwealth},
    description={Plaintiffs challenged the validity of legislation implementing the 1989 Timor Gap Treaty, arguing it was void due to Indonesia's unlawful occupation of East Timor; the High Court held that the external affairs power supports laws applying to geographically external matters, regardless of the treaty's international legality. The case involved a challenge to a treaty recognizing Indonesia's control over East Timor. The principle confirmed is that the external affairs power validates legislation for external matters, even if the underlying treaty is questioned under international law.}
}

% Glossary entry for Island of Palmas Case
\newglossaryentry{summary:island of palmas case}{
    type=summaries,
    name={\textit{Island of Palmas Case (Netherlands v US)} (1928)},
    sort={Island of Palmas Case},
    description={The US claimed Palmas Island based on Spanish discovery and subsequent cession from Spain after the Spanish-American War, while the Netherlands claimed title through continuous exercise of state authority from 1677 onwards. Judge Huber held that discovery alone without subsequent governmental acts cannot prove sovereignty, and that the Netherlands had effectively occupied the island through continuous, peaceful, and public displays of state authority via the Dutch East India Company. The principle established is that territorial sovereignty requires both \textit{animus occupandi} and effective occupation, with the contemporaneous law determining the creation of rights while later law determines their continued existence.}
}

% Glossary entry for Koowarta v Bjelke-Petersen
\newglossaryentry{summary:koowarta v bjelke-petersen}{
    type=summaries,
    name={\textit{Koowarta v Bjelke-Petersen} (1982) 153 CLR 168},
    sort={Koowarta v Bjelke-Petersen},
    description={The Aboriginal Land Fund Commission sued Queensland for refusing to approve a land transfer to Aboriginal purchasers, relying on the Racial Discrimination Act 1975 (Cth); the High Court upheld the Act's validity as it implemented the 1969 International Convention on the Elimination of All Forms of Racial Discrimination, affirming that treaty implementation falls under the external affairs power. The case involved Queensland's refusal to allow Aboriginal land purchase. The principle established is that the external affairs power supports legislation implementing international treaties, including those addressing discrimination.}
}

% Glossary entry for Law Debenture Trust v Ukraine
\newglossaryentry{summary:law debenture trust v ukraine}{
    type=summaries,
    name={\textit{Law Debenture Trust v Ukraine} [2023] UKSC 11},
    sort={Law Debenture Trust v Ukraine},
    description={Ukraine ceased loan repayments to Russia, citing economic and military duress and countermeasures due to Russia's 2014 annexation of Crimea; the UK Supreme Court held that English law, not international law, governed the contract dispute, but customary international law can serve as a source for common law if consistent with domestic law. The case involved a loan dispute amid Russia-Ukraine conflict. The principle clarified is that customary international law is a source for English common law, but only where it aligns with domestic law.}
}

% Glossary entry for Mabo v Queensland (No 2)
\newglossaryentry{summary:mabo v queensland}{
    type=summaries,
    name={\textit{Mabo v Queensland (No 2)} (1992) 175 CLR 1},
    sort={Mabo v Queensland (No 2)},
    description={The High Court recognized native title for Indigenous Australians, rejecting automatic incorporation of international law into Australian law but acknowledging its influence, particularly for universal human values like human rights. The case involved a claim for Indigenous land rights and addressed whether British acquisition of sovereignty extinguished pre-existing Indigenous land rights, with the Court holding that the fiction of \textit{terra nullius} should be rejected and that acquisition of radical title did not necessarily extinguish native title. The principle established is that international law, while not automatically part of Australian law, is a legitimate influence on common law development, especially for universal norms, and that inhabited land cannot be classified as \textit{terra nullius}.}
}

% Glossary entry for Reference Re Secession of Quebec
\newglossaryentry{summary:reference re secession quebec}{
    type=summaries,
    name={\textit{Reference Re Secession of Quebec} (1988) 2 SCR 217},
    sort={Reference Re Secession of Quebec},
    description={The Supreme Court of Canada addressed whether Quebec had a right to unilateral secession under international law; the Court held that international law does not grant component parts of sovereign states the right to unilateral secession, and self-determination is ordinarily fulfilled through internal means. The principle established is that external self-determination through secession arises only in extreme cases of colonial domination, alien subjugation, or denial of internal self-determination.}
}

% Glossary entry for Reparations for Injuries Case
\newglossaryentry{summary:reparations for injuries case}{
    type=summaries,
    name={\textit{Reparations for Injuries Case} [1949] ICJ Rep 174},
    sort={Reparations for Injuries Case},
    description={The ICJ addressed whether the UN could pursue a claim against Israel for the death of a UN official in Jerusalem; the Court held the UN could seek compensation as it possessed international legal personality necessary to perform its functions. The principle established is that international organizations possess functional international personality different from but necessary for their operations, with the ``attribution of international personality'' being ``indispensable'' for discharging their functions.}
}

% Glossary entry for South China Sea Arbitration
\newglossaryentry{summary:south china sea arbitration}{
    type=summaries,
    name={\textit{South China Sea Arbitration (Philippines v China)} [2016] PCA},
    sort={South China Sea Arbitration},
    description={The Philippines challenged China's claims to the entirety of the South China Sea through the ``Nine-Dash Line'' and China's assertion of maritime zones from various features in the Sea. The Tribunal held that China's Nine-Dash Line was incompatible with UNCLOS as it extended well beyond the 200 nautical mile EEZ limit, that none of the disputed maritime features qualified as islands capable of generating an EEZ or continental shelf, and that China had violated the Philippines' EEZ rights. The principle established is that historical maritime claims are extinguished upon signing UNCLOS, and that maritime features must meet specific criteria under UNCLOS to generate extended maritime zones.}
}

% Glossary entry for Texaco Overseas Petroleum Company v Libya
\newglossaryentry{summary:texaco overseas petroleum v libya}{
    type=summaries,
    name={\textit{Texaco Overseas Petroleum Company v Libya} (1977) 53 ILR 389},
    sort={Texaco Overseas Petroleum Company v Libya},
    description={Texaco sought compensation after Libya nationalized the oil industry, and the arbitration was governed by international law rather than Libyan law; the tribunal held that while corporations lack international legal personality, they can be parties to contracts governed by international law. The principle established is that corporations, though not international persons, can be parties to ``internationalised'' contracts under public international law and avail themselves of its principles.}
}

% Glossary entry for Western Sahara Advisory Opinion
\newglossaryentry{summary:western sahara advisory opinion}{
    type=summaries,
    name={\textit{Western Sahara Advisory Opinion} [1975] ICJ Rep 12},
    sort={Western Sahara Advisory Opinion},
    description={The ICJ was asked about Western Sahara's legal status and Morocco's territorial claims; the Court defined self-determination as ``the need to pay regard to the freely expressed will of peoples'' and found no legal ties that would affect the application of decolonization resolutions. The case concerned competing claims by Morocco and Mauritania over the former Spanish colony, with the ICJ holding that territories inhabited by socially and politically organized peoples are not \textit{terra nullius}, even if sparsely populated by nomadic peoples, and that occupation is only valid if territory is truly uninhabited. The principle established is that self-determination requires respect for the freely expressed will of peoples, historical ties do not override this fundamental right, and occupation requires territory to be genuinely \textit{terra nullius}.}
}

% Glossary entry for WHO Advisory Opinion
\newglossaryentry{summary:who advisory opinion}{
    type=summaries,
    name={\textit{WHO Advisory Opinion} [1996] ICJ Rep 66},
    sort={WHO Advisory Opinion},
    description={The ICJ examined requests for advisory opinions on nuclear weapons legality from both the UN General Assembly and WHO; while accepting the UN's request, the Court rejected WHO's request as the question fell outside WHO's scope of activities. The principle established is that international organizations are governed by the principle of speciality and can only act within the limits of their powers as defined by their constitutive treaties.}
}

% Glossary entry for Criminal Complaint against Donald Rumsfeld
\newglossaryentry{summary:criminal complaint against donald rumsfeld}{
    type=summaries,
    name={\textit{Criminal Complaint against Donald Rumsfeld} (2007) German Prosecutor-General},
    sort={Criminal Complaint against Donald Rumsfeld},
    description={A complaint was filed with the German Prosecutor-General against the former U.S. Secretary of Defense for alleged war crimes involving torture at Abu Ghraib and Guantanamo Bay. The Prosecutor-General declined to proceed, ruling that despite Germany's technical ability to exercise universal jurisdiction over such war crimes, the lack of any connection between Germany and the alleged wrongdoing made it more appropriate for the U.S. justice system to handle the matter. The principle established is that universal jurisdiction is permissive rather than mandatory, requiring a ``legitimizing domestic linkage'' to justify prosecution of foreigners for crimes committed abroad against foreigners.}
}

% Glossary entry for Firebird Global Master Fund II Ltd v Nauru
\newglossaryentry{summary:firebird global master fund v nauru}{
    type=summaries,
    name={\textit{Firebird Global Master Fund II Ltd v Nauru} (2015) 326 ALR 396},
    sort={Firebird Global Master Fund II Ltd v Nauru},
    description={Firebird sought to register and enforce in Australia a Japanese judgment against Nauru related to bearer bonds guaranteed by Nauru. The High Court determined that the commercial transaction exception applied to the Japanese judgment, but Nauru's bank accounts in Australia were immune from legal process as they were used exclusively for governmental rather than commercial purposes. The principle established is that while foreign judgments may be recognised under commercial transaction exceptions, enforcement may still fail if available assets are protected for non-commercial governmental functions.}
}

% Glossary entry for I Congresso del Partido
\newglossaryentry{summary:i congresso del partido}{
    type=summaries,
    name={\textit{I Congresso del Partido} (1983) HoL},
    sort={I Congresso del Partido},
    description={Ships owned by the Cuban government failed to deliver sugar cargo to Chile, prompting legal action in the UK when Cuba directed vessels to withhold cargo following Pinochet's coup. The House of Lords rejected Cuba's sovereign immunity claim, applying restrictive immunity by focusing on the nature rather than purpose of the act. The principle established is that commercial transactions are not protected by sovereign immunity even when politically motivated, as the trading relationship remains commercial in character.}
}

% Glossary entry for Kingdom of Spain v Infrastructure Services Luxembourg
\newglossaryentry{summary:kingdom of spain v infrastructure services luxembourg}{
    type=summaries,
    name={\textit{Kingdom of Spain v Infrastructure Services Luxembourg S.à.r.l.} [2023] HCA 11},
    sort={Kingdom of Spain v Infrastructure Services Luxembourg},
    description={A company sought to enforce a €101 million ICSID arbitral award against Spain in Australia, with Spain invoking immunity under the Foreign States Immunities Act. The High Court held that Spain's agreement to ICSID Convention Articles 53-55 constituted an express waiver of immunity for recognition and enforcement proceedings, but not for execution against Spanish sovereign property. The principle established is that waivers of foreign state immunity must be express and derived from international agreement terms, with recognition of awards being separate from their execution against state assets.}
}

% Glossary entry for Moti v R
\newglossaryentry{summary:moti v r}{
    type=summaries,
    name={\textit{Moti v R} (2011) HCA},
    sort={Moti v R},
    description={The High Court permanently stayed prosecution of the former Attorney-General of the Solomon Islands who had been unlawfully removed from Solomon Islands to Australia to face child sex offence charges. Despite being an Australian citizen accused of serious crimes, the Court held that his unlawful removal violated Solomon Islands law and required permanent stay of proceedings. The principle established is that illegally obtained custody can bar criminal prosecution, as the ends of criminal justice do not justify unlawful means of securing an accused's presence.}
}

% Glossary entry for National Commissioner of the South African Police Service
\newglossaryentry{summary:national commissioner sa police v litigation centre}{
    type=summaries,
    name={\textit{National Commissioner of the South African Police Service v Southern African Human Rights Litigation Centre} (2013) SA Const Court},
    sort={National Commissioner of the South African Police Service},
    description={The South African Constitutional Court addressed whether universal jurisdiction could be exercised over alleged torture by Zimbabwe officials without the suspects being present in South Africa. The Court held that investigation of international crimes under universal jurisdiction may occur in the absence of suspects without violating constitutional or international law, but presence is required at more advanced stages of proceedings. The principle established is that universal jurisdiction permits investigation in absentia but requires presence for prosecution.}
}

% Glossary entry for PT Garuda v ACCC (Federal Court version)
\newglossaryentry{summary:pt garuda v accc federal court}{
    type=summaries,
    name={\textit{PT Garuda v ACCC} [2011] FCFCA 52},
    sort={PT Garuda v ACCC (Federal Court)},
    description={The Federal Court examined whether Garuda, an Indonesian state-owned airline, qualified as a separate entity of Indonesia entitled to immunity from ACCC proceedings for anti-competitive conduct. Lander and Greenwood JJ held that determination of agency or instrumentality status requires examining ownership, control, functions performed, and the state's purposes in supporting the entity. The principle established is that state-owned entities may claim foreign state immunity based on their functional relationship with the state rather than mere ownership.}
}

% Glossary entry for Prosecutor v Nikolić
\newglossaryentry{summary:prosecutor v nikolic}{
    type=summaries,
    name={\textit{Prosecutor v Nikolić} (ICTY Appeals Chamber, 2003)},
    sort={Prosecutor v Nikolić},
    description={The ICTY Appeals Chamber addressed whether unlawfully obtained custody of an accused from Bosnia invalidated proceedings for international crimes. The Court held that for universally condemned offences, the legitimate expectation of swift justice and accountability must be weighed against state sovereignty and accused's human rights, with more serious charges making irregularities in custody more likely to be overlooked. The principle established is that international criminal courts may proceed despite illegally obtained custody when prosecuting the most serious international crimes.}
}

% Glossary entry for R v Disun; R v Nardin
\newglossaryentry{summary:r v disun r v nardin}{
    type=summaries,
    name={\textit{R v Disun; R v Nardin} (WASC, 2003)},
    sort={R v Disun; R v Nardin},
    description={Defendants charged with people-smuggling offences after steering the MV Tampa into Australian territorial waters argued they were on Norwegian territory and should be governed by extradition laws. The Western Australian Supreme Court rejected this defence, holding that Australia's territory includes its territorial sea and that states possess jurisdiction over persons and property found within their territory. The principle established is that territorial jurisdiction extends to territorial seas, and foreign-flagged vessels within those waters remain subject to coastal state jurisdiction.}
}

% Glossary entry for R v Turnbull; ex parte Petroff
\newglossaryentry{summary:r v turnbull ex parte petroff}{
    type=summaries,
    name={\textit{R v Turnbull; ex parte Petroff} (ACTSC, 1971)},
    sort={R v Turnbull; ex parte Petroff},
    description={Petroff was charged under ACT law for attempted bombing of the USSR embassy in Canberra and argued he committed no crime within the ACT but rather within the USSR. The ACT Supreme Court held that embassy premises are not outside the territory to which criminal law applies, as missions of foreign states do not become foreign territory despite inviolable protections. The principle established is that embassy grounds remain part of the host state's territory for jurisdictional purposes, though enforcement may be limited by diplomatic immunity.}
}

% Glossary entry for Schooner Exchange v McFaddon
\newglossaryentry{summary:schooner exchange v mcfaddon}{
    type=summaries,
    name={\textit{The Schooner Exchange v McFaddon} (US Supreme Court, 1812)},
    sort={Schooner Exchange v McFaddon},
    description={American owners sought to reclaim their vessel that had been seized by Napoleon and pressed into French naval service, when it later docked in Philadelphia. The U.S. Supreme Court held that the vessel, now a French naval ship, was entitled to absolute immunity based on the principle that ``the perfect equality and absolute independence of sovereigns'' means no state can be made subject to another's jurisdiction against its will. The principle established is absolute sovereign immunity for foreign naval vessels, which remains applicable today.}
}

% Glossary entry for SS Lotus
\newglossaryentry{summary:ss lotus}{
    type=summaries,
    name={\textit{SS Lotus (France v Turkey)} (1927) PCIJ},
    sort={SS Lotus},
    description={After a collision on the high seas between French and Turkish vessels due to criminal negligence by the French master, Turkey asserted criminal jurisdiction over the French officer while France claimed exclusive flag state jurisdiction. The PCIJ held that Turkey was entitled to exercise jurisdiction under the territorial principle, as the effects of the offence were felt on the Turkish vessel, and that states may exercise jurisdiction if one constituent element of an offence takes place in their territory. The principle established is that territorial jurisdiction may be exercised when effects of an offence are felt within a state's territory, though the specific rule regarding high seas collisions has since been superseded by UNCLOS Article 97.}
}

% Glossary entry for State v Ebrahim
\newglossaryentry{summary:state v ebrahim}{
    type=summaries,
    name={\textit{State v Ebrahim} (SupCTSA, 1992)},
    sort={State v Ebrahim},
    description={An ANC member was abducted by South Africa from Swaziland to face charges, raising questions about jurisdiction over illegally obtained custody. The South African Supreme Court held that it had no jurisdiction over the accused, protecting both individual rights and state sovereignty. The principle established is that illegally obtained custody through abduction from another state can deprive courts of jurisdiction, respecting both human rights and international law principles of state sovereignty.}
}

% Glossary entry for Tatchell v Mugabe
\newglossaryentry{summary:tatchell v mugabe}{
    type=summaries,
    name={\textit{Tatchell v Mugabe} (2004) 136 ILR 572},
    sort={Tatchell v Mugabe},
    description={President Mugabe of Zimbabwe was subject to proceedings in the Bow Street Magistrates' Court related to allegations of torture. The UK court held that as a serving head of state, Mugabe was entitled to complete immunity from legal proceedings under customary international law, common law, and the State Immunity Act 1978 (UK). The principle established is that incumbent heads of state enjoy absolute immunity from criminal proceedings in foreign jurisdictions, regardless of the severity of alleged crimes.}
}

% Glossary entry for Thor Shipping v Al Duhail
\newglossaryentry{summary:thor shipping v al duhail}{
    type=summaries,
    name={\textit{Thor Shipping A/S v `Al Duhail'} (2008) 252 ALR 20},
    sort={Thor Shipping v Al Duhail},
    description={The Amir purchased a fishing boat and contracted for its transport, but when the transport company was not paid, they arrested the vessel in Queensland and initiated proceedings. The Federal Court held that the Amir, acting in a private capacity when purchasing and arranging transport, possessed the same immunity as a diplomatic mission head, with no relevant exceptions applying. The principle established is that foreign heads of state enjoy diplomatic-level immunity for both official and private acts while in office.}
}

% Glossary entry for Tokic v Government of Yugoslavia
\newglossaryentry{summary:tokic v government of yugoslavia}{
    type=summaries,
    name={\textit{Tokic v Government of Yugoslavia} (1999) NSWSC Unreported},
    sort={Tokic v Government of Yugoslavia},
    description={The plaintiff was injured by a bullet fired from within the Yugoslav Consulate during a protest, and subsequently sued for damages. The Supreme Court of NSW held that Yugoslavia could not claim immunity under Section 13 of the Foreign States Immunities Act, as the case involved personal injury caused by an act committed in Australia. The principle established is that the local tort exception removes foreign state immunity for personal injury caused within the forum state, though enforcement against non-commercial state property remains problematic.}
}

% Glossary entry for US v Benitez
\newglossaryentry{summary:us v benitez}{
    type=summaries,
    name={\textit{US v Benitez} (US Ct. of A for 11\textsuperscript{th} Cir, 1984)},
    sort={US v Benitez},
    description={The US prosecuted a Colombian national for attempted murder of government agents, asserting jurisdiction under the protective principle. The Court justified jurisdiction by finding that the crime had potential ramifications for broader U.S. security interests. The principle established is that the protective principle allows states to exercise criminal jurisdiction over foreigners whose acts abroad threaten vital state security interests.}
}

% Glossary entry for US v Dire
\newglossaryentry{summary:us v dire}{
    type=summaries,
    name={\textit{US v Dire} (Court of Appeal for the 4\textsuperscript{th} Circuit, 2012)},
    sort={US v Dire},
    description={Somali pirates attacked what they believed was a merchant vessel on the high seas, but it was actually a disguised US frigate, leading to their arrest and prosecution in the US. The Court held that the UNCLOS definition of piracy was broad enough to include attempted acts, rejecting the argument that actual robbery was required. The principle established is that attempted or frustrated piracy on the high seas constitutes piracy jure gentium, with universal jurisdiction applying to incomplete piratical acts.}
}

% Glossary entry for US v Yousef
\newglossaryentry{summary:us v yousef}{
    type=summaries,
    name={\textit{US v Yousef} (2003) US Ct of Ap, 2nd Circuit},
    sort={US v Yousef},
    description={The case addressed the scope of universal jurisdiction over terrorism offences, with the Court noting the difference between traditional universal jurisdiction crimes and terrorism. The Court observed that unlike piracy, war crimes, and crimes against humanity which have precise definitions and universal condemnation, terrorism remains a loosely defined and powerfully charged term. The principle established is that terrorism lacks the definitional precision and universal acceptance required for true universal jurisdiction under customary international law.}
}

% Glossary entry for US v Yunis
\newglossaryentry{summary:us v yunis}{
    type=summaries,
    name={\textit{US v Yunis} (US Ct. of A for 11\textsuperscript{th} Cir, 1991)},
    sort={US v Yunis},
    description={Yunis and four others hijacked a Royal Jordanian flight in Beirut with American citizens aboard, two of whom were detained but unharmed. The US Court asserted jurisdiction over the hijacking based on the passive personality principle, as American nationals were victims of the crime. The principle established is that states may exercise criminal jurisdiction over offences committed against their nationals abroad under the passive personality principle.}
}

% Glossary entry for Ward v R
\newglossaryentry{summary:ward v r}{
    type=summaries,
    name={\textit{Ward v R} (1980) 142 CLR 308},
    sort={Ward v R},
    description={Ward, standing on the Victorian bank of the Murray River, fired a gun and killed a victim standing in NSW waters. The High Court examined relevant Victorian statute and found it adhered to the objective/terminatory theory of territorial jurisdiction, placing venue where the criminal act took effect upon the victim. The principle established is that territorial criminal jurisdiction may be exercised based on where the criminal act takes effect, with Victoria following objective territorial jurisdiction principles.}
}

% Glossary entry for XYZ v Commonwealth
\newglossaryentry{summary:xyz v commonwealth}{
    type=summaries,
    name={\textit{XYZ v Commonwealth} (2006) 227 CLR 532},
    sort={XYZ v Commonwealth},
    description={The High Court considered the constitutional validity of extraterritorial criminal jurisdiction over child sex offences committed abroad by Australian citizens. Gleeson CJ held that the nationality principle covers conduct abroad by both citizens and residents, and that extraterritorial criminal jurisdiction is not contrary to international law principles. The principle established is that the nationality principle supports extraterritorial criminal jurisdiction over citizens and residents, with such jurisdiction being constitutionally valid under Australia's external affairs power.}
}

% Glossary entry for Young v A-G (NZ) and Ministry of Defence (UK)
\newglossaryentry{summary:young v ag nz ministry of defence uk}{
    type=summaries,
    name={\textit{Young v A-G (NZ) and Ministry of Defence (UK)} [2019] NZSC 23},
    sort={Young v A-G (NZ) and Ministry of Defence (UK)},
    description={A claim in tort was brought regarding sexual assault during a Royal Navy posting in the UK, with the plaintiff arguing for an exception to foreign state immunity based on the human right to an effective remedy. The New Zealand Supreme Court held that no exception to immunity arose from the duty to provide an effective remedy for human rights violations. The principle established is that there are no implied exceptions to foreign state immunity even for serious human rights violations, with comprehensive immunity legislation providing exhaustive exceptions.}
}

% Update existing entry for Al Adsani v Kuwait
\newglossaryentry{summary:al adsani v kuwait}{
    type=summaries,
    name={\textit{Al Adsani v Kuwait} (1996) 107 ILR 536},
    sort={Al Adsani v Kuwait},
    description={The European Court of Human Rights held that Kuwait's refusal to allow a tort claim against it in the UK, based on state immunity, did not violate the European Convention on Human Rights, Article 6 (right to a fair trial). The case involved a claim for damages by a Kuwaiti national against Kuwait for torture and detention. The principle confirmed is that state immunity does not bar access to courts for serious human rights violations, but remedies may be limited by the state's sovereign status.}
}

% Update existing entry for Australia International Islamic College Board Inc v Saudi Arabia
\newglossaryentry{summary:australian international islamic college v saudi arabia}{
    type=summaries,
    name={\textit{Australian International Islamic College Board Inc v Saudi Arabia} [2013] QCA 129},
    sort={Australian International Islamic College Board Inc v Saudi Arabia},
    description={Saudi Arabia promised to cover educational costs for students at an Australian college but failed to fulfill this financial obligation, leading to proceedings when the Saudi government invoked sovereign immunity. Holmes JA held that the relationship centred on payment for educational services constituted a commercial transaction under section 11 of the Foreign States Immunities Act, rendering immunity inapplicable. The principle established is that the commercial transactions exception serves dual purposes: ensuring justice for individuals in commercial dealings with states and avoiding challenges to state sovereignty, as commercial transactions neither threaten state dignity nor interfere with sovereign functions.}
}

% Update existing entry for Belgium v Senegal
\newglossaryentry{summary:belgium v senegal}{
    type=summaries,
    name={\textit{Belgium v Senegal} (2007) 45 ILM 1009},
    sort={Belgium v Senegal},
    description={Belgium sought the extradition of a Rwandan suspect from Senegal for involvement in the 1994 genocide; Senegal arrested the suspect but refused extradition, citing lack of jurisdiction and the suspect's right to asylum. The International Court of Justice ruled that Senegal violated its treaty obligation to extradite or prosecute the suspect, establishing that the obligation to prosecute serious international crimes like genocide is universal and not contingent on territorial jurisdiction. The principle confirmed is that states are obligated to prosecute or extradite individuals for serious international crimes, regardless of where the crimes were committed.}
}

% Update existing entry for Jones v Saudi Arabia
\newglossaryentry{summary:jones v saudi arabia}{
    type=summaries,
    name={\textit{Jones v Saudi Arabia} [2006] UKHL 26},
    sort={Jones v Saudi Arabia},
    description={Victims of terrorist attacks in Saudi Arabia sued the country in the UK, alleging negligence for failing to protect them. The House of Lords held that Saudi Arabia was immune from suit under the State Immunity Act 1978 (UK), as the claims were non-justiciable and related to sovereign acts. The principle established is that foreign states have immunity from civil jurisdiction for acts committed in the exercise of sovereign authority, and that the act of state doctrine prevents courts from questioning the validity of foreign acts of state.}
}

% Update existing entry for Pinochet (No 3)
\newglossaryentry{summary:pinochet no 3}{
    type=summaries,
    name={\textit{Pinochet (No 3)} [2000] 1 AC 147},
    sort={Pinochet (No 3)},
    description={The House of Lords addressed whether former Chilean dictator Augusto Pinochet was immune from arrest in the UK for human rights violations. The Court held that Pinochet was not entitled to immunity for acts of torture, as such acts are universally condemned and subject to universal jurisdiction. The principle established is that former heads of state do not enjoy immunity from civil or criminal jurisdiction for international crimes such as torture, and that the act of state doctrine does not apply to violations of jus cogens norms.}
}

% Update existing entry for PT Garuda v ACCC (High Court version)
\newglossaryentry{summary:pt garuda v accc high court}{
    type=summaries,
    name={\textit{PT Garuda v ACCC} [2012] HCA 24},
    sort={PT Garuda v ACCC (High Court)},
    description={The High Court of Australia examined the application of the Foreign States Immunities Act 1985 (Cth) to PT Garuda, an Indonesian state-owned enterprise, in proceedings for anti-competitive conduct. The Court held that Garuda was entitled to immunity as an instrumentality of the Indonesian state, and that the commercial activities exception did not apply to actions taken in the exercise of sovereign authority. The principle established is that state-owned enterprises may claim sovereign immunity in foreign proceedings if they are acting as an organ of the state, and that the commercial activities exception to immunity is narrowly construed.}
}

% Update existing entry for Gaddafi
\newglossaryentry{summary:gaddafi}{
    type=summaries,
    name={\textit{Gaddafi} (2015) EWHC 1662 (Admin)},
    sort={Gaddafi},
    description={The UK High Court ruled on the applicability of state immunity in a case involving the former Libyan leader Muammar Gaddafi. The Court held that Gaddafi was entitled to claim state immunity in the UK for acts performed in his official capacity as head of state, including those alleged to constitute crimes against humanity. The principle established is that serving heads of state enjoy immunity from foreign jurisdiction for acts performed in their official capacity, and that such immunity extends to civil proceedings as well as criminal prosecution.}
}

% Update existing entry for Arrest Warrant Case
\newglossaryentry{summary:arrest warrant case}{
    type=summaries,
    name={\textit{Arrest Warrant Case (Democratic Republic of the Congo v Belgium)} [2002] ICJ Rep 3},
    sort={Arrest Warrant Case},
    description={The ICJ ruled on the legality of an international arrest warrant issued by Belgium against the Minister of Foreign Affairs of the Democratic Republic of the Congo, Abdoulaye Yerodia Ndombasi. The Court held that the warrant violated the principle of sovereign equality of states and the immunity from criminal jurisdiction enjoyed by foreign ministers under international law. The principle established is that sitting foreign ministers have immunity from arrest and prosecution in foreign countries for actions taken in their official capacity, and that this immunity persists even after leaving office.}
}

% Update existing entry for Zhang v Zemin
\newglossaryentry{summary:zhang v zemin}{
    type=summaries,
    name={\textit{Zhang v Zemin} (2004) 128 ILR 145},
    sort={Zhang v Zemin},
    description={The UK House of Lords considered the immunity of a foreign head of state from civil jurisdiction in a case brought by a Chinese national against Jiang Zemin for torture and other human rights abuses. The Court held that Jiang, as a former head of state, was entitled to immunity from civil proceedings in the UK for acts committed in his official capacity, including allegations of torture. The principle established is that former heads of state enjoy immunity from civil jurisdiction for acts performed in their official capacity, and that such immunity extends to acts alleged to constitute international crimes.}
}

% Update existing entries for A-G v Eichmann
\newglossaryentry{summary:ag v eichmann}{
    type=summaries,
    name={\textit{A-G v Eichmann} (Dist Ct. of Jerusalem, 1961)},
    sort={A-G v Eichmann},
    description={Eichmann was abducted by Israeli secret service from Buenos Aires and tried in Israel for Holocaust crimes, with Israeli prosecutors asserting multiple jurisdictional bases including the universality principle for genocide and the protective principle for crimes against the Jewish state. The District Court held that international law recognises universal jurisdiction over grave offences against the law of nations, stating that ``international law, in the absence of an International [Criminal] Court, is in need of the judicial and legislative organs of every country to give effect to its criminal interdictions.'' The principle established is that states may exercise universal jurisdiction over serious international crimes such as genocide, and that illegally obtained custody does not bar prosecution when the state from which the accused was taken waives sovereignty claims.}
}

% Update entry for Joyce v DPP
\newglossaryentry{summary:joyce v dpp}{
    type=summaries,
    name={\textit{Joyce v DPP} (HoL, 1946)},
    sort={Joyce v DPP},
    description={Joyce, born in the US but holding a fraudulently obtained UK passport claiming Irish birth, broadcast German propaganda to Britain during WWII and was tried for treason by British authorities. Despite unclear British nationality status, Jallett LJ proposed an alternate jurisdictional basis under the protective principle, stating that ``no principle of comity demands that a state should ignore the crime of treason committed against it outside its own territory.'' The principle established is that states may exercise jurisdiction under the protective principle over acts of treason and propaganda that threaten vital state security interests, regardless of the perpetrator's nationality.}
}

% Update entry for R v Casement
\newglossaryentry{summary:r v casement}{
    type=summaries,
    name={\textit{R v Casement} (1917) Eng Ct. of Crim. A.},
    sort={R v Casement},
    description={Casement, a British subject, attempted to persuade British prisoners of war in Germany to abandon their allegiance to Britain and join the German military, and was subsequently tried and convicted of treason in the UK. This case demonstrates the protective principle's application to prosecute both nationals and non-nationals where vital state security interests have been infringed. The principle established is that states may exercise criminal jurisdiction over acts committed abroad that threaten their security interests, supporting the protective principle of jurisdiction.}
}

% Update entry for Polyukhovich v Commonwealth
\newglossaryentry{summary:polyukhovich v commonwealth}{
    type=summaries,
    name={\textit{Polyukhovich v Commonwealth (War Crimes Act Case)} (1991) HCA},
    sort={Polyukhovich v Commonwealth},
    description={The High Court considered the constitutional validity of retrospective legislation vesting Australian courts with jurisdiction over war crimes committed in Europe during WWII. Brennan J held that ``a law which vested in an Australian court a jurisdiction recognized by international law as a universal jurisdiction is a law with respect to Australia's external affairs'' and that ``international law recognizes a State to have universal jurisdiction to try suspected war criminals.'' The principle established is that universal jurisdiction over war crimes is recognised by international law and that legislation implementing such jurisdiction is constitutionally valid under Australia's external affairs power.}
}

% New entries to add:

% Glossary entry for 2019 Appeals Chamber Judgement
\newglossaryentry{summary:2019 appeals chamber judgement}{
    type=summaries,
    name={\textit{2019 Appeals Chamber Judgement} (ICC Appeals Chamber)},
    sort={2019 Appeals Chamber Judgement},
    description={The ICC Appeals Chamber addressed Jordan's failure to surrender former Sudanese President al-Bashir to the ICC despite an arrest warrant, with Jordan claiming head of state immunity. The Chamber held that ``the absence of a rule of customary international law recognising Head of State immunity vis-à-vis international courts is relevant [...] also for the horizontal relationship between States when a State is requested by an international court to arrest and surrender the Head of State of another State.'' The principle established is that head of state immunity does not apply in proceedings before international criminal courts, and states parties to the Rome Statute must surrender heads of state when requested by the ICC.}
}

% Glossary entry for Asian Agricultural Products
\newglossaryentry{summary:asian agricultural products}{
    type=summaries,
    name={\textit{Asian Agricultural Products} (1991) CSID Case No. ARB/87/3},
    sort={Asian Agricultural Products},
    description={Sri Lanka was held responsible for damage to a seafood factory owned by a Hong Kong company during counterinsurgency operations against the Tamil Tigers (LTTE), though it was unclear whether government forces or LTTE had caused the damage. The tribunal found that Sri Lanka failed to exercise due diligence to protect foreign-owned property while the factory was under exclusive control of Sri Lankan forces. The principle established is that states have a duty to exercise due diligence to protect foreign investments during internal conflicts, and failure to provide adequate protection can result in state responsibility regardless of which party caused the actual damage.}
}

% Glossary entry for Basfar v Wong
\newglossaryentry{summary:basfar v wong}{
    type=summaries,
    name={\textit{Basfar v Wong} [2022] UKSC 20},
    sort={Basfar v Wong},
    description={Wong, a domestic worker employed by a Saudi diplomatic staff member in the UK, sued her employer alleging modern slavery conditions including confinement to embassy premises and no payment for work. The UK Supreme Court held that employing a domestic worker under conditions of modern slavery constituted commercial activity under VCDR Article 31(1)(c), removing diplomatic immunity as such exploitation was not incidental to diplomatic functions. The principle established is that diplomatic immunity does not protect egregious violations like modern slavery, and such exploitation constitutes commercial activity outside the scope of official diplomatic duties.}
}

% Glossary entry for Bolivar Railways Company Case
\newglossaryentry{summary:bolivar railways company case}{
    type=summaries,
    name={\textit{Bolivar Railways Company Case} (1903) IX RIAA 445},
    sort={Bolivar Railways Company Case},
    description={Venezuela was held responsible for services supplied to a successful revolutionary regime that later became the government. The tribunal held that ``the nation is responsible for the obligations of a successful revolution from its beginning, because in theory, it represented ab initio a changing national will, crystallizing in the finally successful result.'' The principle established is that states are retroactively responsible for the conduct of revolutionary movements that successfully become the new government, as such movements are deemed to represent the national will from their inception.}
}

% Glossary entry for Caire Claim
\newglossaryentry{summary:caire claim}{
    type=summaries,
    name={\textit{Caire Claim} (1929) 5 ADPIL Cases 146},
    sort={Caire Claim},
    description={A French national was killed by two Mexican military officers acting ultra vires in an attempted extortion, using their official uniforms, army barracks, and military weapons. The tribunal held that ``the two officers, even if they are deemed to have acted outside their competence...have involved the responsibility of the State, since they acted under cover of their status as officers and used means placed at their disposal on account of that status.'' The principle established is that states are responsible for ultra vires acts of their organs when performed with apparent authority, using official status and resources, regardless of whether the conduct was authorized.}
}

% Glossary entry for Chorzow Factory Case
\newglossaryentry{summary:chorzow factory case}{
    type=summaries,
    name={\textit{Chorzow Factory Case} (1928) PCIJ (Ser. A) No. 17},
    sort={Chorzow Factory Case},
    description={The PCIJ established the fundamental principle of state responsibility requiring full reparation for wrongful acts. The Court held that ``reparation must, as far as possible, wipe out all the consequences of the illegal act and re-establish the situation which would, in all probability, have existed if that act had not been committed.'' The principle established is that reparation should achieve complete restoration through restitution in kind, or if impossible, payment equivalent to restitution value plus damages for any remaining loss.}
}

% Glossary entry for Corfu Channel
\newglossaryentry{summary:corfu channel}{
    type=summaries,
    name={\textit{Corfu Channel} [1949] ICJ Rep 4},
    sort={Corfu Channel},
    description={UK naval vessels were damaged by mines in Albanian territorial waters in the Corfu Channel, with the ICJ holding Albania responsible for failing to warn of the mines' presence. The Court established that ``every State's obligation not to allow knowingly its territory to be used for acts contrary to the rights of other States,'' finding it unnecessary to determine who placed the mines or why they weren't removed. The principle established is that states have a duty to warn other states of known dangers in their territory that could harm foreign vessels or interests.}
}

% Glossary entry for Dickinson v Del Solar
\newglossaryentry{summary:dickinson v del solar}{
    type=summaries,
    name={\textit{Dickinson v Del Solar} [1930] 1 KB 376},
    sort={Dickinson v Del Solar},
    description={A Peruvian diplomat injured Mr Dickinson in a car accident, and Peru expressly waived the diplomat's civil immunity allowing civil proceedings for damages. The court held that the waiver was valid as it was a sovereign privilege that Peru could exercise. The principle established is that sending states may expressly waive diplomatic immunity, allowing civil proceedings against their diplomatic agents in the receiving state's courts.}
}

% Glossary entry for Diplomatic Immunity Case
\newglossaryentry{summary:diplomatic immunity case}{
    type=summaries,
    name={\textit{Diplomatic Immunity Case} (1973) Family Court of Australia},
    sort={Diplomatic Immunity Case},
    description={The Family Court of Australia considered whether the exceptions to diplomatic immunity under VCDR Article 31 extended to family law disputes. The court held that the three specified exceptions (real property, succession, and commercial activity) do not encompass family law matters. The principle established is that diplomatic immunity from civil jurisdiction is comprehensive except for the three specific exceptions listed in Article 31, and family law disputes fall outside these exceptions.}
}

% Glossary entry for Georgian Diplomat in Washington D.C.
\newglossaryentry{summary:georgian diplomat washington dc}{
    type=summaries,
    name={\textit{Georgian Diplomat in Washington D.C.} (1998)},
    sort={Georgian Diplomat in Washington D.C.},
    description={A Georgian diplomat killed someone while driving intoxicated and speeding, prompting the US to request waiver of criminal immunity, which Georgia granted under VCDR Article 32. However, Georgia did not waive civil immunity, preventing the victim's relatives from successfully claiming damages in civil proceedings. The principle established is that waivers of diplomatic immunity must be express and specific to the type of proceedings (criminal or civil), with waiver for one not implying waiver for the other.}
}

% Glossary entry for Genocide Case (Bosnia/Herzegovina v Serbia/Montenegro)
\newglossaryentry{summary:genocide case bosnia v serbia}{
    type=summaries,
    name={\textit{Genocide Case (Bosnia/Herzegovina v Serbia/Montenegro)} [2007] ICJ Rep 43},
    sort={Genocide Case (Bosnia/Herzegovina v Serbia/Montenegro)},
    description={The ICJ considered whether Serbia and Montenegro were responsible for massacres committed by the Army of Republika Srpska (VRS) in Srebrenica, affirming the ``effective control'' test from Nicaragua v US. The Court found no direct FRY participation, VRS lacked organ status, and there was no ``complete dependence,'' ruling that control must be established ``in respect of each operation in which the alleged violations occurred, not generally in respect of the overall actions.'' The principle established is that attribution under Article 8 requires specific effective control over particular operations where violations occurred, not merely general relationships or support.}
}

% Glossary entry for Home Missionary Society Claim
\newglossaryentry{summary:home missionary society claim}{
    type=summaries,
    name={\textit{Home Missionary Society Claim} (1921) VI RIAA 42},
    sort={Home Missionary Society Claim},
    description={The US claimed against Britain for losses suffered by missionaries in Sierra Leone during a 1898 rebellion prompted by a regressive hut tax imposed by the British colonial government. The tribunal held that ``it is a well-established principle of international law that no government can be held responsible for the acts of rebellious bodies of men committed in violation of its authority, where it is guilty of no breach of good faith, or of no negligence in suppressing insurrection.'' The principle established is that states are not automatically liable for damages caused by rebellious groups, provided the government acted in good faith and was not negligent in suppressing the rebellion.}
}

% Glossary entry for Immunity from Legal Process Advisory Opinion
\newglossaryentry{summary:immunity from legal process advisory opinion}{
    type=summaries,
    name={\textit{Immunity from Legal Process Advisory Opinion} [1999] ICJ Rep 62},
    sort={Immunity from Legal Process Advisory Opinion},
    description={Malaysian courts failed to uphold the immunity of a UN Special Rapporteur (a former Malaysian judge) in relation to statements made in international commercial arbitration. The ICJ held that ``according to a well-established rule of international law, the conduct of any organ of a state [including the courts of the state] must be regarded as an act of that State,'' finding Malaysia in breach of international law. The principle established is that all state organs, including courts, are attributed to the state, and failure by domestic courts to respect international immunities constitutes state responsibility.}
}

% Glossary entry for Jones et al v UK
\newglossaryentry{summary:jones et al v uk}{
    type=summaries,
    name={\textit{Jones et al v UK} (14 January 2014, ECtHR)},
    sort={Jones et al v UK},
    description={Following the House of Lords decision in Jones v Saudi Arabia, plaintiffs took their case to the European Court of Human Rights, arguing that UK recognition of Saudi immunity violated their right to a fair trial under the European Convention. The ECtHR held that ``the grant of immunity to the State officials in this case reflected generally recognised rules of public international law'' and did not amount to an unjustified restriction on court access. The principle established is that recognition of diplomatic immunity in accordance with international law does not violate the right to fair trial or effective remedy under human rights conventions.}
}

% Glossary entry for Jurisdictional Immunities of the State
\newglossaryentry{summary:jurisdictional immunities germany v italy}{
    type=summaries,
    name={\textit{Jurisdictional Immunities of the State (Germany v Italy)} [2012] ICJ Rep 99},
    sort={Jurisdictional Immunities of the State (Germany v Italy)},
    description={Italy allowed civil proceedings against Germany for WWII war crimes and forced deportation, with Italian courts denying Germany immunity due to jus cogens violations. The ICJ held that Italy violated Germany's jurisdictional immunity, ruling that ``the rules of State immunity are procedural in character and are confined to determining whether or not the courts of one State may exercise jurisdiction in respect of another State'' and do not bear upon the lawfulness of the underlying conduct. The principle established is that state immunity is procedural and not affected by jus cogens violations, with immunity not equivalent to impunity but serving as a jurisdictional bar to court proceedings.}
}

% Glossary entry for Kalgoorlie Riots Incident
\newglossaryentry{summary:kalgoorlie riots incident}{
    type=summaries,
    name={\textit{Kalgoorlie Riots Incident} (1934)},
    sort={Kalgoorlie Riots Incident},
    description={Foreign nationals were attacked on the Kalgoorlie goldfields in Western Australia, and when the WA government failed to prevent these attacks, the Commonwealth government was held internationally responsible. The incident demonstrates that the WA government, as an executive organ of an Australian state, was considered an organ of the Australian State under international law. The principle established is that the conduct of any level of government (federal, state, or local) is attributable to the national state, making the central government responsible for the actions or omissions of its territorial units.}
}

% Glossary entry for Law Debenture Trust Corpn plc v Ukraine
\newglossaryentry{summary:law debenture trust v ukraine updated}{
    type=summaries,
    name={\textit{Law Debenture Trust Corpn plc v. Ukraine} [2023] UKSC 11},
    sort={Law Debenture Trust Corpn plc v Ukraine},
    description={Ukraine ceased loan repayments to Russia, citing economic and military duress and countermeasures due to Russia's 2014 annexation of Crimea; the UK Supreme Court held that English law, not international law, governed the contract dispute, but customary international law can serve as a source for common law if consistent with domestic law. The Court also held that ``if the availability of countermeasures at the level of international law were accepted as giving rise to a defence in domestic law, national courts would become the arbiter of inter-state disputes governed by international law which is not their function,'' ruling that countermeasures defences are not available in domestic courts as it would violate principles of state immunity and sovereignty. The principle clarified is that customary international law is a source for English common law, but only where it aligns with domestic law, and domestic courts cannot adjudicate on the validity of international countermeasures.}
}

% Glossary entry for Military and Paramilitary Activities in and against Nicaragua (attribution context)
\newglossaryentry{summary:nicaragua case attribution}{
    type=summaries,
    name={\textit{Military and Paramilitary Activities in and against Nicaragua} [1986] ICJ Rep 14},
    sort={Military and Paramilitary Activities in and against Nicaragua},
    description={Nicaragua sued the US for supporting Contra rebels and laying mines in Nicaraguan ports, with the ICJ finding US violations of the prohibition on use of force while establishing key principles on armed attacks and self-defence. The Court held that armed attacks encompass ``the most grave forms of the use of force'' distinguished from ``less grave forms,'' and that self-defence requires the victim state to suffer an armed attack, with measures being necessary and proportionate. The Court found no armed attack against El Salvador, Honduras, or Costa Rica to justify US collective self-defence, and that even if attacks existed, US responses were neither necessary nor proportionate. The principle established is that self-defence is strictly limited to responses against armed attacks (not mere uses of force), requires necessity and proportionality, and cannot be justified by general support for rebels or failure to establish victim state complaints and assistance requests.}
}

% Glossary entry for Nicaragua v United States (jurisdiction)
\newglossaryentry{summary:nicaragua v united states jurisdiction}{
    type=summaries,
    name={\textit{Nicaragua v United States} [1984] ICJ Rep 392},
    sort={Nicaragua v United States (Jurisdiction)},
    description={In the preliminary objections phase, the US challenged ICJ jurisdiction, but the Court held it had jurisdiction under the optional clause while establishing key principles about declarations accepting compulsory jurisdiction. The ICJ held that ``declarations of acceptance of the compulsory jurisdiction of the Court are facultative, unilateral engagements, that States are absolutely free to make or not to make'' and may include conditions or reservations. The principle established is that states have complete freedom in formulating their acceptance of ICJ compulsory jurisdiction, including temporal, subject matter, and other limitations, with the Court only questioning reservations that are fundamentally incompatible with the ICJ's purpose.}
}

% Glossary entry for Norwegian Loans Case
\newglossaryentry{summary:norwegian loans case}{
    type=summaries,
    name={\textit{Norwegian Loans Case (France v Norway)} [1957] ICJ Rep 9},
    sort={Norwegian Loans Case},
    description={France and Norway had conflicting declarations accepting ICJ jurisdiction with different reservations, leading the Court to determine the scope of common consent between the parties. The ICJ held that ``the French Declaration accepts the Court's jurisdiction within narrower limits than the Norwegian Declaration; consequently the common will of the Parties, which is the basis of the Court's jurisdiction, exists within these narrower limits indicated by the French reservation.'' The principle established is that when states have different reservations to ICJ jurisdiction, the Court's jurisdiction extends only to the overlap between the declarations, with the narrower limitations determining the scope of common consent.}
}

% Glossary entry for Nottebohm Case
\newglossaryentry{summary:nottebohm case}{
    type=summaries,
    name={\textit{Nottebohm Case (Liechtenstein v Guatemala)} [1955] ICJ Rep 4},
    sort={Nottebohm Case},
    description={Liechtenstein sought diplomatic protection for Nottebohm, a German who had acquired Liechtenstein citizenship but maintained strong ties with Guatemala where his property was seized during WWII, with the ICJ rejecting the claim for lack of genuine connection. The Court held that nationality must have ``as its basis a social fact of attachment, a genuine connection of existence, interests and sentiments, together with the existence of reciprocal rights and duties'' between the individual and the protecting state. The principle established is that while states determine their nationality laws, diplomatic protection requires a genuine link between the state and its national when claiming against a state with which the individual has stronger ties, though subsequent ILC commentary suggests this was a relative rule specific to competing connections rather than a general requirement.}
}

% Glossary entry for Nuclear Weapons Advisory Opinion
\newglossaryentry{summary:nuclear weapons advisory opinion}{
    type=summaries,
    name={\textit{Nuclear Weapons Advisory Opinion} [1996] ICJ Rep 226},
    sort={Nuclear Weapons Advisory Opinion},
    description={The ICJ addressed the legality of nuclear weapons threats and use, holding that such threats are unlawful where the actual use would be illegal, while finding no specific prohibition on nuclear weapons use in all circumstances. The Court held that ``where the use of force would be illegal, the threat to use such force is also unlawful,'' establishing the connection between prohibited use and prohibited threats of force. The principle established is that threats of force are unlawful when the threatened use of force would itself violate international law, and that while nuclear weapons are generally contrary to international humanitarian law, extreme circumstances of self-defence might create exceptions to their prohibition.}
}

% Glossary entry for Obligations of States in Respect of Climate Change
\newglossaryentry{summary:obligations states climate change}{
    type=summaries,
    name={\textit{Obligations of States in Respect of Climate Change} [pending]},
    sort={Obligations of States in Respect of Climate Change},
    description={A pending advisory opinion request before the ICJ asking about state obligations regarding climate change, particularly focusing on the legal consequences of causing significant harm to the climate system and obligations toward particularly vulnerable states and future generations. The case represents an attempt to clarify international legal obligations related to climate action and the protection of the global climate system. The principle to be established will concern state responsibility for environmental protection and climate action under international law, potentially expanding understanding of intergenerational and global environmental obligations.}
}

% Glossary entry for Oil Platforms Case
\newglossaryentry{summary:oil platforms case}{
    type=summaries,
    name={\textit{Oil Platforms (Iran v US)} [2003] ICJ Rep 161},
    sort={Oil Platforms Case},
    description={Iran sued the US over attacks on Iranian oil platforms following incidents involving US vessels and mines, with the ICJ examining whether Iran's actions constituted an armed attack justifying US self-defence and whether US responses met necessity and proportionality requirements. The Court found that even cumulative incidents including missile attacks and mine collisions did not constitute an armed attack as they lacked sufficient scale and effects and clear direction against the US, while US responses were retaliatory rather than defensive. The principle established is that armed attacks require sufficient scale and effects to qualify as ``the most grave forms of use of force,'' that cumulative incidents may constitute armed attacks if properly targeted and scaled, and that self-defence responses must be defensive rather than punitive and meet strict necessity and proportionality requirements.}
}

% Glossary entry for Panevezys-Saldutiskis Railway Case
\newglossaryentry{summary:panevezys saldutiskis railway case}{
    type=summaries,
    name={\textit{Panevezys-Saldutiskis Railway Case} (1938) PCIJ (ser A/B) No 76},
    sort={Panevezys-Saldutiskis Railway Case},
    description={This PCIJ case established the foundational theory of diplomatic protection, with the Court holding that when a state resorts to diplomatic action or international judicial proceedings on behalf of its nationals, it is asserting its own right to ensure respect for international law. The Court emphasized that ``by resorting to diplomatic action or international judicial proceedings on his behalf, a State is in reality asserting its own right, the right to ensure in the person of its nationals respect for the rules of international law.'' The principle established is that diplomatic protection is fundamentally about state rights rather than individual rights, with states asserting their own interests in ensuring their nationals receive treatment consistent with international law standards.}
}

% Glossary entry for Quintanilla Case
\newglossaryentry{summary:quintanilla case}{
    type=summaries,
    name={\textit{Quintanilla} (1926) 4 RIAA 60},
    sort={Quintanilla Case},
    description={A Mexican national was taken into custody by US authorities following an alleged assault and later found dead beside a road, with US authorities having no records of what happened to him after custody began. The tribunal held that ``the government can be held liable if it is proven that it has treated him [the foreign national] cruelly, harshly, unlawfully; so much more it is liable if it can say only that it took him into custody...and that it ignores what happened to him.'' The principle established is that states have a duty to account for the safety and whereabouts of foreign nationals taken into their custody, and that failure to maintain proper records or provide adequate explanations for harm can constitute international responsibility.}
}

% Glossary entry for Re (Al Rawi and Others) Secretary of State
\newglossaryentry{summary:re al rawi others secretary of state}{
    type=summaries,
    name={\textit{Re (Al Rawi and Others) Secretary of State} [2006] EWCA Civ 1279},
    sort={Re (Al Rawi and Others) Secretary of State},
    description={The UK initially refused to intervene for British non-citizen residents detained in Guantanamo Bay, with the Court of Appeal confirming the government's position that diplomatic protection is limited to nationals. The government stated that ``it is the long-standing policy of the UK Government not to offer consular or similar assistance to non-British Nationals...the UK does not have the right to exercise diplomatic protection [in respect of non-nationals].'' The principle established is that diplomatic protection is strictly limited to nationals of the protecting state, and that residency or other connections without nationality do not create rights to diplomatic protection under international law.}
}

% Glossary entry for South West Africa Case (Preliminary Objections)
\newglossaryentry{summary:south west africa preliminary objections}{
    type=summaries,
    name={\textit{South West Africa} [1962] ICJ Rep 319},
    sort={South West Africa (Preliminary Objections)},
    description={Ethiopia and Liberia challenged South Africa's administration of South West Africa (Namibia) and its apartheid policies, with South Africa arguing lack of direct negotiation precluded ICJ jurisdiction, but the Court found extensive UN discussions satisfied negotiation requirements. The ICJ held that discussions within international organizations can constitute sufficient negotiation to meet jurisdictional prerequisites, rejecting South Africa's argument about bilateral negotiation requirements. The principle established is that multilateral discussions in international forums can satisfy treaty requirements for prior negotiation, and that disputes may crystallize in institutional settings rather than requiring direct bilateral diplomatic exchanges.}
}

% Glossary entry for South West Africa Case (Second Phase)
\newglossaryentry{summary:south west africa second phase}{
    type=summaries,
    name={\textit{South West Africa} [1966] ICJ Rep 6},
    sort={South West Africa (Second Phase)},
    description={In the second phase of proceedings against South Africa's administration of South West Africa, the ICJ found by the President's casting vote that Ethiopia and Liberia lacked standing to bring the case, effectively dismissing their challenge to South Africa's apartheid policies in the mandate territory. This controversial decision was later superseded by the 1971 Advisory Opinion finding South Africa's continued presence in Namibia illegal. The principle established is that standing requirements in contentious proceedings can be strictly applied even for cases involving important public interests, though the decision was widely criticized and effectively overturned by subsequent proceedings.}
}

% Glossary entry for Ukraine v Russia
\newglossaryentry{summary:ukraine v russia}{
    type=summaries,
    name={\textit{Ukraine v Russia} [2019] ICJ Rep 558},
    sort={Ukraine v Russia},
    description={Ukraine sued Russia for alleged violations of terrorism financing and racial discrimination conventions following events in eastern Ukraine and Crimea, with the ICJ examining whether negotiation requirements had been satisfied and later finding Russia in violation for failing to maintain Ukrainian language education in Crimea. The Court held that negotiation requires ``a genuine attempt by one of the disputing parties to engage in discussions with the other...with a view to resolving the dispute'' and that the precondition is met when negotiations fail or become deadlocked. The principle established is that treaty requirements for prior negotiation demand genuine engagement beyond mere protests or accusatory exchanges, with the failure or futility of negotiations satisfying the jurisdictional prerequisite for international adjudication.}
}

% Update existing entry for Application of the Convention on the Prevention and Punishment of Crime of Genocide in the Gaza Strip
\newglossaryentry{summary:application genocide convention gaza strip updated}{
    type=summaries,
    name={\textit{Application of the Convention on the Prevention and Punishment of Crime of Genocide in the Gaza Strip (South Africa v Israel)} [2024] ICJ Rep 1},
    sort={Application of the Convention on the Prevention and Punishment of Crime of Genocide in the Gaza Strip},
    description={South Africa sought provisional measures against Israel alleging genocide in Gaza, with the ICJ finding prima facie jurisdiction, plausible rights of Palestinians to protection from genocidal acts, and real and imminent risk of irreparable prejudice due to the catastrophic humanitarian situation. The Court issued three sets of provisional measures (January, March, and May 2024) ordering Israel to take steps to prevent genocidal acts, ensure humanitarian assistance, and halt military operations in Rafah. The principle established is that provisional measures can be ordered to preserve plausible rights pending final judgment when there is prima facie jurisdiction and real risk of irreparable harm, with the Court emphasizing the erga omnes nature of obligations under the Genocide Convention allowing any state party to invoke protection for affected populations.}
}

% Update existing entry for La Grand Case
\newglossaryentry{summary:la grand case updated}{
    type=summaries,
    name={\textit{La Grand (Germany v United States)} [2001] ICJ Rep 466},
    sort={La Grand Case},
    description={Germany sought diplomatic protection for two German nationals sentenced to death in the US after being denied consular assistance under the Vienna Convention on Consular Relations, with the ICJ establishing the binding nature of provisional measures when the US failed to prevent their execution. The Court held that provisional measures orders are legally binding and that states must comply with them, rejecting US arguments that federal structure prevented compliance with ICJ orders. The principle established is that ICJ provisional measures are legally binding rather than merely recommendatory, that states cannot invoke domestic legal constraints to avoid compliance with international court orders, and that violations of consular notification rights can constitute breaches of treaty obligations warranting diplomatic protection.}
}

% Update existing entry for Whaling in the Antarctic Case
\newglossaryentry{summary:whaling in the antarctic case updated 2}{
    type=summaries,
    name={\textit{Whaling in the Antarctic Case} [2014] ICJ Rep 226},
    sort={Whaling in the Antarctic Case},
    description={Australia challenged Japan's JARPA II whaling program, and the ICJ held that it constituted commercial whaling, violating the International Convention for the Regulation of Whaling, as it was not for scientific research, and resolutions, while not binding, inform treaty interpretation. The case involved Australia's claim that Japan's whaling was commercial, not scientific, establishing that treaty interpretation under the VCLT considers objective program design, and non-binding resolutions can guide interpretation. In the context of state responsibility, Australia successfully argued that Japan breached international whaling conventions not based on any effect on Australia, but on Japan's general breach of the convention, demonstrating wide acceptance of ARSIWA Article 48 allowing any party state to take action against breaches of multilateral treaty obligations. The ICJ also briefly considered Australia's Article 36(2) reservation, holding that the maritime delimitation exclusion did not apply as the dispute was not about delimiting maritime borders, and that the combined declarations and reservations of both parties comprised the parameters of jurisdiction. The principle clarified is that treaty interpretation under the VLT considers objective program design with non-binding resolutions guiding interpretation, that states may invoke responsibility for breaches of multilateral obligations even without direct injury supporting Article 48's provisions for community interest enforcement, and that ICJ jurisdiction reservations must be precisely interpreted to determine their scope and application.}
}