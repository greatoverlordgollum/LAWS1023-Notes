\newglossary[slg]{summaries}{syi}{syg}{Case Summaries}

\newglossaryentry{summary:ure v commonwealth}{
    type=summaries,
    name={\textit{Ure v Commonwealth} (2016) 329 ALR 452},
    sort={Ure v Commonwealth},
    description={[Page \pageref{case:Ure v Commonwealth}] Australian citizens tried to assert sovereignty over unclaimed islands; under international law, only states can assert claims of sovereignty, with the Federal Court holding that \statute{\textit{ICJ Statute} Art 38} constitutes good law on the sources of international law.}
}

\newglossaryentry{summary:north sea continental shelf}{
    type=summaries,
    name={\textit{North Sea Continental Shelf} (1969) ICJ Rep 3},
    sort={North Sea Continental Shelf},
    description={[Page \pageref{case:North Sea Continental Shelf}] Germany, Denmark and the Netherlands had a dispute over how their continental boundaries were to be delineated; the ICJ held that customary international law requires states to negotiate in good faith to delimit continental shelf boundaries, and that the principle of equidistance is not a rule of customary international law but a method of delimitation.}
}

\newglossaryentry{summary:nicaragua case}{
    type=summaries,
    name={\textit{Nicaragua v United States} (1986) ICJ Rep 14},
    sort={Nicaragua v United States},
    description={[Page \pageref{case:Military and paramilitary activities in Nicaragua}] Nicaragua brought a case against the US for supporting Contra rebels; the ICJ held that the US had violated international law by using force against Nicaragua, and that customary international law prohibits the use of force in international relations. Moreover, it held that state practice does not need to be exactly uniform, but should be consistent.}
}

\newglossaryentry{summary:nicaragua v colombia}{
    type=summaries,
    name={\textit{Nicaragua v Colombia} (2012) ICJ Rep 624},
    sort={Nicaragua v Colombia},
    description={[Page \pageref{case:Nicaragua v Colombia}] Nicaragua claimed that Colombia had violated its rights to the continental shelf and exclusive economic zone; the ICJ held that Nicaragua had not established its claim to the continental shelf, but that it had rights to an exclusive economic zone, and that Colombia's actions violated Nicaragua's rights under international law. The ICJ also held that treaties can be binding upon states that are not parties to them if they are indicative of \textit{opinio juris}.}
}

\newglossaryentry{summary:asylum case}{
    type=summaries,
    name={\textit{Asylum Case (Colombia v Peru)} (1950) ICJ Rep 266},
    sort={Asylum Case},
    description={[Page \pageref{case:Asylum case}] Colombia granted de la Torre political asylum, which was challenged by Peru; Colombia raised a defence in regional customary norm, but the ICJ rejected this, holding that regional standards are held to a higher degree of stability and continuity to apply as international law.}
}

\newglossaryentry{summary:al-saadoon v sec. def}{
    type=summaries,
    name={\textit{Al-Saadoon v Secretary of State for Defence} [2010] EWCA Civ 776},
    sort={Al-Saadoon v Secretary of State for Defence},
    description={[Page \pageref{case:R (Al-Saadoon v Sec. of Defence)}] Al-Saadoon and others were detained in Iraq by British forces; the UK Court of Appeal held that the UK was bound by the Geneva Conventions, and that it had a duty to ensure that detainees were treated in accordance with international law, including the prohibition on torture. This case shows that regional customary norms can be applied as international law if they clear the high threshold of stability and continuity.}
}

\newglossaryentry{summary:anglo norwegian fisheries}{
    type=summaries,
    name={\textit{Anglo-Norwegian Fisheries Case} (1951) ICJ Rep 116},
    sort={Anglo-Norwegian Fisheries Case},
    description={[Page \pageref{case:UK v Norway Fisheries}] The UK challenged Norway's baselines, holding that there was a rule of custom forbidding Norway's method of drawing baselines; however, as Norway had persistently objected to this rule from its inception, they were exempt from it.}
}

\newglossaryentry{summary:bay of bengal (bangladesh v myanmar)}{
    type=summaries,
    name={\textit{Bay of Bengal (Bangladesh v Myanmar)} (2012) ITLOS 12},
    sort={Bay of Bengal (Bangladesh v Myanmar)},
    description={[Page \pageref{case:Bangladesh v Myanmar}] Bangladesh and Myanmar disputed the delimitation of their maritime boundary in the Bay of Bengal; the International Tribunal for the Law of the Sea held that the dispute could be decided through implicitly adopting general principles of international law into the decision.}
}

\newglossaryentry{summary:chagos marine protected area}{
    type=summaries,
    name={\textit{Chagos Marine Protected Area Arbitration (Mauritius v UK)} (2015) XXXI RIAA 359},
    sort={Chagos Marine Protected Area Arbitration},
    description={[Page \pageref{case:Chagos Marine Protected Area Arbitration}] To resolve a dispute between Mauritius and the UK over the establishment of a Marine Protected Area in the Chagos Archipelago, the Tribunal held that it was able to look at the doctrine of estoppel in various domestic legal systems to determine the position under international law.}
}

\newglossaryentry{summary:paquete habana}{
    type=summaries,
    name={\textit{The Paquete Habana} (1900) 175 US 677},
    sort={Paquete Habana},
    description={[Page \pageref{case:Paquete Habana}] The US Supreme Court held that customary international law is part of US law, and that the US must respect the rights of foreign fishing vessels in its territorial waters. This case illustrates the principle that customary international law can be applied in domestic courts. Moreover, decisions taken by judicial tribunals are compelling as they constitute trustworthy evidence of what the international law is.}
}

\newglossaryentry{summary:nuclear weapons case}{
    type=summaries,
    name={\textit{Legality of the Threat or Use of Nuclear Weapons} (1996) ICJ Rep 226},
    sort={Legality of the Threat or Use of Nuclear Weapons},
    description={[Page \pageref{case:Legality of Nuclear Weapons [1996] ICJ Rep 254}] The ICJ held that the use of nuclear weapons would generally be contrary to international law, but that there is no specific rule prohibiting their use in all circumstances. The Court also stated that the right to self-defence does not include the right to use nuclear weapons, and that states must comply with their obligations under international humanitarian law. This case held that UNGA resolutions are not normally binding on states, but can have normative value and can provide evidence of customary international law.}
}

% Add more case summary entries here following the same pattern
% \newglossaryentry{summary:casename}{
%     type=summaries,
%     name={Case Name},
%     description={Case summary and legal principles}
% }