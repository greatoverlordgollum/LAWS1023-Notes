\section{Immunity and International Criminal Courts}
\begin{itemize}
    \item One of the core principles of international law is that the official capacity of an individual is irrelevant in proceedings before an international criminal court
    \item This is stated in \statute{\textit{Rome Statute of the International Criminal Court} Art 27}, with similar provisions stated in the statutes of the ICTY, ICTR, the Extraordinary Chambers in the Courts of Cambodia, etc.
\end{itemize}

\begin{statutedetails}{\textit{Rome Statute of the International Criminal Court} Article 27}
    \flushleft
    \textbf{Irrelevance of official capacity}
    \begin{enumerate}
        \item This Statute shall apply equally to all persons without any distinction based on official capacity. In particular, official capacity as a Head of State or Government, a member of a Government or parliament, an elected representative or a government official shall in no case exempt a person from criminal responsibility under this Statute, nor shall it, in and of itself, constitute a ground for reduction of sentence. 
        \item Immunities or special procedural rules which may attach to the official capacity of a person, whether under national or international law, shall not bar the Court from exercising its jurisdiction over such a person.
    \end{enumerate}
\end{statutedetails}

\begin{itemize}
    \item This principle becomes hard to apply when it clashes with the immunity that some individuals may enjoy in the jurisdiction of a particular country
    \item This was exemplified in a dispute between the ICC and Sudan (alongside other African states) over arrest warrants for Sudan's former President al-Bashir, and the meaning of Articles 27 and 28 of the \statute{\textit{Rome Statute of the International Criminal Court}}
    \begin{itemize}
        \item This case was complicated by Sudan not being a party to the \statute{\textit{Rome Statute}}
        \item There were several ICC decisions regarding the failure by states party to the \statute{\textit{Rome Statute}} to surrender al-Bashir to the ICC, including the \case{\textit{2019 Appeals Chamber Judgement}} (which concerned the failure of Jordan to surrender al-Bashir)
        \item ``The absence of a rule of customary international law recognising Head of State immunity vis-à-vis international courts is relevant [...] also for the horizontal relationship between States when a State is requested by an international court to arrest and surrender the Head of State of another State.''
        \item This case explained that the basis of the former President's immunity arose as a matter of international law and as a matter between states
    \end{itemize}
    \item Dapo Akande has held that ``The issue of the immunity of heads of state before international criminal courts is not what is at issue in these cases. What was is at issue is the immunity of heads of states from arrest by other states acting at the request of an international criminal court. That the head of state may not have immunity before the international criminal court does not, without more, say anything about whether he or she may have immunity before a foreign state.'', illuminating the pitfalls of the ICC's decision in this area
    \item There are other, less controversial routes to jurisdiction
    \begin{itemize}
        \item An example is the UN Security Council, which has the capacity to issue binding resolutions, and moreover has the ability to make direct referrals to the ICC under \statute{\textit{Rome Statute} Art 13(b)} - if this is done, then immunity can be overridden
    \end{itemize}
\end{itemize}

\begin{statutedetails}{\textit{Rome Statute of the International Criminal Court} Article 13}
    \flushleft
    \textbf{Exercise of Jurisdiction}

    \vspace{\baselineskip}

    The Court may exercise its jurisdiction with respect to a crime referred to in article 5 in  accordance with the provisions of this Statute if: 
    \begin{enumerate}[label=(\alph*)]
        \item A situation in which one or more of such crimes appears to have been committed is referred to the Prosecutor by a State Party in accordance with article 14; 
        \item A situation in which one or more of such crimes appears to have been committed is referred to the Prosecutor by the Security Council acting under Chapter VII of the Charter of the United Nations; or 
        \item The Prosecutor has initiated an investigation in respect of such a crime in accordance with article 15.
    \end{enumerate}
\end{statutedetails}

\section{Foreign State Immunity and \textit{Jus Cogens}}

\begin{casedetails}{\textit{Jurisdictional Immunities of the State (Germany v Italy)} [2012] ICJ Rep 99}
    \flushleft
    This case concerned damages proceedings in the Italian courts, and arose out of multiple distinct fact patters, as outlined in the table below:
    \begin{longtable}{p{0.03\textwidth}|>{\raggedright\arraybackslash}p{0.9\textwidth}}
        1 & From 1943, when Italy surrendered to the Allies, to the end of WW2 in 1945, the armed forces of the German Reich were in belligerent occupation of much of Italy (especially Northern Italy from 1943-1945). During their occupation, German forces committed war crimes and crimes against humanity, including the murder of civilians and the forced deportation of civilians for use as slave labour in Germany. In 1998, Luigi Ferrini, an Italian national forcibly deported to Germany in 1944 and forced to work in a munitions factory, instituted compensation proceedings in Italy against Germany. In 2004, the Corte Suprema di Cassazione (Italian Supreme Court) held that the Italian courts had jurisdiction in respect to the claim, and that Germany was not entitled to foreign state immunity as the mistreatment of Mr Ferrini was a crime under international law. Following this decision, hundreds of similar claims were brought in Italian courts. \\[0.5cm]\hline
        2 & The relatives of Greek civilians murdered by German forces in the Greek village of Distomo in 1944 brought a claim for compensation against Germany in the Greek courts in 1997. The Greek claimants in this case sought to enforce the judgement of the Greek Court of First Instance of Livadia in Italy, and this was allowed by a decision of 2006 of the Court of Appeal of Florence. In 2007, the Greek claimants registered a charge over a German state-owned property in Italy, the Villa Vigoni, near Lake Como; Germany owned and operated this property to promote cross-cultural initiatives.
    \end{longtable} 

    In both cases, Germany objected to Italy's assertion of jurisdiction, and took Italy to the ICJ.

    \tcblower
    \flushleft
    The ICJ held that Italy had violated foreign state immunity, and as such had violated its obligation to respect the jurisdictional immunity of Germany under customary international law. The court very neatly explained the rationale and the origins of the immunity afforded to Germany.

    \begin{longtable}{p{0.07\textwidth}|>{\raggedright\arraybackslash}p{0.85\textwidth}}
        [51] & ... the rule of State immunity ... derives from the principle of sovereign equality of States, which, as Article 2, paragraph 1, of the Charter of the United Nations makes clear, is one of the fundamental principles of the international legal order. \\\hline
        [65] & Court is not called upon…to resolve the question whether there is in customary international law a ``tort exception" to State immunity applicable to acta \textit{jure imperii} ... The issue before the Court is confined to acts committed ... by the armed forces of a foreign State. \\\hline
        [77] & State practice in the form of judicial decisions supports the proposition that State immunity for acta jure imperii continues to extend to civil proceedings for [torts] committed by the armed forces ... of a State in the conduct of armed conflict. \\\hline
        [91] & A State is not deprived of immunity by ... the fact that it is accused of serious violations of international human rights law or the international law of armed conflict ... the Court must emphasize that ... the question of whether ... immunity might apply in criminal proceedings against an official of the State is not in issue. [\case{\textit{Pinochet (No 3)}} distinguished]. \\\hline
        [92] & [Italy's argument concerning violation of \textit{jus cogens}] rests on the premise that there is a conflict between jus cogens ... and according immunity to Germany. \\\hline
        [93] & ... however, no such conflict exists ... \textbf{The rules of State immunity are procedural in character and are confined to determining whether or not the courts of one State may exercise jurisdiction in respect of another State.} They do not bear upon ... whether or not the conduct in respect of which the proceedings are brought was lawful or unlawful. [\case{\textit{Arrest Warrant}} compared]\\\hline
        [101] & ... [T]he Court cannot accept Italy's contention that the alleged shortcomings in Germany's provisions for reparation to Italian victims, entitled the Italian courts to deprive Germany of jurisdictional immunity. \textbf{The Court can find no basis in the State practice from which customary international law is derived that makes the entitlement of a State to immunity dependent upon the existence of effective alternative means of redress.}\\
    \end{longtable} 

    \begin{itemize}
        \item At [65], it was accepted that acts committed by the armed forces of a foreign state, regardless of where they are committed, are subject to immunity
        \item At [77], the court did not examine the local torts exception at all
        \item At [91], the central question of the case was considered; the court rejected taking the approach of denying a state civil immunity, distinguishing \case{\textit{Pinochet}} as this case was for civil damages, not criminal damages; the approach of denial was rooted in the desire to uphold the peremptory norms of international law/prohibitions of on the commissions of the most serious crimes
        \item At [93], the court held that state immunity does not go to the underlying character of the acts subject to jurisdiction; in \case{\textit{Arrest Warrant}}, the ICJ held that immunity is not equivalent to impunity - the mere presence of immunity does not absolve or erase the underlying jurisdiction over a crime, but only acts as a procedural bar to the exercise of jurisdiction
        \item Italy held that another exception to immunity could be derived from a failure of a plaintiff to receive appropriate compensation
        \item Here, the victims received some compensation from the government of Germany, but Italy argued that this was insufficient (this is a last-resort avenue)
    \end{itemize}
\end{casedetails}

\begin{itemize}
    \item The case of \case{\textit{Jones et al v UK} (14 January 2014, ECtHR)} (European Court of Human Rights) is also relevant
    \begin{itemize}
        \item ``The Court is satisfied that the grant of immunity to the State officials in this case reflected generally recognised rules of public international law. The application of the provisions of the 1978 Act to grant immunity to the State officials in the applicants' civil cases did not therefore amount to an unjustified restriction on the applicants' access to a court.''
        \item This is the sequel to \case{\textit{Jones v Saudi Arabia} [2007] AC 270}; after losing in the House of Lords, the plaintiff went to the European Court of Human Rights, arguing that their right to a fair trial had been denied as the UK courts had recognised immunity
        \item The right to a fair trial/remedy is an internationally recognised right under the UCHR, and the \convention{\textit{International Covenant on Civil and Political Rights}}
        \item The ECtHR, in a similar manner to the ICJ, had to recognise the immunity rules and the protection of human rights, and held that the UK's grant of immunity under the \statute{\textit{State Immunity Act}} did not amount to an unjustified prescription on the applicant's access to the Court 
    \end{itemize}
\end{itemize}

\section{Foreign Act of State Doctrine}\label{sec:Foreign Act of State Doctrine}
\begin{itemize}
    \item This refers to when proceedings against a private person or entity, or the Australian government, may raise issues concerning the actions of a foreign state
    \item This is a distinct principle of common law and of judicial restraint, but is not really a principle of public international law
    \item The proceedings are somewhat related to the actions of a foreign state, but are not related to the foreign state itself
\end{itemize}

\begin{casedetails}{\textit{Spycatcher Case} (1988) 165 CLR 30}
    \flushleft
    This case related to a former MI5 officer who wanted to publish a book revealing the secrets of MI5 and MI6. The author was banned from publishing in the UK, but was able to be published in Australia after the High Court ruled that the book could not be withheld in Australia.

    \vspace{\baselineskip}

    ``In general, courts will not adjudicate upon the validity of acts and transactions of
    a foreign sovereign state within that sovereign's own territory.'' This quote summarises the foreign act of state doctrine.
\end{casedetails}

\begin{itemize}
    \item The foreign act of state doctrine does not apply where the relevant conduct of a foreign state involves a breach of an established principle of international law, as determined in \case{\textit{Hicks v Ruddock} (2007) 156 FCR 574} and \case{\textit{Habib v Commonwealth} (2010) 183 FCR 62}
    \item In recent years, this doctrine has contracted in scope in jurisdictions around the world
\end{itemize}

\begin{casedetails}{\textit{Hicks v Ruddock} (2007) 156 FCR 574}
    \flushleft
    In this case, David Hicks, an Australian citizen, initiated legal proceedings against the Australian Attorney-General, challenging the government's failure to request his release from Guantanamo Bay. Hicks had been captured by United States forces during the Afghanistan war in 2001 and detained at Guantanamo Bay for five years without being formally charged. His claims were based on habeas corpus and administrative law, seeking judicial review of the Australian government's inaction in advocating for his release from US custody. Hicks argued that the government had a legal obligation to protect its citizen, particularly given the prolonged detention without a lawful trial.

    \vspace{\baselineskip}

    The Australian government moved for a summary dismissal of the proceedings, asserting that the case engaged the foreign act of state doctrine. This doctrine generally prevents domestic courts from adjudicating on the lawfulness of a foreign government's actions conducted outside the court's jurisdiction—in this case, the US detention of Hicks at Guantanamo Bay. The government argued that allowing the case to proceed would require the Australian court to improperly sit in judgement on the actions of the United States, which occurred extraterritorially, and thus the matter was non-justiciable.

    \vspace{\baselineskip}

    The Federal Court, presided over by Tamberlin J, rejected the government's application for summary dismissal. The judge ruled that the foreign act of state doctrine did not apply when the conduct of the foreign state constituted a breach of an international convention. The court found that Hicks' detention for an extended period without a lawful trial amounted to a war crime, violating international legal standards. This determination permitted the case to advance, as the court was willing to scrutinise the lawfulness of Hicks' detention, notwithstanding the involvement of a foreign state's actions.

    \vspace{\baselineskip}

    Ultimately, Hicks was released from Guantanamo Bay in 2007 before the case could fully progress through the judicial system, rendering further litigation moot. His release followed significant domestic and international advocacy, as well as diplomatic efforts. The Hicks v Ruddock case remains notable for its examination of the Australian government's responsibilities towards citizens detained abroad, the limits of the foreign act of state doctrine, and the intersection of domestic judicial review with international law obligations.
\end{casedetails}

\begin{casedetails}{\textit{Habib v Commonwealth} (2010) 183 FCR 62}
    \flushleft
    In this case, Mamdouh Habib, an Australian citizen, sought damages from the Commonwealth of Australia for alleged torts of misfeasance in public office and intentional infliction of harm. Habib claimed he suffered acts of torture while detained in multiple locations, including Pakistan, Egypt, Afghanistan, and Guantanamo Bay, between 2001 and 2005. His proceedings targeted the Commonwealth, alleging that Australian officials were complicit in or failed to prevent the torture he endured during his detention by foreign authorities, asserting that such conduct caused him significant physical and psychological harm.
    
    \vspace{\baselineskip}
    
    The Australian government argued that the court lacked jurisdiction to hear the case and that the claim was non-justiciable, invoking the foreign act of state doctrine and the Foreign States Immunities Act 1985 (Cth). The government contended that adjudicating the case would require the court to scrutinise the actions of foreign states on their own territory, which was impermissible under the doctrine, as it would involve sitting in judgement on the legality of foreign sovereign acts. The Commonwealth sought to have the proceedings dismissed on these grounds, asserting that the court could not lawfully examine the conduct of foreign agents.
    
    \vspace{\baselineskip}
    
    The Full Court of the Federal Court of Australia, with Jagot J delivering the judgement, rejected the Commonwealth's arguments. The court held that the foreign act of state doctrine does not preclude judicial scrutiny by an Australian court of the conduct of a foreign state's agents within their own territory when the alleged conduct involves acts of torture. Such acts are illegal under both international law and domestic Australian law, rendering the doctrine inapplicable in these circumstances. The court emphasised that torture constitutes a serious violation of fundamental legal norms, allowing Australian courts to exercise jurisdiction over claims involving such conduct, even if perpetrated abroad.
    
    \vspace{\baselineskip}
    
    Furthermore, the court ruled that the foreign act of state doctrine has no application as a matter of Australian constitutional law when it is alleged that Commonwealth officials acted beyond the scope of their authority under Commonwealth law. The court identified constitutional reasons for not applying the doctrine, particularly where the actions of Australian officials are alleged to contravene legal limits on their power. This finding reinforced the justiciability of Habib's claims, as it permitted the court to examine whether Commonwealth officials had acted unlawfully, irrespective of the involvement of foreign states. The decision allowed Habib's case to proceed, marking a significant precedent regarding the accountability of Australian officials in cases involving extraterritorial torture.
\end{casedetails}

\section{Diplomatic Immunity and Inviolability}
\begin{itemize}
    \item The law on diplomatic immunity is one of the oldest and most settled areas of international law, and is now codified in the \convention{\textit{1961 Vienna Convention on Diplomatic Relations} (\textit{VCDR})}  (there are similar rules for consular immunity under \convention{\textit{1963 Vienna Convention on Consular Relations}}, but they are not entirely identical and thus cannot be conflated)
    \item There is a strong history of compliance with the rules of diplomatic immunity, given the strong reciprocal benefits available to nations in respecting immunity
    \item The \convention{\textit{VCDR}} is given effect in Australia by the \statute{\textit{Diplomatic Privileges and Immunities Act 1967 (Cth)}}
    \item Immunities are accorded to diplomatic agents sent by the `sending state' and received by the `receiving state'
    \begin{itemize}
        \item The receiving state is under obligations to respect the immunity of diplomatic agents
    \end{itemize}
    \item The \convention{\textit{VCDR}} establishes a graduated regime of jurisdictional immunity, as opposed to a uniform set of rules for all diplomats
    \item As mentioned in the preamble of the \convention{\textit{VCDR}}, diplomatic immunity is ``not to benefit individuals, but to ensure the efficient performance of the functions of diplomatic missions''
    \begin{itemize}
        \item This point has been explored by Rosalyn Higgins in a journal article [(1985) 79 \textit{American Journal of International Law} 641], where she held that ``diplomatic law governs the conduct of relations between representative organs of a state operating within the territory of another state, and the receiving state. Its purpose is to facilitate international diplomacy, balancing the pursuit of the foreign policy interests of the sending state with respect for the territorial sovereignty of the receiving state. Diplomatic immunity is an exception to the general rule of territorial jurisdiction. It allows diplomats to be able to carry out their functions within the framework of necessary security and confidentiality.''
    \end{itemize}
    \item \convention{\textit{VCDR} Art 1} defines several key terms in the area of diplomatic immunity, including those of `diplomatic agent' and `premises of the mission'
    \begin{itemize}
        \item Commonwealth countries have High Commissions between each other (which are akin to embassies), and High Commissioners as the top-level diplomatic agents (which are akin to ambassadors); there is no distinction drawn between the two sets of terms for the purposes of the \convention{\textit{VCDR}}
    \end{itemize}
\end{itemize}

\begin{conventiondetails}{\textit{1961 Vienna Convention on Diplomatic Relations} Article 1}
    \flushleft
    For the purpose of the present Convention, the following expressions shall have the meanings hereunder assigned to them:
    \begin{enumerate}[label=(\alph*)]
        \item The ``head of the mission" is the person charged by the sending State with the duty of acting in that capacity;
        \item The ``members of the mission" are the head of the mission and the members of the staff of the mission
        \item The ``members of the staff of the mission" are the members of the diplomatic staff, of the administrative and technical staff and of the service staff of the mission;
        \item The ``members of the diplomatic staff" are the members of the staff of the mission having diplomatic rank;
        \item A ``diplomatic agent" is the head of the mission or a member of the diplomatic staff of the mission;
        \item The ``members of the administrative and technical staff" are the members of the staff of the mission employed in the administrative and technical service of the mission;
        \item The ``members of the service staff" are the members of the staff of the mission in the domestic service of the mission;
        \item A ``private servant" is a person who is in the domestic service of a member of the mission and who is not an employee of the sending State;
        \item The ``premises of the mission" are the buildings or parts of buildings and the land ancillary thereto, irrespective of ownership, used for the purposes of the mission including the residence of the head of the mission.
    \end{enumerate}
\end{conventiondetails}

\begin{itemize}
    \item \convention{\textit{VCDR} Article 3} sets out the primary functions of a diplomatic mission, which includes:
    \begin{itemize}
        \item Representing the sending state
        \item Protecting the interests of the sending state
        \item Negotiating with the receiving state
        \item Ascertaining and reporting to the sending state the conditions and developments in the receiving state
        \item Promoting friendly relations between the sending state and the receiving state
    \end{itemize}
\end{itemize}

\begin{conventiondetails}{\textit{1961 Vienna Convention on Diplomatic Relations} Article 3}
    \flushleft
    \begin{enumerate}
        \item The functions of a diplomatic mission consist, inter alia, in:
        \begin{enumerate}
            \item Representing the sending State in the receiving State;
            \item Protecting in the receiving State the interests of the sending State and of its nationals, within the limits permitted by international law;
            \item Negotiating with the Government of the receiving State;
            \item Ascertaining by all lawful means conditions and developments in the receiving State, and reporting thereon to the Government of the sending State;
            \item Promoting friendly relations between the sending State and the receiving State, and developing their economic, cultural and scientific relations.
        \end{enumerate}
        \item  Nothing in the present Convention shall be construed as preventing the performance of consular functions by a diplomatic mission.
    \end{enumerate}
\end{conventiondetails}

\begin{itemize}
    \item \convention{\textit{VCDR} Art 4-19} address several mostly procedural matters, such as the appointment and accreditation of diplomatic agents (note that the consent of the receiving state is required for the appointment of a head of a mission, but not for the appointment of other diplomatic agents)
    \begin{itemize}
        \item If a state is seeking to appoint an ambassador to a receiving state, it must first receive the consent of the foreign state before any such posting can be made
    \end{itemize}
\end{itemize}

\begin{conventiondetails}{\textit{1961 Vienna Convention on Diplomatic Relations} Article 4}
    \flushleft
    \begin{enumerate}
        \item The sending State must make certain that the \textit{agrément} of the receiving State has been given for the person it proposes to accredit as head of the mission to that State.
        \item The receiving State is not obliged to give reasons to the sending State for a refusal of \textit{agrément}.
    \end{enumerate}
\end{conventiondetails}

\begin{conventiondetails}{\textit{1961 Vienna Convention on Diplomatic Relations} Article 5}
    \flushleft
    \begin{enumerate}
        \item The sending State may, after it has given due notification to the receiving States concerned, accredit a head of mission or assign any member of the diplomatic staff, as the case may be, to more than one State, unless there is express objection by any of the receiving States.
        \item If the sending State accredits a head of mission to one or more other States it may establish a diplomatic mission headed by a chargé d'affaires ad interim in each State where the head of mission has not his permanent seat.
        \item A head of mission or any member of the diplomatic staff of the mission may act as
        representative of the sending State to any international organization.
    \end{enumerate}
\end{conventiondetails}

\begin{conventiondetails}{\textit{1961 Vienna Convention on Diplomatic Relations} Article 6}
    \flushleft
    Two or more States may accredit the same person as head of mission to another State, unless objection is offered by the receiving State.
\end{conventiondetails}

\begin{conventiondetails}{\textit{1961 Vienna Convention on Diplomatic Relations} Article 7}
    \flushleft
    Subject to the provisions of articles 5, 8, 9 and 11, the sending State may freely appoint the members of the staff of the mission. In the case of military, naval or air attachés, the receiving State may require their names to be submitted beforehand, for its approval.
\end{conventiondetails}

\begin{conventiondetails}{\textit{1961 Vienna Convention on Diplomatic Relations} Article 8}
    \flushleft
    \begin{enumerate}
        \item Members of the diplomatic staff of the mission should in principle be of the nationality of the sending State.
        \item Members of the diplomatic staff of the mission may not be appointed from among persons having the nationality of the receiving State, except with the consent of that State which may be withdrawn at any time.
        \item The receiving State may reserve the same right with regard to nationals of a third State who are not also nationals of the sending State.
    \end{enumerate}
\end{conventiondetails}

\begin{conventiondetails}{\textit{1961 Vienna Convention on Diplomatic Relations} Article 9}
    \flushleft
    \begin{enumerate}
        \item The receiving State may at any time and without having to explain its decision, notify the sending State that the head of the mission or any member of the diplomatic staff of the mission is persona non grata or that any other member of the staff of the mission is not acceptable. In any such case, the sending State shall, as appropriate, either recall the person concerned or terminate his functions with the mission. A person may be declared non grata or not acceptable before arriving in the territory of the receiving State.
        \item If the sending State refuses or fails within a reasonable period to carry out its obligations under paragraph 1 of this article, the receiving State may refuse to recognize the person concerned as a member of the mission.
    \end{enumerate}
\end{conventiondetails}

\begin{conventiondetails}{\textit{1961 Vienna Convention on Diplomatic Relations} Article 10}
    \flushleft
    \begin{enumerate}
        \item The Ministry for Foreign Affairs of the receiving State, or such other ministry as may be agreed, shall be notified of:
        \begin{enumerate}
            \item The appointment of members of the mission, their arrival and their final departure or the termination of their functions with the mission;
            \item The arrival and final departure of a person belonging to the family of a member of the mission and, where appropriate, the fact that a person becomes or ceases to be a member of the family of a member of the mission;
            \item The arrival and final departure of private servants in the employ of persons referred to in subparagraph (a) of this paragraph and, where appropriate, the fact that they are leaving the employ of such persons;
            \item The engagement and discharge of persons resident in the receiving State as members of the mission or private servants entitled to privileges and immunities.
        \end{enumerate}
        \item Where possible, prior notification of arrival and final departure shall also be given.
    \end{enumerate}
\end{conventiondetails}

\begin{conventiondetails}{\textit{1961 Vienna Convention on Diplomatic Relations} Article 11}
    \flushleft
    \begin{enumerate}
        \item In the absence of specific agreement as to the size of the mission, the receiving State may require that the size of a mission be kept within limits considered by it to be reasonable and normal, having regard to circumstances and conditions in the receiving State and to the needs of the particular mission.
        \item The receiving State may equally, within similar bounds and on a non-discriminatory basis, refuse to accept officials of a particular category.
    \end{enumerate}
\end{conventiondetails}

\begin{conventiondetails}{\textit{1961 Vienna Convention on Diplomatic Relations} Article 12}
    \flushleft
    The sending State may not, without the prior express consent of the receiving State, establish offices forming part of the mission in localities other than those in which the mission itself is established.
\end{conventiondetails}

\begin{conventiondetails}{\textit{1961 Vienna Convention on Diplomatic Relations} Article 13}
    \flushleft
    \begin{enumerate}
        \item The head of the mission is considered as having taken up his functions in the receiving State either when he has presented his credentials or when he has notified his arrival and a true copy of his credentials has been presented to the Ministry for Foreign Affairs of the receiving State, or such other ministry as may be agreed, in accordance with the practice prevailing in the receiving State which shall be applied in a uniform manner.
        \item The order of presentation of credentials or of a true copy thereof will be determined by the date and time of the arrival of the head of the mission.
    \end{enumerate}
\end{conventiondetails}

\begin{conventiondetails}{\textit{1961 Vienna Convention on Diplomatic Relations} Article 14}
    \flushleft
    \begin{enumerate}
        \item Heads of mission are divided into three classes, namely:
        \begin{enumerate}
            \item That of ambassadors or nuncios accredited to Heads of State, and other heads of mission of equivalent rank;
            \item That of envoys, ministers and internuncios accredited to Heads of State;
            \item That of chargés d'affaires accredited to Ministers for Foreign Affairs.
        \end{enumerate}
        \item Except as concerns precedence and etiquette, there shall be no differentiation between heads of mission by reason of their class.
    \end{enumerate}
\end{conventiondetails}

\begin{conventiondetails}{\textit{1961 Vienna Convention on Diplomatic Relations} Article 15}
    \flushleft
    The class to which the heads of their missions are to be assigned shall be agreed between States.
\end{conventiondetails}

\begin{conventiondetails}{\textit{1961 Vienna Convention on Diplomatic Relations} Article 16}
    \flushleft
    \begin{enumerate}
        \item Heads of mission shall take precedence in their respective classes in the order of the date and time of taking up their functions in accordance with article 13.
        \item Alterations in the credentials of a head of mission not involving any change of class shall not affect his precedence.
        \item This article is without prejudice to any practice accepted by the receiving State regarding the precedence of the representative of the Holy See.
    \end{enumerate}
\end{conventiondetails}

\begin{conventiondetails}{\textit{1961 Vienna Convention on Diplomatic Relations} Article 17}
    \flushleft
    The precedence of the members of the diplomatic staff of the mission shall be notified by the head of the mission to the Ministry for Foreign Affairs or such other ministry as may be agreed.
\end{conventiondetails}

\begin{conventiondetails}{\textit{1961 Vienna Convention on Diplomatic Relations} Article 18}
    \flushleft
    The procedure to be observed in each State for the reception of heads of mission shall be uniform in respect of each class.
\end{conventiondetails}

\begin{conventiondetails}{\textit{1961 Vienna Convention on Diplomatic Relations} Article 19}
    \flushleft
    \begin{enumerate}
        \item  If the post of head of the mission is vacant, or if the head of the mission is unable to perform his functions a chargé d'affaires ad interim shall act provisionally as head of the mission. The name of the chargé d'affaires ad interim shall be notified, either by the head of the mission or, in case he is unable to do so, by the Ministry for Foreign Affairs of the sending State to the Ministry for Foreign Affairs of the receiving State or such other ministry as may be agreed.
        \item In cases where no member of the diplomatic staff of the mission is present in the receiving State, a member of the administrative and technical staff may, with the consent of the receiving State, be designated by the sending State to be in charge of the current administrative affairs of the mission.
    \end{enumerate}
\end{conventiondetails}

\subsection{Diplomatic Inviolability}
\begin{conventiondetails}{\textit{1961 Vienna Convention on Diplomatic Relations} Article 22}
    \flushleft
    \begin{enumerate}
        \item The premises of the mission shall be inviolable. The agents of the receiving State may not enter them, except with the consent of the head of the mission.
        \item The receiving State is under a special duty to take all appropriate steps to protect the premises of the mission against any intrusion or damage and to prevent any disturbance of the peace of the mission or impairment of its dignity.
        \item The premises of the mission, their furnishings and other property thereon and the means of transport of the mission shall be immune from search, requisition, attachment or execution.
    \end{enumerate}
\end{conventiondetails}

\begin{itemize}
    \item Under \convention{\textit{VCDR} Article 22(1)}, the premises of the diplomatic mission are inviolable and may not be entered without the consent of the head of the mission
    \item This obligation has been strictly adhered to (e.g., the 1984 St James Square incident, where the UK police were unable to enter the Iranian embassy in London to arrest a suspect following the murder of WPC Fletcher, and Julian Assange seeking refuge in the Ecuadorian embassy in London)
    \item It was held in \case{\textit{Saudi Arabian Cultural Mission v Alramadi} [2024] FCA 1060} that serving proceedings on a diplomatic mission without the consent of the head of the mission breaches the inviolability of the mission
\end{itemize}

\begin{casedetails}{\textit{Saudi Arabian Cultural Mission v Alramadi} [2024] FCA 1060}
    \flushleft

    In this case, 17 former employees initiated legal proceedings against the Saudi Arabian Cultural Mission (SACM) in South Australia, alleging violations of the Fair Work Act 2009 (Cth). The employees claimed that the SACM had failed to pay their entitlements, giving rise to an employment dispute. The case centred on the proper procedure for serving legal proceedings on a foreign state entity and the implications of failing to adhere to diplomatic protocols under international law.
    
    \vspace{\baselineskip}

    Under section 24 of the Foreign States Immunities Act 1985 (Cth), when suing a foreign state, the originating process must be sent to the Commonwealth Attorney-General, who is then responsible for forwarding it to the foreign state's relevant ministry of foreign affairs. In this instance, the Attorney-General did not transmit the originating process as required. Instead, the legal documents were served directly on the deputy head of the Saudi embassy, bypassing the formal channels outlined in the legislation and raising questions about the propriety of the service method.
    
    \vspace{\baselineskip}

    Raper J of the Federal Court of Australia ruled that the direct service on the Saudi embassy was invalid. The court held that, under section 24 of the Foreign States Immunities Act, service on a diplomatic mission can only be effected if the foreign state itself authorises such service. Without explicit consent from the head of the mission, serving the deputy head breached the inviolability of the diplomatic mission, as protected by Article 22 of the Vienna Convention on Diplomatic Relations 1961 (VCDR). This article ensures that diplomatic missions are immune from intrusion, including the service of legal processes, unless consented to by the mission.
    
    \vspace{\baselineskip}

    This decision underscored the importance of adhering to statutory and international legal frameworks when initiating proceedings against foreign state entities. The court found that the failure to follow the prescribed process under section 24, combined with the lack of consent from the Saudi mission, rendered the service invalid. This ruling highlighted the intersection of domestic law and diplomatic immunity, affirming the protections afforded to diplomatic missions under the VCDR and the procedural requirements for engaging foreign states in Australian courts.
\end{casedetails}

\begin{itemize}
    \item Under \convention{\textit{VCDR} Article 22(2)}, the receiving state is under a special duty to take all appropriate steps to protect the premises of the mission against any intrusion or damage and to prevent any disturbance of the peace of the mission or impairment of its dignity
    \item This is a strict obligation on the receiving state, and it is not a matter of discretion, or of claiming that they had tried their best (e.g., in the attacks on the Iranian Embassy in Canberra by protestors is 1992, the Australian government had failed to protect the embassy, and so assumed liability and compensated the Iranian government for the damage caused)
    \item \case{\textit{Tehran Hostages} [1980] ICJ Rep 3} exemplifies the importance of protecting the premises of a diplomatic mission from damage, but \case{\textit{Minister for Foreign Affairs \& Trade v Magno} (1992) 37 FCR 298} holds that merely peaceful protests outside a diplomatic mission do not amount to a breach of the inviolability of the mission, so long as it does not interfere with the functions of the mission
\end{itemize}

\begin{casedetails}{\textit{Tehran Hostages} [1980] ICJ Rep 3}
    \flushleft
    This case arose during the 1979 Iranian Revolution, when Iranian students overthrew the government and targeted foreign entities, including the US. On 4 November 1979, protestors stormed the US embassy in Tehran, taking 52 staff members hostage for 444 days. The US brought the matter before the International Court of Justice (ICJ), alleging Iran violated its obligations under international law.

    \vspace{\baselineskip}

    The ICJ ruled that Iran breached Article 22 of the \convention{\textit{1961 Vienna Convention on Diplomatic Relations 1961 (VCDR)}}, which requires host states to protect diplomatic missions. The court found Iran failed to prevent the embassy takeover and even supported the protestors, exacerbating the violation. The ICJ stressed the stringent nature of Article 22, holding that host states must proactively ensure the safety of diplomatic premises, regardless of internal unrest.
    
    \vspace{\baselineskip}

    This landmark decision reinforced the inviolability of diplomatic missions and the absolute duty of host states under the \convention{\textit{VCDR}}, highlighting that domestic instability does not excuse failures to uphold international legal standards.
\end{casedetails}

\begin{casedetails}{\textit{Minister for Foreign Affairs \& Trade v Magno} (1992) 37 FCR 298}
    \flushleft
    Here, the Full Court of the Federal Court of Australia addressed a peaceful protest outside the Indonesian embassy in Canberra. The demonstration, held in 1992, was organised by groups protesting a massacre that had occurred days earlier. Protestors placed approximately 120 small white crosses on the grass outside the embassy as a symbolic gesture but refused to remove them when requested. The Australian government responded by adopting a new regulation, empowering the Minister for Foreign Affairs to order the removal of such symbols if deemed offensive to the peace and dignity of the diplomatic mission.

    \vspace{\baselineskip}

    The protestors challenged the validity of the regulation in the Federal Court, arguing it infringed on their rights to free expression. The court upheld the regulation, finding it a lawful exercise of governmental authority to balance diplomatic obligations with public expression. However, French J provided significant observations on the application of \convention{\textit{1961 Vienna Convention on Diplomatic Relations} (\textit{VCDR}) Article 22}, which requires receiving states to protect the peace and dignity of diplomatic missions. He noted that not all activities outside a mission necessarily breach Article 22, particularly peaceful protests.

    \vspace{\baselineskip}

    French J elaborated on what might impair a mission's dignity, stating at page 326: ``However, the dignity of the mission may be impaired by activity that would not amount to a disturbance of its peace. Offensive or insulting behaviour in the vicinity of and directed to the mission may fall into this category. The burning of the flag of the sending State or the mock execution of its leader in effigy if committed in the immediate vicinity of the mission could well be construed as attacks upon its dignity. So too might the depositing of some offensive substance and perhaps also the dumping of farm commodities outside mission premises in protest against subsidy policies of the sending State. Any such incident would have to be assessed in the light of the surrounding circumstances. But subject to protection against those classes of conduct, the sending State takes the receiving State as it finds it. If it finds it with a well established tradition of free expression, including public comment on matters of domestic and international politics, it cannot invoke either Article 22(2) or 29 against manifestations of that tradition." This highlighted that acts like burning flags or staging mock executions could violate a mission's dignity, but peaceful protests generally do not.

    \vspace{\baselineskip}

    French J concluded that a mere peaceful protest, such as the lawful placement of a reproachful and dignified symbol like the crosses, does not impair the dignity of the mission or breach its peace. The decision affirmed Australia's tradition of free expression, noting that sending states must accept the receiving state's cultural and legal norms, including the right to peaceful protest, unless the conduct explicitly undermines the mission's dignity or peace as outlined in Article 22.
\end{casedetails}

\begin{itemize}
    \item Article 29 holds that the person of a diplomatic agent is inviolable, and they cannot be arrested or detained
    \item Article 30 holds that the private residence of a diplomatic agent is inviolable, and their papers and correspondence are also inviolable
    \item All diplomatic agents, not just the heads of mission, are entitled to the protections outlined in Articles 29 and 30
\end{itemize}

\begin{conventiondetails}{\textit{1961 Vienna Convention on Diplomatic Relations} Article 29}
    \flushleft
    The person of a diplomatic agent shall be inviolable. He shall not be liable to any form of arrest or detention. The receiving State shall treat him with due respect and shall take all appropriate steps to prevent any attack on his person, freedom or dignity.
\end{conventiondetails}

\begin{conventiondetails}{\textit{1961 Vienna Convention on Diplomatic Relations} Article 30}
    \flushleft
    \begin{enumerate}
        \item The private residence of a diplomatic agent shall enjoy the same inviolability and protection as the premises of the mission.
        \item His papers, correspondence and, except as provided in paragraph 3 of article 31, his property, shall likewise enjoy inviolability.
    \end{enumerate}
\end{conventiondetails}

\subsection{Diplomatic Agents}

\begin{conventiondetails}{\textit{1961 Vienna Convention on Diplomatic Relations} Article 31}
    \flushleft
    \begin{enumerate}
        \item A diplomatic agent shall enjoy immunity from the criminal jurisdiction of the receiving State. He shall also enjoy immunity from its civil and administrative jurisdiction, except in the case of:
        \begin{enumerate}
            \item A real action relating to private immovable property situated in the territory of the receiving State, unless he holds it on behalf of the sending State for the purposes of the mission;
            \item An action relating to succession in which the diplomatic agent is involved as executor, administrator, heir or legatee as a private person and not on behalf of the sending State;
            \item An action relating to any professional or commercial activity exercised by the diplomatic agent in the receiving State outside his official functions.
        \end{enumerate}
        \item A diplomatic agent is not obliged to give evidence as a witness.
        \item No measures of execution may be taken in respect of a diplomatic agent except in the cases coming under subparagraphs (a), (b) and (c) of paragraph 1 of this article, and provided that the measures concerned can be taken without infringing the inviolability of his person or of his residence.
        \item The immunity of a diplomatic agent from the jurisdiction of the receiving State does not exempt him from the jurisdiction of the sending State.
    \end{enumerate}
\end{conventiondetails}

\begin{itemize}
    \item Article 31 deals with immunity from civil and criminal jurisdictions as applied to diplomatic agents
    \item Under Article 31(1), diplomatic agents enjoy absolute immunity \textit{ratione personae} (immunity that derives from their status as diplomats, not from the nature of their activities) from the criminal jurisdiction of the receiving state
    \item They also enjoy a broad immunity from the receiving state's civil and administrative jurisdiction, with three key exceptions
    \begin{enumerate}
        \item A real action relating to private immovable property situated in the receiving state (unless property held for purposes of mission)
        \item An action relating to succession (where agent is executor, administrator, heir, or legatee, as a private person)
        \item An action relating to any professional or commercial activity exercised by the diplomatic agent in the receiving state outside his or her official functions
    \end{enumerate}
    \begin{itemize}
        \item E.g., if a diplomatic agent is running a side business, they will not be immune in civil proceedings in relation to the business
    \end{itemize}
    \item Under \case{\textit{Diplomatic Immunity Case} (1973) Family Court of Australia}, the exceptions made to diplomatic agents under under Article 31 do not extend to family law disputes
    \item The Article 31 exceptions also do not extend to employment that is a form of modern slavery, following \case{\textit{Basfar v Wong} [2022] UKSC 20} (and also \case{\textit{Reyes v Al-Malki} [2017] UKSC 61; [2019] AC 735})
    \begin{itemize}
        \item The employment relationship is between the diplomatic agent and another individual
        \item Modern slavery includes forced labour, servitude, trafficking, etc., so long as it amounts to commercial activity
    \end{itemize}
\end{itemize}

\begin{casedetails}{\textit{Basfar v Wong} [2022] UKSC 20}
    \flushleft
    In this case, Josephine Wong, a domestic worker employed by a member of the diplomatic staff at the Saudi embassy in the UK, brought a claim against her employer, Mr Basfar. Wong alleged that she was subjected to exploitative conditions amounting to modern slavery. She was confined to the mission premises, unable to leave, and received no payment for her work. These conditions were deemed to constitute modern slavery under both UK and international law, prompting Wong to seek compensation for her mistreatment.

    \vspace{\baselineskip}
    
    The case raised questions about diplomatic immunity under \convention{\textit{1961 Vienna Convention on Diplomatic Relations} (\textit{VCDR}) Article 31}, which generally grants diplomats immunity from civil jurisdiction in the receiving state. Basfar argued that this immunity barred Wong's claim. However, Wong contended that her treatment fell outside the scope of diplomatic functions, invoking an exception to immunity. The dispute centred on whether employing a domestic worker under conditions of modern slavery could be considered a ``commercial activity" under Article 31(1)(c) of the VCDR, which excludes immunity for such activities.
    
    \vspace{\baselineskip}
    
    The UK Supreme Court (UKSC) ruled in Wong's favour, holding that her employment under conditions of modern slavery constituted a commercial activity beyond the ordinary operation of a diplomatic mission. The court found that the exploitation of Wong for domestic labour, in a manner amounting to slavery, was not incidental to the diplomatic functions of the mission but rather a form of commercial exploitation. Consequently, the exception to immunity under Article 31(1)(c) applied, as the activity was not part of the official duties of the diplomat.
    
    \vspace{\baselineskip}
    
    This decision clarified that diplomatic immunity does not extend to egregious violations like modern slavery, even when occurring within a mission. The UKSC's ruling ensured that diplomats cannot rely on immunity to shield exploitative practices that contravene fundamental human rights, allowing Wong to pursue her claim for compensation and setting limits for  diplomatic immunity in cases of serious misconduct.
\end{casedetails}

\begin{conventiondetails}{\textit{1961 Vienna Convention on Diplomatic Relations} Article 37}
    \flushleft
    \begin{enumerate}
        \item The members of the family of a diplomatic agent forming part of his household shall, if they are not nationals of the receiving State, enjoy the privileges and immunities specified in articles 29 to 36.
        \item Members of the administrative and technical staff of the mission, together with members of their families forming part of their respective households, shall, if they are not nationals of or permanently resident in the receiving State, enjoy the privileges and immunities specified in articles 29 to 35, except that the immunity from civil and administrative jurisdiction of the receiving State specified in paragraph 1 of article 31 shall not extend to acts performed outside the course of their duties. They shall also enjoy the privileges specified in article 36, paragraph 1, in respect of articles imported at the time of first installation.
        \item Members of the service staff of the mission who are not nationals of or permanently resident in the receiving State shall enjoy immunity in respect of acts performed in the course of their duties, exemption from dues and taxes on the emoluments they receive by reason of their employment and the exemption contained in article 33.
        \item Private servants of members of the mission shall, if they are not nationals of or permanently resident in the receiving State, be exempt from dues and taxes on the emoluments they receive by reason of their employment. In other respects, they may enjoy privileges and immunities only to the extent admitted by the receiving State. However, the receiving State must exercise its jurisdiction over those persons in such a manner as not to interfere unduly with the performance of the functions of the mission.
    \end{enumerate}
\end{conventiondetails}

\begin{itemize}
    \item For members of the diplomatic agent's family, if they are not nationals of the receiving state, and are part of the agent's household, they enjoy the same privileges and immunities as the diplomatic agent, under \convention{\textit{VCDR} Art 37(1)}
    \item For members of the administrative and technical staff of the mission, if they are not nationals or permanent residents of the receiving state, they enjoy the same immunities as the diplomatic agent, except that the immunity from civil and administrative jurisdiction under \convention{\textit{VCDR} Art 31(1)} does not extend to acts performed outside the course of their duties, under \convention{\textit{VCDR} Art 37(2)}
    \begin{itemize}
        \item These are important staff members in a mission
    \end{itemize}
    \item For members of the service staff of the mission, if they are not nationals or permanent residents of the receiving state, they enjoy immunity in respect of acts performed in the course of their duties, and exemption from dues and taxes on the emoluments they receive by reason of their employment, and the exemption contained in \convention{\textit{VCDR} Art 33}, under \convention{\textit{VCDR} Art 37(3)}
    \begin{itemize}
        \item These generally include staff members such as the kitchen staff, cleaners, and other support staff
    \end{itemize}
    \item For private servants in the mission, if they are not nationals or permanent residents of the receiving state, they are exempt from dues and taxes on their wages, under \convention{\textit{VCDR} Art 37(4)}
    \begin{itemize}
        \item Examples of private servants include butlers
    \end{itemize}
\end{itemize}

\subsection{Abuse of Diplomatic Immunities}
\begin{itemize}
    \item Under \convention{\textit{VCDR} Art 9}, if a diplomat abuses their privileges and immunities, the receiving state may, at any time, declare that the head of the mission or any other member of the mission is \textit{persona non grata}
    \begin{itemize}
        \item The sending state must then recall the person, or terminate their functions (if this is not completed in a reasonable period of time, then the receiving state may treat the person may treat the person as no longer enjoying diplomatic privileges and immunities) (reasons do not have to be given)
        \item The sending state is under an obligation to recall the person from the mission before they are forcibly removed by the receiving state after a reasonable period of time
    \end{itemize}
\end{itemize}

\begin{conventiondetails}{\textit{1961 Vienna Convention on Diplomatic Relations} Article 32}
    \flushleft
    \begin{enumerate}
        \item The immunity from jurisdiction of diplomatic agents and of persons enjoying immunity under article 37 may be waived by the sending State.
        \item Waiver must always be express.
        \item The initiation of proceedings by a diplomatic agent or by a person enjoying immunity from jurisdiction under article 37 shall preclude him from invoking immunity from jurisdiction in respect of any counterclaim directly connected with the principal claim.
        \item Waiver of immunity from jurisdiction in respect of civil or administrative proceedings shall not be held to imply waiver of immunity in respect of the execution of the judgement, for which a separate waiver shall be necessary.
    \end{enumerate}
\end{conventiondetails}

\begin{itemize}
    \item Under \convention{\textit{VCDR} Art 32}, the sending state may waive the immunity of a diplomatic agent, but this must be express
    \begin{itemize}
        \item A waiver relating to civil and administrative jurisdiction does not imply a waiver of immunity in respect of the execution of the judgement, for which a separate waiver is necessary, and vice versa
    \end{itemize}
    \item Waiver can sometimes be waived with respect to car accidents; e.g., \case{\textit{Dickinson v Del Solar} [1930] 1 KB 376} (a waiver of immunity in relation to civil proceedings), and \case{\textit{Georgian Diplomat in Washington D.C.} (1998)} (a waiver of immunity in relation to criminal proceedings)
\end{itemize}

\begin{casedetails}{\textit{Dickinson v Del Solar} [1930] 1 KB 376}
    \flushleft
    This case involved a Peruvian diplomat running over and subsequently injuring Mr Dickinson in his car. Ordinarily in civil proceedings, the diplomat would have civil immunity. However, the Peruvian government waived the diplomat's immunity in relation to civil proceedings, allowing Mr Dickinson to sue the diplomat for damages. The court held that the waiver of immunity was valid and that Mr Dickinson could proceed with his claim against the diplomat, as it was a sovereign privilege that the Peruvian government was exercising.
\end{casedetails}

\begin{casedetails}{\textit{Georgian Diplomat in Washington D.C.} (1998)}
    \flushleft
    In this case, a diplomat was intoxicated and speeding whilst driving, subsequently crashing his car and killing a young person. The US Department of State requested Georgia waive criminal immunity, and as such, Georgia invoked Art 32 and waived criminal immunity. However, as there had been no waiver of civil immunity, the plaintiff's relatives were unsuccessful in claiming damages against the diplomat in civil proceedings.
\end{casedetails}

\begin{conventiondetails}{\textit{1961 Vienna Convention on Diplomatic Relations} Article 39}
    \flushleft
    \begin{enumerate}
        \item Every person entitled to privileges and immunities shall enjoy them from the moment he enters the territory of the receiving State on proceeding to take up his post or, if already in its territory, from the moment when his appointment is notified to the Ministry for Foreign Affairs or such other ministry as may be agreed.
        \item When the functions of a person enjoying privileges and immunities have come to an end,  such privileges and immunities shall normally cease at the moment when he leaves the country, or on expiry of a reasonable period in which to do so, but shall subsist until that time, even in case of armed conflict. However, with respect to acts performed by such a person in the exercise of his functions as a member of the mission, immunity shall continue to subsist.
        \item In case of the death of a member of the mission, the members of his family shall continue to enjoy the privileges and immunities to which they are entitled until the expiry of a reasonable period in which to leave the country.
        \item  In the event of the death of a member of the mission not a national of or permanently resident in the receiving State or a member of his family forming part of his household, the receiving State shall permit the withdrawal of the movable property of the deceased, with the exception of any property acquired in the country the export of which was prohibited at the time of his death. Estate, succession and inheritance duties shall not be levied on movable property the presence of which in the receiving State was due solely to the presence there of the deceased as a member of the mission or as a member of the family of a member of the mission.
    \end{enumerate}
\end{conventiondetails}

\begin{itemize}
    \item Under \convention{\textit{VCDR} Art 39(1)}, diplomatic agents enjoy privileges and immunities from the moment they enter the territory of the receiving state
    \item Under \convention{\textit{VCDR} Art 39(2)}, when the functions of a person enjoying immunities come to an end, the immunities cease the moment he or she leaves the country
    \begin{itemize}
        \item However, immunity \textit{ratione materiae} continues to subsist with respect to acts performed by the person in the exercise of his or her functions as a member of the mission
        \item When the diplomat leaves the country, they are no longer protected by a personal immunity, and are instead only protected by a residual functional immunity for the acts that they had performed in the exercise of their functions as a member of the mission
        \item This immunity applies both with respect to civil and criminal proceedings
        \item To escape civil/criminal prosecution, a diplomat must characterise their actions as within the scope of their functions as a member of the mission
    \end{itemize}
\end{itemize}