\begin{itemize}
    \item In contrast to domestic systems of law, the sources of public international law are often more challenging to ascertain, as there is a wide variety of material sources, and limited machinery for formal law-making (e.g., there's no global legislature, no global court with universal compulsory jurisdiction, and a lack of precedent)
    \item The source doctrine in international law is state-centric, and hinges on states consenting to be bound by the sources of international law
    \item The International Court of Justice (ICJ) is the primary judicial organ of the United Nations, with the \textit{Statute of the International Court of Justice} forming an integral part of the \textit{Charter of the United Nations}
\end{itemize}

\begin{conventiondetails}{\textit{Charter of the United Nations} Art 92}
    \flushleft
    The International Court of Justice shall be the principal judicial organ of the United Nations. It shall function in accordance with the annexed Statute, which is based upon the Statute of the Permanent Court of International Justice and forms an integral part of the present Charter.
\end{conventiondetails}

\begin{itemize}
    \item The accepted sources of public international law are set out in Article 38(1) of the \textit{Statute of the International Court of Justice}
\end{itemize}

\begin{statutedetails}{\textit{Statute of the International Court of Justice} Art 38}\label{ICJ Statute Art 38}
    \flushleft
    \begin{enumerate}
        \item The Court, whose function is to decide in accordance with international law such disputes as are submitted to it, shall apply:
        \begin{enumerate}[label=\alph*.]
            \item international conventions, whether general or particular, establishing rules expressly recognized by the contesting states;
            \item international custom, as evidence of a general practice accepted as law;
            \item the general principles of law recognized by civilized nations;
            \item subject to the provisions of Article 59, judicial decisions and the teachings of the most highly qualified publicists of the various nations, as subsidiary means for the determination of rules of law.
        \end{enumerate}
        \item This provision shall not prejudice the power of the Court to decide a case \textit{ex aequo et bono}, if the parties agree thereto.
    \end{enumerate}
\end{statutedetails}

\begin{itemize}
    \item There is an emphasis on state consent to be bound to the jurisdiction of the ICJ
    \item There is no clear hierarchy governing which facet or source is to be applied first (e.g., treaties do not always trump international conventions)
    \begin{itemize}
        \item However, the sources in \statute{art 38(1)(d)} (judicial decisions and the writings of publicists) are `subsidiary' rather than direct sources
    \end{itemize}
    \item States always remain the primary actors in the application of international law
    \item \statute{Art 38(1)} is `generally regarded as a complete statement of the sources of international law' in Australia, following \case{\textit{Ure v Commonwealth} (2016) 329 ALR 452}
\end{itemize}

\begin{casedetails}{\textit{Ure v Commonwealth} (2016) 329 ALR 452}\label{case:Ure v Commonwealth}
    \flushleft
    This case involved two highly remote islands (Elizabeth Reef and Middleton Reef) in the southwest Pacific Ocean, around 80 nautical miles north of Lord Howe Island. In 1970, Mr Ure erected a sign on the bridge of a ship wrecked on Middleton Reef above the high-tide mark, and claimed title to the islands.

    \vspace{\baselineskip}

    10 years later, his son brought proceedings against the Commonwealth (Mr Ure had died), the question for determination on the assumed facts (including the assumed fact that, in 1970, the islands were unoccupied and constituted \textit{terra nullius} in respect of which no state had claimed sovereignty) was whether, under public international law, there existed a rule that an individual may acquire proprietary title in unoccupied land not claimed by any sovereign state.

    \vspace{\baselineskip}

    The Federal Court examined the sources of the aforementioned rule. At paragraph [15], it was held that ``Australian courts have accepted that Art[icle] 38(1) [of the \textit{Statute of the International Court of Justice} 1945] sets out the sources of international law: \textit{Polyukhovich v. The Commonwealth }(1991) 172 CLR 501 at 559 per Brennan J … .” The Court held that a rule of customary international law requires proof of both (following \case{\textit{North Sea Continental Shelf Cases (FRG v Denmark; FRG v The Netherlands)} at [29] - [31]}):
    \begin{itemize}
        \item ``Extensive and virtually uniform'' state practice
        \item \textit{Opinio juris} (a belief by states that state practice is rendered obligatory on it)
    \end{itemize}

    Whilst the plaintiff attempted to show evidence of state practice supporting their rule, this was rejected by the Court, which held that the rule was not a general principle of law recognised in municipal legal systems within the meaning of Art 38(1)(c). The plaintiff's appeal was dismissed.
\end{casedetails}

\section{Treaties}
\begin{itemize}
    \item Under Art 38(1)(a) of the \textit{Statute of the International Court of Justice}, `conventions' (embodying all binding international agreements) are a source of public international law (e.g., treaties, protocols, statutes, charters, covenants, etc.)
    \item The law of treaties is governed in the 1969 \convention{\textit{Vienna Convention on the Law of Treaties} (VLCT)} (see Topic \ref{sec:Topic 3} on page \pageref{sec:Topic 3}), and now constitutes the most voluminous source of PIL (due to a rapid increase in the number of treaties in the 20\textsuperscript{th} and 21\textsuperscript{st} centuries)
    \item Treaties can be either bilateral (between two states) or multilateral (between more than two states)
    \item Some treaties may be a mere source of obligation (often known as `treaty contracts'), whilst others form a more generalised source of law (`law-making treaties', especially those that contribute to the generation of customary law, such as the \textit{Charter of the United Nations})
    \begin{itemize}
        \item Some treaties, such as the \textit{Universal Declaration of Human Rights}, can adopt a constitutional tone
    \end{itemize}
\end{itemize}

\section{International Custom}
\begin{itemize}
    \item Under Art 38(1)(b) of the \textit{Statute of the International Court of Justice}, international custom can be evidence of a general practice accepted as law, and comprises two components:
    \begin{itemize}
        \item An objective element (`general practice')
        \item A subjective or psychological element (`accepted as law'), known as \textit{opinio juris sive necessitatis} (the belief that the practice is obligatory)
        \begin{itemize}
            \item This is often shortened to \textit{opinio juris}
        \end{itemize}
    \end{itemize}
    \item Customary international law binds all states, even if they have not participated in its creation (with the very narrow exception of the `persistent objector')
    \item Certain customary norms are \textit{jus cogens}, which are peremptory norms from which no derogation is permitted
\end{itemize}

\subsection{Elements of Custom}
\begin{itemize}
    \item Evidence of state practice
    \begin{itemize}
        \item This constitutes any material which demonstrates the activities and views of states and state officials (e.g., legislation, statements of officials, court decisions, voting records in international forums, etc.)
    \end{itemize}
    \item Requirements for practice to generate custom, which include
    \begin{itemize}
        \item Consistency of the practice over time
        \item The practice being widespread
        \item The practice being representative of multiple states (including the states most likely being affected)
        \item Having developed over a lengthy period (but customary norms may still emerge rapidly if there is an overwhelming practice of it)
        \item The practice does not need to be entirely uniform
    \end{itemize}
    \item Practice must be accompanied by \textit{opinio juris}
    \begin{itemize}
        \item \textit{Opinio juris} refers to the belief that the state practice is obligatory
        \item This is notionally as important as state practice, and the two are often weighed on a sliding scale:
        \begin{itemize}
            \item If there is extensive state practice, then \textit{opinio juris} tends to be less important (which gives rise to a rebuttable presumption that there is sufficient \textit{opinio juris})
            \item If there is limited state practice, then \textit{opinio juris} may be more important
        \end{itemize}
    \end{itemize}
    \item Treaties may codify custom in order to reduce ambiguity
\end{itemize}

\begin{casedetails}{\textit{North Sea Continental Shelf Cases (Germany v Denmark; Germany v Netherlands)} [1969] ICJ Rep 3}\label{case:North Sea Continental Shelf}
    \flushleft
    \textbf{ICJ Summary}: These cases concerned the delimitation of the continental shelf of the North Sea as between Denmark and the Federal Republic of Germany, and as between the Netherlands and the Federal Republic, and were submitted to the Court by Special Agreement. The Parties asked the Court to state the principles and rules of international law applicable, and undertook thereafter to carry out the delimitations on that basis. By an Order of 26 April 1968 the Court, having found Denmark and the Netherlands to be in the same interest, joined the proceedings in the two cases. In its Judgment, delivered on 20 February 1969, the Court found that the boundary lines in question were to be drawn by agreement between the Parties and in accordance with equitable principles in such a way as to leave to each Party those areas of the continental shelf which constituted the natural prolongation of its land territory under the sea, and it indicated certain factors to be taken into consideration for that purpose. The Court rejected the contention that the delimitations in question had to be carried out in accordance with the principle of equidistance as defined in the 1958 Geneva Convention on the Continental Shelf. The Court took account of the fact that the Federal Republic had not ratified that Convention, and held that the equidistance principle was not inherent in the basic concept of continental shelf rights, and that this principle was not a rule of customary international law.

    \tcblower

    \flushleft

    In this case, the ICJ held that treaty norms could become custom (but not in this instance), and that treaty provisions may become customary norm/customary international law (however, this was also not made out in this instance). Additionally, a short time frame is not a bar to establishing custom, but the practice needs to be extensive and virtually uniform (this was not the case here).
\end{casedetails}

\begin{casedetails}{\textit{Military and Paramilitary Activities in and against Nicaragua} [1986] ICJ Rep 14}\label{case:Military and paramilitary activities in Nicaragua}
    \flushleft
    \textbf{ICJ Summary}: On 9 April 1984 Nicaragua filed an Application instituting proceedings against the United States of America, together with a Request for the indication of provisional measures concerning a dispute relating to responsibility for military and paramilitary activities in and against Nicaragua. On 10 May 1984 the Court made an Order indicating provisional measures. One of these measures required the United States immediately to cease and refrain from any action restricting access to Nicaraguan ports, and, in particular, the laying of mines. The Court also indicated that the right to sovereignty and to political independence possessed by Nicaragua, like any other State, should be fully respected and should not be jeopardized by activities contrary to the principle prohibiting the threat or use of force and to the principle of non-intervention in matters within the domestic jurisdiction of a State. The Court also decided in the aforementioned Order that the proceedings would first be addressed to the questions of the jurisdiction of the Court and of the admissibility of the Nicaraguan Application. Just before the closure of the written proceedings in this phase, El Salvador filed a declaration of intervention in the case under Article 63 of the Statute, requesting permission to claim that the Court lacked jurisdiction to entertain Nicaragua's Application. In its Order dated 4 October 1984, the Court decided that El Salvador's declaration of intervention was inadmissible inasmuch as it related to the jurisdictional phase of the proceedings.

    \vspace{\baselineskip}

    After hearing argument from both Parties in the course of public hearings held from 8 to 18 October 1984, on 26 November 1984 the Court delivered a Judgment stating that it possessed jurisdiction to deal with the case and that Nicaragua's Application was admissible. In particular, it held that the Nicaraguan declaration of 1929 was valid and that Nicaragua was therefore entitled to invoke the United States declaration of 1946 as a basis of the Court's jurisdiction (Article 36, paragraphs 2 and 5, of the Statute). The subsequent proceedings took place in the absence of the United States, which announced on 18 January 1985 that it “intends not to participate in any further proceedings in connection with this case”. From 12 to 20 September 1985, the Court heard oral argument by Nicaragua and the testimony of the five witnesses it had called. On 27 June 1986, the Court delivered its Judgment on the merits. The findings included a rejection of the justification of collective self-defence advanced by the United States concerning the military or paramilitary activities in or against Nicaragua, and a statement that the United States had violated the obligations imposed by customary international law not to intervene in the affairs of another State, not to use force against another State, not to infringe the sovereignty of another State, and not to interrupt peaceful maritime commerce. The Court also found that the United States had violated certain obligations arising from a bilateral Treaty of Friendship, Commerce and Navigation of 1956, and that it had committed acts such to deprive that treaty of its object and purpose.

    \vspace{\baselineskip}
    
    It decided that the United States was under a duty immediately to cease and to refrain from all acts constituting breaches of its legal obligations, and that it must make reparation for all injury caused to Nicaragua by the breaches of obligations under customary international law and the 1956 Treaty, the amount of that reparation to be fixed in subsequent proceedings if the Parties were unable to reach agreement. The Court subsequently fixed, by an Order, time-limits for the filing of written pleadings by the Parties on the matter of the form and amount of reparation, and the Memorial of Nicaragua was filed on 29 March 1988, while the United States maintained its refusal to take part in the case. In September 1991, Nicaragua informed the Court, inter alia, that it did not wish to continue the proceedings. The United States told the Court that it welcomed the discontinuance and, by an Order of the President dated 26 September 1991, the case was removed from the Court's List.

    \tcblower

    \flushleft
    In this case, the ICJ affirmed that to give rise to a custom, state practice does not need to be ``perfect, in the Sense that States should have refrained, with complete consistency, from the use of force of from intervention in each other's internal affairs''. Here, the norms relied upon by Nicaragua were part of customary international law, which had a separate applicability to the \textit{Charter of the United Nations}. Additionally, if a treaty gives rise to custom, the custom exists independently of the treaty. Moreover, if it has been pointed out that there has been a number of instances of states contravening a treaty, the courts have held that there does not need to be consistently correct conduct from the states to affirm the custom, and that some variation in practice is acceptable, especially when these variations are treated as breaches of the rule rather than emergence of a new rule (thereby affirming the existing rule).

    \begin{longtable}{r|>{\raggedright\arraybackslash}p{0.95\textwidth}}
        [186] & The Court does not consider that, for a rule to be established as customary, the corresponding practice must be in absolutely rigorous conformity with the rule. In order to deduce the existence of customary rules, the Court deems it sufficient that the conduct of States should, in general, be consistent with such rules, and that instances of State conduct inconsistent with a given rule should generally have been treated as breaches of that rule, not as indications of the recognition of a new rule. If a State acts in a way \textit{prima facie} incompatible with a recognized rule, but defends its conduct by appealing to exceptions or justifications contained within the rule itself, then whether or not the State's conduct is in fact justifiable on that basis, the significance of that attitude is to confirm rather than to weaken the rule. 
    \end{longtable}    
\end{casedetails}

\begin{casedetails}{\textit{Nicaragua v Colombia} [2023] ICJ Rep 413}
    \flushleft

    The Court addressed Nicaragua's request to define the maritime boundary in areas beyond the 2012 Judgment's limits, focusing on two key legal questions posed in its October 4, 2022 Order. The primary question was whether, under customary international law, a State's entitlement to an extended continental shelf beyond 200 nautical miles could extend within 200 nautical miles of another State's baselines. The Court concluded that it could not, based on the interrelationship between the exclusive economic zone and continental shelf regimes under customary law, as reflected in UNCLOS, and widespread State practice showing opinio juris against such overlap. Consequently, the Court rejected Nicaragua's submissions for delimiting overlapping continental shelf areas with Colombia's mainland and islands (San Andrés, Providencia, Serranilla, Bajo Nuevo, and Serrana), finding no overlapping entitlements to delimit, thus rendering the second question on criteria for outer limits unnecessary to address. The decision, supported by a majority of thirteen to four votes on key points, reaffirmed Colombia's maritime entitlements within 200 nautical miles and upheld the 2012 Judgment's findings, dismissing Nicaragua's claims without needing further proceedings.

    \tcblower
    \flushleft

    Even though Colombia is not a party to the \textit{UN Convention on the Law of the Sea}, the ICJ held it still provided key evidence of custom, even though Colombia wasn't a party to it. The practices under it were ``indicative of \textit{opinio juris}'', even if such practice may have been motivated in part by considerations other than a sense of legal obligation".
\end{casedetails}

\subsection{Regional Customary International Law}
\begin{casedetails}{\textit{Asylum Case (Colombia v Peru)} [1950] ICJ Rep 226}
    \flushleft
    Colombia granted de la Torre (who was the head of an unsuccessful revolutionary group in Peru) political asylum in the Colombian Embassy in Lima, Peru. Colombia invoked `American international law' to allow it to grant political asylum (referring to the Americas as a continent, not the United States); this supposed custom was that a unilateral decision to hold something was politically motivated was sufficient. The ICJ held that there was insufficient evidence of such regional customary norm, as such a practice had too much contradiction and fluctuation to be a regional standard. The ICJ concluded that regional standards need a higher standard of stability and continuity to apply as international law.
\end{casedetails}

\begin{casedetails}{\textit{R (app. Al-Saadoon v Sec. of Defence)} [2010] 1 All ER 271}\label{case:R (Al-Saadoon v Sec. of Defence)}
    \flushleft
    In this case, there was a serious risk that the plaintiff would face death at the hands of the Iraqi system if they were deported. The question at hand was whether an obligation of non-refoulement (non-return) to countries where the death penalty available as a rule of regional customary international law in Europe? There is a concept of regional customary international law that bound the UK and other states in the Council of Europe that prevented a European state from transferring a person to a third state where the death penalty was a possibility. The English Court of Appeal found that this had not been established in this instance by the materials cited (including the European Union Charter of Fundamental Rights). The Court accepted that there could be such a rule of regional customary international law, but on the evidence presented, this rule had not been established, and the relevant elements invoked by the claimants did not establish this rule of regional custom.

\end{casedetails}

\subsection{Persistent Objection}
\begin{itemize}
    \item This is a fairly narrow doctrine under customary international law
    \item States which consistently object to the emergence of a rule of custom from its earliest point of gestation will not be bound by this custom should the rule emerge; otherwise, they will be bound to it
    \item A state cannot be a persistent objector to a \textit{jus cogens} principle, following the International Law Commission's 2019 report, at Conclusion 14 of Chapter V
\end{itemize}

\begin{conventiondetails}{\textit{International Law Commission 2019 Report} Chapter V Conclusion 14}\label{report:2019 ILC Conc. 14}
    \flushleft
    \textbf{Rules of customary international law conflicting with a peremptory norm of general international law (\textit{jus cogens})}

    \begin{enumerate}
        \item A rule of customary international law does not come into existence if it conflicts with a peremptory norm of general international law (\textit{jus cogens}). This is without prejudice to the possible modification of a peremptory norm of general international law (\textit{jus cogens}) by a subsequent norm of general international law having the same character.
        \item A rule of customary international law not of a peremptory character ceases to exist if and to the extent that it conflicts with a new peremptory norm of general international law (\textit{jus cogens}).
        \item The persistent objector rule does not apply to peremptory norms of general international law (\textit{jus cogens}).
    \end{enumerate}
\end{conventiondetails}

\begin{casedetails}{\textit{Anglo Norwegian Fisheries Case (UK v Norway)} [1951] ICJ Rep 116}\label{case:UK v Norway Fisheries}
    \flushleft
    To determine its coastal baselines, Norway drew a system of straight baselines. The UK objected to Norway's straight baselines, as Norway had suddenly closed the waters that were open to British fishing vessels (the regular practice was to not have a system of straight baselines). The question before the ICJ was whether there was a rule of custom prohibiting baselines more than 10 nautical miles in length. The ICJ held that there was no such rule, but even it did exist, Norway was a persistent objector (and so even if it did exist, it wouldn't apply to Norway). Accordingly, the UK lost.
\end{casedetails}

\section{General Principles of Law}
\begin{itemize}
    \item Under Art 38(1)(c) of the \textit{Statute of the International Court of Justice}, general principles of law recognised by civilised nations form a source of public international law
    \item The objective of including the general principles of law is to avoid the \textit{non liquet} (the situation where `it is not clear')
    \item This includes general principles of both international law and municipal law
    \item Examples of this include \textit{res judicata} (the principle of finality, holding that once a case is decided, it is final and cannot be relitigated), the principle that a breach of an obligation is accompanied by an obligation to make reparations, and the principles of acquiescence and estoppel
\end{itemize}

\begin{casedetails}{\textit{Bay of Bengal (Bangladesh/Myanmar)} [2012] ILTOS 12}
    \flushleft

    This judgement by the International Tribunal for the Law of the Sea addressed the delineation of maritime zones (territorial sea, exclusive economic zone (EEZ), and continental shelf) between the two states. The Tribunal, affirming its jurisdiction under the \textit{United Nations Convention on the Law of the Sea} (UNCLOS), rejected Bangladesh's claim that the 1974 and 2008 Agreed Minutes constituted a binding agreement for the territorial sea, finding them non-binding due to their conditional nature and lack of formal approval, resulting in there being no estoppel. It delimited the territorial sea using equidistance adjusted for St. Martin's Island, and for the EEZ and continental shelf within 200 nautical miles (nm), it applied a provisional equidistance line adjusted for Bangladesh's concave coast to avoid a cut-off effect. The Tribunal also asserted jurisdiction over the continental shelf beyond 200 nm, delimiting it based on geological entitlement and equity, resulting in a single maritime boundary, though creating a ``grey area" where Bangladesh's continental shelf overlapped Myanmar's EEZ, which it left unresolved for future negotiation.

    \vspace{\baselineskip}

    Regarding Article 38(1)(c), the ITLOS Judgment implicitly engaged such principles, particularly equity, in its delimitation process. While the Judgment primarily applied UNCLOS provisions (Articles 15, 74, 83, and 76), the Tribunal's adjustment of the equidistance line to achieve an "equitable solution", notably to mitigate the cut-off effect of Bangladesh's concave coast, reflects the general principle of equity, a concept widely accepted across legal systems and frequently invoked in maritime delimitation (e.g., \textit{North Sea Continental Shelf} cases). The dissenting opinion of Judge Lucky explicitly references Article 38 in the context of Articles 74 and 83, advocating the angle-bisector method over equidistance to ensure fairness, underscoring equity \textit{infra legem} as a method to interpret and apply the law justly. Thus, the Judgment's reliance on equity to balance the parties' rights demonstrates how general principles under Article 38(1)(c) supplement treaty law in achieving a fair outcome specific to this case.

    \begin{longtable}{r|>{\raggedright\arraybackslash}p{0.95\textwidth}}
        [124] & The Tribunal observes that, in international law, a situation of estoppel exists when a State, by its conduct, has created the appearance of a particular situation and another State, relying on such conduct in good faith, has acted or abstained from an action to its detriment. The effect of the notion of estoppel is that a State is precluded, by its conduct, from asserting that it did not agree to, or recognize, a certain situation. \\[2.5cm]
        [125] & In the view of the Tribunal, the evidence submitted by Bangladesh to demonstrate that the Parties have administered their waters in accordance with the  limits set forth in the 1974 Agreed Minutes is not conclusive. There is no indication that Myanmar's conduct caused Bangladesh to change its position to its  detriment or suffer some prejudice in reliance on such conduct. For these reasons, the Tribunal finds that Bangladesh's claim of estoppel cannot be upheld.
    \end{longtable}    
\end{casedetails}

\begin{casedetails}{\textit{Chagos Marine Protected Area Arbitration (Mauritius v United Kingdom)} (2015) XXXI RIAA 359}
    \flushleft
    The arbitral tribunal, constituted under Annex VII of the United Nations Convention on the Law of the Sea (UNCLOS), addressed a dispute between Mauritius and the United Kingdom (UK) concerning the UK's establishment of a Marine Protected Area (MPA) around the Chagos Archipelago on 1 April 2010. Mauritius argued that the UK, as the administering power of the British Indian Ocean Territory (BIOT), lacked the authority to unilaterally declare the MPA, violating UNCLOS and international law by disregarding Mauritius' rights, including fishing rights and the UK's undertakings to return the Archipelago and share resource benefits when no longer needed for defense purposes. The Tribunal found it lacked jurisdiction over Mauritius' sovereignty claims (First and Second Submissions) and a related dispute (Third Submission), but unanimously asserted jurisdiction over the Fourth Submission, concluding that the UK breached Articles 2(3), 56(2), and 194(4) of UNCLOS due to insufficient consultation and failure to balance Mauritius' rights, rendering the MPA's declaration incompatible with the Convention. The Tribunal emphasised procedural inadequacies rather than the MPA's environmental merits, urging further negotiations, and ordered costs to be borne equally by the parties.

    \vspace{\baselineskip}

    Here, the Tribunal's interpretation of UNCLOS provisions, such as Article 2(3), relied on general principles like good faith and due regard, which are widely accepted across legal systems and reflect fundamental norms ensuring equitable conduct between states. These principles, derived from domestic legal traditions and adapted to the international context, served to evaluate the UK's obligations to consult and balance Mauritius' rights, demonstrating their role as a gap-filling mechanism where treaty or customary rules are ambiguous or silent. The Tribunal's reference to good faith in Article 2(3) and the balancing requirement in Article 56(2) underscores how general principles, as per Article 38(1)(c), provide a flexible yet authoritative basis for resolving disputes, reinforcing the coherence and fairness of international legal obligations beyond specific treaty terms.

    \begin{longtable}{r|>{\raggedright\arraybackslash}p{0.95\textwidth}}
        [438] & Further to this jurisprudence, estoppel may be invoked where (a) a State has made clear and consistent representations, by word, conduct, or silence; (b) such representations were made through an agent authorized to speak for the State with respect to the matter in question; (c) the State invoking estoppel was induced by such representations to act to its detriment, to suffer a prejudice, or to convey a benefit upon the representing State; and (d) such reliance was legitimate, as the representation was one on which that State was entitled to rely.
    \end{longtable}  

\end{casedetails}

\section{Judicial Decisions and the Teaching of Publicists}
\begin{itemize}
    \item Under Art 38(1)(d) of the \textit{Statute of the International Court of Justice}, judicial decisions taken at both a domestic and an international level, and the teaching of publicists can be considered as sources of public international law
    \item However, these are `subsidiary means' for the determination of rules of law, and are treated as having lesser significance than other sources
    \item Publicists generally constitute academics who are distinguished in the field, and probably have been dead for a long period of time
    \item Decisions taken by the ICJ do not constitute binding precedent in future decisions, and remain merely persuasive, following Art 59 of the \textit{Statute of the International Court of Justice}
    \item It has been held that these other sources are ``resorted to by judicial tribunals not for the speculations of their authors concerning what the law ought to be, but for trustworthy evidence of what the law really is", per \case{\textit{The Paquete Habana} 175 US 677 (1900)}
\end{itemize}
\begin{statutedetails}{\textit{Statute of the International Court of Justice} Art 59}
    \flushleft
    The decision of the Court has no binding force except between the parties and in respect of that particular case.
\end{statutedetails}

\begin{casedetails}{\textit{The Paquete Habana} 175 US 677 (1900) (United State Supreme Court)}
    \flushleft
    The U.S. Supreme Court reviewed the capture of two Spanish fishing vessels, the Paquete Habana and the Lola, by U.S. naval forces during the Spanish-American War. Both vessels, owned by Spanish subjects in Havana and crewed by Cuban fishermen, were engaged in coast fishing off Cuba and Yucatan, carrying live fish caught by their crews. Captured in April 1898 near Havana by U.S. blockading ships, they were unarmed, unaware of the war or blockade, and made no attempt to resist or aid the enemy. The District Court for the Southern District of Florida condemned them as prizes of war on May 30, 1898, selling them for \$490 and \$800, respectively, asserting no legal exemption existed without a treaty or proclamation. The Supreme Court reversed this, finding their capture unlawful under international law, which exempts coast fishing vessels pursuing a peaceful trade from war prizes, and ordered restitution with compensatory damages.

    \vspace{\baselineskip}

    This decision illustrates how customary international law integrates with other legal sources when treaties or domestic acts are absent. It ruled that the exemption of coast fishing vessels is an established rule of customary international law, derived from the consistent practice and \textit{opinio juris} of civilised nations, evidenced by historical treaties (e.g., 1521 Charles V-Francis I treaty), state practice (e.g., U.S. in the Mexican War), and jurists' writings. Absent a controlling treaty, executive order, or statute (none of which existed here), the Court relied on this custom, distinguishing it from the non-binding UNGA resolutions in the 1996 ICJ Nuclear Weapons case, which lacked sufficient state practice to form custom. The decision aligns with treaty-based exemptions, but asserts judicial authority to enforce customary norms directly. This case therefore underscores custom's enforceability in U.S. courts, complementing treaties and executive discretion in wartime.

    \begin{longtable}{r|>{\raggedright\arraybackslash}p{0.9\textwidth}}
        Pg. 175 & International law is part of our law, and must be ascertained and administered by the courts of justice of appropriate jurisdiction as often as questions of right depending upon it are duly presented for their determination. For this purpose, where there is no treaty and no controlling executive or legislative act or judicial decision, resort must be had to the customs and usages of civilized nations, and, as evidence of these, to the works of jurists and commentators who by years of labor, research, and experience have made themselves peculiarly well acquainted with the subjects of which they treat. Such works are resorted to by judicial tribunals not for the speculations of their authors concerning what the law ought to be, but for trustworthy evidence of what the law really is.
    \end{longtable}
\end{casedetails}

\subsection{United Nations General Assembly Resolutions}
\begin{itemize}
    \item The United Nations General Assembly (UNGA) is the plenary body of the UN, generating a large amount of documents, of which the most important are the UNGA Resolutions, since:
    \begin{itemize}
        \item All UN members have a seat and can thus contribute to the formation of these resolutions
        \item The UNGA has many different capacities, and can adopt different decisions (however, these are recommendatory, and not legally binding)
        \item The UNGA has generally influenced PIL as it is a great forum for state practice and \textit{opinio juris}
    \end{itemize}
    \item Decisions of the UNGA are not binding, except in the key areas of admission of member states, suspension of member states, and matters related to the UN budget (if these wre not binding, the UN would not be able to function)
    \item These resolutions provide evidence on the state of customary international law, as it is a great forum to evidence what states are doing
    \item The UNGA can also be far more responsive than the traditional case-by-case process of implementing customary international law, and ultimately serves to advance the norms of international law
    \item UNGA resolutions can influence international law in three main ways:
    \begin{enumerate}
        \item Interpreting the \textit{Charter of the United Nations}
        \item Affirming recognised customary norms (this is done by a resolution of the UNGA)
        \item Influencing the creation of new customary norms (e.g., a resolution can be the spark that creates a new customary norm)
    \end{enumerate}
    \item Furthermore, it has been held in \case{\textit{Legality of the Threat or Use of Nuclear Weapons} [1996] ICJ Rep 254} that UNGA resolutions may ``sometimes have normative value'' (at [70]), and can provide ``evidence important for the establishing the existence of a rule or the emergence of a \textit{opinio juris}'' (at [70])
    \item Such evidence can include:
    \begin{itemize}
        \item The voting records of the UNGA
        \item Transcripts of what was said on the floor of the UNGA
        \item Margins of the votes undertaken in the UNGA
    \end{itemize}
\end{itemize}

\begin{casedetails}{\textit{Legality of the Threat or Use of Nuclear Weapons} [1996] ICJ Rep 254}\label{case:Legality of Nuclear Weapons [1996] ICJ Rep 254}
    \flushleft
    
    The Court was asked whether the threat or use of nuclear weapons was permitted under international law. They found no specific authorisation or comprehensive prohibition of nuclear weapons in customary or conventional international law. It ruled that any such threat or use must comply with the UN Charter, prohibiting unlawful force (Article 2(4)) and regulating self-defense (Article 51), and international humanitarian law (IHL), which requires distinguishing between combatants and civilians and avoiding unnecessary suffering. While the Court concluded that nuclear weapons' indiscriminate effects would ``generally" violate IHL, it could not definitively rule on their legality in extreme self-defense scenarios threatening a state's survival. It unanimously affirmed an obligation under Article VI of the Non-Proliferation Treaty (NPT) to pursue nuclear disarmament in good faith.

    \vspace{\baselineskip}
    
    The ICJ clarified the role of UNGA resolutions in international law, particularly in the context of nuclear weapons. \textbf{Resolutions are not legally binding on their own but may have normative value as evidence of customary law if supported by state practice and \textit{opinio juris} ([70]-[73]).} The Court found that these resolutions, despite large majorities, did not establish a customary prohibition due to opposition from nuclear states, abstentions, and the lack of consistent practice, reflecting a divide between emerging \textit{opinio juris} and the deterrence policy adhered to by some states. They signal deep concern and a desire for a ban, but alone, they fall short of creating a legal rule.

    \vspace{\baselineskip}
    
    The ICJ's analysis underscores that \textbf{UNGA resolutions complement, rather than independently create, binding norms}. Their significance depends on content, adoption conditions, and state acceptance, but in this case, they did not overcome the absence of universal consensus ([70] - [71]). In contrast, the NPT's Article VI imposes a clear legal duty on its 182 parties to negotiate disarmament, reinforced by UNGA resolutions but distinct in its binding force ([99] - [103]). Thus, while UNGA resolutions highlight an evolving legal consciousness and support treaty obligations, they were insufficient in 1996 to resolve the legality of nuclear weapons definitively, illustrating the Court's cautious approach to law-making based solely on such instruments.

    \begin{longtable}{r|>{\raggedright\arraybackslash}p{0.9\textwidth}}
        [70] & The Court notes that General Assembly resolutions, even if they are not binding, may sometimes have normative value. They can, in certain circumstances, provide evidence important for establishing the existence of a rule or the emergence of an \textit{opinio juris}. To establish whether this is true of a given General Assembly resolution, it is necessary to look at its content and the conditions of its adoption; it is also necessary to see whether an \textit{opinio juris} exists as to its normative character. Or a series of resolutions may show the gradual evolution of the \textit{opinio juris} required for the establishment of a new rule. 
    \end{longtable}
\end{casedetails}

\subsection{UN Security Council}
\begin{itemize}
    \item The UN Security Council can adopt a direct role in international law making (e.g., following the September 11 attacks, Resolution 1373 was deemed a form of `international legislation')
    \item However, the UN Security Council has limited law-making capacity, and can adopt certain binding resolutions, but these may have expedited impacts
    \begin{itemize}
        \item Under art 25 of the \textit{Charter of the United Nations}, these resolutions are only binding on members of the UN
    \end{itemize}
\end{itemize}

\begin{conventiondetails}{\textit{Charter of the United Nations} Art 25}\label{UN Charter Art 25}
    \flushleft
    The Members of the United Nations agree to accept and carry out the decisions of the Security Council in accordance with the present Charter.
\end{conventiondetails}

\section{Soft Law}
\begin{itemize}
    \item Soft law refers to rules that are binding but vague, and/or `rules' that are clear but not binding
    \item They serve as a convenient encompassment of a variety of non-legally binding instruments used in contemporary international relation
    \item Whilst soft law instruments are not in and of themselves legally binding, they can articular standards or norms that will, over time, become concrete and be transformed into international law
    \item They can also be used to interpret other sources of international law (e.g., treaty or custom)
    \item An example is the precautionary principle, which is central to international environmental law:
    \begin{itemize}
        \item These constitute cost-effective measures to protect the environment, with their implementation to not be delayed whilst there is uncertainty to their efficacy (i.e., protect the environment now rather than wait for complete certainty)
        \item This was articulated in 1992 in the United Nations General Assembly, and can now be found in different areas of international law (an example of soft law becoming hard law over time)
    \end{itemize}
\end{itemize}