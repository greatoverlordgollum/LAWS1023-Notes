\begin{itemize}
    \item This topic refers to the use of force between states
    \item It is not a straightforward area of international law, but there are developed rules around how to regulate force, and how to deal with defensive force when force happens to be used
    \item International law imposes prohibitions and/or restrictions on the use of force, subject to certain exceptions
\end{itemize}

\section{Use of Armed Force in International Law}
\begin{itemize}
    \item There are two `areas' of international law that deal with the use of armed force:
    \begin{itemize}
        \item \textit{Jus ad bellum}, which is the law relating to whether there is a right to use force
        \begin{itemize}
            \item This domain has a very long history, and seeks to minimise the use of force to only those situations where it is absolutely necessary
        \end{itemize}
        \item \textit{Jus in bello}, which refers to the law governing the rules of combat
        \begin{itemize}
            \item This set of rules refers to what can be done once hostilities have commenced
            \item It is agnostic as to why a conflict started, but instead civilises an existing conflict by regulating what objects can be targeted, what weapons can be used, etc.
        \end{itemize}
    \end{itemize}
    \item The `Just War' tradition has influenced the development of international law relating to the use of force
    \begin{itemize}
        \item There are seven just war criteria:
        \begin{enumerate}
            \item Just cause
            \item Right authority
            \item Right intention
            \item Proportionality
            \item Last resort
            \item Hope of success
            \item Aim of peace
        \end{enumerate}
        \item These criteria must be satisfied before a state can wage war upon another state, and have all influenced contemporary international law relating to armed force
        \item This also connects with `humanitarian intervention' and the `responsibility to protect', which is a notion that has been adopted by the UN to describe situations where states may be authorised by the UN to intervene in another state to prevent an overwhelming humanitarian crisis
    \end{itemize}
\end{itemize}

\section{Prohibition on the Use of Force}
\subsection{Historical Development of the Law on the Use of Force}
\begin{itemize}
    \item At the beginning of the 20th century, the use of force was regarded as lawful, irrespective of the purpose for which it was used (e.g., conquest of territory, repayment of debts, etc.)
    \item Over the 20\textsuperscript{th} century, the law around state force was developed, slowly restricting 
    \item In 1919, the League of Nations sought to moderate the use of force
    \begin{itemize}
        \item `Rupture' was coded language for a `hot war'
        \item This convention sought to hold off on physical conflict until three months after arbitration had been attempted, thereby acting as a cooling-off period, but not prohibiting war altogether
    \end{itemize}
\end{itemize}

\begin{conventiondetails}{\textit{1919 Covenant of the League of Nations} Article 12}
    \flushleft
    The Members of the League agree that, if there should arise between them any dispute likely to lead to a rupture they will submit the matter either to arbitration or judicial settlement or to enquiry by the Council, and they agree in no case to resort to war until three months after the award by the arbitrators or the judicial decision, or the report by the Council.
\end{conventiondetails}

\begin{itemize}
    \item The \convention{\textit{1928 General Treaty for the Renunciation of War} (Kellogg-Briand Pact)} was a clear statement in the law that war should no longer be used as a means for undertaking conflict resolution
    \begin{itemize}
        \item Article 1 held that ``[t]he High Contracting Parties solemnly declare…that they condemn recourse to war for the solution of international controversies, and renounce it as an instrument of national policy in their relations with one another.''
        \item However, it is not clear if this amounts to a prohibition on the use of force in general (including `armed reprisals'), or only if it outlawed full-scale war
        \begin{itemize}
            \item The use of the term `war' suggests that it only prohibits full-scale war, and not lesser uses of force, such as armed reprisals (e.g., retaliation)
        \end{itemize}
    \end{itemize}
    \item The \convention{\textit{1945 Charter of the United Nations}} is the most important source of law on the use of force, and encapsulates an all-encompassing prohibition on the use of force, with only a few exceptions
    \begin{itemize}
        \item The central purpose of the \convention{\textit{Charter}} is to prevent armed conflict, by establishing a system of collective security in the aftermath of WWII and the Holocaust
    \end{itemize}
\end{itemize}

\subsection{Charter of the United Nations, and the Present Law}

\begin{conventiondetails}{\textit{1945 Charter of the United Nations} Preamble (First Recital)}
    \flushleft
    To save succeeding generations from the scourge of war, which twice in our lifetime has brought untold sorrow to mankind.
\end{conventiondetails}

\begin{conventiondetails}{\textit{1945 Charter of the United Nations} Article 2(4)}
    \flushleft
    All Members shall refrain in their international relations from the threat or use of force against the territorial integrity or political independence of any state, or in any other manner inconsistent with the Purposes of the United Nations.
\end{conventiondetails}

\begin{itemize}
    \item \convention{Art 2(4)} is the central provision relating to the use of force in the \convention{\textit{UN Charter}}
    \item The \convention{\textit{Charter}} is a treaty, and therefore binding on all states that have ratified it
    \item It is also now a customary norm of international law (and hence is also binding on non-UN members), as well as being a \gls{jus cogens} norm, per \case{\textit{Nicaragua} (1968)}
    \begin{itemize}
        \item Since the \convention{\textit{UN Charter}} prevails over other treaties, it is regarded as having `quasi-statal' status
        \item \convention{Article 2(4)} has also been regarded as a `cornerstone of the United Nations Charter' in \case{\textit{Armed Activities (DRC v Uganda)} (2005)}
    \end{itemize}
    \item There are two recognised exceptions to \convention{Article 2(4)}:
    \begin{itemize}
        \item Self-defence, as set out in \convention{\textit{UN Charter} Article 51}, which is the inherent right of a state that can be exercised in response to an armed attack (which upholds the unilateral right of a state's independence)
        \item Collective security, as set out in \convention{\textit{UN Charter} Chapter VII}, which refers to the right of the UN Security Council to authorise the use of force in response to threats to international peace and security in the interests of collective security
    \end{itemize}
    \item There is relatively little state practice on the meaning of \convention{Article 2(4)}, but since it refers to the `threat or use of force' rather than `war', it has been interpreted to cover all threats or uses of armed force, and also covers uses of force that fall short of full-scaled armed conflict
    \item However, \convention{Article 2(4)} only applies to inter-state force (`in their international relations'; i.e., not intra-state force), and as such, does not prohibit civil war within a country (although other areas of international law may do so)
    \item \convention{Article 2(4)} moreover prohibits the threat or use of force `against the territorial integrity or political independence of any state', and applies to `any state' (not just UN members)
    \begin{itemize}
        \item The prevailing view, supported by the \textit{travaux préparatoires}, is that these are specific examples that elaborate and illuminate what falls within the prohibition on the use of force
    \end{itemize}
    \item Relevant aspects of the \convention{\textit{UN Charter}} use different terms:
    \begin{itemize}
        \item Article 2(4) - prohibition of `the use of force'
        \item Article 51 - self-defence against `in armed attack'
        \item Article 39 - Security Council may determine there has been a `threat to the peace, breach of the peace, or act of aggression'
    \end{itemize}
    \item However, none of these terms are defined in the \convention{\textit{UN Charter}}
    \item As such, there are several resolutions of the UN General Assembly that are aimed at clarifying the scope of the prohibition:
    \begin{itemize}
        \item \convention{\textit{Declaration on Friendly Relations} 1970 (Resolution 2625)}
        \item \convention{\textit{Definition of Aggression 1974} (Res 3314)}
        \item \convention{\textit{Declaration on the Non-Use of Force 1987} (Res 42/22)}
    \end{itemize}
\end{itemize}

\subsubsection{Declaration on Friendly Relations 1970}
\begin{itemize}
    \item This is a declaration that reaffirms the core of the \convention{\textit{UN Charter}}; Principle 1 of the declaration reaffirms the prohibition on the threat or use of force against any state as set out in the \convention{\textit{UN Charter}}
\end{itemize}

\begin{conventiondetails}{\textit{Declaration on Friendly Relations 1970} (Res 2625) Principle 1}
    \flushleft
    States shall refrain in their international relations from the threat or use of force against the territorial integrity or political independence of any State, or in any other manner inconsistent with the purposes of the United Nations.

    \vspace{\baselineskip}

    ...

    \vspace{\baselineskip}

    ... organising or encouraging the organisation of irregular forces or armed bands, including mercenaries, for incursion into the territory of another State.

    \vspace{\baselineskip}

    ...

    \vspace{\baselineskip}

    organising, instigating, assisting or participating in acts of civil strife or terrorist acts in another State or acquiescing in organised activities within its directed towards the commission of such acts, when the acts ... involve a threat or use of force.
\end{conventiondetails}

\begin{itemize}
    \item The second aspect of Principle 1 above holds that the prohibition on the use of force encompasses acts conducted by paramilitary or irregular forces (mercenaries)
    \item This is not present in the \convention{\textit{UN Charter}}, and so the Declaration serves to flesh out the meaning of the charter
    \item Moreover, the Declaration clarifies that state-sanctioned terrorism isn't permitted under \convention{\textit{UN Charter} Art 2(4)}
    \item The Declaration has been held to reflect customary international law, under \case{\textit{Nicaragua} (1986) [188], [191]} and \case{\textit{Armed Activities} (2005) [162], [300]}
    \item Note that whilst resolutions of the UN General Assembly are generally not binding, this one is as it has evolved into being a reflection of customary international law
\end{itemize}

\subsubsection{Resolution on Definition of Aggression 1974}
\begin{conventiondetails}{\textit{Resolution on Definition of Aggression 1974} Article 1}
    \flushleft
    Aggression is the use of armed force by a State against the sovereignty, territorial integrity or political independence of another State, or in any other manner inconsistent with the Charter of the United Nations, as set out in this Definition.
\end{conventiondetails}

\begin{itemize}
    \item This resolution picks up most of the acts prohibited under \convention{\textit{UN Charter} Article 2(4)}
    \item Article 3 outlines a non-exhaustive list of acts that constitute aggression
\end{itemize}

\begin{conventiondetails}{\textit{Resolution on Definition of Aggression 1974} Article 3}
    \flushleft
    Any of the following acts, regardless of a declaration of war, shall, subject to and in accordance with the provisions of article 2, qualify as an act of aggression:

    \begin{enumerate}[label=(\alph*)]
        \item The invasion or attack by the armed forces of a State of the territory of another State, or any military occupation, however temporary, resulting from such invasion or attack, or any annexation by the use of force of the territory of an- other State or part thereof;
        \item Bombardment by the armed forces of a State against the territory of another State or the use of any weapons by a State against the territory of another State;
        \item The blockade of the ports or coasts of a State by the armed forces of another State;
        \item An attack by the armed forces of a State on the land, sea or air forces, or marine and air fleets of another State;
        \item The use of armed forces of one State which are within the territory of another State with the agreement of the receiving State, in contravention of the conditions provided for in the agreement or any extension of their presence in such territory beyond the termination of the agreement;
        \item The action of a State in allowing its territory, which it has placed at the disposal of another State, 10 be used by that other State for perpetrating an act of aggression against a third State;
        \item The sending by or on behalf of a State of armed bands, groups, irregulars or mercenaries, which carry out acts of armed force against another State of such gravity as to amount to the acts listed above, or its substantial involvement therein.
    \end{enumerate}
\end{conventiondetails}

\begin{itemize}
    \item The Resolution reflects customary international law, as per \case{\textit{Nicaragua} (1986), [195]} and \case{\textit{Armed Activities} (2005), [146]}
\end{itemize}

\subsection{Use of Force}
\begin{itemize}
    \item A use of force, prohibited under \convention{\textit{UN Charter} Article 2(4)}, refers to the use, by one state against another, of armed force (not political or economic pressure)
    \begin{itemize}
        \item Whilst this force is generally kinetic (e.g., firing weapons), it is possible that non-kinetic forces (e.g., cyber-attacks) can also amount to the use of force under the Charter, although this is currently unclear
        \item Political and/or economic pressure will be prohibited under the Charter if it rises to a very serious level such that it constitutes an unlawful intervention in another country, but is otherwise not an issue
    \end{itemize}
    \item Examples of direct armed force include invasion, missile attacks, the laying or mines, etc.
    \item Examples of indirect armed force include:
    \begin{itemize}
        \item Sending `armed bands' (irregular forces) into another state's territory, following \case{\textit{Nicaragua} (1986), [195], [247]}
        \item `Actively extending military, logistic, economic and financial support to irregular forces' (so long as it is directed to the war-fighting capacity of a state), following \case{\textit{Armed Activities} (2005), [161]-[165]}
        \item Providing weapons, logistical and/or other other support to armed insurgents, following \case{\textit{Nicaragua} (1986), [195], [205], [247], [251]}
    \end{itemize}
    \item The `mere supply of funds' to irregular forces does \textit{not} constitute a use of force, following \case{\textit{Nicaragua} (1986), [228]}
\end{itemize}

\subsection{Threat of Force}
\begin{itemize}
    \item A threat of force is a threat to use armed force against another state, and is also prohibited under \convention{\textit{UN Charter} Article 2(4)}
    \item It was held in \case{\textit{Nuclear Weapons Advisory Opinion} (1996), [47]} that where the use of force would be illegal, the threat to use such force is also unlawful
    \item An example of an unlawful threat of force is an ultimatum to attack a state if it does not comply
    \item \case{\textit{Nicaragua} (1986), [227] held that US military manoeuvres near the Nicaraguan border did not constitute a threat of force}
    \item The UK has argued that the aiming of Iraqi artillery and tanks at Kuwait did constitute a threat of force, but this was rejected by the ICJ in \case{\textit{Nicaragua} (1986)}
    \item \case{\textit{Nicaragua} (1986)} has held that an accumulation of forces can constitute a threat of force, depending on its accompaniments, such as statement of intent, relevant exercises, etc.
\end{itemize}

\section{Self-Defence}

\begin{conventiondetails}{\textit{UN Charter} Article 51}
    \flushleft
    Nothing in the present Charter shall impair the inherent right of individual or collective self-defence if an armed attack occurs against a Member of the United Nations, until the Security Council has taken measures necessary to maintain international peace and security. Measures taken by Members in the exercise of this right of self-defence shall be immediately reported to the Security Council and shall not in any way affect the authority and responsibility of the Security Council under the present Charter to take at any time such action as it deems necessary in order to maintain or restore international peace and security.
\end{conventiondetails}

\begin{itemize}
    \item `Inherent right' is a reference to and codification of the right of self-defence at customary international law
    \item `Armed attack' does not have a clear definition, but it has been taken to mean a grave use of force beyond the ordinary use of force
    \item `Until the Security Council has taken measures' acts to preserve the right of self-defence on a temporary basis until the UN Security Council is able to take action
\end{itemize}

\subsection{Armed Attack}
\begin{itemize}
    \item An armed attack concerns the `most grave forms of the use of force', which is distinguished from `other less grave forms', in \case{\textit{Nicaragua} (1986), [191]} (approved in \case{\textit{Oil Platforms} (1984)}, [51])
    \item Unless the force is grave, there will be no right to self-defence; this does have the effect of a state possibly having to endure an unlawful use of force whilst being deprived of the right to respond by way of self-defence until it becomes grave enough to amount to an armed attack (international law seeks to restrain the use of force where it can, and so this is a policy reason)
    \item Whether an action will be an `armed attack' will depend on the `scale and effects' of the operation, as per \case{\textit{Nicaragua} (1986), [195]}
    \item As such, an operation can be a use of force contrary to Article 2(4) of the \convention{\textit{UN Charter}}, but not an armed attack, and hence not a use of force that can be responded to by way of self-defence, per \case{\textit{Nicaragua} (1986), [191]} and \case{\textit{Oil Platforms} (2003), [51], [61]}
    \item The ICJ has stated that the placing of mines on a single military vessel might be sufficient to amount to an armed attack, per \case{\textit{Oil Platforms} (2003), [72]}
    \begin{itemize}
        \item Whilst it wasn't an armed attack in that case, the ICJ held that on the scales and effects test, it theoretically could constitute an armed attack
        \item As such, even an isolated incident using force can amount to an armed attack if it is serious enough (for this reason, targeting military vessels is more serious than targeting civilian vessels, as the former is more likely to be an armed attack and the latter is not)
    \end{itemize}
    \item Ugandan military assaults that resulted in the taking of several towns near the DRC/Ugandan border constituted an `armed attack', per \case{\textit{Armed Activities} (2005), [110], [147]}, as a result of the excessive scale and effects of the attacks
    \item An armed attack can be conducted by armed bands, irregulars, mercenaries, etc. sent by on behalf of a state onto another state (i.e., a state sends proxies rather than its own army), per \case{\textit{Nicaragua} (1986), [195]}, citing the \convention{\textit{Resolution on Definition of Aggression 1974} Article 3(g)} as customary international law
    \item An armed attack does \textit{not} include:
    \begin{itemize}
        \item `A mere frontier incident', per \case{\textit{Nicaragua} (1986), [195]}
        \begin{itemize}
            \item This is likely a minor dispute about the border by the armed forces of the two states (the conflict only lasts a few hours or days)
        \end{itemize}
        \item Armed incursions where the victim states did not raise any complaints, per \case{\textit{Nicaragua} (1986), [231]-[234]}
        \begin{itemize}
            \item Here, Nicaragua had undertaken armed incursions into Costa Rica and Honduras, but as neither state complained about these incursions, it was presumed they were not serious enough to amount to an armed attack
        \end{itemize}
        \item Providing assistance to rebels or insurgents in the form of weapons, logistical, or other support, per \case{\textit{Nicaragua} (1986), [195], [230]} (the second paragraph concerns Nicaragua's support of rebels in El Salvador)
        \begin{itemize}
            \item Whilst this is a use of force, it is not an armed attack as it directly was not the most grave use of force 
        \end{itemize}
        \item Even if taken cumulatively, a missile attack, the laying of mines on a ship and then firing on a US ship by Iran did not amount to an armed attack, as there was no evidence that the attacks were aimed specifically at the US, per \case{\textit{Oil Platforms} (2003), [64]}
    \end{itemize}
\end{itemize}

\begin{casedetails}{\textit{Oil Platforms} (Iran v US) (2003)}
    \flushleft
    This case related to a series of incidents between the two states whereby several US flag vessels, one a civilian vessel and one a military vessel, were the victim of (i) a missile attack and (ii) a collision with a sea mine. The issue before the Court was whether or not Iran had engaged in an armed attack on the US which justified the US's response of bombing Iranian oil rigs, with the Court being asked whether there was an armed attack.

    \vspace{\baselineskip}

    An argument made by the US was that all acts had to be observed cumulatively as violations of US interests, which would therefore cumulatively amount to an armed attack (even if one wasn't enough, the combination of them was sufficient). The Court accepted the possibility that an armed attack could be satisfied if there were accumulations of uses of force, but held on these facts that such an accumulation could not be made out. This is because it wasn't clear on the totality of the evidence that Iran's actions were \textit{directed} at the US, especially as Iran was indiscriminately shooting at all vessels.

    \vspace{\baselineskip}

    From this case, it can be inferred that an armed attack generally needs to be targeted at a state, but this requirement is not strict and can be waived if the scale and effects of the attack are severe enough (e.g., nuclear weapons are indiscriminate, and the scale and effects of their use would be so severe that they would amount to an armed attack even if they were not directed at a state).
    
\end{casedetails}

\subsection{Timing of an Armed Attack}
\begin{itemize}
    \item An armed attack does not begin until a state's territory is affected, but it is possible for it to begin when an irreversible course of action has begun (e.g., the launch of missiles)
    \item An exception may lie in the `accumulation of events' theory
    \begin{itemize}
        \item It is often the case that there is not a singular armed attack, but that there is instead a number of armed attacks
        \item This series of attacks should be viewed as a whole, so that action to prevent future attacks in the series is not anticipatory self-defence but is rather self-defence against an ongoing attack, and moreover is not retaliation for past attacks (i.e., the state is not unlawfully retaliating for past attacks but rather is lawfully responding for an accumulation of attacks)
    \end{itemize}
    \item This argument has been implicitly accepted by the ICJ, in \case{\textit{Nicaragua} (1986), [231]}, where it held that it was ``difficult to decide whether [incursions by Nicaragua] may be treated for legal purposes as amounting, singly or collectively, to an armed attack''; this position was supported in \case{\textit{Oil Platforms} (2003), [64]}, and \case{\textit{Armed Activities} (2005), [146]}
\end{itemize}

\subsubsection{Anticipatory Self-Defence}
\begin{itemize}
    \item The ICJ in \case{\textit{Nicaragua} (1986), [194]} and \case{\textit{Armed Activities} (DRC v Uganda) (2005), [143]}, has expressly reserved its opinion around \textbf{anticipatory self-defence}, and as such this question has never been addressed by the ICJ
    \item There is some state practice and some scholarly commentary around the concept that states do not have to wait to be the subject of an attack, but can take action when the attack is imminent so as to protect themselves
\end{itemize}

\subsubsection{Pre-Emptive Self-Defence}
\begin{itemize}
    \item Following 9/11, the \article{\textit{2002 US National Security Strategy}} brought  ``pre-emptive self-defence back into global discussion''
    \item this is generally done in cases of terrorism
    \item Pre-emptive self-defence is based on the idea that armed attacks are not imminent, but it is known that they are coming, and so as an insurance strategy, pre-emptive self-defence is used to disable or reduce the threatening military capacity of the other state before an attack occurs
    \item Arguments in favour of pre-emptive self-defence include the changed nature of threats since 1945 (the rise of weapons of mass destruction and terrorism), the impractical requirement of making states wait for an attack, and the lack of a guarantee that the UN Security Council would respond (the US was in favour of these arguments)
    \item Arguments against pre-emptive self-defence include that it was too vague and arbitrary, there was ambiguity on how to assess necessity and proportionality, and that it may also erode the prohibition on the use of force (this is the view accepted by most states and scholars)
    \item It is generally accepted in state practice that there must be at least an actual or imminent threat of attack to enliven the right of self-defence (i.e., it is not pre-emptive)
\end{itemize}

\subsection{Necessity and Proportionality}
\begin{itemize}
    \item Measures taken in self-defence must be `proportional to the armed attack and necessary to respond to it', per \case{\textit{Nicaragua} (1986), [176]}, with these requirements applying to self-defence under both \convention{\textit{UN Charter} Article 51} and customary international law
    \item Necessity requires that there are no alternative means of repelling the attack
    \item Proportionality requires that the response is not excessive in relation to the armed attack
    \begin{itemize}
        \item This is measured against what is required to stop the attack, although it has been argued that proportionality relates to the scale and effects of the attack
        \item A state must ascertain what is necessary to stop the attack, and do nothing more than that
        \item This is a superior idea that the response can bear proportionality to the original attack, but simply because the state has been attacked particularly badly does not give it the right in self-defence to respond in a `tit-for-tat' manner; rather, self-defence authorises what is necessary for self-defence and to repel the attack only, not undertake countermeasures or a retaliatory act (i.e., it is not punitive or for revenge)
    \end{itemize}
\end{itemize}

\begin{casedetails}{\textit{Caroline Case} (1841/1842)}
    \flushleft

    This case never went to court, and so it is just an incident, but nonetheless involved a Canadian rebellion against Great Britain. Canadian rebels, with the support of US nationals, were attacking British ships in US waters, and some of these rebels used the \textit{Caroline} (a US flag ship) to transport supplies and personnel from a base to an island in the middle of the Hudson river, from which they were launching attacks on the British. The British untied the \textit{Caroline}, set it on fire, and allowed it to fly over the Niagara falls, resulting in the death of 2 people (including a US citizen). The question that arose was whether the British had lawfully/legitimately exercised the right of self-defence.

    \vspace{\baselineskip}

    This case did not go before a court/tribunal, and so there is no decision of any judge/arbitrator, but there is the exchange of diplomatic correspondence between the US and the British as to what the right of self-defence entailed. The states agreed that the relevant test for self-defence is in the paragraph below; that necessity is reflexive, but has to be limited by what is absolutely required.

    \begin{quote}
        It will be for the British Government to show a necessity of self-defence, instant, overwhelming, leaving no choice of means, and no moment of deliberation ... the act, justified by the necessity of self-defence, must be limited by that necessity, and kept clearly within it.
    \end{quote}
\end{casedetails}

\begin{casedetails}{\textit{Oil Platforms} (Iran v US) (2003)}
    \flushleft
    The ICJ determined that the United States was not subject to an armed attack when Iran launched a missile attack on a US-flagged civilian vessel, as the attack did not meet the required scale and effects threshold to qualify as an armed attack, though it constituted a use of force. The ICJ further ruled that the US response, which involved attacking Iranian oil platforms several days later, did not meet the criteria of necessity or proportionality. The US actions were not aimed at preventing future attacks but instead exhibited characteristics of retaliation, targeting Iranian state infrastructure in a manner that suggested a punitive lashing out rather than a defensive measure.
\end{casedetails}

\begin{casedetails}{\textit{Armed Activities Case} (DRC v Uganda) (2005)}
    \flushleft
    The ICJ held that since there was no armed attack, there was no right to self-defence, rendering the consideration of necessity and proportionality of the response unnecessary. However, in \textit{obiter dicta}, the ICJ noted that the US actions of capturing airports and towns hundreds of kilometres inside Uganda's territory would not appear proportionate or necessary in response to the series of transient, cross-border attacks that the US claimed justified self-defence. These actions lacked a reasonable relationship to the seriousness of the initial attacks and were not necessary to address them.
\end{casedetails}

\begin{casedetails}{\textit{Nicaragua} (Nicaragua v US) (1986)}
    \flushleft
    \textit{This is a canonical case on this area.}
    \tcbsubtitle{Background}
    In 1979, the right-wing Somoza Government in Nicaragua was overthrown by the left-wing Sandinista Government, prompting the United States, starting in 1981, to implement measures against the new regime, including supporting the Contra rebels aiming to overthrow the Sandinistas. The ICJ found it had jurisdiction over the case, despite the US reservation under Article 36(2) of the \statute{\textit{ICJ Statute}} regarding multilateral treaties, by determining that the relevant legal principles were rooted in customary international law, which the parties accepted as reflected in the UN Charters rules on the use of force.

    \tcbsubtitle{Applicable Law}
    The parties took the view that the principles on the use of force in the UN Charter correspond, in essentials, to those found in custom. They found that the general rule prohibiting force allows for certain exceptions, and held that the inherent right of self-defence under \convention{\textit{UN Charter} Article 51} covered both collective and individual self-defence.

    \tcbsubtitle{Use of Force and Self-Defence}

    The ICJ did not express a view on the concept of anticipatory self-defence in situations where an armed attack has not yet occurred but an imminent threat exists, leaving the issue of anticipatory self-defence unaddressed. The Court clarified that for a state to exercise individual self-defence, it must be the victim of an armed attack. An armed attack, as defined by the ICJ, encompasses not only actions by regular armed forces crossing an international border but also the deployment by or on behalf of a state of armed bands, groups, irregulars, or mercenaries. Such actions constitute an armed attack if their scale and effects would qualify as an armed attack had they been carried out by regular forces. Specifically, if a state employs irregular forces instead of its own military, the operation's scale and effects determine whether it qualifies as an armed attack, equivalent to one conducted by regular forces.

    \vspace{\baselineskip}

    The ICJ found that certain US actions, such as laying mines in Nicaraguan ports and conducting attacks on those ports, constituted infringements of the prohibition on the use of force. However, US military manoeuvres near Nicaraguan borders were not deemed a threat or use of force in violation of this prohibition, as they did not amount to direct uses of force. Additionally, the US breached the prohibition on the use of force by providing assistance to the Contra rebels in Nicaragua through arming and training them. However, not all forms of support to the Contras were considered a use of force; for instance, the mere supply of funds, while an unlawful act of intervention, did not constitute a use of force. The US argued that these actions were justified as exercises of self-defence.

    \vspace{\baselineskip}

    Regarding the US claim of exercising collective self-defence in support of El Salvador, Honduras, or Costa Rica, the ICJ ruled that these states were not subject to armed attacks, nor did they declare themselves attacked or request assistance at the relevant time. Consequently, the US actions could not be justified as collective self-defence. Even if an armed attack against these states could be established, the US response was neither necessary nor proportionate, failing to meet the legal requirements for self-defence. The ICJ further noted that the US itself was not the subject of an armed attack, precluding individual self-defence. Moreover, there was no evidence that El Salvador, Honduras, or Costa Rica complained about Nicaraguan actions or sought US assistance. Even if such complaints or requests existed, the US response was deemed disproportionate and unnecessary. The Court also clarified that an unlawful use of force that does not rise to the level of an armed attack may still invoke state responsibility, but it does not justify self-defence, as it lacks the proportionality required to constitute a breach of international law permitting such a response.
\end{casedetails}

\subsection{Terrorist Attacks}
\begin{itemize}
    \item Terrorist attacks may be armed attacks, depending on their scale and effects, and their cumulative effect
    \item A question arises as to whether the attack must be by a `state':
    \begin{itemize}
        \item In \case{\textit{Israel Wall} (2004)}, the ICJ said yes
        \item In \case{\textit{Armed Activities} (2005), [147]}, the ICJ expressly reserved its opinion
        \item An alternative view was implicit in the separate opinions of Kooijmans and Simma JJ in \case{\textit{Armed Activities} (2005)}, where self-defence against non-state actors might be permitted, especially where no real governmental authority exists in a target state and so there is no attribution of state responsibility possible (this is a limited, failed-state exception)
    \end{itemize}
    \item Following 9/11, the UNSC passed Resolutions 1368 and 1373, which both referred to ``the inherent right of individual or collective self-defence'' in respect to the terrorist attacks on the US, possibly indicating an implied authority by the UNSC
    \item Where a state is unwilling or unable to prevent its territory from being used to harbour terrorist groups or from being used to launch cross-border attacks against other states, this gives the victim state the right to use force within the territory of that unwilling/unable state against the non-state group, and justify that \textit{prima facie} unlawful use of force (however, state practice on this area is divided)
\end{itemize}

\section{Collective Security}
\begin{itemize}
    \item Under \convention{\textit{UN Charter} Chapter VII}, the UN Security Council is empowered to take actions necessary in response to threats of the peace, breaches of the peace, and acts of aggression
    \begin{itemize}
        \item This creates an agency relationship, where the UNSC authorises member states to use specific instances of force/power on its behalf
    \end{itemize}
    \item \convention{\textit{UN Charter} Article 39} outlines the threshold that has to be met; if it is met, turn to Article 41
\end{itemize}

\begin{conventiondetails}{\textit{UN Charter} Article 39}
    \flushleft
    The Security Council shall determine the existence of any threat to the peace, breach of the peace, or act of aggression and shall make recommendations, or decide what measures shall be taken in accordance with Articles 41 and 42, to maintain or restore international peace and security.
\end{conventiondetails}
    
\begin{conventiondetails}{\textit{UN Charter} Article 41}
    \flushleft
    The Security Council may decide what measures not involving the use of armed force are to be employed to give effect to its decisions, and it may call upon the Members of the United Nations to apply such measures. These may include complete or partial interruption of economic relations and of rail, sea, air, postal, telegraphic, radio, and other means of communication, and the severance of diplomatic relations.
\end{conventiondetails}

\begin{conventiondetails}{\textit{UN Charter} Article 42}
    \flushleft
    Should the Security Council consider that measures provided for in Article 41 would be inadequate or have proved to be inadequate, it may take such action by air, sea, or land forces as may be necessary to maintain or restore international peace and security. Such action may include demonstrations, blockade, and other operations by air, sea, or land forces of Members of the United Nations.
\end{conventiondetails}

\begin{itemize}
    \item Under \convention{\textit{UN Charter} Article 42}, the Security Council can authorise the use of force by UN member states, and this is a collective security measure, with this arising from the fact that the UN itself does not have its own army, air force or navy
    \item The UN Security Council has taken an increasingly broad approach to determining threats and breaches of the peace under \convention{Article 39}, to include serious internal conflicts (e.g., Rwanda, Libya, Syria) and acts of terrorism (e.g., 9/11) (as such, this goes beyond just classical interstate conflict, and can include purely internal, humanitarian disasters)
    \item The UNSC can take action so long the veto is not exercised by at least one of its five permanent members against threats to international peace and security
\end{itemize}

\section{Humanitarian Intervention and the Responsibility to Protect}
\subsection{Humanitarian Intervention}
\begin{itemize}
    \item This is a purported exception to the prohibition on the use of force to allow forcible intervention in another state to protect that state's own nationals from extreme cruelty or persecution
    \item It has been argued by some states for a third exception to the prohibition to allow one or more states to intervene in another state to protect that state's nationals (thus adopting a humanitarian angle)
    \item There has been preponderance of practice and commentary against its legality (it may be legitimate, but unlawful - this is yet to be determined)
    \item Practice in this area is scant: in 1990, NATO intervened in Kosovo to prevent genocide by means of a bombing campaign, despite the lack of UN Security Council authorisation, and more recently, military action was taken against the former Syrian regime for its use of chemical weapons
    \item The UK's position was enunciated in 2013 in the following material by its Foreign Office, but the majority of governments do not accept intervention on the rights of humanitarian issues, on the grounds that it is an unacceptable violation of sovereignty:
    \begin{quote}
        If action in Security Council is blocked, the UK would still be permitted…to take exceptional measures in order to alleviate the scale of the overwhelming humanitarian catastrophe in Syria by deterring and disrupting the further use of chemical weapons by the Syrian regime. Such a legal basis is available, under the doctrine of humanitarian intervention…[if] (i) there is convincing evidence…of extreme humanitarian distress on a large scale, requiring immediate and urgent relief; (ii) it must be objectively clear that there is no practicable alternative to the use of force ... and (iii) proposed use of force must be necessary and proportionate to the aim of relief of humanitarian need and strictly limited in time and scope to this aim.
    \end{quote}
\end{itemize}

\subsection{Responsibility to Protect}
\begin{itemize}
    \item The \textit{2001 International Commission on Intervention and State Responsibility} report held that force should be used only as a very last resort, and that `there is no better or more appropriate authority than the United Nations Security Council to authorise military intervention for humanitarian protection purposes'
    \item The Responsibility to Protect (R2P) doctrine seeks to achieve objectives of humanitarian intervention, but within the legal framework of the \convention{\textit{UN Charter}}
    \item It was endorsed and adopted in UNGA Resolution 61/1 (2005 World Summit Outcome)
    \item It is referred to obliquely in UNSC resolutions (e.g., the UNSC Resolution in 1973 concerning Libya's responsibility to protect its own nationals)
\end{itemize}