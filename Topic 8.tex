\begin{itemize}
    \item Whilst states enjoy complete and exclusive jurisdiction within their own territories, foreign states (including foreign state officials and diplomatic representatives) enjoy a range of immunities from the jurisdiction in the forum
    \item The foreign state and diplomatic immunities operate as a procedural bar to the exercise of jurisdiction (i.e., there is still jurisdiction, but it cannot be imposed)
    \begin{itemize}
        \item This immunity can be waived by the foreign agent or their state
    \end{itemize}
    \item Foreign state immunity is less extensive than it once was, with the predominant mode of operation having shifted to `restrictive immunity', where states can be sued in relation to commercial transactions (although diplomatic immunity remains largely unchanged over the years)
\end{itemize}

\section{Foreign State Immunity}
\begin{itemize}
    \item The principle of foreign state immunity is that one sovereign state (the forum state) may not adjudicate on the conduct of a foreign state (e.g., in an Australian court, foreign states would have immunity from proceedings)
    \begin{itemize}
        \item It is no longer the case that foreign states enjoy complete immunity; there are now a variety of circumstances in which a foreign state may be subject to the jurisdiction of a court that is not protected by an immunity
    \end{itemize}
    \item Subject to certain exceptions, the foreign state and its officials enjoy procedural immunity from criminal and civil proceedings in the forum state (i.e., foreign states, their officials and representatives (e.g., head of state, head of government, ministers, etc.) can enjoy immunity from proceedings within the forum), following Lord Brown-Wilkinson in \case{\textit{Pinochet (No 3)} [1999] 2 All ER 97} (see Page \pageref{case:Pinochet (No 3)})
    \item This is derived from the idea that all sovereign states are equal, and that one state does not have power over another (\textit{par in partem non habet imperium})
    \begin{itemize}
        \item This has been held to be ``one of the fundamental principles of the international legal order'' in \case{\textit{Jurisdictional Immunities Case} [2012] ICJ Rep 99}
    \end{itemize}
    \item There are two approaches/theories to foreign state immunity: `absolute immunity' and `restrictive immunity'
\end{itemize}

\subsection{Absolute Immunity Principle}
\begin{itemize}
    \item The absolute immunity principle holds that state are completely immune from the jurisdiction of another state; i.e., they are immune from any and all proceedings in the forum state
    \item Whilst this has now been eroded, it was the classical position
    \begin{itemize}
        \item ```[T]he perfect equality and absolute independence of sovereigns' mean no state can be made subject to the jurisdiction of another against its will'' - \case{\textit{The Schooner Exchange v McFaddon} (US Supreme Court, 1812)}
        \item In the modern day, only a few developing states tend to adhere to this absolute view of state immunity
    \end{itemize}  
\end{itemize}

\begin{casedetails}{\textit{The Schooner Exchange v McFaddon} (US Supreme Court, 1812)}
    \flushleft
    In a notable maritime case, a private vessel that sailed from Baltimore to Spain was seized by Napoleon en route and pressed into service for the French navy. The vessel later docked in Philadelphia, where its original American owners sought to reclaim it. The U.S. Supreme Court, in addressing the case, articulated the principle of absolute immunity for foreign naval vessels. The Court ruled that, regardless of the original ownership, the vessel was now a French naval ship and thus entitled to immunity, depriving the plaintiffs of jurisdiction to pursue their claim. This principle of sovereign immunity for naval vessels remains applicable today, shielding such ships from legal actions in foreign courts.
\end{casedetails}

\begin{itemize}
    \item In recent years, China abandoned this principle in favour of the restrictive immunity principle by adopting the \statute{\textit{Foreign States Immunity Law 2024}}, which largely follows the \convention{\textit{UN Convention on Jurisdictional Immunities of States and Their Property}} (this is noteworthy as the Convention has not yet entered into force, but is still influential as a statement of general principle)
    \begin{itemize}
        \item Prior to this, China's adherence to the absolute immunity principle was evident in \case{\textit{DRC v F C Hemisphere} [2011] HKFCA 43}, which applied absolute immunity in commercial proceedings against the government of the Democratic Republic of Congo
    \end{itemize}
\end{itemize}

\subsection{Restrictive Immunity Principle}
\begin{itemize}
    \item The restrictive immunity principle holds that foreign states are afforded immunity only with respect to acts of a sovereign or governmental character (\textit{acta jure imperii}), not for acts of a commercial character (\textit{acta jure gestionis}) (or any other non-sovereign acts); this is the prevailing approach across the world
    \item The restrictive immunity principle has been deemed necessary in the interests of justice to allow individuals to bring commercial transactions before the courts, thereby resolving the unfairness in commercial disputes between states and individuals as it does not impede the sovereignty of the foreign state
    \item This principle has been rationalised on the basis of not infringing on state sovereignty, as it only affects acts which have a commercial aspect to them and does not prevent states from undertaking sovereign acts
\end{itemize}

\begin{casedetails}{\textit{I Congresso del Partido} (1983) HoL}
    \flushleft
    A dispute arose when ships owned by the Cuban government failed to deliver a sugar cargo to Chile, prompting the cargo owners to initiate legal action against the ships and the Cuban government in the United Kingdom. The case stemmed from a contract between a Cuban state trading company and a Chilean company for the sale and delivery of sugar. Following a coup in Chile that brought Augusto Pinochet to power, Cuba directed its vessels to withhold the cargo, leading to proceedings in the UK courts. The Cuban government claimed sovereign immunity to shield itself from the lawsuit. However, the House of Lords, led by Lord Wilberforce and with the agreement of the other Law Lords, rejected this claim. The court applied a restrictive immunity approach, focusing on the nature of the act rather than its purpose. Despite the political motivation behind Cuba's decision to halt the sugar delivery, the transaction was deemed a commercial one, rooted in a trading relationship. Consequently, the court held that Cuba was not entitled to immunity, allowing the legal action to proceed against the Cuban vessel, and upholding the restrictive immunity principle.
\end{casedetails}

\begin{itemize}
    \item The majority of states have adopted the restrictive immunity principle
    \item This approach was moreover adopted in the \convention{\textit{2004 Convention on Jurisdictional Immunities of States and Their Property}}
    \begin{itemize}
        \item This Convention had, as of 14/04/2025, 24 parties, falling short of the 30 required for it to enter into force
        \item As such, it acts as a partial codification of customary international law, especially in relation to the provisions around restrictive immunity
    \end{itemize}
    \item In Australia, the relevant legislation is the \statute{\textit{Foreign States Immunities Act 1985 (Cth)}}, which was enacted following an ALRC inquiry on Foreign State Immunity in 1984
    \begin{itemize}
        \item The ALRC stated that ``In the interests of avoiding possible foreign relations problems, Australia should articulate to foreign states more precise rules governing their liability to the jurisdiction of the Australian courts."
        \item This Act adopts the restrictive immunity principle in Australia
    \end{itemize}
\end{itemize}

\section{Foreign States Immunities Act 1985 (Cth)}

\subsection{Section 9}
\begin{statutedetails}{\textit{Foreign States Immunities Act 1985 (Cth)} Section 9}
    \flushleft
    \textbf{General immunity from jurisdiction}

    \vspace{\baselineskip}

    Except as provided by or under this Act, a foreign State is immune from the jurisdiction of the courts of Australia in a proceeding.
\end{statutedetails}

\begin{itemize}
    \item Under \statute{s 3(1)}, a proceeding means a civil proceeding, not a prosecution for an offence (i.e., not a criminal proceeding)
    \item It was held in \case{\textit{PT Garuda v ACCC} [2012] HCA 33} that a proceeding includes a proceeding commenced by the ACCC against a defendant seeking penalties for anti-competitive conduct
\end{itemize}

\begin{casedetails}{\textit{Pt Garuda v ACCC} [2012] HCA 33}
    \flushleft
    Garuda was an Indonesia state-owned airline. The ACCC brought proceedings against them, alleging that Garuda had entered into anti-competitive conduct in relation to air-freight services into Australia. The Court held that a proceeding under \statute{\textit{Foreign States Immunities Act 1985 (Cth)} s 3} did extend to include not just proceedings in matters such as tort, but also proceedings brought by a regulator (including the ACCC0 seeking civil penalties for breaches of the Australian consumer law). From this case, it is established that there is a general immunity of foreign states from the jurisdiction of a state, subject to certain identified immunities.
\end{casedetails}

\begin{itemize}
    \item \statute{\textit{Foreign States Immunities Act 1985 (Cth)} ss 10-21} establish exceptions to the general immunity afforded under \statute{s 9}
    \item The general immunity afforded under \statute{s 9} applies to foreign states (including organs of foreign states), and also `separate entities' of foreign states (agencies/instrumentalities of state)
    \begin{itemize}
        \item This applies to all parts of a foreign state's government and organs of that government
        \item This serves to capture the case where states have established statutory organisations to carry out activities usually done by the government
    \end{itemize}
    \item The classic example of this is foreign airlines, such as in the case of \case{\textit{PT Garuda v ACCC} [2011] FCFCA 52} (this is the Federal Court, not the High Court version on Page \pageref{case:ACCC v Garuda})
    \begin{itemize}
        \item The key question in the case was whether Garuda was to be considered a separate entity of Indonesia?
        \begin{itemize}
            \item If it was, it was entitled to immunity unless there was an exception enlivened
        \end{itemize}
        \item Lander and Greenwood JJ held that ``[t]he most relevant factor in determining whether a natural person or a corporation is an agency or instrumentality is whether that body is carrying out the foreign State's functions or purposes ... regard will be had to ownership, control, the functions which the natural person or corporation perform, the foreign State's purposes in supporting the [alleged separate entity] and the manner in which the [separate entity] conducts itself or its business.''
        \begin{itemize}
            \item It is imperative to look at the ownership of the entity and the functions that are being executed, as well as the overriding functional purpose
            \item A company with some association with the State may fall under this, depending on how close it is to the state
        \end{itemize}
    \end{itemize}
    \item Under \statute{ss 30-32}, unless there has been a waiver, only the commercial property of a state may be the subject of execution of a judgement (commercial property is property in use by the foreign state substantially for commercial purposes, and does not include diplomatic or military property)
    \begin{itemize}
        \item Judgement cannot be enforced against sovereign property
    \end{itemize}
\end{itemize}

\begin{casedetails}{\textit{Firebird Global Master Fund II Ltd v Nauru} (2015) 326 ALR 396}
    \flushleft
    Firebird sought to register and enforce in Australia a Japanese judgment obtained against Nauru, related to bearer bonds guaranteed by Nauru. The High Court of Australia determined that the commercial transaction exception applied, as the Japanese judgment pertained to a commercial transaction. However, Nauru's bank accounts in Australia were deemed immune from any legal process or order, as they were used exclusively for governmental purposes rather than commercial activities. Consequently, the cash held in these accounts could not be subjected to a garnishing order or used to enforce the judgment.

    \flushleft

    Although Firebird successfully had the foreign judgment recognised in Australia and overcame the initial hurdle of \statute{\textit{Foreign States Immunities Act 1985 (Cth)} Section 9}, which governs immunity from jurisdiction, the enforcement effort ultimately failed due to the absence of available assets. The immunity of Nauru's bank accounts, reserved for governmental functions, presented an insurmountable barrier, leaving Firebird unable to execute the judgment. This case highlights the challenges of enforcing foreign judgments against sovereign states when their assets are protected for non-commercial purposes.

\end{casedetails}

\subsection{Section 10}
\begin{statutedetails}{\textit{Foreign States Immunities Act 1985 (Cth)} Section 10}
    \flushleft
    \textbf{Submission to jurisdiction}

    \begin{enumerate}[label=(\arabic*)]
        \item A foreign State is not immune in a proceeding in which it has submitted to the jurisdiction in accordance with this section.
        \item A foreign State may submit to the jurisdiction at any time, whether by agreement or otherwise, but a foreign State shall not be taken to have so submitted by reason only that it is a party to an agreement the proper law of which is the law of Australia.
        \item A submission under subsection (2) may be subject to a specified limitation, condition or exclusion (whether in respect of remedies or otherwise).
        \item Without limiting any other power of a court to dismiss, stay or otherwise decline to hear and determine a proceeding, the court may dismiss, stay or otherwise decline to hear and determine a proceeding if it is satisfied that, by reason of the nature of a limitation, condition or exclusion to which a submission is subject (not being a limitation, condition or exclusion in respect of remedies), it is appropriate to do so.
        \item An agreement by a foreign State to waive its immunity under this Part has effect to waive that immunity and the waiver may not be withdrawn except in accordance with the terms of the agreement.
        \item Subject to subsections (7), (8) and (9), a foreign State may submit to the jurisdiction in a proceeding by:
        \begin{enumerate}[label=(\alph*)]
            \item instituting the proceeding; or
            \item intervening in, or taking a step as a party to, the proceeding.           
        \end{enumerate}
        \item A foreign State shall not be taken to have submitted to the jurisdiction in a proceeding by reason only that:
        \begin{enumerate}[label=(\alph*)]
            \item it has made an application for costs; or
            \item it has intervened, or has taken a step, in the proceeding for the purpose or in the course of asserting immunity.
        \end{enumerate}
        \item Where the foreign State is not a party to a proceeding, it shall not be taken to have submitted to the jurisdiction by reason only that it has intervened in the proceeding for the purpose or in the course of asserting an interest in property involved in or affected by the proceeding.
        \item Where:
        \begin{enumerate}[label=(\alph*)]
            \item the intervention or step was taken by a person who did not know and could not reasonably have been expected to know of the immunity; and
            \item the immunity is asserted without unreasonable delay;
        \end{enumerate}
        the foreign State shall not be taken to have submitted to the jurisdiction in the proceeding by reason only of that intervention or step.
        \item Where a foreign State has submitted to the jurisdiction in a proceeding, then, subject to the operation of subsection (3), it is not immune in relation to a claim made in the proceeding by some other party against it (whether by way of set - off, counter - claim or otherwise), being a claim that arises out of and relates to the transactions or events to which the proceeding relates.
        \item In addition to any other person who has authority to submit, on behalf of a foreign State, to the jurisdiction:
        \begin{enumerate}[label=(\alph*)]
            \item the person for the time being performing the functions of the head of the State's diplomatic mission in Australia has that authority; and
            \item a person who has entered into a contract on behalf of and with the authority of the State has authority to submit in that contract, on behalf of the State, to the jurisdiction in respect of a proceeding arising out of the contract.
        \end{enumerate}
    \end{enumerate}
\end{statutedetails}

\begin{itemize}
    \item \statute{s 10} provides an exception to immunity where the foreign state has expressly submitted to the jurisdiction of the forum state (i.e., they've waived their immunity at their will)
    \item At times, foreign states will appeal in proceedings, but will be appearing only to contest jurisdiction; at times, they may just submit to the jurisdiction of the court
\end{itemize}

\begin{casedetails}{\textit{Kingdom of Spain v Infrastructure Services Luxembourg S.à.r.l.} [2023] HCA 11}
    \flushleft
    A company was awarded €101 million against Spain in an arbitration under the \convention{\textit{ICSID Convention}} and sought to enforce this award in Australia under the \statute{\textit{International Arbitration Act 1974 (Cth)}}. The \convention{\textit{ICSID Convention}} allows enforcement of arbitral awards in any member state's jurisdiction, including Australia. Spain invoked immunity under the Foreign States Immunities Act 1985 (Cth) to block the enforcement. The High Court of Australia (HCA) considered Article 54 of the \convention{\textit{ICSID Convention}}, which requires member states to recognize ICSID awards as binding and enforce them as if they were final domestic court judgments. The HCA interpreted section 10(2) of the Foreign States Immunities Act in accordance with international law, which mandates that any waiver of immunity must be express.
    
    \vspace{\baselineskip}
    
    The court held that such a waiver must be explicitly derived from the terms of international agreements, either as an express provision or as a term implied by necessity, as noted in paragraph [25] of the judgment. The HCA determined that Spain's agreement to Articles 53, 54, and 55 of the \convention{\textit{ICSID Convention}} constituted an express waiver of immunity from Australian court jurisdiction for the recognition and enforcement of the award, but not for its execution against Spanish sovereign property. As a result, the company could proceed with recognition and enforcement proceedings, but Spain's immunity from execution rendered the victory pyrrhic, as the award could not be executed against Spanish assets.
\end{casedetails}

\begin{casedetails}{\textit{Republic of India v CCDM Holdings, LLC} [2025] FCAFC 2}
    \flushleft
    A US\$111 million arbitral award was issued in favour of Mauritian investors in an Indian company under the India-Mauritius Bilateral Investment Treaty (BIT). The investors sought to enforce this award in Australia, relying on the 1958 New York Convention on the Recognition and Enforcement of Arbitration Awards, to which both Australia and India are parties. India had made a reservation to the Convention, limiting its application to disputes arising from legal relationships, contractual or otherwise, deemed commercial under Indian law. The investors initiated proceedings in Australia against India to enforce the award.
    
    \vspace{\baselineskip}
    
    The Federal Court of Australia (FCA) overturned the decision of Jackman J, ruling that India's reservation applied reciprocally, affecting the obligations of both Australia and India. The FCA held that Australia was not obligated to enforce awards related to disputes not considered commercial under Indian law. The court found that the award did not stem from a commercial legal relationship, as the BIT created a relationship under public international law, and the underlying dispute arose from the cancellation of a contract on public interest grounds.
    
    \vspace{\baselineskip}
    
    Furthermore, the FCA determined that India's ratification of the New York Convention did not constitute a waiver of foreign state immunity under the ``unmistakable" test established in \case{\textit{Kingdom of Spain v Infrastructure Services Luxembourg S.à.r.l.} [2023] HCA 11}, which requires a clear and explicit waiver of immunity. Consequently, the enforcement proceedings against India could not proceed in Australia.
\end{casedetails}

\begin{itemize}
    \item This Act comprehensively examines immunity issues, and there are no implied exceptions to immunity, as seen in the following cases:
    \begin{itemize}
        \item \case{\textit{Zhang v Zemin} [2010] NSWCA 255}, which was a damages proceeding against Chinese officials regarding torture
        \begin{itemize}
            \item This case held that there was no implied exception to foreign state immunity based on the \textit{jus cogens} prohibition of torture
            \item Spigelman J stated that there was no additional justification for exceptions, even in such a grave case
        \end{itemize}
        item \case{\textit{Young v A-G (NZ) and Ministry of Defence (UK)} [2019] NZSC 23}, which was a claim in tort in respect of sexual assault during a posting of the Royal Navy in the UK
        \begin{itemize}
            \item In this case, there was no exception to immunity arising from the duty of a state to provide the human right of a right to an effective remedy
            \item The plaintiff claimed to be the victim of assault whilst in the UK, working for the Royal Navy
        \end{itemize}
    \end{itemize}
\end{itemize}

\subsection{Section 11}
\begin{statutedetails}{\textit{Foreign States Immunities Act 1985 (Cth)} Section 11}
    \flushleft
    \textbf{Commercial Transactions}

    \begin{enumerate}[label=(\arabic*)]
        \item A foreign State is not immune in a proceeding in so far as the proceeding concerns a commercial transaction.
        \item Subsection (1) does not apply:
        \begin{enumerate}
            \item if all the parties to the proceeding:
            \begin{enumerate}[label=(\roman*)]
                \item are foreign States or are the Commonwealth and one or more foreign States; or
                \item have otherwise agreed in writing; or
            \end{enumerate}
            \item in so far as the proceeding concerns a payment in respect of a grant, a scholarship, a pension or a payment of a like kind.
        \end{enumerate}
        \item In this section, \textit{\textbf{commercial transaction}} means a commercial, trading, business, professional or industrial or like transaction into which the foreign State has entered or a like activity in which the State has engaged and, without limiting the generality of the foregoing, includes:
        \begin{enumerate}[label=(\alph*)]
            \item a contract for the supply of goods or services;
            \item an agreement for a loan or some other transaction for or in respect of the provision of finance; and
            \item a guarantee or indemnity in respect of a financial obligation; but does not include a contract of employment or a bill of exchange.
        \end{enumerate} 
    \end{enumerate}
\end{statutedetails}

\begin{itemize}
    \item This section states that a foreign sate is not immune in a proceeding which concerns a commercial transaction, clearly expressing the restrictive immunity principle
    \item This is the most important section of the Act, and is invoked the most frequently
    \item `The definition of ``commercial transaction" fixes upon entry and engagement by the foreign State. It does not have any limiting terms which would restrict the immunity…to a proceeding instituted against the foreign State by a party to the commercial transaction in question ... \textbf{The arrangements and understandings into which the ACCC alleges Garuda entered were \underline{dealings of a commercial, trading and business character}}, respecting the conduct of airline freight services to Australia.', following \case{\textit{PT Garuda v ACCC} [2012] HCA 33 per French CJ, Gummow, Hayne and Crennan JJ}
    \begin{itemize}
        \item In this case, it was held that for proceedings to continue, there needed to be established some exception to immunity
        \item In this case, the Court held that the commercial transaction exception applies, as the whole character of the proceedings at hand related to commercial dealings, even though the commercial dealings were not between the ACCC and Garuda
        \item Thus, it is irrelevant if the parties were directly engaged in a commercial relationship; the exception applies if the underlying character of the case is commercial in nature
    \end{itemize}
\end{itemize}

\begin{casedetails}{\textit{Australian International Islamic College Board Inc v Saudi Arabia} [2013] QCA 129}
    \flushleft
    The government of Saudi Arabia promised to cover the costs of educating students at a college in Australia, a commitment that constituted a commercial transaction for the recipients of these scholarships. However, the Saudi government failed to fulfill this financial obligation. When legal proceedings were initiated in Queensland, Saudi Arabia invoked sovereign immunity to avoid liability. In the resulting case, Holmes JA delivered the leading judgment, with the other justices in agreement, characterising the relationship as one centred on the payment for educational services. The court determined that this was indeed a commercial transaction, and thus, section 11 of the relevant legislation applied, rendering the immunity claim inapplicable.

    \vspace{\baselineskip}

    Holmes JA further elaborated on the rationale behind the commercial transactions exception, providing clarity on its purpose and justification. As stated in the judgment: ``The relevant exception…appears to have two main foundations: (a) It is necessary in the interest of justice to individuals having such transactions with states to allow them to bring such transactions before the courts. (b) To require a state to answer a claim based upon such transactions does not involve a challenge to or inquiry into any act of sovereignty or governmental act of that state. It is ... neither a threat to the dignity of that state, nor any interference with its sovereign functions." (Holmes JA at [24]). This reasoning underscores the balance between ensuring justice for individuals and respecting state sovereignty, emphasising that commercial dealings do not infringe upon a state's governmental authority or dignity.
\end{casedetails}

\subsection{Section 12}
\begin{statutedetails}{\textit{Foreign States Immunities Act 1985 (Cth)} Section 12}
    \flushleft
    \textbf{Contracts of employment}

    \begin{enumerate}[label=(\arabic*)]
        \item A foreign State, as employer, is not immune in a proceeding in so far as the proceeding concerns the employment of a person under a contract of employment that was made in Australia or was to be performed wholly or partly in Australia.
        \item A reference in subsection (1) to a proceeding includes a reference to a proceeding concerning
        \begin{enumerate}[label=(\alph*)]
            \item a right or obligation conferred or imposed by a law of Australia on a person as employer or employee; or
            \item a payment the entitlement to which arises under a contract of employment.
        \end{enumerate}
        \item Where, at the time when the contract of employment was made, the person employed was:
        \begin{enumerate}[label=(\alph*)]
            \item a national of the foreign State but not a permanent resident of Australia ; or
            \item an habitual resident of the foreign State;
        \end{enumerate}
        subsection (1) does not apply.
        \item Subsection (1) does not apply where:
        \begin{enumerate}[label=(\alph*)]
            \item an inconsistent provision is included in the contract of employment; and
            \item a law of Australia does not avoid the operation of, or prohibit or render unlawful the inclusion of, the provision.
        \end{enumerate}
        \item Subsection (1) does not apply in relation to the employment of:
        \begin{enumerate}
            \item a member of the diplomatic staff of a mission as defined by the Vienna Convention on Diplomatic Relations, being the Convention the English text of which is set out in the Schedule to the Diplomatic Privileges and Immunities Act 1967 ; or
            \item a consular officer as defined by the Vienna Convention on Consular Relations, being the Convention the English text of which is set out in the Schedule to the Consular Privileges and Immunities Act 1972 .
        \end{enumerate}
        \item Subsection (1) does not apply in relation to the employment of:
        \begin{enumerate}[label=(\alph*)]
            \item a member of the administrative and technical staff of a mission as defined by the Convention referred to in paragraph (5)(a); or
            \item a consular employee as defined by the Convention referred to in paragraph (5)(b);
        \end{enumerate}
        unless the member or employee was, at the time when the contract of employment was made, a permanent resident of Australia.
        \item In this section, permanent resident of Australia means:
        \begin{enumerate}
            \item an Australian citizen; or
            \item a person resident in Australia whose continued presence in Australia is not subject to a limitation as to time imposed by or under a law of Australia.
        \end{enumerate}
    \end{enumerate}
\end{statutedetails}

\begin{itemize}
    \item This section states that a foreign state is not immune in a proceeding which concerns the employment of a person under a contract of employment made in Australia or performed in Australia (i.e., if you are a foreign state and employ Australians, you cannot claim immunity in this area)
\end{itemize}

\subsection{Section 13}
\begin{statutedetails}{\textit{Foreign States Immunities Act 1985 (Cth)} Section 13}
    \flushleft
    \textbf{Personal injury and damage to property}

    \vspace{\baselineskip}

    A foreign State is not immune in a proceeding in so far as the proceeding concerns:
    \begin{enumerate}[label=(\alph*)]
        \item the death of, or personal injury to, a person; or
        \item loss of or damage to tangible property;
    \end{enumerate}
    caused by an act or omission done or omitted to be done in Australia.
\end{statutedetails}

\begin{itemize}
    \item This section states that a foreign state is not immune in a proceeding concerning death, personal injury, or loss or damage of tangible property, caused by an act or omission \underline{in Australia} (the `local torts' exception)
\end{itemize}

\begin{casedetails}{\textit{Tokic v Government of Yugoslavia} (1999) NSWSC Unreported}
    \flushleft
    The plaintiff sustained injuries after being struck by a bullet fired from within the Yugoslav Consulate in Sydney during a protest outside the building. In subsequent civil proceedings seeking damages for personal injury, the Supreme Court of New South Wales ruled that the Government of Yugoslavia could not claim immunity. The court determined that the case fell under Section 13, which applies to proceedings concerning personal injury caused by an act committed in Australia. The evidence established that the plaintiff was shot as a result of a weapon being fired from the consulate, leading the judge to award \$50,000 in damages in favour of the plaintiff against the foreign state of Yugoslavia.

    \vspace{\baselineskip}
    
    Despite the court's ruling, the victory was pyrrhic, as there was no commercial property available against which the judgement could be enforced. It is possible that the Commonwealth made an \textit{ex-gratia} payment to the plaintiff, though this is secondary to the case's significance. This case serves as a clear illustration of the local tort exception under Australian law, which encompasses both personal injury and property damage, demonstrating how such exceptions apply when a foreign state's actions cause harm within Australia.
\end{casedetails}

\subsection{Section 14}
\begin{statutedetails}{\textit{Foreign States Immunities Act 1985 (Cth)} Section 14}
    \flushleft
    \textbf{Ownership, possession and use of property etc.}
    \begin{enumerate}
        \item A foreign State is not immune in a proceeding in so far as the proceeding concerns:
        \begin{enumerate}
            \item an interest of the State in, or the possession or use by the State of, immovable property in Australia; or
            \item an obligation of the State that arises out of its interest in, or its possession or use of, property of that kind.
        \end{enumerate}
        \item A foreign State is not immune in a proceeding in so far as the proceeding concerns an interest of the State in property that arose by way of gift made in Australia or by succession.
        \item A foreign State is not immune in a proceeding in so far as the proceeding concerns:
        \begin{enumerate}
            \item bankruptcy, insolvency or the winding up of a body corporate; or
            \item the administration of a trust, of the estate of a deceased person or of the estate of a person of unsound mind.
        \end{enumerate}
    \end{enumerate}
\end{statutedetails}

\begin{itemize}
    \item This section states that a foreign state is not immune in a proceeding concerning:
    \begin{enumerate}
        \item The interest in or possession or use by a state of immovable property
        \item The interest in property that arose by way of gift or by succession
        \item Bankruptcy, insolvency or winding up of a body corporate, or the administration of a trust, the estate of a deceased person or estate of a person of unsound mind
    \end{enumerate}
\end{itemize}

\subsection{Section 15}
\begin{statutedetails}{\textit{Foreign States Immunities Act 1985 (Cth)} Section 15}
    \flushleft
    \textbf{Copyright, patents, trade marks etc.}

    \begin{enumerate}[label=(\arabic*)]
        \item A foreign State is not immune in a proceeding in so far as the proceeding concerns:
        \begin{enumerate}[label=(\alph*)]
            \item the ownership of a copyright or the ownership, or the registration or protection in Australia, of an invention, a design or a trade mark;
            \item an alleged infringement by the foreign State in Australia of copyright, a patent for an invention, a registered trade mark or a registered design; or
            \item the use in Australia of a trade name or a business name.
        \end{enumerate}
        \item Subsection (1) does not apply in relation to the importation into Australia, or the use in Australia, of property otherwise than in the course of or for the purposes of a commercial transaction as defined by subsection 11(3).
    \end{enumerate}
\end{statutedetails}

\begin{itemize}
    \item This section states that a foreign state is not immune in a proceeding concerning copyright, invention, design or trademark in Australia, or use in Australia of a trade name or business name
\end{itemize}

\subsection{Section 16}
\begin{statutedetails}{\textit{Foreign States Immunities Act 1985 (Cth)} Section 16}
    \flushleft
    \textbf{Membership of bodies corporate etc.}
    \begin{enumerate}
        \item A foreign State is not immune in a proceeding in so far as the proceeding concerns its membership, or a right or obligation that relates to its membership, of a body corporate, an unincorporated body or a partnership that:
        \begin{enumerate}
            \item has a member that is not a foreign State or the Commonwealth; and
            \item is incorporated or has been established under the law of Australia or is controlled from, or has its principal place of business in, Australia;
        \end{enumerate}
        being a proceeding arising between the foreign State and the body or other members of the body or between the foreign State and one or more of the other partners.
        \item Where a provision included in:
        \begin{enumerate}
            \item the constitution or other instrument establishing or regulating the body or partnership; or
            \item an agreement between the parties to the proceeding;
        \end{enumerate}
        is inconsistent with subsection (1), that subsection has effect subject to that provision.
    \end{enumerate}
\end{statutedetails}

\begin{itemize}
    \item This section states that a foreign state is not immune in certain proceedings concerning its membership of a body corporate or unincorporated body or partnership established under the law of Australia or operating in Australia
\end{itemize}

\subsection{Section 17}
\begin{statutedetails}{\textit{Foreign States Immunities Act 1985 (Cth)} Section 17}
    \flushleft
    \textbf{Arbitrations}
    \begin{enumerate}
        \item Where a foreign State is a party to an agreement to submit a dispute to arbitration, then, subject to any inconsistent provision in the agreement, the foreign State is not immune in a proceeding for the exercise of the supervisory jurisdiction of a court in respect of the arbitration, including a proceeding:
        \begin{enumerate}
            \item by way of a case stated for the opinion of a court;
            \item to determine a question as to the validity or operation of the agreement or as to the arbitration procedure; or
            \item to set aside the award.
        \end{enumerate}
        \item Where:
        \begin{enumerate}[label=(\alph*)]
            \item part from the operation of subparagraph 11(2)(a)(ii), subsection 12(4) or subsection 16(2), a foreign State would not be immune in a proceeding concerning a transaction or event; and
            \item the foreign State is a party to an agreement to submit to arbitration a dispute about the transaction or event;
        \end{enumerate}
        then, subject to any inconsistent provision in the agreement, the foreign State is not immune in a proceeding concerning the recognition as binding for any purpose, or for the enforcement, of an award made pursuant to the arbitration, wherever the award was made.
        \item Subsection (1) does not apply where the only parties to the agreement are any 2 or more of the following:
        \begin{enumerate}[label=(\alph*)]
            \item a foreign State;
            \item the Commonwealth;
            \item an organisation the members of which are only foreign States or the Commonwealth and one or more foreign States.
        \end{enumerate}
    \end{enumerate}
\end{statutedetails}

\begin{itemize}
    \item This section states that a foreign state is not immune in proceedings concerning an arbitration agreement to which foreign state is a party
\end{itemize}

\subsection{Section 18}
\begin{statutedetails}{\textit{Foreign States Immunities Act 1985 (Cth)} Section 18}
    \flushleft
    \textbf{Actions in rem}
    \begin{enumerate}[label=(\arabic*)]
        \item A foreign State is not immune in a proceeding commenced as an action in rem against a ship concerning a claim in connection with the ship if, at the time when the cause of action arose, the ship was in use for commercial purposes.
        \item A foreign State is not immune in a proceeding commenced as an action in rem against a ship concerning a claim against another ship if:
        \begin{enumerate}[label=(\alph*)]
            \item at the time when the proceeding was instituted, the ship that is the subject of the action in rem was in use for commercial purposes; and
            \item at the time when the cause of action arose, the other ship was in use for commercial purposes.
        \end{enumerate}
        \item A foreign State is not immune in a proceeding commenced as an action in rem against cargo that was, at the time when the cause of action arose, a commercial cargo.
        \item The preceding provisions of this section do not apply in relation to the arrest, detention or sale of a ship or cargo.
        \item A reference in this section to a ship in use for commercial purposes or to a commercial cargo is a reference to a ship or a cargo that is commercial property as defined by subsection 32(3).
    \end{enumerate}
\end{statutedetails}

\begin{itemize}
    \item This section states that a foreign state is not immune in proceedings concerning actions in rem against a ship in connection with a ship in use for commercial purposes
\end{itemize}

\subsection{Section 19}
\begin{statutedetails}{\textit{Foreign States Immunities Act 1985 (Cth)} Section 19}
    \flushleft
    \textbf{Bills of exchange}

    \vspace{\baselineskip}

    Where:
    \begin{enumerate}[label=(\alph*)]
        \item a bill of exchange has been drawn, made, issued or indorsed by a foreign State in connection with a transaction or event; and
        \item the foreign State would not be immune in a proceeding in so far as the proceeding concerns the transaction or event;
    \end{enumerate}
    the foreign State is not immune in a proceeding in so far as the proceeding concerns the bill of exchange.
\end{statutedetails}

\subsection{Section 20}
\begin{statutedetails}{\textit{Foreign States Immunities Act 1985 (Cth)} Section 20}
    \flushleft
    \textbf{Taxes}

    \vspace{\baselineskip}

    A foreign State is not immune in a proceeding in so far as the proceeding concerns an obligation imposed on it by or under a provision of a law of Australia with respect to taxation, being a provision that is prescribed, or is included in a class of provisions that is prescribed, for the purposes of this section.
\end{statutedetails}

\begin{itemize}
    \item This section states that a foreign state is not immune in a proceeding concerning a taxation obligation under Australian law (i.e., a foreign state cannot escape Australian tax law)
\end{itemize}

\section{Immunity of Foreign State Officials}
\begin{itemize}
    \item Heads of state, government and foreign affairs ministers are generally afforded immunity, alongside other officials
    \item This immunity is not for their individual benefit, but for the benefit of the state that they are employed by or represent
    \item Generally, the immunity of individuals as representatives of a state is confined to acts performed in their official capacity (immunity \textit{ratione materiae})
    \begin{itemize}
        \item This refers to immunity by reference to whether or not an individual is acting in an official capacity
        \item Officials of a state generally enjoy functional immunity in relation to their functional acts, as it protects the state
    \end{itemize}
    \item However, some high-ranking individuals are entitled to immunity by virtue of their position (immunity \textit{ratione personae}), on the rationale that such immunity is essential for the performance of their official functions representing a state
    \begin{itemize}
        \item Unlike immunity \textit{ratione materiae}, this immunity is far broader, and covers these individuals for all acts that they commit, on the basis that it is essential for the performance of their official functions
    \end{itemize}
\end{itemize}

\subsection{Incumbent Heads of State}
\subsubsection{Civil Proceedings in Representative Capacity}
\begin{itemize}
    \item The ordinary principles of state immunity apply (e.g., \statute{\textit{Foreign States Immunities Act 1985 (Cth)} s 3(3)(b)}, which holds that a reference to a foreign state includes a reference to the head of that state in their public capacity)
    \item The Australian principle, as stated in \statute{\textit{Foreign States Immunities Act 1985 (Cth)}}, mostly reflects international law
\end{itemize}

\subsubsection{Civil or Criminal Proceedings in Personal Capacity}
\begin{itemize}
    \item If they are acting in their personal capacity, the foreign head of state is entitled to the same immunity as the head of a diplomatic mission, which grants them complete personal immunity for criminal proceedings, and the same immunity as heads of diplomatic missions for civil proceedings, as outlined in \statute{\textit{Foreign States Immunities Act 1985 (Cth)} s 36}
    \item It is not possible to have a criminal proceeding against a foreign head of state at all, and there are only very limited circumstances in which a civil proceeding can be brought against them
\end{itemize}

\begin{casedetails}{\textit{Thor Shipping A/S v `Al Duhail'} (2008) 252 ALR 20}
    \flushleft

    The Amir purchased a fishing boat constructed in New Zealand and entered into a contract with a company to transport the vessel from New Zealand to an island destination. However, the company was not paid for the carriage of the vessel. Instead of being transported by a cargo vessel, the fishing boat travelled under its own power to Queensland, Australia, where it was arrested by the company, which initiated legal proceedings against the vessel in an Australian court.

    \vspace{\baselineskip}

    The Amir claimed immunity from the proceedings, and the Federal Court of Australia (FCA) agreed. The court determined that the Amir was acting in a private, personal capacity, not as the head of state, when he purchased and arranged for the transport of the boat. Consequently, the FCA ruled that the Amir possessed the same immunity as the head of a diplomatic mission, and none of the relevant exceptions to immunity applied. As a result, the court ordered the release of the vessel.
\end{casedetails}

\begin{casedetails}{\textit{Tatchell v Mugabe} (2004) 136 ILR 572}
    \flushleft
    This case was heard in the Bow Street Magistrates' Court.

    \vspace{\baselineskip}

    As a matter of customary international law, common law, and the State Immunity Act 1978 (UK), President Mugabe of Zimbabwe was entitled to immunity in the United Kingdom and could not be arrested or detained, including in relation to allegations of torture. At the time the proceedings were brought against him, he was the serving President of Zimbabwe. The UK court applied the State Immunity Act, which is similar to equivalent legislation in Australia, and ruled that, as a serving head of state, President Mugabe enjoyed complete immunity from legal proceedings in the UK.
\end{casedetails}

\begin{casedetails}{\textit{Gaddafi} (2001) 125 ILR 490}
    \flushleft

    This case was heard in the Court of Cassation in France.

    \vspace{\baselineskip}

    As the incumbent head of state of Libya, Colonel Gaddafi was entitled to immunity from French criminal jurisdiction in relation to a terrorist bombing of a French airliner. When criminal proceedings were initiated by the victims of the terrorist bombing, the French court held that, as a serving head of state, Colonel Gaddafi enjoyed immunity \textit{ratione personae}. Consequently, the court ruled that the case could not proceed against him due to his immunity.
\end{casedetails}

\begin{itemize}
    \item \textit{Jus cogens} norms do not override immunity
    \item The \case{\textit{Jurisdictional Immunities of State Case} (ICJ)} held that immunity rules and \textit{jus cogens} exist on two different levels, and that there is no direct conflict between them
    \item There are certain rules around immunity that go to core questions around sovereign equality such that if immunity disappears then judgement will be forthcoming, relatively quickly
\end{itemize}

\subsection{Former Heads of State}
\begin{itemize}
    \item A lot of the protections afforded to incumbent individuals is lost when they give up office
    \item They no longer possess immunity \textit{ratione personi} (this only exists when they are in office), but they still continue to possess immunity \textit{ratione materiae} for acts performed in their official capacity when they were in office
    \item This position is reflected in Australia under \statute{\textit{Foreign States Immunities Act 1985 (Cth)} s 36}, which affords a broad immunity \textit{ratione materiae}
    \item Following \case{\textit{Harb v Prince Abdul Aziz Bin Fahd Bin Abdul Aziz} [2015] EWCA Div 481}, a former head of state or the estate of a deceased head of foreign state is not entitled to jurisdictional immunity in civil proceedings in respect of a private act done whilst in office as head of a foreign state
\end{itemize}

\begin{casedetails}{\textit{Harb v Prince Abdul Aziz Bin Fahd Bin Abdul Aziz} [2015] EWCA Div 481}
    \flushleft
    A former head of state, or the estate of a deceased head of state, does not enjoy jurisdictional immunity in civil proceedings for private acts committed while in office. This principle was central to a case involving the estate of a deceased Saudi king, who died while serving as head of state. The claimant, who had married a prince of the Saudi royal family before he ascended to the throne, entered into a contract with the king. Under this agreement, the king promised to provide financial support for her in London, including the purchase of two properties.

    \vspace{\baselineskip}

    Following the king's death, the claimant initiated legal proceedings against his estate to enforce the contract. The estate argued that it was entitled to immunity from such claims. However, the Court of Appeal (CoA) ruled that no immunity applied. The court determined that the contract constituted a private act, distinct from official governmental functions. As a result, the estate could not claim immunity ratione personae, which applies to personal acts rather than official ones. Had the act been an official governmental function, the estate might have been protected by immunity for such acts. Consequently, the estate was subject to the proceedings and had no legal protection in this context.
\end{casedetails}

\begin{itemize}
    \item In respect of torture and other serious international crimes, former heads of foreign states have no immunity, even in respect of official acts
    \begin{itemize}
        \item This is dependent on what the crime is
    \end{itemize}
    \item \case{\textit{R v Bow Street Stipendary Magistrate; ex parte Pinochet (No 3)} [1992] 2 All ER 97} held that ``international law could not without absurdity require criminal jurisdiction to he assumed and exercised where the Torture Convention conditions were satisfied and, at the same time, require immunity to be granted to those properly charged'' (in respect of torture; this was explained by Lord Bingham in \case{\textit{Jones v Saudi Arabia} [2007] AC 270})
    \begin{itemize}
        \item See the facts of \case{\textit{Pinochet} [1992] 2 All ER 97} on Page \pageref{case:Pinochet (No 3)}
        \item This case held that as torture at the hands of a state are acts of an official character, which means that they would be protected by a functional immunity
    \end{itemize}
    \item \convention{Draft Article 7 of the \textit{ILC Draft Articles} (2022)} considers what are \textit{jus cogens} crimes, and whether immunity applies
    \item It was stated in \case{\textit{Pinochet} [1992] 2 All ER 97} that if a serious international crime was committed, then a former head of state, even if they committed that crime whilst in office as part of their official functions, would not be protected by an immunity
\end{itemize}

\subsection{Incumbent Foreign Affairs Ministers}
\begin{casedetails}{\textit{Arrest Warrant Case} [2002] ICJ Rep 3}
    \flushleft

    On 17 October 2000, the Democratic Republic of the Congo (DRC) initiated proceedings against Belgium at the International Court of Justice (ICJ) over an international arrest warrant issued by Belgium on 11 April 2000 against the DRC's acting Foreign Minister, Abdoulaye Yerodia Ndombasi, for alleged war crimes and crimes against humanity. The DRC argued that the warrant, circulated globally, violated customary international law granting immunity and inviolability to incumbent foreign ministers, and sought its cancellation and reparations. Belgium contested the Court's jurisdiction, mootness, and admissibility. On 14 February 2002, the ICJ rejected Belgium's objections, affirming its jurisdiction and ruling that incumbent foreign ministers enjoy full immunity from criminal jurisdiction under customary international law to ensure their effective functioning, regardless of whether acts were official or private, or committed before or during their tenure. The Court found that the issuance and circulation of the warrant violated Belgium's obligations to respect Yerodia's immunity, constituting a breach of international law. While immunity does not equate to impunity, as criminal responsibility remains, the Court ordered Belgium to cancel the warrant and notify relevant authorities, deeming this and the judgment itself sufficient to address the DRC's moral injury.

    \tcblower
    \flushleft

    The International Court of Justice (ICJ) ruled that the serving Minister for Foreign Affairs of the Democratic Republic of the Congo was immune from criminal proceedings in Belgium. The allegations against the Minister involved inciting war crimes and crimes against humanity in the Congo. Belgium sought to arrest him, but the Congo claimed immunity on behalf of its Foreign Affairs Minister. The ICJ's judgement addressed the broader issue of immunity for foreign state representatives, clarifying that such immunity is granted for the benefit of the state, not the individual. This immunity is essential to enable the Minister for Foreign Affairs to effectively perform their functions without interference.

    \vspace{\baselineskip}

    The ruling further clarified several key aspects of this immunity. It does not distinguish between acts performed in an official or personal capacity, as the immunity is a broad personal one, eliminating the need to examine the nature of the acts. Additionally, it is irrelevant whether the Minister was in the arresting state on an official or private visit, and there is no exception for serious international crimes. However, immunity does not equate to impunity, as it is a procedural protection rather than a substantive one. The Minister could still face prosecution in their home state if the home state waives immunity, though this is highly unlikely. Furthermore, once the Minister is no longer in office, a more limited residual immunity applies, and there is no immunity before international criminal courts.
\end{casedetails}

\begin{itemize}
    \item Under this case, immunity is granted for the benefit of the state, not the individual
    \item Immunity is essential for the Minister for Foreign Affairs to effectively perform their functions without interference
    \item There was no distinction between acts performed in an official or personal capacity, as the immunity is a broad personal one
    \item It is not relevant whether the Minister is in the arresting state on an official or a private visit
    \item There is no exception in relation to serious international crimes (\textit{jus cogens})
\end{itemize}

\subsection{The Troika}
\begin{itemize}
    \item The ``Troika" refers to the Head of State, Head of Government, and Minister for Foreign Affairs
    \item The concept of the `troika' has been codified in \convention{\textit{ILC Draft Article 3}}; this concept found its footing in state practice and in the \case{\textit{Arrest Warrant Case} [2002] ICJ Rep 3}
    \item Immunity may be extended to the Minister for Defence, following \case{\textit{Re Mofaz} (UK, 2004)}
    \begin{itemize}
        \item In this case, there was an application made for an arrest warrant against the Israeli Defence Minister for alleged war crimes committed in the West Bank
        \item The application was refused by the Bow Street Magistrates' Court, which held that the Minister for Defence was entitled to immunity, by drawing analogy to the Foreign Affairs Minister
    \end{itemize}
    \item Immunity may be extended to the Minister for International Trade, following \case{\textit{Re Bo Xilai} (UK, 2005)}
    \begin{itemize}
        \item In this case, the Bow Street Magistrates' Court held refused an application for an arrest warrant against the Chinese Minister for International Trade, who was accused of torture
        \item The Court found that the Trade Minister had functions equivalent to those of a Foreign Affairs Minister, and therefore was entitled to immunity
    \end{itemize}
\end{itemize}

\subsection{Proceedings Concerning State Torture}
\subsubsection{Civil Proceedings in the United Kingdom}
\begin{casedetails}{\textit{Al Adsani v Kuwait} (1995) 103 ILR 420 (QBD and CA)}
    \flushleft
    In a civil case brought by a UK national in UK courts against Kuwait for alleged torture in a Kuwaiti state prison, the court ruled that the local tort exception did not apply, as the alleged torture occurred in Kuwait. Furthermore, the court found no implied exception to the general rule of state immunity, even when the proceedings involve conduct constituting an offence under public international law for which the state may be responsible.
\end{casedetails}

\begin{casedetails}{\textit{Jones v Saudi Arabia} [2007] AC 270}
    \flushleft
    In civil claims for damages for personal injury brought by UK plaintiffs alleging torture while imprisoned in Saudi Arabia, the defendants included Saudi Arabia and named Saudi public officials. The court held that while UK legislation did not specifically address civil proceedings against foreign public officials acting in their official capacity, ``there is ... a wealth of authority to show that in such case the foreign state is entitled to claim immunity for its servants as it could if sued itself. The foreign state's right to immunity cannot be circumvented by suing its servants or agents", as stated by Lord Bingham.
\end{casedetails}

\subsubsection{Australia}
\begin{casedetails}{\textit{Zhang v Zemin} [2010] NSWCA 255}
    \flushleft
    In a case similar to \case{\textit{Jones v Saudi Arabia}}, involving a claim for damages for the tort of trespass due to alleged torture in China, the defendants included the former President of the People's Republic of China, the Falun Gong Control Office, and Luo Gan, a senior member of the Chinese Communist Party. The court, with Spigelman CJ delivering the judgment and Allsop P and McClellan CJ at CL agreeing, held that the former President was entitled to immunity under the Foreign States Immunities Act 1985 (Cth). The court determined that whether a person or entity qualifies as a foreign state, as defined under s 3(3)(b) of the Act, is assessed at the time of the impugned conduct. Additionally, s 3(3)(c) includes individuals acting as public or government officials within the definition of a foreign state, as excluding them would render the Act ``virtually devoid of practical significance". Consequently, Luo Gan was immune for conduct performed in his official capacity. The court further ruled that there was no implied exception to immunity, as s 9 of the Act is clear, rendering the \textit{Polites} interpretive principle inapplicable.
\end{casedetails}