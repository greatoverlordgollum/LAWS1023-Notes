There was no tutorial problem for Topic 1, and the tutorial problem for Topic 2 consisted of source evaluation only.
\setcounter{section}{2}

\section{Treaties}
\begin{tutorialquestion}
    \flushleft
    Astra, Benthos and Ceres are neighbouring states on Torrent Island. Astra is located  in the eastern and mountainous part of Torrent Island and is highly industrialised. Benthos andCeres are developing states located on the western and low-lying part of Torrent Island.

    \vspace{\baselineskip}

    The River Delta flows from Astra into Benthos and then on to Ceres. In 1977 Astra, Benthos and Ceres concluded the River Delta Treaty (`RDT'). Article 1 of the RDT commits the parties to `cooperate to achieve the reasonable and equitable use of the River Delta'. In Article 10 of the RDT, Benthos and Ceres are guaranteed defined minimum volumes of water per annum. On signing the RDT, Astra lodged the following declaration:

    \begin{quote}
        `The Government of the Astra Republic, in approving the Treaty, declares that reference to the concept of reasonable and equitable use of transboundary waters does not constitute recognition of a principle of customary law, but illustrates a general principle of cooperation between Parties to the Treaty.'
    \end{quote}

    In 2024, due to an ongoing drought exacerbated by climate change, Astra announced that it was unable to supply any water to Benthos. Benthos in turn was unable to comply with its obligations under Article 10 to supply a required volume of water to Ceres.

    \vspace{\baselineskip}

    In January 2025, Astra announced that the RDT was `hereby terminated with immediate effect'. In doing so Astra referred to the persistent drought conditions, and the record of negotiations for the RDT during which former President of Benthos commented that `of course whatever the treaty says must be subject to the vagaries of nature and we may have to put the treaty on hold if the river dries up.' The RDT includes no provision relating to suspension or termination.

    \vspace{\baselineskip}

    Benthos and Ceres contend that the RDT remains in force and that Astra is also bound by a customary law obligation to provide reasonable and equitable access to a shared freshwater resource.

    \vspace{\baselineskip}

    Astra, Benthos and Ceres are parties to the 1997 United Nations Convention on the Law of the Non-Navigational uses of International Watercourses (the UN Watercourses Convention) which provides, in Article 3, that `In the absence of an agreement to the contrary, nothing in the present Convention shall affect the rights or obligations of a watercourse State arising from agreements in force for it on the date on which it became a party to the present Convention.'

    \vspace{\baselineskip}

    Astra, Benthos and Ceres are parties to the 1969 Vienna Convention on the Law of Treaties.

    \vspace{\baselineskip}

    At the urging of the UN Secretary-General, the three States have agreed to mediation to resolve their dispute. You have been asked to prepare a legal brief advising the mediator on the legal issues that arise under the law of treaties from these facts.
\end{tutorialquestion}

\subsection{Objectives of the Parties}
\begin{itemize}
    \item \textbf{Astra}: Seeks to terminate the River Delta Treaty (RDT).
    \item \textbf{Benthos}: Wishes the RDT to remain in force.
    \item \textbf{Ceres}: Wishes the RDT to remain in force.
\end{itemize}

\subsection{Applicability of the Vienna Convention on the Law of Treaties (VCLT)}
\begin{itemize}
    \item All parties (Astra, Benthos, Ceres) are parties to the VCLT.
    \item Many VCLT provisions reflect customary international law, applicable even if the VCLT (enforced in 1980) does not apply directly to the RDT (enforced in 1977) due to non-retroactivity.
    \item VCLT Articles 60–62 (material breach, supervening impossibility, fundamental change of circumstances) are customary law and apply regardless.
\end{itemize}

\subsection{Can Astra Terminate the RDT?}
\begin{itemize}
    \item \textbf{Treaty Provisions}:
    \begin{itemize}
        \item RDT lacks provisions on suspension or termination, excluding internal grounds for termination.
        \item Object and purpose (Article 1, RDT): Cooperation for reasonable and equitable use of the River Delta.
        \item No clear material breach, as facts do not indicate a complete failure to supply water.
    \end{itemize}
    \item \textbf{Grounds for Termination (VCLT)}:
    \begin{itemize}
        \item \textbf{Material Breach (Article 60, VCLT)}:
        \begin{itemize}
            \item Astra and Benthos failed to provide minimum water quantities downstream (Article 10, RDT), breaching Article 1’s cooperative purpose.
            \item Not a material breach, as drought reduces water availability but does not eliminate it entirely.
            \item Astra cannot rely on its own breach to terminate (Article 60).
            \item \textit{Gabcíkovo-Nagymaros Project} (Slovakia case): Material breach requires significant violation of treaty’s core obligations, not met here.
        \end{itemize}
        \item \textbf{Supervening Impossibility (Article 61, VCLT)}:
        \begin{itemize}
            \item Astra’s strongest argument: Ongoing drought may render performance impossible.
            \item Requires permanent disruption radically transforming obligations.
            \item Facts suggest drought is severe but potentially cyclical or temporary, not permanent.
            \item Drought linked to climate change does not fully eliminate water sources (e.g., Astra could use desalination or mountain water).
            \item Impossibility threshold not met, as obligations remain feasible, albeit challenging.
        \end{itemize}
        \item \textbf{Fundamental Change of Circumstances (Article 62, VCLT)}:
        \begin{itemize}
            \item Requires: (1) unforeseen circumstances, (2) radically transforming obligations.
            \item \textbf{Unforeseen Circumstances}:
            \begin{itemize}
                \item Benthos’ president’s statement (“if the river dries up”) suggests droughts were contemplated, undermining Astra’s claim of unforeseeability.
                \item Preparatory work or declaration (not a reservation per VCLT Article 2(1)(d), as it does not modify legal effects) indicates awareness of drought risks.
                \item Drought not entirely unforeseen, given climate change knowledge in 2024/2025.
            \end{itemize}
            \item \textbf{Radical Transformation}:
            \begin{itemize}
                \item Obligations to supply water remain, though strained by drought.
                \item \textit{Fisheries Jurisdiction} case: Economic instability does not constitute a fundamental change; treaty object (cooperation) persists.
                \item Threshold not met, as river has not permanently dried up, and alternative water sources exist.
            \end{itemize}
        \end{itemize}
    \end{itemize}
    \item \textbf{Article 56(1), VCLT}:
    \begin{itemize}
        \item Absent termination provisions in the RDT, withdrawal is permissible only under Articles 60–62, which Astra fails to satisfy.
    \end{itemize}
    \item \textbf{Good Faith (Article 26, VCLT)}:
    \begin{itemize}
        \item Pacta sunt servanda requires parties to perform treaties in good faith.
        \item Astra, as a developed state, could explore alternatives (e.g., desalination, sharing mountain water) rather than withholding resources.
        \item Failure to seek reasonable alternatives breaches Article 26, undermining termination claims.
    \end{itemize}
\end{itemize}

\subsection{Treaty Interpretation}
\begin{itemize}
    \item \textbf{VCLT Interpretation Rules}:
    \begin{itemize}
        \item Article 31(1): Interpret in good faith, considering ordinary meaning, context, and object/purpose (cooperation for equitable water use, Article 1, RDT).
        \item Article 32: Supplementary means (e.g., preparatory work like Benthos’ statement) confirm drought contemplation.
    \end{itemize}
    \item \textbf{UN Watercourses Convention (1997)}:
    \begin{itemize}
        \item Provides interpretative guidance with similar goals to RDT (Article 5, equitable and reasonable use).
        \item Reinforces customary expectations of cooperation in water management.
    \end{itemize}
    \item \textbf{Astra’s Declaration}:
    \begin{itemize}
        \item Interpretative declaration, not a reservation (VCLT Article 2(1)(d)), as it clarifies Astra’s understanding without modifying legal effects.
        \item Not a persistent objection to customary law, as Astra’s later signing of the 1997 Watercourses Convention negates such status.
    \end{itemize}
\end{itemize}

\subsection{Customary International Law Obligations}
\begin{itemize}
    \item \textbf{Post-Termination Obligations}:
    \begin{itemize}
        \item If RDT is terminated, customary law may still bind parties.
        \item Article 1 (RDT) is vague, but Article 10 (minimum water supply) reflects specific state practice.
        \item UN Watercourses Convention (1997): 103 votes in favour, 40 parties, growing acceptance indicates developing customary law (state practice and opinio juris, \textit{North Sea Continental Shelf}).
        \item Similarities between RDT (Article 1) and UN Convention (Article 5) suggest customary obligations for equitable water use.
        \item Astra and Benthos’ enforcement efforts imply opinio juris, supporting customary law status.
    \end{itemize}
    \item \textbf{Coexistence of RDT and UN Watercourses Convention}:
    \begin{itemize}
        \item Article 3, UN Watercourses Convention: Pre-existing agreements (e.g., RDT) remain valid and coexist without conflict.
        \item No agreement exists to supersede RDT, so both instruments apply.
    \end{itemize}
\end{itemize}

\subsection{Benthos’ Presidential Statement}
\begin{itemize}
    \item \textbf{Unilateral Declaration Assessment}:
    \begin{itemize}
        \item Not a unilateral declaration, as it lacks clear intent to bind Benthos (\textit{Nuclear Tests} case).
        \item Vague terminology (“if the river dries up” vs. “drought”) and lack of specificity distinguish it from a binding declaration.
        \item Serves as preparatory work or interpretative statement, not a reservation or persistent objection.
    \end{itemize}
\end{itemize}

\subsection{Socioeconomic and Geographic Considerations}
\begin{itemize}
    \item \textbf{Astra’s Responsibilities}:
    \begin{itemize}
        \item As a developed state, Astra bears greater responsibility to provide water due to its industrial capacity (e.g., desalination).
        \item Socioeconomic disparities do not alter Benthos and Ceres’ equal obligations under the RDT.
    \end{itemize}
    \item \textbf{Obligations Under RDT}:
    \begin{itemize}
        \item Astra has primary obligation to supply water to Benthos, who has a secondary obligation to supply Ceres.
        \item No guaranteed minimum water for Astra, emphasizing cooperative purpose (Article 1).
    \end{itemize}
\end{itemize}

\subsection{Conclusion}
\begin{itemize}
    \item Astra’s termination attempt is invalid under VCLT Articles 60–62, as no material breach, supervening impossibility, or fundamental change of circumstances is established.
    \item Article 26 (pacta sunt servanda) requires good-faith performance; Astra should explore alternatives (e.g., water sharing) rather than terminate.
    \item Customary law, reflected in the UN Watercourses Convention, reinforces RDT obligations, ensuring continued cooperation for equitable water use.
\end{itemize}

\section{International Law and Australian Law}
\begin{tutorialquestion}
    \flushleft

    The United Nations Declaration on the Rights of Indigenous Peoples (`the Declaration') was adopted by the United Nations General Assembly on 13 September 2007, by a majority of 143 states in favour, 4 votes against (Australia, Canada, New Zealand and the United States) and 11 abstentions. The four states that voted against subsequently declared their support for the Declaration (including Australia in 2007, following the election of the Rudd Government).
    
    \vspace{\baselineskip}

    The Declaration mentions Treaties between Indigenous Peoples and States in its Preamble and in Article 37. The latter provides:
    \begin{enumerate}
        \item Indigenous peoples have the right to the recognition, observance and enforcement of treaties, agreements and other constructive arrangements concluded with States or their successors and to have States honour and respect such treaties, agreements and other constructive arrangements.
        \item Nothing in this Declaration may be interpreted as diminishing or eliminating the rights of indigenous peoples contained in treaties, agreements and other constructive arrangements.
    \end{enumerate}

    You are a legal advisor in the independent Treaty Authority established in Victoria to oversee negotiations between the Victorian Government and the First Peoples' Assembly of Victoria to ensure a fair process in the conclusion of a Treaty that delivers self-determination for Victoria's First Peoples. Victoria was the first State to commit to all three elements of the Uluru Statement from the Heart (Voice, Treaty and Truth). The relevance of the Declaration to the Treaty-making process has been considered in detail by Dr Harry Hobbs. Statewide Treaty negotiations began in Victoria in November 2024.

    \vspace{\baselineskip}

    You have been asked to provide legal advice addressing the following questions:

    \begin{enumerate}
        \item What is the status of the Declaration under international law?
        \item What is the status of the Declaration under Australian law?
        \item In \case{\textit{Love v Commonwealth} [2020] HCA 3}, Bell J (at [73]) cited the Declaration when making the following observation:
        \begin{quote}
            ``It is not offensive, in the context of contemporary international understanding, to recognise the cultural and spiritual dimensions of the distinctive connection between indigenous peoples and their traditional lands, and in light of that recognition to hold that the exercise of the sovereign power of this nation does not extend to the exclusion of the indigenous inhabitants from the Australian community.''
        \end{quote}

        What legal effect was Bell J ascribing to the Declaration under Aboriginal law?
        \item In what respects does the Advancing the Treaty Process with Aboriginal Victorians Act 2018 (Vic) implement the Declaration in Victorian law?
        \item What status will a Treaty or Treaties between Indigenous Victorians and the State of Victoria have as a matter of Victorian, Australian and international law?
    \end{enumerate}
\end{tutorialquestion}

\subsection*{Question 1}
\begin{itemize}
    \item UNGA Resolutions are not binding, but they can be evidence of state practice and \gls{opinio juris} for customary law (and may possibly be a piece of soft law)
    \item This declaration is not a treaty, as there is no objective evidence that the parties sought to be bound nor that they made any decisions
    \item To determine whether it is a source of law, \statute{Art 38(1)} must be looked at:
    \begin{itemize}
        \item It does not fit in with any of the sources mentioned in \statute{Art 38(1)}
        \item It might be possible to argue that it is customary international law under \statute{Art 38(1)(b)}; nonetheless, it would be hard to make out \gls{opinio juris} on the facts
        \item It is still possible that there is state practice present, as there are 143 states voting in favour of the declaration; moreover, the reversal in position by the states that voted against it, which all had high levels of Indigenous populations, indicates a strengthening of support for the declaration and hence may evidence state practice
        \item Moreover, there is some form of international consensus, but there is no specific consensus/practice present
    \end{itemize}
    \item Some aspects of UN Declarations have attained customary international law status (\case{\textit{Horta v Commonwealth} (1994) 181 CLR 183} (Page \pageref{case:Horta v Commonwealth})), but it is unlikely that this is the case here
    \item Given the declaration has not been signed and ratified, it is fundamentally not directly binding on states
    \item Moreover, the UN sets out legal principles; the consensual, non-binding status of its decisions makes it especially active to bring states together
    \item Declarations are a serious matter; since states do not wish to get on the wrong side of the international community, they are an easy way of getting them all on the same page
    \item Any detailed analysis of this point must begin with why the declaration is non-binding, but as soft law, it nonetheless evidences state practice and \gls{opinio juris}
    \begin{itemize}
        \item A link to \case{\textit{Horta v Commonwealth} (1994) 181 CLR 183} (Page \pageref{case:Horta v Commonwealth}) is then required to evidence how UN principles may become customary international law
        \item A connection to self-determination must then be made to explore why states continue to nonetheless accept these declarations
    \end{itemize}
\end{itemize}

\subsection*{Question 2}
\begin{itemize}
    \item There is no Act directly implementing the Declaration, but there is statute, such as the \statute{\textit{Native Title Act 1993}}; moreover, aspects of the declaration had already been implemented into Australian law (which had occurred before the signing of this declaration)
    \item As there is no specific Act in Australia that implements this declaration, it is not binding on Australia's domestic law (since Australia takes a hard transformation approach, it is only binding if the legislature enacts legislation that aligns with the declaration)
    \begin{itemize}
        \item For full marks, discuss the theories and process that Australia adopts (i.e., monism vs dualism, and the resultant transformation approach)
        \item It is possible for customary international law to be a source of domestic law, or otherwise influence Australian law (\case{\textit{Mabo}})
        \item Under these cases, a universally recognised principle of international law will be applied, but does not explicitly form a part of Australian law
    \end{itemize}
    \item The declaration is quite possibly a source of non-binding soft law
    \item Under \statute{\textit{Constitution} s 51}, international legislation can be implemented into domestic legislation
    \item In \case{\textit{Chow Hung Ching v R} (1949) 77 CLR 449} (Page \pageref{case:Chow Hung Ching}), Latham CJ held that a universally recognised principle could be applied by the courts
    \begin{itemize}
        \item This approach was followed in \case{\textit{Mabo}}, where it was held that international law is an influence on development; if it is customary, it would take effect, but not be directly applicable
    \end{itemize}
    \item The \textit{Polites} principle is of relevance here: in the absence of express words to the contrary, it is assumed that domestic legislation is intended to conform with international law and should be taken as such
    \item A treaty will only be ratified when legislation with respect to it has been passed in the Commonwealth parliament
    \item Customary international law can be applied in Australian Courts (\case{\textit{Chow Hung Ching v R} (1949) 77 CLR 449} (Page \pageref{case:Chow Hung Ching})), but when there is a conflict between domestic and international law, domestic law will prevail
\end{itemize}

\subsection*{Question 3}
\begin{itemize}
    \item Bell J referred to the Declaration's legal effect, and for the capacity for it to be used as a source of international law
    \item This is a demonstration of contemporary understanding of indigenous rights, which can be a source of evidence for the current global understanding around this area
    \item \case{\textit{Al-Kateb v Godwin} (2004) 219 CLR 562} (Page \pageref{case:Al-Kateb v Godwin}) holds that a direct application of international law to interpret/modify the \textit{Constitution} would contradict \statute{\textit{Constitution} s 128}, which holds that it can only be changed by Australia; thus, any external influence cannot be detrimental to this, nor can it be used to assist in the interpretation of Australia
    \begin{itemize}
        \item Bell J was using the declaration here to inform her understanding about whether Indigenous people have a connection to their land (thus, it was not a source of law, but rather a supplementary material to help understand the context)
        \item In this minority opinion in \case{\textit{Al-Kateb v Godwin} (2004) 219 CLR 562}, Kirby J stated that the law should be pragmatic and evolve with the times (``the complete isolation of the constitution from international law is not possible/desirable''; ``national courts and constitutional courts have a duty to interpret constitutional texts with respect to international sources'')
    \end{itemize}
    \item Moreover, the exploration of this treaty does not have any legal effect; only the legislature is able to enact treaties/declarations into binding law
    \item The main issue in \case{\textit{Love v Commonwealth} [2020] HCA 3} was whether indigenous peoples could be deported as aliens
    \begin{itemize}
        \item The common law recognises native title, and hence they cannot
        \item The use of the Declaration was purely for interpretive purposes
        \item The use of extrinsic material is permissible, but has no legal effect
    \end{itemize}
    \item Furthermore, the approach taken here was noe one of soft transformation
\end{itemize}

\subsection*{Question 4}
\begin{itemize}
    \item The Victorian parliament can implement the Declaration into Victoria law using its plenary power
    \item The Declaration was not used in the Act itself, but references were instead placed in the preamble to ensure its relevance, thereby making it highly influential
    \item The terminology of Art 37 of the declaration is also used to connect the Act to the Declaration (moreover, there are similarities in the Act and the Declaration, giving it substantial adoption but not making it binding)
    \item The Act does not implement the Declaration, but instead provides the Victorian government a framework to be consistent with the Declaration; it still does not have any binding effect (in any case, it would be the Federal Government who would enact the Declaration, as Federal Parliament has the external affairs power, per \statute{\textit{Constitution} s 51(xxix)})
    \item The Act is just a framework and shows good faith on the part of the Victorian government, rather than anything of important or of relevant substance
    \item The treaty upholds the representative process, and recognises the right to self-determination
    \item It does not directly implement the declaration in Victorian law, but continues to have a strong influence
\end{itemize}

\subsection*{Question 5}
\begin{itemize}
    \item The treaty will be binding in Victoria unless there is federal law that conflicts with it, or the commonwealth legislature legislates against it (\case{\textit{Commonwealth v Tasmania} (1983) 185 CLR 1}) (Page \pageref{case:Commonwealth v Tasmania})
    \item However, this is not binding on any other territories in Australia, following \statute{\textit{Constitution} s 109}
    \item Since the Commonwealth is not a party to this treaty, it is not binding upon the Commonwealth
    \item Moreover, this treaty is not binding internationally, as the state of Victoria (being a state of a State) and Indigenous Peoples both are not peoples of international law
    \begin{itemize}
        \item Under \statute{\textit{VCLT} Art 2(1)(a)}, only states have the power to enter into a treaty, as only they can consent to its principles
        \item There is no evidence of state practice, as Victoria is once again not an international state
    \end{itemize}
\end{itemize}

\section{Personality, Statehood and Self-Determination}
\begin{tutorialquestion}
    \flushleft
    
    Aquisita is a large state that acquired a number of colonies, including Bulgan, a mountainous region, in the early twentieth century. Bulgan is home to three nomadic tribes that ranged across the territory's expansive alpine plains during the summer months and then based themselves in three loosely defined areas known as Chinggis, Delkhii and Erenhot. All three nomadic groups are ethnically and racially similar and speak a common language (although
    each group also speaks a distinct dialogue unique to their group).

    \vspace{\baselineskip}

    In 1963, the Aquisita Parliament passed a law which defined the borders of Delkhii and established the territory as a `Special Delkhii Administrative Zone'. The Aquisita government described this measure as necessary to maintain public order given past histories of conflict between the Delkhii and other tribes. There was no consultation with the nomadic tribes in Bulgan prior to this law being passed. In 1978, Bulgan was granted independence by Aquisita
    and became a member of the United Nations, however the status of the `Special Delkhii Administrative Zone' was unchanged, and ethnic Delkhii peoples were excluded from holding public office or serving in the new Bulgan government.

    \vspace{\baselineskip}

    Over the past five years the nomadic people of Delkhii have been engaged in an armed uprising against the Bulgan administration in Delkhii, and in early 2025 were successful in seizing control of Delkhii's largest city and taking control of Bulgan government offices there. While the group had control of the city, Bulgan forces remained in command of several more remote parts of the Special Delkhii Administrative Zone. The Delkhii have received very extensive financial and other support from Imperia, another state in the region, over several years. Imperia has been accused by Asquisita and the Bulgan government of `controlling the Delkhii for its own purposes.'

    \vspace{\baselineskip}

    In a public broadcast from the Delkhii Television Centre in March 2025, the leaders of the Delkhii declared that the territory was now the Independent State of Delkhii (ISD), that new boundaries for the state would be announced in coming months, and that action will be taken against `unfriendly' governments that have not supported the right of the Delkhii to self-determination. Several governments, including Imperia, have recognised ISD, however others have not.

    \vspace{\baselineskip}

    The ISD wishes to apply for membership of the United Nations and seeks your advice on the main issues of international law arising from these facts.
    
\end{tutorialquestion}

\subsection{Right to Self-Determination}
\begin{itemize}
    \item \textbf{Existence of a Right to Secession}:
    \begin{itemize}
        \item International law does not explicitly recognize a right to secession but addresses self-determination as a principle of customary international law, binding on all states (erga omnes, Lecture Note p. 47).
        \item No legal bar exists to the Delkhii people declaring independence (ISD) from Bulgan, but other issues (e.g., genocide) must be exhausted first.
    \end{itemize}
    \item \textbf{Delkhii as a ‘People’}:
    \begin{itemize}
        \item The right to self-determination under the ICCPR requires identifying the group as a ‘people’.
        \item UNESCO Final Report (1990): Provides non-exhaustive criteria for ‘peoples’, including unique dialect, ethnicity, territory, and distinct identity.
        \item Delkhii meet these criteria: unique dialect, original inhabitants, distinct group within Bulgan.
    \end{itemize}
    \item \textbf{Internal vs. External Self-Determination}:
    \begin{itemize}
        \item \textbf{Internal Self-Determination}: Exercised within the state without separation.
        \item \textbf{External Self-Determination}: Involves separation, primarily linked to decolonization (UNGA Declaration on the Granting of Independence to Colonial Countries and Peoples, Articles 1–2).
        \item Delkhii’s situation (segregation in SDAZ, exclusion from governance, lack of consultation) supports external self-determination (\textit{Western Sahara}, \textit{Quebec}).
        \item \textit{Quebec} case: Clarifies internal vs. external self-determination; serves as a subsidiary source of law, unopposed by the ICJ, reinforcing customary law status.
    \end{itemize}
    \item \textbf{Uti Possidetis Juris}:
    \begin{itemize}
        \item Respect for existing borders at the time of decolonization applies.
        \item Bulgan’s borders were defined in 1963 during decolonization, with SDAZ established; independence in 1978 fixed these boundaries.
        \item ISD cannot unilaterally declare new boundaries; 1963 borders must be respected, and 1978 changes are problematic.
    \end{itemize}
\end{itemize}

\subsection{Statehood of ISD}
\begin{itemize}
    \item \textbf{Montevideo Convention (Article 1)}:
    \begin{itemize}
        \item Reflects customary international law due to widespread state practice and opinio juris (\textit{North Sea Continental Shelf}).
        \item Criteria for statehood:
        \begin{itemize}
            \item \textbf{Permanent Population}:
            \begin{itemize}
                \item Delkhii resemble organized tribes in \textit{Western Sahara}, where nomadic populations qualified as permanent.
                \item SDAZ segregation provides some permanence, despite nomadic nature.
            \end{itemize}
            \item \textbf{Defined Territory}:
            \begin{itemize}
                \item SDAZ constitutes a defined territory (first sentence, second paragraph).
                \item ISD’s declaration of boundaries (second-last paragraph) indicates a loose territorial claim, though problematic under uti possidetis.
            \end{itemize}
            \item \textbf{Government}:
            \begin{itemize}
                \item Requires an organized and effective government (\textit{Kosovo}: clear leadership suffices).
                \item ISD’s control of government offices (third paragraph) and media (DTC) demonstrates governance.
                \item Effectiveness of government is questionable but not a barrier to statehood if basic criteria are met.
            \end{itemize}
            \item \textbf{Capacity to Enter into Relations with Other States}:
            \begin{itemize}
                \item No direct evidence, but ISD’s recognition by other states and threats against unfriendly governments suggest capacity.
                \item Declaration of independence and boundaries implies intent to engage internationally.
            \end{itemize}
        \end{itemize}
    \end{itemize}
    \item \textbf{Additional Statehood Considerations}:
    \begin{itemize}
        \item \textbf{Use of Force}: ISD’s armed uprising may violate Article 2(4) of the UN Charter, prohibiting use of force in international disputes, as states must be peace-loving.
        \item \textbf{Foreign Interference}: Imperia’s support (potentially military) to ISD may breach Article 2(4), undermining Bulgan’s sovereignty.
        \item \textbf{Racist Practices}: No evidence of racist practices by ISD.
        \item \textbf{Permanence}: ISD’s actions suggest intent for permanent statehood.
    \end{itemize}
\end{itemize}

\subsection{Recognition of ISD}
\begin{itemize}
    \item \textbf{Types of Recognition}:
    \begin{itemize}
        \item \textbf{Constitutive Recognition}: Recognition creates statehood (minority view).
        \item \textbf{Declaratory Recognition}: Recognition acknowledges existing statehood (preferred view).
    \end{itemize}
    \item \textbf{Application to ISD}:
    \begin{itemize}
        \item Several governments have recognized ISD (last sentence, second-last paragraph), indicating declaratory recognition.
        \item Declaratory recognition aligns with the facts, as ISD meets Montevideo criteria, and recognition confirms rather than creates statehood.
    \end{itemize}
\end{itemize}

\subsection{Violation of Bulgan’s Territorial Integrity}
\begin{itemize}
    \item \textbf{Assessment}:
    \begin{itemize}
        \item ISD’s declaration of independence and armed uprising may violate Bulgan’s territorial integrity under Article 2(4) of the UN Charter.
        \item Imperia’s support to ISD further infringes Bulgan’s sovereignty, constituting foreign interference.
    \end{itemize}
\end{itemize}

\subsection{Lawfulness of SDAZ Establishment}
\begin{itemize}
    \item \textbf{Assessment}:
    \begin{itemize}
        \item SDAZ’s creation in 1963 segregated the Delkhii, denying governance participation and consultation.
        \item This violates self-determination principles, as it restricts the Delkhii’s ability to govern themselves, supporting their claim for external self-determination (\textit{Quebec}).
    \end{itemize}
\end{itemize}

\section{Title to Territory}
\begin{tutorialquestion}
    \flushleft
    In their original state, the Celestials were a cluster of five small reefs (submerged at high tide) and two rocks (above water at high tide) all within 100 nautical miles (M) of Arema's coast. 
    
    \vspace{\baselineskip}
    
    The Celestials are subject to competing claims by Arema and Binton. Arema asserts sovereignty over the Celestials on the basis that they are proximate to Arema's coast, were first discovered by a private explorer from Arema in the 1800s (who left a small brass plaque on one of the rocks), and were regularly visited until the 1950s by the Arema navy. As a result of a civil war in 1960 in Arema these naval visits ceased. Arema did not establish a permanent physical presence on Celestials, apart from constructing a small lighthouse on one of the rocks.
    
    \vspace{\baselineskip}

    From 1965, Binton regularly deployed naval vessels to the Celestials and the brass plaque and lighthouse were removed. Binton publicly claimed the Celestials in 1970, on the ground that they had been abandoned by Arema. Binton then began an extensive program of land reclamation which has transformed the small features into one island. Binton has constructed several military installations on the Celestials and stationed a garrison of troops there. Arema has repeatedly protested the statements and actions of Binton, although in 1980 the Foreign Minister of Arema did make a speech in which he stated that `the time has come to recognise that Arema no longer has any interest in the remote and inhospitable Celestials.'
    
    \vspace{\baselineskip}

    In January 2025, Binton asserted a 50 nautical mile `economic and security zone' around the Celestials, and the Parliament of Binton passed a law making it a criminal offence for `any foreign national to be present in the economic and security zone of Celestials unless in possession of a valid permit issued by the Binton Department of Homeland Integrity.'
    
    \vspace{\baselineskip}

    Assess the competing claims of Arema and Binton and the legality of Binton's assertion of sovereignty and jurisdiction over the Celestials.
\end{tutorialquestion}

\subsection{Issue for Arema}
\begin{itemize}
    \item \textbf{Terra Nullius}:
    \begin{itemize}
        \item Facts indicate the Celestial rocks were uninhabited, supporting Arema’s claim of terra nullius.
        \item Uninhabitable nature strengthens Arema’s claim, as the rocks were not suitable for habitation.
        \item \textit{Mabo} rejected terra nullius for inhabited lands, but the uninhabited status of the rocks supports Arema’s position.
    \end{itemize}
    \item \textbf{Discovery}:
    \begin{itemize}
        \item Discovered by a private explorer, not a state authority.
        \item \textit{Island of Palmas}: Discovery alone does not confer title; it provides only an inchoate (imperfect) title.
        \item Private citizens cannot acquire territorial title (\textit{Ure v Commonwealth}); only state-authorized actions are valid.
        \item If the private explorer did not act on behalf of Arema, no rights are conferred to the state.
    \end{itemize}
    \item \textbf{Effective Control}:
    \begin{itemize}
        \item Evidence of Arema’s governmental actions includes naval visits and a lighthouse.
        \item \textit{Indonesia v Malaysia} and \textit{Malaysia v Singapore} (\textit{Pedra Branca}): A lighthouse demonstrates governmental function and intent to claim sovereignty.
        \item Regular and consistent naval visits indicate ongoing exercise of authority.
        \item \textit{Island of Palmas}: For small or remote uninhabitable lands, intermittent governmental presence (e.g., less frequent visits) is sufficient to establish effective control.
    \end{itemize}
\end{itemize}

\subsection{Binton’s Claims}
\begin{itemize}
    \item \textbf{Prescription}:
    \begin{itemize}
        \item Prescription involves a state acquiring territory through peaceful, public, uninterrupted exercise of sovereignty over time, unchallenged by the original claimant.
        \item \textbf{Peaceful Exercise}:
        \begin{itemize}
            \item Binton’s destruction of Arema’s lighthouse and plaque may be seen as non-peaceful.
            \item Argue it was largely peaceful: no lives were lost, only a structure and plaque (in poor condition) were removed, and Arema was absent during the act.
            \item Facts are silent on Arema’s response to the destruction, weakening claims of conflict.
        \end{itemize}
        \item \textbf{Public Exercise}:
        \begin{itemize}
            \item In 1970, Binton publicly claimed the territory, satisfying the publicity requirement.
        \end{itemize}
        \item \textbf{Uninterrupted Exercise}:
        \begin{itemize}
            \item Binton’s control was uninterrupted, as Arema’s protests were minimal and ceased in the 1980s.
            \item Arema’s lack of action post-1980s suggests acquiescence or abandonment.
        \end{itemize}
        \item \textbf{Length of Time}:
        \begin{itemize}
            \item Binton’s control since 1970 (approximately 5 years by 1975) may be insufficient compared to Arema’s longer historical claim since the 1800s.
            \item Relative duration weakens Binton’s prescription claim.
        \end{itemize}
        \item \textbf{No Public Challenge}:
        \begin{itemize}
            \item Arema protested but did not mount a military or significant challenge.
            \item By the 1980s, Arema’s public statement suggested abandonment of the claim.
        \end{itemize}
    \end{itemize}
    \item \textbf{Terra Nullius Argument}:
    \begin{itemize}
        \item Binton could claim the land was terra nullius if Arema failed to show effective control.
        \item Presence of a plaque and lighthouse indicates human activity, undermining terra nullius.
    \end{itemize}
    \item \textbf{Critical Dates}:
    \begin{itemize}
        \item \textit{Indonesia v Malaysia} and \textit{Malaysia v Singapore}: Only acts before the dispute’s crystallization are relevant.
        \item 1965 is not a critical date, as Binton’s actions (e.g., lighthouse/plaque removal) were not public.
        \item 1970 is the primary critical date, when Binton’s public claim made the dispute known.
        \item 1980 is a secondary critical date, as Arema’s cessation of protests suggests acquiescence or abandonment.
        \item By 1980, the dispute fully crystallized, with both parties’ positions clear, strengthening Binton’s claim of Arema’s abandonment.
    \end{itemize}
    \item \textbf{Unilateral Declaration}:
    \begin{itemize}
        \item \textit{Nuclear Tests Case}: A state’s authorized public declaration, if clear and unambiguous, is binding under international law.
        \item Arema’s Foreign Minister’s 1980 statement, made in a state capacity, suggests abandonment but is ambiguous (lacking explicit intent to cede the territory).
        \item Ambiguity weakens its binding effect compared to the clear declaration in \textit{Nuclear Tests}.
    \end{itemize}
\end{itemize}

\subsection{Validity of 2025 Law and Maritime Claims}
\begin{itemize}
    \item \textbf{UNCLOS and Maritime Zones}:
    \begin{itemize}
        \item Article 121, UNCLOS: Only naturally formed land above water at high tide generates maritime zones (e.g., territorial sea, exclusive economic zone).
        \item The Celestial rocks, if submerged at high tide, do not qualify as land under Article 121 and cannot generate a 12-nautical-mile (NM) territorial sea.
        \item Reclaimed land (e.g., combining rocks into an island) does not count for coastal baselines (\textit{South China Sea Arbitration}).
        \item Article 5, UNCLOS: Coastal baselines are measured from the low-tide coast, excluding fully submerged features.
    \end{itemize}
    \item \textbf{Territorial Sea and Exclusive Security Zone (ESZ)}:
    \begin{itemize}
        \item A 12-NM territorial sea is permissible from qualifying rocks (above high tide).
        \item Binton’s claimed 50-NM ESZ exceeds UNCLOS limits, as only a 12-NM territorial sea is allowed.
        \item The 12-NM zone is measured from each qualifying rock, not the mainland or combined landmass.
        \item Within the 12-NM territorial sea, innocent passage must be granted to foreign vessels.
    \end{itemize}
    \item \textbf{Legal Assessment}:
    \begin{itemize}
        \item The 2025 law establishing a 50-NM ESZ is invalid under UNCLOS, as it exceeds the 12-NM limit.
        \item Only rocks consistently above high tide generate a 12-NM territorial sea, measured from their natural baselines.
    \end{itemize}
\end{itemize}

\section{State Jurisdiction}
\begin{tutorialquestion}
    \flushleft
    Following a military coup in 2010, Istan has become a secretive state which systematically represses any opposition to its military leadership. In 2020, Istan issued a decree which made it a criminal offence, subject to punishment of life imprisonment, to protest against the government of Istan, including in any place outside the territory of Istan. The decree also gave authority to the Istan Justice Minister to sue nationals of Istan for damages for any `defamation of the Istan state'.
    
    \vspace{\baselineskip}
    
    Renee fled to Australia from Istan in 2020, was granted a permanent protection visa and renounced her citizenship of Istan. Renee has been organising weekly protests outside the Istan Embassy in Canberra. At one protest in 2024 Renee threw a can of red paint over one of the walls of the Istan Embassy.

    \vspace{\baselineskip}

    In January 2025, the Istan Prosecutor's Office (IPO) commenced criminal proceedings against Renee in the Supreme Court of Istan. After a summary hearing, Renee was convicted and sentenced to imprisonment in absentia. The IPO also prosecuted several members of Renee's family in Istan for `aiding an enemy of the Istan state.' Renee's father was detained and repeatedly tortured at the direction of Istan's Chief Prosecutor, who has overseen a large- scale program targeting opponents of the Istan government and their family members. This has involved the use of arbitrary imprisonment, enforced disappearance, torture, and extra- judicial killings. The Justice Minister also commenced civil proceedings, seeking damages against Renee for statements and activities in Australia harmful to the reputation of Istan.

    \vspace{\baselineskip}

    In February 2025 Istan's Chief Prosecutor travelled to Australia in a private capacity. The Australian Government is considering what action, if any, it may take under international law against him.

    \vspace{\baselineskip}

    You are a legal officer in Australia's Department of Foreign Affairs and Trade and have been asked to assess the issues of jurisdiction raised by these facts.
\end{tutorialquestion}

\subsection{Istan’s Jurisdiction to Prosecute Renee}

\subsubsection{Criminal Jurisdiction}
\begin{itemize}
    \item \textbf{Extraterritorial Jurisdiction Overview}:
    \begin{itemize}
        \item Three types: judicial, legislative, and executive.
        \item Extraterritorial jurisdiction allows a state to take action against its nationals or non-nationals for acts committed outside its territory.
        \item Example: \textit{XYZ v Commonwealth} – Australia applied extraterritorial jurisdiction to convict an Australian for sexual offences in Thailand, where the act was illegal under Australian law (Criminal Code 1995 (Cth)) but not Thai law.
    \end{itemize}
    \item \textbf{Principles of Jurisdiction}:
    \begin{itemize}
        \item \textbf{Territorial Principle}:
        \begin{itemize}
            \item Renee’s act (throwing paint/protesting) is illegal under Istan’s law, making her liable under the territorial principle if committed within Istan.
        \end{itemize}
        \item \textbf{Nationality Principle}:
        \begin{itemize}
            \item Renunciation of citizenship does not automatically remove nationality under international law; it is a municipal law matter determined by the state.
            \item Facts do not confirm Renee’s Australian citizenship, only her renunciation of Istan citizenship.
            \item Istan can assert extraterritorial jurisdiction based on Renee’s nationality, as renunciation is not recognised unless Istan accepts it (\textit{Nottebohm}).
            \item Cases like \textit{XYZ v Commonwealth} and \textit{Joyce v DPP} support extraterritorial jurisdiction over nationals.
            \item Istan has a strong basis to apply the nationality principle, as Renee’s renunciation is not conclusive.
        \end{itemize}
        \item \textbf{Protective Principle}:
        \begin{itemize}
            \item Throwing paint/protesting may be seen as a violation of Istan’s security but does not significantly impact vital state interests (\textit{Eichmann}, \textit{Benitez}).
            \item Unlikely to justify jurisdiction, as the act is minor and does not affect core state functions.
        \end{itemize}
        \item \textbf{Passive Personality Principle}:
        \begin{itemize}
            \item Applies if the victim is an Istan national, but the “victim” here is the embassy.
            \item No evidence of harm to individuals, though potential exists (e.g., paint hitting someone).
            \item Weak basis for jurisdiction under this principle.
        \end{itemize}
        \item \textbf{Universality Principle}:
        \begin{itemize}
            \item Does not apply, as throwing paint/protesting is not a crime of universal concern (e.g., piracy, genocide).
        \end{itemize}
    \end{itemize}
\end{itemize}

\subsubsection{Civil Jurisdiction}
\begin{itemize}
    \item \textbf{Limits on Civil Jurisdiction}:
    \begin{itemize}
        \item Australian courts will not enforce foreign civil judgments contrary to public policy.
        \item Istan can assert civil jurisdiction within its territory for claims (e.g., defamation) under its own laws.
        \item Issue: Enforcing Istan’s civil judgment in Australia requires Australian consent, and international law is largely silent on civil jurisdiction enforcement.
    \end{itemize}
    \item \textbf{Civil Action Against Renee}:
    \begin{itemize}
        \item Istan can bring a civil action against Renee as a national (though her nationality is contested).
        \item Facts do not suggest defamation, but liability depends on Istan’s municipal law.
    \end{itemize}
\end{itemize}

\subsubsection{Enforcement Jurisdiction}
\begin{itemize}
    \item \textbf{Consent Requirement}:
    \begin{itemize}
        \item Australia must consent to Istan’s enforcement actions against Renee within its territory (\textit{R v Turnbull}).
        \item Without consent, Istan’s jurisdiction cannot be enforced, and extradition attempts would violate Australia’s sovereignty.
    \end{itemize}
\end{itemize}

\subsection{Australia’s Jurisdiction to Prosecute Istan’s Chief Prosecutor}

\subsubsection{Criminal Jurisdiction}
\begin{itemize}
    \item \textbf{Principles of Jurisdiction}:
    \begin{itemize}
        \item \textbf{Territorial Principle}: Not applicable, as no crimes were committed on Australian territory.
        \item \textbf{Nationality Principle}: Not relevant, as the Chief Prosecutor is not nationals of Australia.
        \item \textbf{Protective Principle}: No vital Australian interests were breached by the Chief Prosecutor’s actions.
        \item \textbf{Passive Personality Principle}: Not applicable, as no Australian victims are involved.
        \item \textbf{Universality Principle}:
        \begin{itemize}
            \item Applicable to grave crimes (e.g., torture, crimes against humanity) under customary international law.
            \item Facts reference torture and crimes against humanity, which fall under the Rome Statute and UN Convention Against Torture (UNCAT).
            \item Australia, as a party to the Rome Statute, is obligated to prosecute crimes against humanity.
            \item UNCAT, Article 7: Requires states to prosecute or extradite for torture (Article 1(1) defines torture; facts suggest the Chief Prosecutor’s actions qualify).
            \item Australia has standing to arrest the Chief Prosecutor under both the Rome Statute and UNCAT Article 7(1).
        \end{itemize}
    \end{itemize}
    \item \textbf{Jus Cogens and Customary Law}:
    \begin{itemize}
        \item If Australia were not a party to the Rome Statute or UNCAT, torture as a jus cogens norm may still apply, though its status as jus cogens is contested.
        \item Customary international law would permit prosecution for torture, but the jus cogens status of torture remains debated (ILC is developing a list of accepted jus cogens norms).
    \end{itemize}
    \item \textbf{Diplomatic Immunity Considerations}:
    \begin{itemize}
        \item Diplomatic immunity prevents prosecution of the Chief Prosecutor while in office (see VCDR and Topic 9).
        \item Immunity must be considered alongside jurisdictional grounds.
    \end{itemize}
    \item \textbf{Treaty Signing vs. Ratification}:
    \begin{itemize}
        \item States (e.g., India) may sign treaties like UNCAT to signal agreement but not ratify to avoid binding obligations.
        \item Australia’s ratification of UNCAT and the Rome Statute imposes mandatory obligations to address torture and crimes against humanity.
    \end{itemize}
\end{itemize}

\section{Immunity from Jurisdiction I}
\begin{tutorialquestion}
    \flushleft
    The Foreign States Immunities Act (`the Act') was enacted in 1985 following an \href{http://www.austlii.edu.au/au/other/lawreform/ALRC/1984/24.html}{inquiry by the Australian Law Reform Commission} (ALRC) on the common law and statutory rules addressing foreign state immunity.

    \vspace{\baselineskip}
    
    With 2025 being the 40th anniversary of the Act, the Australian Government is considering 
    whether the Act should be reformed to keep up to date with developments in international law.
    
    \vspace{\baselineskip}
    
    You are a Legal Officer at the ALRC. Anticipating a potential referral from the Attorney General, the ALRC has requested that you write a memorandum addressing the following issues:

    \begin{enumerate}
        \item Whether the international law relating to foreign state immunity has materially changed since 1985;
        \item Whether additional exceptions to foreign state immunity should be created;
        \item Whether the provisions of the Act relating to the immunity of foreign heads of state and other government officials should be clarified, and if so, how; and
        \item Whether the Act should be amended to allow victims of human rights abuses by foreign government opportunities for redress against those victims and their serving and former government officials.
    \end{enumerate}
\end{tutorialquestion}

\subsection{Question 1: Evolution of State Immunity Since 1985}
\begin{itemize}
    \item \textbf{Material Change in Immunity Approach}:
    \begin{itemize}
        \item Since the Foreign States Immunities Act 1985 (Cth) (FSIA), there has been a shift from absolute to restrictive immunity.
        \item \textbf{Absolute Immunity}: Provides blanket protection for states against civil and criminal proceedings, regardless of the offence.
        \item \textbf{Restrictive Immunity}: Retains criminal immunity but allows civil liability, reflecting a more limited scope of protection.
        \item State practice examples:
        \begin{itemize}
            \item Russia adopted restrictive immunity in 2006.
            \item China adopted restrictive immunity in 2024, incorporating reciprocity (if one state grants immunity, the other reciprocates).
        \end{itemize}
    \end{itemize}
    \item \textbf{International Codification and Developments}:
    \begin{itemize}
        \item UN Convention on Jurisdictional Immunities of States and Their Property (2004): Codifies the shift to restrictive immunity, though not in force (24 ratifications, requires 30).
        \item \textit{Pinochet} case and UN Convention Against Torture (1987): Demonstrate a move towards restrictive immunity by limiting immunity for serious international crimes, reflecting evolving state practice since 1985.
    \end{itemize}
\end{itemize}

\subsection{Question 2: Exceptions to Immunity and Potential Criminal Exceptions}
\begin{itemize}
    \item \textbf{Current Exceptions Under FSIA}:
    \begin{itemize}
        \item Section 11, FSIA: Commercial transactions are exempt from immunity (e.g., \textit{Firebird v Nauru}).
        \item No exceptions for human rights violations, as seen in \textit{Al-Adsani v Kuwait} (torture claims rejected due to immunity), \textit{Jones v Saudi Arabia}, and \textit{Zhang v Zemin}.
    \end{itemize}
    \item \textbf{Adding Criminal Exceptions to FSIA}:
    \begin{itemize}
        \item \textbf{Permissibility Under International Law}:
        \begin{itemize}
            \item Section 9, FSIA: Establishes general state immunity in Australia.
            \item International law does not prohibit domestic criminal exceptions, as state sovereignty allows countries to enact internal laws.
            \item Domestic law cannot override international obligations (e.g., treaty-based immunities).
            \item Extraterritorial enforcement of such laws is limited by international law principles.
        \end{itemize}
        \item \textbf{Australia’s Legal System}:
        \begin{itemize}
            \item Australia follows a dualist approach with hard transformation, requiring domestic legislation to incorporate international law.
            \item Australia could add criminal exceptions to FSIA, even if global consensus is lacking.
        \end{itemize}
        \item \textbf{Jus Cogens Crimes}:
        \begin{itemize}
            \item Article 7, ILC Draft Articles on Immunity of State Officials from Foreign Criminal Jurisdiction (2017): Proposes exceptions for jus cogens crimes (e.g., genocide, torture).
            \item Reflects a potential direction for customary international law, though not yet binding.
        \end{itemize}
    \end{itemize}
\end{itemize}

\subsection{Question 3: Ambiguities in FSIA and Immunity Scope}
\begin{itemize}
    \item \textbf{Ambiguities in FSIA}:
    \begin{itemize}
        \item Sections 3(3)(b) and 36, FSIA: Lack clarity on the scope and applicability of immunity.
        \item Silent on immunity for non-troika officials (e.g., \textit{Tatchell v Mugabe}, where immunity depended on function, not title).
        \item Need for clarification on which officials (beyond Heads of State, Government, and Foreign Ministers) receive immunity.
    \end{itemize}
    \item \textbf{Personal vs. Functional Immunity}:
    \begin{itemize}
        \item \textbf{Personal Immunity (Ratione Personae)}: Applies to high-ranking officials (e.g., troika) during office, covering all acts (\textit{Arrest Warrant Case}).
        \item \textbf{Functional Immunity (Ratione Materiae)}: Applies to official acts, regardless of office status, and persists post-tenure (e.g., UK cases granting functional immunity to a Defence Minister and personal immunity to a Trade Minister).
        \item Former officials retain functional immunity for official acts but are liable for private acts or pre/post-tenure crimes.
    \end{itemize}
    \item \textbf{Arresting a Head of State with an ICJ Warrant}:
    \begin{itemize}
        \item \textbf{Domestic Law Priority}: \textit{Chow Hung Ching} principle: Construe domestic law to align with international law; otherwise, follow domestic law (e.g., FSIA s 36(1) on treatment of incumbent heads of state).
        \item \textbf{International Obligations}:
        \begin{itemize}
            \item \textit{Arrest Warrant Case}: Heads of state enjoy personal immunity in foreign jurisdictions, even for international crimes, unless waived.
            \item Rome Statute, Article 27: Removes immunity before the ICC for state parties, but Israel’s non-party status limits its application.
            \item UN Convention Against Torture (1987): Waives head of state immunity for torture, as in \textit{Pinochet}.
            \item Dapo Akande and Talita de Souza Dias: Article 27 may allow state parties to exercise domestic jurisdiction over international crimes, bypassing immunity, or enable ICC primacy when immunity is involved.
        \end{itemize}
        \item \textbf{Legal Position}: Australia is bound by state and diplomatic immunity under FSIA and customary law. An ICJ warrant does not automatically override immunity, especially for non-ICC states like Israel, but torture-related exceptions may apply.
    \end{itemize}
\end{itemize}

\subsection{Question 4: Enforcement Challenges and Broader Immunity Trends}
\begin{itemize}
    \item \textbf{ICSID Enforcement Issues}:
    \begin{itemize}
        \item \textit{Kingdom of Spain v Infrastructure Services Luxembourg} (2023 HCA 11): Australia and Spain, as ICSID members, recognize arbitral awards, but execution against state assets raises sovereignty concerns.
        \item ICSID signatories waive immunity for recognition and enforcement, but execution remains problematic (e.g., seizing state property violates sovereignty).
        \item First High Court judgment clarifying FSIA waiver for ICSID awards, distinguishing recognition, enforcement, and execution.
    \end{itemize}
    \item \textbf{Broader Trends in State Immunity}:
    \begin{itemize}
        \item \textbf{Move to Restrictive Immunity}: Codified by the UN Convention (2004), though not in force, and adopted by Russia (2006) and China (2024).
        \item \textbf{Municipal vs. International Law}: Australia’s dualist system requires transformation of international obligations, but domestic law cannot override treaty commitments.
        \item \textbf{Jus Cogens Exceptions}: ILC Draft Article 7 (2017) advocates exceptions for jus cogens crimes, reflecting a push to limit immunity for human rights abuses.
        \item \textbf{Rome Statute vs. State Immunity}: Article 27 removes immunity (;;before the ICC, but its effect on non-parties or domestic courts is limited (\textit{Arrest Warrant Case}).
        \item \textbf{Human Rights Abuses}: Cases like \textit{Germany v Italy}, \textit{R v Jones}, and \textit{Zhang v Zemin} highlight tensions between immunity and accountability for serious violations.
    \end{itemize}
    \item \textbf{Fundamental Cases}:
    \begin{itemize}
        \item \textit{Arrest Warrant Case} (2002): Affirms personal immunity for high-ranking officials in foreign jurisdictions.
        \item \textit{Germany v Italy} (2012): Rejects jus cogens exceptions to state immunity in civil cases.
        \item \textit{R v Jones}: Limits domestic prosecution of international crimes due to immunity.
        \item \textit{Zhang v Zemin}: Upholds FSIA immunity with no implied exceptions.
        \item \textit{Kingdom of Spain v Infrastructure}: Clarifies FSIA application to ICSID awards.
    \end{itemize}
\end{itemize}

\section{Immunity from Jurisdiction II}
\begin{tutorialquestion}
    \flushleft
    Siegfried Private Military Company (the `Siegfried Group') is a company incorporated in the Cayman Islands that provides security services. It is widely known for supplying mercenaries for use by authoritarian governments worldwide, and there are credible reports by Human Rights Watch that members of the Siegfried Group, including its leadership, have been directly involved in torture and other crimes in multiple jurisdictions.

    \vspace{\baselineskip}

    The head of Siegfried Group, Dimitri Atkin, is currently visiting Australia and staying at the Embassy of Istan in Canberra. Following a military coup in 2010, Istan has become a secretive state which systematically represses any opposition to its military leadership. A group of protesters gathered outside the Istan Embassy shortly after Atkin arrived, to protest against the Istan government and the Siegfried Group, which has been engaged by the Istan government to support the systematic persecution of opponents of the Istan regime.
    
    \vspace{\baselineskip}
    
    Atkin, the Istan Ambassador, and the Ambassador's Executive Assistant, fired rounds from assault rifles into the crowd, with several bullets striking and injuring multiple protesters.

    \vspace{\baselineskip}

    In addition to his official role, the Istan Ambassador operates a business in Australia that imports agricultural products from Istan.

    \vspace{\baselineskip}

    Assess the immunity and other international legal issues arising from the following:

    \begin{enumerate}
        \item The protest at the Istan Embassy;
        \item Proceedings commenced by a plaintiff in the Supreme Court of NSW seeking damages for alleged torture by Atkin when the plaintiff was in Istan;
        \item Proceedings commenced by plaintiffs in the Supreme Court of NSW against Istan, the Istan Ambassador, the Ambassador’s Executive Assistant and Atkin seeking damages for personal injuries suffered while protesting outside the Istan Embassy;
        \item A criminal investigation in the Australian Capital Territory of the Istan Ambassador, the Ambassador's Executive Assistant and Atkin in relation to the events at the Istan Embassy; and
        \item Damages proceedings in the Supreme Court of NSW against the Istan Ambassador arising out of a contractual dispute with his importing business.
    \end{enumerate}
\end{tutorialquestion}

\subsection{Question 1: Diplomatic Inviolability and Obligations of the Receiving State}
\begin{itemize}
    \item \textbf{Inviolability of Embassy Premises}:
    \begin{itemize}
        \item Article 22(1) of the Vienna Convention on Diplomatic Relations (VCDR): Embassy premises are inviolable.
        \item Australia, as the receiving state, cannot enter or search the embassy without consent (\textit{R v Turnbull; Killing of British Police Officer}).
        \item Embassy premises remain under Australia’s jurisdiction but are inviolable (\textit{Magno}).
    \end{itemize}
    \item \textbf{Obligations of the Receiving State}:
    \begin{itemize}
        \item Article 22(2) of the VCDR: Australia has a strict obligation to protect embassy premises and ensure the peace of diplomatic missions.
        \item This is an obligation of result, not conduct (\textit{Tehran Hostages}, \textit{Magno}): Australia must achieve the outcome of no harm, regardless of efforts taken.
        \item Mere protests or demonstrations do not breach Article 22(2), but insulting behaviour impairing the mission’s dignity may constitute a violation.
        \item Australia must prohibit protests that impair the mission’s dignity or cause harm.
    \end{itemize}
\end{itemize}

\subsection{Question 2: Civil Proceedings and Potential Defendants}
\begin{itemize}
    \item \textbf{Nature of Proceedings}:
    \begin{itemize}
        \item Claims involve damages, indicating civil proceedings.
    \end{itemize}
    \item \textbf{Atkin’s Status and Immunity}:
    \begin{itemize}
        \item Articles 1(e), 4, and 7 of the VCDR: Atkin’s diplomatic status is unclear, as facts are silent on whether he is accredited by Australia as a head of mission or formally appointed.
        \item Likely not a diplomat, as he appears to be visiting, and thus enjoys no diplomatic immunity.
        \item Universal jurisdiction may apply for grave violations.
        \item Potential defences for Atkin:
        \begin{itemize}
            \item \textit{Jones v Saudi Arabia} and \textit{Jones v UK}: If Atkin acted with state complicity, suing him cannot circumvent state immunity; torture is not excluded from civil immunity.
            \item Foreign act of state doctrine (\textit{Habib v Commonwealth}): Unlikely to apply due to grave violations of public international law.
            \item Functional immunity: May apply for acts performed in service of Istan, but facts suggest his actions were not official.
        \end{itemize}
    \end{itemize}
    \item \textbf{Potential Defendants}:
    \begin{itemize}
        \item \textbf{Atkin as a Private Citizen}:
        \begin{itemize}
            \item Foreign States Immunities Act 1985 (Cth), s 9: Atkin is immune unless a separate entity from Istan.
            \item \textit{PT Garuda}: Factors to determine separate entity status include ownership, control, function, purpose, and manner of conduct (latter three are more significant).
            \item Atkin’s engagement to conduct Istan’s bidding suggests possible state connection, but no conclusive determination can be made.
        \end{itemize}
        \item \textbf{State of Istan}:
        \begin{itemize}
            \item Foreign States Immunities Act, s 9: Istan enjoys immunity.
            \item \textit{Zhang v Zemin}: No implied exceptions exist; the Act’s exceptions are exhaustive.
            \item Istan is immune from civil proceedings.
        \end{itemize}
        \item \textbf{Siegfried Group}:
        \begin{itemize}
            \item \textit{Habib}, \textit{Hicks}: Foreign act of state doctrine does not apply where public international law is breached.
            \item Doctrine applies only to acts by nationals on foreign sovereign territory, which may not fit here.
        \end{itemize}
    \end{itemize}
    \item \textbf{Approach to Defendants}:
    \begin{itemize}
        \item All possible defendants (Atkin, Istan, Siegfried Group) should be addressed due to uncertainty in the facts.
    \end{itemize}
\end{itemize}

\subsection{Question 3: Liability of Istan, Ambassador, and Executive Assistant (EA)}
\begin{itemize}
    \item \textbf{State of Istan}:
    \begin{itemize}
        \item Foreign States Immunities Act, s 9: Istan is generally immune.
        \item Exception under s 13: Personal injury caused by a state’s actions allows liability (\textit{Yugoslavia} case: States can be vicariously liable for diplomatic agents).
        \item Facts analogous to \textit{Tokic}: Istan can be sued for damages caused by its agents.
    \end{itemize}
    \item \textbf{Ambassador and Executive Assistant (EA)}:
    \begin{itemize}
        \item VCDR Articles 1(a) and 1(d): Both may be diplomatic agents, but analysis should treat them separately.
        \item \textbf{Ambassador}:
        \begin{itemize}
            \item Article 31(1): Enjoys immunity from civil jurisdiction, with exceptions (e.g., commercial activities) that do not apply here.
            \item Immunity likely applies unless declared persona non grata (Article 9).
        \end{itemize}
        \item \textbf{Executive Assistant (EA)}:
        \begin{itemize}
            \item Article 37(2): EA enjoys functional immunity for acts within official duties.
            \item Shooting may fall outside official duties, potentially negating immunity.
            \item \textit{Yugoslavia}: State may be vicariously liable for EA’s actions.
            \item If EA is an Australian national, no immunity applies under VCDR.
            \item Need to clarify EA’s staff category and whether the shooting was within official functions.
        \end{itemize}
    \end{itemize}
    \item \textbf{Atkin}:
    \begin{itemize}
        \item Likely not a diplomatic agent, so no VCDR immunity applies.
        \item Functional immunity may apply if acting as a state agent, but shooting random individuals (not state enemies) falls outside governmental services.
        \item No immunity for Atkin in civil proceedings.
    \end{itemize}
\end{itemize}

\subsection{Question 4: Criminal Jurisdiction and Immunity}
\begin{itemize}
    \item \textbf{Ambassador}:
    \begin{itemize}
        \item Articles 31(1) and 37(2): Enjoys complete criminal immunity, regardless of the crime’s severity.
        \item Inviolability extends to the ambassador’s person and embassy premises.
    \end{itemize}
    \item \textbf{Executive Assistant (EA)}:
    \begin{itemize}
        \item Article 37(2): EA enjoys functional criminal immunity for acts within official duties.
        \item Same analysis as Question 3: Shooting likely falls outside official duties, but immunity applies if within official functions.
        \item No immunity if EA is an Australian national.
    \end{itemize}
    \item \textbf{Atkin}:
    \begin{itemize}
        \item Not a diplomatic agent, so no VCDR immunity applies.
        \item If Atkin remains in the embassy:
        \begin{itemize}
            \item Embassy’s inviolability prevents arrest (\textit{Assange} case).
            \item Australia may declare Atkin persona non grata (Article 9), negating any protection from embassy inviolability.
        \end{itemize}
        \item No criminal immunity otherwise.
    \end{itemize}
\end{itemize}

\subsection{Question 5: Civil Jurisdiction for Commercial Activities}
\begin{itemize}
    \item \textbf{Ambassador’s Immunity}:
    \begin{itemize}
        \item Article 31(1)(c) of the VCDR: Ambassador has immunity from civil jurisdiction, except for commercial activities outside official functions.
        \item If the ambassador’s actions were for personal profit (e.g., conducted ``in addition to his official role''), they fall outside official functions, negating immunity.
        \item More facts needed to confirm the nature of the actions.
    \end{itemize}
    \item \textbf{Foreign States Immunities Act}:
    \begin{itemize}
        \item Section 11: Commercial transactions are an exception to state immunity.
        \item If the ambassador’s actions constitute a commercial transaction, liability applies with no exceptions.
    \end{itemize}
    \item \textbf{Conclusion}:
    \begin{itemize}
        \item The ambassador is likely liable for civil claims if the actions were commercial and outside official duties.
    \end{itemize}
\end{itemize}

\section{State Responsibility I}
\begin{tutorialquestion}
    \flushleft

    Agricola, Pacifica, and Machina are neighbouring states. Agricola has been experiencing a period of internal strife for some years generated by a revolutionary movement within Agricola that is seeking to overthrow the current government. The guerrilla arm of the movement is known as `Insurcs', and there is evidence that the government of Pacifica has been providing funding, training and small arms to the Insurcs, and has directed the Insurcs to attack certain government facilities in Agricola.
    
    \vspace{\baselineskip}

    The military forces of Agricola have been stretched in dealing with the internal strife caused by the Insurcs and in January 2024, the State Department of Agricola engaged Control Inc, a private security company incorporated in Machina, to assist the Agricolan military in suppressing the Insurcs.

    \vspace{\baselineskip}
    
    Shoemakers Ltd is a limited liability company incorporated in Machina. Several years ago Shoemakers Ltd established a shoe manufacturing plant in a small town in Agricola near the border with Pacifica. In April 2024, the factory was badly damaged by grenades in a military operation carried out by Control Inc against Insurcs, and all shoe production has had to cease.
    
    \vspace{\baselineskip}
    
    During the April 2024 military operation, some members of Control Inc’s security forces left the frontline of the battle, expressly against orders from their commander, and began looting property and attacking citizens in the `Pacifican Quarter', an area of the town where many persons of Pacifica nationality lived and worked. During these attacks several rockets were fired across the border into Pacifica.

    \vspace{\baselineskip}

    In May 2024 the government of Agricola issued the following statement:
    
    \vspace{\baselineskip}
    
    ``Agricola regrets any damage that has been occasioned to foreign nationals and their property as a result of recent counter insurgency operations against the Insurcs. The Insurcs pose a grave threat to the very existence of Agricola, and the government of Agricola is fully entitled to use all means at its disposal, including the services of the private security firm Control Inc, to protect the Agricolan nation."

    \vspace{\baselineskip}
    
    The government of Agricola seeks your advice concerning the issues of state responsibility raised by these events.

\end{tutorialquestion}

\subsection{General Principles of State Responsibility}
\begin{itemize}
    \item \textbf{Key Elements}:
    \begin{itemize}
        \item Violation or breach of a rule of public international law (PIL).
        \item Attribution of the conduct to a state.
        \item Availability of defences under the Articles on State Responsibility (ASR).
        \item Invocation and standing to raise the matter in an international forum.
        \item Consequences, including remedies such as cessation, reparation, or apology.
    \end{itemize}
    \item \textbf{Attribution Focus}:
    \begin{itemize}
        \item Attribution to the state is a critical element for establishing responsibility.
        \item A state with standing/jurisdiction must exist to raise the violation in a forum.
    \end{itemize}
    \item \textbf{Common Remedy}:
    \begin{itemize}
        \item An apology is the most common remedy for state responsibility.
    \end{itemize}
    \item \textbf{Legal Framework}:
    \begin{itemize}
        \item The ILC Articles on State Responsibility (ASR) codify customary international law, providing the framework for analysis.
    \end{itemize}
\end{itemize}

\subsection{Responsibility of Pacific (P) for Funding/Training Insurcs in Arcadia (A)}
\begin{itemize}
    \item \textbf{Attribution to Pacific}:
    \begin{itemize}
        \item Issue: Does funding/training the Insurcs by P’s government trigger state responsibility under Article 8 of the ASR?
        \item Article 8: Conduct of private persons/entities is not attributable to a state unless they act under state direction or control (strict test, per \textit{Nicaragua} and \textit{Bosnia} genocide cases).
        \item Two factors for attribution:
        \begin{itemize}
            \item P’s government provided arms to the Insurcs.
            \item P’s government directed the Insurcs to attack facilities in A.
        \end{itemize}
        \item \textbf{Effective Control Test} (\textit{Nicaragua}):
        \begin{itemize}
            \item Conduct must be an integral part of a state-directed operation (ILC Commentary on Article 8).
            \item Insurcs’ actions may not meet the threshold of complete dependency on P, as in \textit{Bosnia} (where entities must be wholly dependent on the state).
            \item Insurcs did not function as a quasi-state organ, suggesting independence from P’s funding for the specific attack.
        \end{itemize}
        \item Conclusion: Attribution to P is unlikely due to insufficient evidence of complete control or dependency.
    \end{itemize}
    \item \textbf{Breach of International Law}:
    \begin{itemize}
        \item Per \textit{Corfu Channel}, states must not allow their territory to be used to harm others, constituting a significant breach if P supported the Insurcs’ actions.
        \item \textit{Nicaragua}: States are responsible for non-interference in other states’ affairs and for their own actions.
    \end{itemize}
    \item \textbf{Defences}:
    \begin{itemize}
        \item No defence under the ASR applies, particularly necessity (\textit{Hungary v. Slovakia} test).
        \item Supporting rebels is not essential for P’s survival, and no facts support other defences (e.g., consent or force majeure).
    \end{itemize}
    \item \textbf{Consequences}:
    \begin{itemize}
        \item Cessation of funding (Article 30, ASR).
        \item Reparation (Article 31), including compensation (Article 36) or satisfaction (Article 37), as restitution (Article 35) is infeasible.
        \item Similarity to \textit{Trail Smelter}: Compensation for damages may apply.
    \end{itemize}
\end{itemize}

\subsection{Responsibility of Arcadia (A) for Damages to Shoemakers (S)}
\begin{itemize}
    \item \textbf{Attribution to Arcadia}:
    \begin{itemize}
        \item Issue: Can actions of Control Inc. (C), a private company incorporated in Maris (M) but operating in A, be attributed to A for damages to Shoemakers’ factory?
        \item Article 4 (ASR): C is not a state organ, as it is a private company engaged for military support.
        \item Article 5 (ASR): C may be attributable if empowered by A’s law to exercise governmental authority and acted in that capacity.
        \item Facts confirm C was engaged to assist A’s military operations, satisfying the empowerment criterion.
        \item A’s expression of regret may trigger Article 11 (ASR), where state acknowledgment or adoption of private conduct results in attribution (distinct from mere support, per ILC Commentary).
        \item Conclusion: C’s actions are likely attributable to A under Article 5 or Article 11.
    \end{itemize}
    \item \textbf{Standing and Invocation}:
    \begin{itemize}
        \item M has standing to invoke responsibility on behalf of S under Article 42(a) (ASR), as S is incorporated in M (\textit{Barcelona Traction}).
        \item M’s connection to S’s damages (as S’s state of incorporation) grants M the right to bring a claim against A.
    \end{itemize}
    \item \textbf{Breach of International Law}:
    \begin{itemize}
        \item A’s failure to prevent C’s actions may breach its obligation to protect foreign nationals’ property within its territory.
    \end{itemize}
    \item \textbf{Defences}:
    \begin{itemize}
        \item No consent from S exists to excuse A’s responsibility.
        \item Self-defence or necessity may be argued, as C was responding to attacks by Insurcs, but necessity requires an essential interest and grave peril (\textit{Rainbow Warrior}).
        \item Necessity is unlikely, as C’s actions were not the only means to protect A’s interests.
    \end{itemize}
    \item \textbf{Consequences}:
    \begin{itemize}
        \item Cessation of wrongful acts (Article 30, ASR).
        \item Reparation (Article 31, ASR):
        \begin{itemize}
            \item Restitution (Article 35) is unlikely due to the warzone context.
            \item Compensation (Article 36) is most appropriate, covering factory damages and lost trade (see \textit{Chorzów Factory} case).
            \item Satisfaction (Article 37), such as A’s statement of regret, is insufficient, as it does not address specific damages.
        \end{itemize}
    \end{itemize}
\end{itemize}

\subsection{Responsibility of Arcadia (A) for Looting in Pacifica (PQ) and Rocket Attacks}
\begin{itemize}
    \item \textbf{Attribution to Arcadia}:
    \begin{itemize}
        \item Issue: Is A responsible for looting and rocket attacks by Control Inc. (C) in PQ, conducted against A’s orders?
        \item Article 5 (ASR): C, a parastatal entity, is empowered by A’s law to act in a governmental capacity.
        \item Article 7 (ASR): Actions exceeding authority or contravening instructions remain attributable if performed in an official capacity.
        \item Per \textit{Caire} and \textit{Youmans v. Mexico}, disobedience does not preclude attribution if C acted in its state-empowered capacity.
        \item Looting: C’s shift to looting does not remove state capacity, as they retained their role as a military contractor.
        \item Rocket Attacks: Likely not attributable, as they may exceed C’s empowered role, but facts are unclear on who conducted them.
        \item Distinction: Unlike \textit{Caire} and \textit{Youmans} (where actors were state organs under Article 4), C operates under Article 5. However, Article 7 applies, and attribution holds if C acted in its empowered capacity.
    \end{itemize}
    \item \textbf{Breach of International Law}:
    \begin{itemize}
        \item \textit{Trail Smelter}: States are responsible for preventing harmful acts originating from their territory.
        \item Article 2(4) of the UN Charter: Rocket attacks into PQ may constitute a prohibited use of force, violating territorial integrity.
        \item Looting and attacks on PQ citizens breach A’s obligation to prevent harm by entities under its control.
    \end{itemize}
    \item \textbf{Defences}:
    \begin{itemize}
        \item Self-defence is precluded, as the actions were not a proportionate response to an armed attack (\textit{Caroline} case).
        \item Necessity is weak, as looting and rocket attacks were not the only means to protect A’s interests, and disobedience suggests a disciplinary failure, not a grave peril.
        \item Force majeure has a high threshold and is inapplicable, as disobedience reflects A’s failure to control C, not an external force.
        \item No circumstances precluding wrongfulness are established.
    \end{itemize}
    \item \textbf{Standing and Invocation}:
    \begin{itemize}
        \item PQ has standing to bring a claim for looting and rocket attacks, as they infringe its sovereignty.
        \item M may bring a claim for damages to S’s factory, as S is its national.
        \item A’s statement of regret does not negate standing for specific claims.
    \end{itemize}
    \item \textbf{Consequences}:
    \begin{itemize}
        \item Cessation of wrongful acts (Article 30, ASR).
        \item Reparation (Article 31, ASR), including compensation for damages (Article 36, see \textit{Chorzów Factory}) and possibly satisfaction (Article 37, e.g., a targeted apology).
        \item A’s existing statement of regret is insufficient, as it lacks specificity to the incidents.
        \item Reference \textit{Rainbow Warrior} for cessation and reparation obligations.
    \end{itemize}
\end{itemize}

\section{State Responsibility II}
\begin{tutorialquestion}
    \flushleft
    Article 19 of the Articles on Diplomatic Protection provides as follows:

    \begin{center}
        *** \\

        \vspace{\baselineskip}

        Article 19

        \vspace{\baselineskip}

        Recommended Practice
    \end{center}
    
    \vspace{\baselineskip}

    A State entitled to exercise diplomatic protection according to the present draft articles, should:
    \begin{enumerate}[label=(\alph*)]
        \item give due consideration to the possibility of exercising diplomatic protection, especially when a significant injury has occurred;
        \item take into account, wherever feasible, the views of injured persons with regard to resort to diplomatic protection and the reparation to be sought; and
        \item transfer to the injured person any compensation obtained for the injury from the responsible State subject to any reasonable deductions.
    \end{enumerate}
    \centerline{***}

    \vspace{\baselineskip}
    
    Critically assess the object and purpose of Article 19, its legal status, and explain how it could be given effect under Australian law.
\end{tutorialquestion}

\subsection{Objective Purpose}
\begin{itemize}
    \item \textbf{Freedom of State Action}:
    \begin{itemize}
        \item The principle of state freedom of action is central, as established in \textit{Barcelona Traction}, affirming states' discretionary right to exercise diplomatic protection.
        \item The draft article is framed as a recommendation, per Commentary 1, using non-binding language such as ``it is recommended''.
        \item The article's title, ``Recommended Practice'', underscores its non-binding, persuasive nature.
    \end{itemize}
    \item \textbf{Commentary and Intent}:
    \begin{itemize}
        \item Commentary 2 adopts an individualistic approach but remains non-binding on states.
        \item The purpose is to shift from a state-based to a human rights-centric approach, recognising diplomatic protection as a state right rather than an obligation (\textit{Barcelona Traction}).
        \item The recommendatory nature reflects the drafters' intent, with the title explicitly labelling it as ``Recommended Practice''.
        \item Ambiguous language, such as ``should'' instead of ``must'', indicates soft law, aiming to persuade states towards a human rights-centric approach.
    \end{itemize}
    \item \textbf{Purpose and Cost Recovery}:
    \begin{itemize}
        \item The article allows states to recoup costs incurred in exercising diplomatic protection, reflecting varying levels of state effort.
        \item This aligns with the goal of codifying existing state practice, though it remains recommendatory.
    \end{itemize}
    \item \textbf{Australian and Comparative Perspectives}:
    \begin{itemize}
        \item Australia rejected a human rights-centric approach in the \textit{Hicks} case, prioritising state discretion.
        \item Contrast with the UK’s \textit{Secretary of State} case, which adopted a more individual-focused approach, highlighting divergent state practices.
    \end{itemize}
    \item \textbf{Customary International Law Status}:
    \begin{itemize}
        \item The article attempts to codify existing customary international law but lacks binding force.
        \item States are not ready to accept it as binding, reflecting their preference for discretionary application.
        \item The draft articles evidence state practice but lack \textit{opinio juris}, undermining claims of customary international law status.
    \end{itemize}
    \item \textbf{Why No Treaty or Convention?}:
    \begin{itemize}
        \item Codifying the Articles of State Responsibility (ASR) as a treaty would impose obligations, risking state liability for violations.
        \item States resist binding commitments, fearing rejection or non-compliance, as noted in relevant textbooks.
        \item The recommendatory nature avoids imposing obligations, aligning with state reluctance to be bound.
    \end{itemize}
\end{itemize}

\subsection{Legal Status}
\begin{itemize}
    \item \textbf{Customary Law Assessment}:
    \begin{itemize}
        \item Commentary 4 suggests Article 19(b) may reflect customary international law, with evidence of state practice supporting its application absent the recommendatory framework.
        \item Article 19(c) is problematic, as it does not align with customary law due to inconsistent state practice.
    \end{itemize}
    \item \textbf{Counterexamples and Precedents}:
    \begin{itemize}
        \item The \textit{Rainbow Warrior} case (France vs. New Zealand) contradicts Article 19(c), highlighting divergent state practices.
        \item In \textit{Administrative Decision No. 5} (US), compensation received was treated as a trust, supporting cost recovery.
        \item The ECtHR’s \textit{Beau Martin} case (France) also permits states to recoup costs incurred in diplomatic protection.
    \end{itemize}
\end{itemize}

\subsection{How to Give Effect}
\begin{itemize}
    \item \textbf{Legislative Implementation}:
    \begin{itemize}
        \item Draft and pass legislation to incorporate the draft articles into Australian law.
        \item The Commonwealth can use the external affairs power (s 51(xxix) of the Constitution) to implement the article through the Department of Foreign Affairs and Trade (DFAT).
    \end{itemize}
    \item \textbf{Common Law Development}:
    \begin{itemize}
        \item Administrative law, as in the \textit{Abbasi} case, recognises state obligations to provide diplomatic protection.
        \item \textit{Hicks v Ruddock} did not adopt this approach, but future cases could explore it.
    \end{itemize}
    \item \textbf{Judicial Activism}:
    \begin{itemize}
        \item The High Court of Australia could apply the draft articles, adopting a human rights-centric approach through judicial interpretation.
    \end{itemize}
    \item \textbf{Incorporation in a Dualist System}:
    \begin{itemize}
        \item Australia, as a dualist state, requires domestic incorporation of international law.
        \item Implementation involves all three branches of government:
        \begin{itemize}
            \item \textbf{Legislative}: Enact laws to give effect to the articles.
            \item \textbf{Executive}: DFAT can develop policies under the external affairs power.
            \item \textbf{Judicial}: Courts can interpret and apply the articles if incorporated or persuasive.
        \end{itemize}
    \end{itemize}
\end{itemize}

\section{Use of Force}
\begin{tutorialquestion}
    \flushleft
    Xeta and Zenith are two rivalrous states vying for economic and military superiority. Tensions have arisen between the two countries over several issues, including the treatment by Zenith of a substantial population of ethnic Xetans (known as the `Xetani') in Zenith.

    \vspace{\baselineskip}

    Xeta has raised the plight of the Xetani in the United Nations General Assembly and the Security Council for over a decade. Both Xeta and Zenith are members of the United Nations.

    \vspace{\baselineskip}
    
    In 2024 the General Assembly adopted a resolution calling on `Zenith to uphold the human rights of the Xetani and respect their right to self-determination'. The situation in Zenith has also been considered by the Security Council on several occasions. However, one permanent member has repeatedly voted against the adoption of a resolution that would (a) recognise the situation as a `breach of international peace and security' and (b) call on all UN members `to take all measures necessary to protect the Xetani in Zenith.'

    \vspace{\baselineskip}

    Concerned at the welfare of the Xetani, the Xetan Defence Force (`XDF') commenced the highly secretive `Operation Panoptes', which involved a suite of activities designed to protect this population. The XDF:
    \begin{itemize}
        \item Deployed uncrewed high-altitude balloons to undertake surveillance of prison camps in Zenith holding Xetani. During one operation a balloon exploded, and the equipment it was carrying fell on a military barracks in Zenith, killing dozens of soldiers;
        \item Deployed an uncrewed aerial vehicle (`UAV') to drop munitions on the home of a Zenith miliary commander suspected of the indiscriminate killing of Xetani in the prison camps;
        \item Provided funding and small arms to the Xetani Revolutionary Force (`XRF'), a small guerrilla group in Zenith; and
        \item Undertook a cyberattack against internet servers in Zenith with the aim of disrupting the operation of the prison camps holding Xetani. The attack had wider effects, disabling all Zenith government computer systems for over a week, resulting in disruption to multiple essential services including the water supply to Zenith's largest city.
    \end{itemize}

    On detecting Xetan government involvement in the cyberattack, Zenith released a statement that `condemned the assault on Zenith's territorial sovereignty' and promised `that severe retaliatory action would be taken against Xeth'. The Zenith air force bombed several military facilities in Xeta. Several weeks later, the Zenith air force undertook a second round of bombing against a wider range of targets, including the Xetan Parliament. 
    
    \vspace{\baselineskip}
    
    You are an advisor to the United Nations Secretary General, who has asked you to identify the main issues of international law arising from this situation.

\end{tutorialquestion}

\section{Implementation, Enforcement and Accountability}
\begin{tutorialquestion}
    \flushleft
    Locrian and Mixolydian are parties to a treaty that seeks to promote sustainable fishing for greenfin tuna in the Dorian Sea (the Treaty for the Conservation of Greenfin Tuna or TCGT).

    \vspace{\baselineskip}

    The treaty provides in Article 1 that the parties `commit to the conservation and sustainable use of greenfin tuna'. Article 2 states that the parties will meet annually to agree on a total allowable catch for greenfin tuna and national fishing quotas. Article 10 provides that `in the event of a dispute concerning the implementation of the treaty that has not been settled through negotiation, a party may submit the matter to the International Court of Justice.'

    \vspace{\baselineskip}

    Mode Island is a large island approximately in the middle of the Dorian Sea. It is subject to competing sovereignty claims by Locrian and another state, Aeolian. It has been occupied by Aeolian for several decades. In 2023, in order to sustain the growing population on Mode Island, fishing vessels from Aeolian began catching greenfin tuna, and continue to do so. Some of these vessels have been offloading their catches in Mixolydian ports for processing and canning. These canning factories are poorly regulated, and release large volumes of pollutants into the Dorian Sea, damaging the greenfin tuna spawning grounds.

    \vspace{\baselineskip}

    At a meeting of the Dorian Regional Economic Organisation in July 2024, a regional forum for economic development, the Locrian Foreign Minister gave a speech attacking Mixolydian for undermining the TCGT `by stealth' through the `laundering of the endangered greenfin tuna in Mixolydian ports'. Mixolydian responded with a statement insisting that the government was complying with its obligations and that it would engage in no further discussions on the matter outside the annual meeting of parties under the TCGT.

    \vspace{\baselineskip}

    Locrian, Mixolydian and Aeolian are all members of the United Nations, and Locrian and Mixolydian have deposited declarations under Article 36(2) of the Statute of the International Court of Justice (`ICJ') accepting the compulsory jurisdiction of the Court. Mixolydian's declaration includes no reservations. Locrian's declaration, lodged in January 2024, provides that Locrian does not accept the jurisdiction of the ICJ with respect to disputes `(a) where the parties have agreed to utilise another method of binding dispute settlement, or (b) arising out of facts or situations occurring prior to the making of this declaration.' Locrian wishes to commence proceedings against Mixolydian in the ICJ seeking orders that:

    \begin{enumerate}[label=(\alph*)]
        \item Mixolydian has breached the TCGT by enabling fishing for greenfin tuna in excess of the agreed national quotas;
        \item Mixolydian has breached the TCGT by failing to control pollution from canning factories within its territory; and
        \item Mixolydian has violated Locrian's sovereignty over Mode Island;
    \end{enumerate}

    Locrian also wishes to request provisional measures from the ICJ. Advise Locrian.
\end{tutorialquestion}

\subsection{Part A: Establishing ICJ Jurisdiction}
\begin{itemize}
    \item \textbf{Jurisdictional Basis}:
    \begin{itemize}
        \item Assess whether the International Court of Justice (ICJ) has jurisdiction.
        \item Jurisdiction may arise under Article 10 of the Treaty on Conservation and Governance of Tuna (TCGT).
        \item Dispute crystallised in July 2024, marking the relevant timeline for jurisdiction.
    \end{itemize}
    \item \textbf{Treaty-Based Jurisdiction}:
    \begin{itemize}
        \item Jurisdiction stems from Article 10 of the TCGT, as both states are parties to the treaty.
        \item Requires confirmation that the dispute falls within the treaty's scope.
    \end{itemize}
    \item \textbf{Precondition of Negotiations}:
    \begin{itemize}
        \item Negotiations are a prerequisite for ICJ jurisdiction (per \textit{Russia v. Ukraine}, 2019).
        \item Facts are silent on whether negotiations occurred or failed.
        \item If negotiations took place, jurisdiction may be established (per \textit{Marshall Islands} case for dispute crystallisation).
        \item Absence of negotiation details may challenge jurisdiction, as negotiation failure is fundamental.
    \end{itemize}
\end{itemize}

\subsection{Part B: Substantive Claims and Jurisdictional Considerations}
\begin{itemize}
    \item \textbf{Violation of Treaty Obligations}:
    \begin{itemize}
        \item Locrian (L) argues a violation of Article 1 of the TCGT by Maris (M) due to failure in conservation and sustainable use of tuna.
        \item Last two lines of paragraph 3 of the facts suggest a breach of Article 1, enabling jurisdiction under Article 10.
    \end{itemize}
    \item \textbf{Scope of Pollution}:
    \begin{itemize}
        \item Determine whether pollution caused by M falls within the TCGT's scope.
        \item Pollution's impact on tuna conservation is central to the claim.
    \end{itemize}
    \item \textbf{Maris's Counterargument}:
    \begin{itemize}
        \item M may raise defences against L's claims, potentially arguing the pollution's irrelevance to the treaty or lack of direct causation.
    \end{itemize}
    \item \textbf{Compulsory Jurisdiction under Article 36(2)}:
    \begin{itemize}
        \item Both L and M are parties to the ICJ Statute and have accepted compulsory jurisdiction under Article 36(2).
        \item L can bring the case to the ICJ under this provision.
    \end{itemize}
    \item \textbf{Reservations and Declarations}:
    \begin{itemize}
        \item Analyse overlapping declarations of L and M for jurisdictional limits.
        \item L's declaration includes reservations that may restrict ICJ jurisdiction.
        \item Reservation (b) imposes a time limit, potentially barring L's claim if the dispute falls outside the temporal scope.
        \item M has no reservations, fully submitting to ICJ jurisdiction.
        \item L's declaration can be invoked by M (reciprocity principle), limiting jurisdiction based on L's own terms.
        \item If negotiations have not occurred, L may argue this precludes jurisdiction, further limiting the ICJ's role.
    \end{itemize}
\end{itemize}

\subsection{Part C: Alternative Jurisdictional Pathways and Third-Party Issues}
\begin{itemize}
    \item \textbf{Inapplicability of Article 10}:
    \begin{itemize}
        \item Article 10 of the TCGT does not apply, affecting claims under Articles 1 and 2.
        \item Jurisdiction must rely on other grounds, such as Article 36(2).
    \end{itemize}
    \item \textbf{Article 36(2) Application}:
    \begin{itemize}
        \item Dispute's crystallisation in July 2024 falls outside L's reservation, enabling jurisdiction under Article 36(2).
        \item L's reservations may support their argument here, unlike in Part B.
    \end{itemize}
    \item \textbf{Aeolian as an Indispensable Third Party}:
    \begin{itemize}
        \item Aeolian's absence is critical, as paragraph 2 raises a sovereignty claim involving them.
        \item Aeolian must be a party to the dispute for the ICJ to proceed (\textit{Monetary Gold} principle).
        \item Without Aeolian's consent, the case cannot be decided due to the \textit{Monetary Gold} principle (absence of an essential third party).
        \item Reference \textit{Portugal v. Australia}, where Indonesia's absence as an indispensable party barred adjudication.
    \end{itemize}
\end{itemize}

\subsection{Provisional Measures}
\begin{itemize}
    \item \textbf{Legal Basis and Precedents}:
    \begin{itemize}
        \item Provisional measures governed by Article 41 of the ICJ Statute.
        \item Measures are legally binding (\textit{LaGrand}, Germany v. US).
        \item \textit{India v. Pakistan} case: ICJ stayed execution (urgent measure) but declined release (substantive issue, not urgent).
    \end{itemize}
    \item \textbf{Necessity for Part B}:
    \begin{itemize}
        \item Irreparable harm to the tuna population due to climate degradation justifies provisional measures in Part B.
        \item Urgency exists due to the immediate threat to L's fishing interests.
        \item Part A and Part C lack comparable urgency or irreparable harm, precluding provisional measures.
        \item Only Part B satisfies conditions for provisional measures due to severe environmental impact.
    \end{itemize}
\end{itemize}