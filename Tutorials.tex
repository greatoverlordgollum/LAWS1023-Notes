There was no tutorial problem for Topic 1, and the tutorial problem for Topic 2 consisted of source evaluation only.
\setcounter{section}{2}

\section{Treaties}
\begin{tutorialquestion}
    \flushleft
    Astra, Benthos and Ceres are neighbouring states on Torrent Island. Astra is located  in the eastern and mountainous part of Torrent Island and is highly industrialised. Benthos andCeres are developing states located on the western and low-lying part of Torrent Island.

    \vspace{\baselineskip}

    The River Delta flows from Astra into Benthos and then on to Ceres. In 1977 Astra, Benthos and Ceres concluded the River Delta Treaty (`RDT'). Article 1 of the RDT commits the parties to `cooperate to achieve the reasonable and equitable use of the River Delta'. In Article 10 of the RDT, Benthos and Ceres are guaranteed defined minimum volumes of water per annum. On signing the RDT, Astra lodged the following declaration:

    \begin{quote}
        `The Government of the Astra Republic, in approving the Treaty, declares that reference to the concept of reasonable and equitable use of transboundary waters does not constitute recognition of a principle of customary law, but illustrates a general principle of cooperation between Parties to the Treaty.'
    \end{quote}

    In 2024, due to an ongoing drought exacerbated by climate change, Astra announced that it was unable to supply any water to Benthos. Benthos in turn was unable to comply with its obligations under Article 10 to supply a required volume of water to Ceres.

    \vspace{\baselineskip}

    In January 2025, Astra announced that the RDT was `hereby terminated with immediate effect'. In doing so Astra referred to the persistent drought conditions, and the record of negotiations for the RDT during which former President of Benthos commented that `of course whatever the treaty says must be subject to the vagaries of nature and we may have to put the treaty on hold if the river dries up.' The RDT includes no provision relating to suspension or termination.

    \vspace{\baselineskip}

    Benthos and Ceres contend that the RDT remains in force and that Astra is also bound by a customary law obligation to provide reasonable and equitable access to a shared freshwater resource.

    \vspace{\baselineskip}

    Astra, Benthos and Ceres are parties to the 1997 United Nations Convention on the Law of the Non-Navigational uses of International Watercourses (the UN Watercourses Convention) which provides, in Article 3, that `In the absence of an agreement to the contrary, nothing in the present Convention shall affect the rights or obligations of a watercourse State arising from agreements in force for it on the date on which it became a party to the present Convention.'

    \vspace{\baselineskip}

    Astra, Benthos and Ceres are parties to the 1969 Vienna Convention on the Law of Treaties.

    \vspace{\baselineskip}

    At the urging of the UN Secretary-General, the three States have agreed to mediation to resolve their dispute. You have been asked to prepare a legal brief advising the mediator on the legal issues that arise under the law of treaties from these facts.
\end{tutorialquestion}
\begin{itemize}
    \item What does each country want?
    \begin{itemize}
        \item Astra: to terminate the treaty
        \item Benthos: to remain in force
        \item Ceres: to remain in force
    \end{itemize}
    \item Are all of countries party to the VCLT?
    \begin{itemize}
        \item Here, they are all party to the VCLT
        \item If a country is not, many of the VCLT provisions can apply as customary law regardless (Table \ref{tab:VCLT Articles that can apply as customary international law} on Page \pageref{tab:VCLT Articles that can apply as customary international law})
        \item \textit{In this instance}, as the treaty was ratified in 1977, the VCLT would not apply as it was ratified in 1980 and does not have retroactive application (its provisions are being applied as customary international law)
    \end{itemize}
\end{itemize}


\section{International Law and Australian Law}
\begin{tutorialquestion}
    \flushleft

    The United Nations Declaration on the Rights of Indigenous Peoples (`the Declaration') was adopted by the United Nations General Assembly on 13 September 2007, by a majority of 143 states in favour, 4 votes against (Australia, Canada, New Zealand and the United States) and 11 abstentions. The four states that voted against subsequently declared their support for the Declaration (including Australia in 2007, following the election of the Rudd Government).
    
    \vspace{\baselineskip}

    The Declaration mentions Treaties between Indigenous Peoples and States in its Preamble and in Article 37. The latter provides:
    \begin{enumerate}
        \item Indigenous peoples have the right to the recognition, observance and enforcement of treaties, agreements and other constructive arrangements concluded with States or their successors and to have States honour and respect such treaties, agreements and other constructive arrangements.
        \item Nothing in this Declaration may be interpreted as diminishing or eliminating the rights of indigenous peoples contained in treaties, agreements and other constructive arrangements.
    \end{enumerate}

    You are a legal advisor in the independent Treaty Authority established in Victoria to oversee negotiations between the Victorian Government and the First Peoples’ Assembly of Victoria to ensure a fair process in the conclusion of a Treaty that delivers self-determination for Victoria’s First Peoples. Victoria was the first State to commit to all three elements of the Uluru Statement from the Heart (Voice, Treaty and Truth). The relevance of the Declaration to the Treaty-making process has been considered in detail by Dr Harry Hobbs. Statewide Treaty negotiations began in Victoria in November 2024.

    \vspace{\baselineskip}

    You have been asked to provide legal advice addressing the following questions:

    \begin{enumerate}
        \item What is the status of the Declaration under international law?
        \item What is the status of the Declaration under Australian law?
        \item In \case{\textit{Love v Commonwealth} [2020] HCA 3}, Bell J (at [73]) cited the Declaration when making the following observation:
        \begin{quote}
            ``It is not offensive, in the context of contemporary international understanding, to recognise the cultural and spiritual dimensions of the distinctive connection between indigenous peoples and their traditional lands, and in light of that recognition to hold that the exercise of the sovereign power of this nation does not extend to the exclusion of the indigenous inhabitants from the Australian community.''
        \end{quote}

        What legal effect was Bell J ascribing to the Declaration under Aboriginal law?
        \item In what respects does the Advancing the Treaty Process with Aboriginal Victorians Act 2018 (Vic) implement the Declaration in Victorian law?
        \item What status will a Treaty or Treaties between Indigenous Victorians and the State of Victoria have as a matter of Victorian, Australian and international law?
    \end{enumerate}
\end{tutorialquestion}

\subsection*{Question 1}
\begin{itemize}
    \item UNGA Resolutions are not binding, but they can be evidence of state practice and \gls{opinio juris} for customary law (and may possibly be a piece of soft law)
    \item This declaration is not a treaty, as there is no objective evidence that the parties sought to be bound nor that they made any decisions
    \item To determine whether it is a source of law, \statute{Art 38(1)} must be looked at:
    \begin{itemize}
        \item It does not fit in with any of the sources mentioned in \statute{Art 38(1)}
        \item It might be possible to argue that it is customary international law under \statute{Art 38(1)(b)}; nonetheless, it would be hard to make out \gls{opinio juris} on the facts
        \item It is still possible that there is state practice present, as there are 143 states voting in favour of the declaration; moreover, the reversal in position by the states that voted against it, which all had high levels of Indigenous populations, indicates a strengthening of support for the declaration and hence may evidence state practice
        \item Moreover, there is some form of international consensus, but there is no specific consensus/practice present
    \end{itemize}
    \item Some aspects of UN Declarations have attained customary international law status (\case{\textit{Horta v Commonwealth} (1994) 181 CLR 183} (Page \pageref{case:Horta v Commonwealth})), but it is unlikely that this is the case here
    \item Given the declaration has not been signed and ratified, it is fundamentally not directly binding on states
    \item Moreover, the UN sets out legal principles; the consensual, non-binding status of its decisions makes it especially active to bring states together
    \item Declarations are a serious matter; since states do not wish to get on the wrong side of the international community, they are an easy way of getting them all on the same page
    \item Any detailed analysis of this point must begin with why the declaration is non-binding, but as soft law, it nonetheless evidences state practice and \gls{opinio juris}
    \begin{itemize}
        \item A link to \case{\textit{Horta v Commonwealth} (1994) 181 CLR 183} (Page \pageref{case:Horta v Commonwealth}) is then required to evidence how UN principles may become customary international law
        \item A connection to self-determination must then be made to explore why states continue to nonetheless accept these declarations
    \end{itemize}
\end{itemize}

\subsection*{Question 2}
\begin{itemize}
    \item There is no Act directly implementing the Declaration, but there is statute, such as the \statute{\textit{Native Title Act 1993}}; moreover, aspects of the declaration had already been implemented into Australian law (which had occurred before the signing of this declaration)
    \item As there is no specific Act in Australia that implements this declaration, it is not binding on Australia's domestic law (since Australia takes a hard transformation approach, it is only binding if the legislature enacts legislation that aligns with the declaration)
    \begin{itemize}
        \item For full marks, discuss the theories and process that Australia adopts (i.e., monism vs dualism, and the resultant transformation approach)
        \item It is possible for customary international law to be a source of domestic law, or otherwise influence Australian law (\case{\textit{Mabo}})
        \item Under these cases, a universally recognised principle of international law will be applied, but does not explicitly form a part of Australian law
    \end{itemize}
    \item The declaration is quite possibly a source of non-binding soft law
    \item Under \statute{\textit{Constitution} s 51}, international legislation can be implemented into domestic legislation
    \item In \case{\textit{Chow Hung Ching v R} (1949) 77 CLR 449} (Page \pageref{case:Chow Hung Ching}), Latham CJ held that a universally recognised principle could be applied by the courts
    \begin{itemize}
        \item This approach was followed in \case{\textit{Mabo}}, where it was held that international law is an influence on development; if it is customary, it would take effect, but not be directly applicable
    \end{itemize}
    \item The \textit{Polites} principle is of relevance here: in the absence of express words to the contrary, it is assumed that domestic legislation is intended to conform with international law and should be taken as such
    \item A treaty will only be ratified when legislation with respect to it has been passed in the Commonwealth parliament
    \item Customary international law can be applied in Australian Courts (\case{\textit{Chow Hung Ching v R} (1949) 77 CLR 449} (Page \pageref{case:Chow Hung Ching})), but when there is a conflict between domestic and international law, domestic law will prevail
\end{itemize}

\subsection*{Question 3}
\begin{itemize}
    \item Bell J referred to the Declaration's legal effect, and for the capacity for it to be used as a source of international law
    \item This is a demonstration of contemporary understanding of indigenous rights, which can be a source of evidence for the current global understanding around this area
    \item \case{\textit{Al-Kateb v Godwin} (2004) 219 CLR 562} (Page \pageref{case:Al-Kateb v Godwin}) holds that a direct application of international law to interpret/modify the \textit{Constitution} would contradict \statute{\textit{Constitution} s 128}, which holds that it can only be changed by Australia; thus, any external influence cannot be detrimental to this, nor can it be used to assist in the interpretation of Australia
    \begin{itemize}
        \item Bell J was using the declaration here to inform her understanding about whether Indigenous people have a connection to their land (thus, it was not a source of law, but rather a supplementary material to help understand the context)
        \item In this minority opinion in \case{\textit{Al-Kateb v Godwin} (2004) 219 CLR 562}, Kirby J stated that the law should be pragmatic and evolve with the times (``the complete isolation of the constitution from international law is not possible/desirable''; ``national courts and constitutional courts have a duty to interpret constitutional texts with respect to international sources'')
    \end{itemize}
    \item Moreover, the exploration of this treaty does not have any legal effect; only the legislature is able to enact treaties/declarations into binding law
    \item The main issue in \case{\textit{Love v Commonwealth} [2020] HCA 3} was whether indigenous peoples could be deported as aliens
    \begin{itemize}
        \item The common law recognises native title, and hence they cannot
        \item The use of the Declaration was purely for interpretive purposes
        \item The use of extrinsic material is permissible, but has no legal effect
    \end{itemize}
    \item Furthermore, the approach taken here was noe one of soft transformation
\end{itemize}

\subsection*{Question 4}
\begin{itemize}
    \item The Victorian parliament can implement the Declaration into Victoria law using its plenary power
    \item The Declaration was not used in the Act itself, but references were instead placed in the preamble to ensure its relevance, thereby making it highly influential
    \item The terminology of Art 37 of the declaration is also used to connect the Act to the Declaration (moreover, there are similarities in the Act and the Declaration, giving it substantial adoption but not making it binding)
    \item The Act does not implement the Declaration, but instead provides the Victorian government a framework to be consistent with the Declaration; it still does not have any binding effect (in any case, it would be the Federal Government who would enact the Declaration, as Federal Parliament has the external affairs power, per \statute{\textit{Constitution} s 51(xxix)})
    \item The Act is just a framework and shows good faith on the part of the Victorian government, rather than anything of important or of relevant substance
    \item The treaty upholds the representative process, and recognises the right to self-determination
    \item It does not directly implement the declaration in Victorian law, but continues to have a strong influence
\end{itemize}

\subsection*{Question 5}
\begin{itemize}
    \item The treaty will be binding in Victoria unless there is federal law that conflicts with it, or the commonwealth legislature legislates against it (\case{\textit{Commonwealth v Tasmania} (1983) 185 CLR 1}) (Page \pageref{case:Commonwealth v Tasmania})
    \item However, this is not binding on any other territories in Australia, following \statute{\textit{Constitution} s 109}
    \item Since the Commonwealth is not a party to this treaty, it is not binding upon the Commonwealth
    \item Moreover, this treaty is not binding internationally, as the state of Victoria (being a state of a State) and Indigenous Peoples both are not peoples of international law
    \begin{itemize}
        \item Under \statute{\textit{VCLT} Art 2(1)(a)}, only states have the power to enter into a treaty, as only they can consent to its principles
        \item There is no evidence of state practice, as Victoria is once again not an international state
    \end{itemize}
\end{itemize}