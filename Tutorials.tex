There was no tutorial problem for Topic 1, and the tutorial problem for Topic 2 consisted of source evaluation only.
\setcounter{section}{2}

\section{Treaties}
\begin{tutorialquestion}
    \flushleft
    Astra, Benthos and Ceres are neighbouring states on Torrent Island. Astra is located  in the eastern and mountainous part of Torrent Island and is highly industrialised. Benthos andCeres are developing states located on the western and low-lying part of Torrent Island.

    \vspace{\baselineskip}

    The River Delta flows from Astra into Benthos and then on to Ceres. In 1977 Astra, Benthos and Ceres concluded the River Delta Treaty (`RDT'). Article 1 of the RDT commits the parties to `cooperate to achieve the reasonable and equitable use of the River Delta'. In Article 10 of the RDT, Benthos and Ceres are guaranteed defined minimum volumes of water per annum. On signing the RDT, Astra lodged the following declaration:

    \begin{quote}
        `The Government of the Astra Republic, in approving the Treaty, declares that reference to the concept of reasonable and equitable use of transboundary waters does not constitute recognition of a principle of customary law, but illustrates a general principle of cooperation between Parties to the Treaty.'
    \end{quote}

    In 2024, due to an ongoing drought exacerbated by climate change, Astra announced that it was unable to supply any water to Benthos. Benthos in turn was unable to comply with its obligations under Article 10 to supply a required volume of water to Ceres.

    \vspace{\baselineskip}

    In January 2025, Astra announced that the RDT was `‘'hereby terminated with immediate effect'. In doing so Astra referred to the persistent drought conditions, and the record of negotiations for the RDT during which former President of Benthos commented that `of course whatever the treaty says must be subject to the vagaries of nature and we may have to put the treaty on hold if the river dries up.' The RDT includes no provision relating to suspension or termination.

    \vspace{\baselineskip}

    Benthos and Ceres contend that the RDT remains in force and that Astra is also bound by a customary law obligation to provide reasonable and equitable access to a shared freshwater resource.

    \vspace{\baselineskip}

    Astra, Benthos and Ceres are parties to the 1997 United Nations Convention on the Law of the Non-Navigational uses of International Watercourses (the UN Watercourses Convention) which provides, in Article 3, that `In the absence of an agreement to the contrary, nothing in the present Convention shall affect the rights or obligations of a watercourse State arising from agreements in force for it on the date on which it became a party to the present Convention.'

    \vspace{\baselineskip}

    Astra, Benthos and Ceres are parties to the 1969 Vienna Convention on the Law of Treaties.

    \vspace{\baselineskip}

    At the urging of the UN Secretary-General, the three States have agreed to mediation to resolve their dispute. You have been asked to prepare a legal brief advising the mediator on the legal issues that arise under the law of treaties from these facts.

\end{tutorialquestion}

\section{International Law and Australian Law}
\begin{tutorialquestion}
    \flushleft

    The United Nations Declaration on the Rights of Indigenous Peoples (`the Declaration') was adopted by the United Nations General Assembly on 13 September 2007, by a majority of 143 states in favour, 4 votes against (Australia, Canada, New Zealand and the United States) and 11 abstentions. The four states that voted against subsequently declared their support for the Declaration (including Australia in 2007, following the election of the Rudd Government).
    
    \vspace{\baselineskip}

    The Declaration mentions Treaties between Indigenous Peoples and States in its Preamble and in Article 37. The latter provides:
    \begin{enumerate}
        \item Indigenous peoples have the right to the recognition, observance and enforcement of treaties, agreements and other constructive arrangements concluded with States or their successors and to have States honour and respect such treaties, agreements and other constructive arrangements.
        \item Nothing in this Declaration may be interpreted as diminishing or eliminating the rights of indigenous peoples contained in treaties, agreements and other constructive arrangements.
    \end{enumerate}

    You are a legal advisor in the independent Treaty Authority established in Victoria to oversee negotiations between the Victorian Government and the First Peoples’ Assembly of Victoria to ensure a fair process in the conclusion of a Treaty that delivers self-determination for Victoria’s First Peoples. Victoria was the first State to commit to all three elements of the Uluru Statement from the Heart (Voice, Treaty and Truth). The relevance of the Declaration to the Treaty-making process has been considered in detail by Dr Harry Hobbs. Statewide Treaty negotiations began in Victoria in November 2024.

    \vspace{\baselineskip}

    You have been asked to provide legal advice addressing the following questions:

    \begin{enumerate}
        \item What is the status of the Declaration under international law?
        \item What is the status of the Declaration under Australian law?
        \item In \case{\textit{Love v Commonwealth} [2020] HCA 3}, Bell J (at [73]) cited the Declaration when making the following observation:
        \begin{quote}
            ``It is not offensive, in the context of contemporary international understanding, to recognise the cultural and spiritual dimensions of the distinctive connection between indigenous peoples and their traditional lands, and in light of that recognition to hold that the exercise of the sovereign power of this nation does not extend to the exclusion of the indigenous inhabitants from the Australian community.''
        \end{quote}

        What legal effect was Bell J ascribing to the Declaration under Aboriginal law?
        \item In what respects does the Advancing the Treaty Process with Aboriginal Victorians Act 2018 (Vic) implement the Declaration in Victorian law?
        \item What status will a Treaty or Treaties between Indigenous Victorians and the State of Victoria have as a matter of Victorian, Australian and international law?
    \end{enumerate}
\end{tutorialquestion}

\subsection*{Question 1}
\subsection*{Question 2}
\subsection*{Question 3}
\subsection*{Question 4}
\subsection*{Question 5}