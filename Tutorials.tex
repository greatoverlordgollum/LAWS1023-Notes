There was no tutorial problem for Topic 1, and the tutorial problem for Topic 2 consisted of source evaluation only.
\setcounter{section}{2}

\section{Treaties}
\begin{tutorialquestion}
    \flushleft
    Astra, Benthos and Ceres are neighbouring states on Torrent Island. Astra is located  in the eastern and mountainous part of Torrent Island and is highly industrialised. Benthos andCeres are developing states located on the western and low-lying part of Torrent Island.

    \vspace{\baselineskip}

    The River Delta flows from Astra into Benthos and then on to Ceres. In 1977 Astra, Benthos and Ceres concluded the River Delta Treaty (`RDT'). Article 1 of the RDT commits the parties to `cooperate to achieve the reasonable and equitable use of the River Delta'. In Article 10 of the RDT, Benthos and Ceres are guaranteed defined minimum volumes of water per annum. On signing the RDT, Astra lodged the following declaration:

    \begin{quote}
        `The Government of the Astra Republic, in approving the Treaty, declares that reference to the concept of reasonable and equitable use of transboundary waters does not constitute recognition of a principle of customary law, but illustrates a general principle of cooperation between Parties to the Treaty.'
    \end{quote}

    In 2024, due to an ongoing drought exacerbated by climate change, Astra announced that it was unable to supply any water to Benthos. Benthos in turn was unable to comply with its obligations under Article 10 to supply a required volume of water to Ceres.

    \vspace{\baselineskip}

    In January 2025, Astra announced that the RDT was `hereby terminated with immediate effect'. In doing so Astra referred to the persistent drought conditions, and the record of negotiations for the RDT during which former President of Benthos commented that `of course whatever the treaty says must be subject to the vagaries of nature and we may have to put the treaty on hold if the river dries up.' The RDT includes no provision relating to suspension or termination.

    \vspace{\baselineskip}

    Benthos and Ceres contend that the RDT remains in force and that Astra is also bound by a customary law obligation to provide reasonable and equitable access to a shared freshwater resource.

    \vspace{\baselineskip}

    Astra, Benthos and Ceres are parties to the 1997 United Nations Convention on the Law of the Non-Navigational uses of International Watercourses (the UN Watercourses Convention) which provides, in Article 3, that `In the absence of an agreement to the contrary, nothing in the present Convention shall affect the rights or obligations of a watercourse State arising from agreements in force for it on the date on which it became a party to the present Convention.'

    \vspace{\baselineskip}

    Astra, Benthos and Ceres are parties to the 1969 Vienna Convention on the Law of Treaties.

    \vspace{\baselineskip}

    At the urging of the UN Secretary-General, the three States have agreed to mediation to resolve their dispute. You have been asked to prepare a legal brief advising the mediator on the legal issues that arise under the law of treaties from these facts.
\end{tutorialquestion}
\begin{itemize}
    \item What does each country want?
    \begin{itemize}
        \item Astra: to terminate the treaty
        \item Benthos: to remain in force
        \item Ceres: to remain in force
    \end{itemize}
    \item Are all of countries party to the VCLT?
    \begin{itemize}
        \item Here, they are all party to the VCLT
        \item If a country is not, many of the VCLT provisions can apply as customary law regardless (Table \ref{tab:VCLT Articles that can apply as customary international law} on Page \pageref{tab:VCLT Articles that can apply as customary international law})
        \item \textit{In this instance}, as the treaty was ratified in 1977, the VCLT would not apply as it was ratified in 1980 and does not have retroactive application (its provisions are being applied as customary international law)
    \end{itemize}
    \item Can Astra terminate the treaty?
    \begin{itemize}
        \item The first point to start at is the treaty itself
        \begin{itemize}
            \item Since the treaty contains ``no provision relating to suspension or termination'', internal grounds are excluded, and we must turn to external grounds (VCLT Art 60-62)
            \item The purpose of the treaty is to cooperate to achieve a reasonable and equitable use of the River Delta (under Art 1)
            \begin{itemize}
                \item There is not a clear circumstance where there will be a breach in the first place
                \item There is most likely not a material breach here
            \end{itemize}
            \item Supervening impossibility is most likely the strongest argument present
            \begin{itemize}
                \item The river has dried up, and the drought could possibly continue forever
                \item Droughts tend towards climate change -- even if it ends in 100 years, it is still a supervening event
                \item Refer to VCLT Art 61
                \item This needs to be a permanent disruption - the drought could be cyclical or temporary
                \item Must radically transform the obligations to an extent where the obligation itself is completely changed                
            \end{itemize}
        \end{itemize}
        \item Was there a fundamental change of circumstances?
        \begin{itemize}
            \item Statement by the President of Benthos
            \begin{itemize}
                \item Drought was not foreseen, but there was the possibility planned for in this circumstance - doesn't go for a supervening impossibility
                \item This could be evidence of preparatory work
                \item Was this a reservation or not?
                \item Look at the definition of a reservation in the VCLT - likely that this is a declaration and not a reservation, as it was not made to exclude or modify the legal effect of that provision (thereby, they're not bound)
            \end{itemize}
            \item Not foreseen by parties
            \begin{itemize}
                \item Benthos - they could've possibly foreseen it as “if the river dries up”
                \item Art 26 - pacta serva sunda - should've looked for alternative measures or made changes to account for this provision (put this in the conclusion! Go through termination/suspension/etc, then say art 26 could apply and suggest what else could be done)
                \item On the facts - not completely foreseen by the parties                
            \end{itemize}
            \item Radical change in obligations
            \begin{itemize}
                \item Could still give less water
                \item Drought is a dire situation, but isn't an actual disaster - the countries can bounce back
                \item Countries are not unaware that droughts can happen - climate change causes a lot of unforeseen events (we're in 2024/2025, so it's reasonable to make that assumption)                
            \end{itemize}
            \item A fundamental change of circumstances is very difficult threshold to reach in the first place (\case{\textit{Anglo Norwegian Fisheries Case (UK v Norway)} [1951] ICJ Rep 116} on Page \pageref{case:UK v Norway Fisheries}) threat of economic instability isn't considered to be a fundamental change
            \begin{itemize}
                \item To pass this threshold, the agreement in question must have disappeared (e.g., if a state sank away, the treaty would be radically transformed)
            \end{itemize}
        \end{itemize}
        \item Art 26 is about good faith - need to look for other alternatives - the parties should've looked at reasonable alternative (Astra's keeping all resources to themselves, even though they're highly industrialised and can produce more (e.g., desalination/mountain water))
        \item Art 26 is used - when a party can't perform obligations fully, they should still try to fulfil them to their best extent, following \textit{pacta sunt servanda} (e.g., if astra tried to give water/gave water, this would be met and art 26 isn't made out)
        \item If the treaty is terminated, are they still bound by customary law?
        \begin{itemize}
            \item Art 1 is very vague, but art 10 is very specific (both can't be general practice)
            \item Here, it isn't fundamentally non-creating, but state practice and opinio juris is still conductional
            \item Might be some form of state practice under the 1997 convention - compliance over a long period of time
            \item Opinio juris - not much on the facts, but that A and B are attempting to enforce suggests some form of uniformity/state practice            
        \end{itemize}
        \item If the text being examined is a treaty per the definition in the VCLT, one sentence is needed to acknowledge this
    \end{itemize}
    \item Consider geographic and socioeconomic inequality
    \item Astra is a developed state, and so they have more responsibilities and obligations when it comes to giving water
    \item In this treaty, Astra is not guaranteed a minimum amount of water
    \item The object and purpose for this treaty is defined in paragraph 1
    \item 1 - socioeconomic status, 2 - who has the obligations (A and B, to B and C respectively), 3 - object and purpose of the treaty
    \item To interpret the treaty, use VCLT (look at the ordinary meaning in terms of the context of the treaty) (Art 31(1) - good faith; and Art 32)
    \item Also include the interpretative assistant provided by the UN's Watercourses convention
    \item Astra's declaration is an interpretative declaration, as it doesn't modify the text of the treaty but instead clarifies Astra's position
    \begin{itemize}
        \item This could be a reservation (Art 2(1)(d)), but it is not as it does not change the terms of the treaty, but merely is interpreting the law as they are understanding it (whether it applies or not depends - is this considered `persistent objector'? - the later signing of the 1997 treaty interferes any issue with objections, and so not a persistent objector)
        \item Astra could show this is a persistent objection to this becoming customary international law        
    \end{itemize}
    \item Lack of water supply shows both A and B have breached the treaty - not providing water = breach of Article 10 (which shows that they have to give minimum amounts of water downstream)
    \item Astra had an attempt to terminate - was this valid?
    \begin{itemize}
        \item RDT is silent, so look to the VCLT
        \item Options - material breach, impossibility, change of circumstance (art 60-62 respectively)
        \item Astra can't rely on its own breach to terminate - art 60
        \item Look at benthos - art 10 was not a material breach as the object and purpose of the treaty had been breached by not supplying minimum water quantity
        \item Material breach is no water, can't argue drought, as the fact don't show that there is no water, but only that there is a drought (you can still supply water, Art 1 talks about cooperation, reasonable and equitable)
        \item They could terminate if there was no water
        \item A and B both breached Art 1, as they didn't fulfil their supply obligations to the next country
        \item Drought means that there is no rain, not that the river has dried up
    \end{itemize}
    \item ``Ongoing drought'' - in the future, there could be a fundamental change of circumstances (and hence no material breach)
    \item Was the declaration of Benthos a unilateral declaration?
    \begin{itemize}
        \item Here, it says dried up vs drought (not the same), so not a unilateral declaration
        \item Unilateral declaration - if its against the treaty, it's just a statement
        \item B also uses vague terminology in their declaration, so not a unilateral declaration
        \item Head of state says something - unilateral declaration - can be an acceptance of the law by a state        
    \end{itemize}
    \item There are no provisions of suspension and termination
    \begin{itemize}
        \item Astra will argue these under Art 61 and 62 (impossibility and fundamental breach)
        \item Can argue this as the treaty doesn't have a provision on this - see VCLT 56(1) - if a treaty doesn't have these clauses, then they may withdraw, but that withdrawal cannot be on an understanding other than art 61/62        
    \end{itemize}
    \item B/C contends that the treaty will remain on force and that all parties are bound by customary international law
    \begin{itemize}
        \item Was there a customary obligation?
        \begin{itemize}
            \item See Art 38(1) of ICJ statute - requires state practice and opinio juris
            \item UN watercourse had 103 in favour, 3 against and 27 abstentions
            \item UN watercourse is in force, has 40 parties, and 16 states sign at launch, so more states joining over time indicates a growing acceptance
            \item Similarities in goals between the un convention and the RDT - almost the same wording in regards to reasonable and equitable water use - trend in law making by states - UN watercourses has general language that has the qualities of law making treaties (sets out how to interpret), indicating a wider phenomena and acceptance
            \item Art 1 of this treaty and art 5 of UN is similar
            \item 106 states and 40 parties showing that there is a developing customary law (state practice and opinio juris - north sea continental shelf (anywhere talking about customary international law and opinio juris, put this case!))
            \item Which one of the two will prevail?
            \begin{itemize}
                \item In absence of an agreement to the contrary
                \item Due to art 3 of the UN treaty, RDT remains valid as it came before and there was no agreement to subsume/supersede it
                \item Art 3 doesn't say that all other agreements are repealed (on the contrary, it says that it will coexist and continue in peace)
                \item RDT and watercourse convention both continue to exist, and hence there is no conflict between the two
                \item Art 3 - when they join, it will not influence any previous agreements that were in force when the states entered into the treaty
                \item If in any place a state is a party (they have ratified, making sure it has become law), or states who have signed the agreement (they have not ratified) - key distinction as to whether an agreement is applicable or not and the obligations that apply to those states
                \item If people have signed, its not binding, but the signing shows evidence of state practice (good for customary international law)                
            \end{itemize}       
        \end{itemize}
    \end{itemize}
    \item To answer the whole question:
    \begin{itemize}
        \item Look at the whole question
        \item Determine the obligations and responsibilities of each state
        \item Is there a declaration/reservation/attempt to be a persistent objector? (e.g., could the Presidential statement be an example of persistent objection?)
        \item Was suspension/termination allowed or not? (here, no internal grounds, and so turn to external grounds in the VCLT)
        \item Discuss what could be argued by Astra under VCLT Art 60-62
        \item Compare the convention to the RDT, analysing the similarities and differences (how is it similar to customary law?)
        \item With one line, note that the VCLT does not apply as a treaty, but rather its provisions do as a matter of customary law
    \end{itemize}
\end{itemize}

\section{International Law and Australian Law}
\begin{tutorialquestion}
    \flushleft

    The United Nations Declaration on the Rights of Indigenous Peoples (`the Declaration') was adopted by the United Nations General Assembly on 13 September 2007, by a majority of 143 states in favour, 4 votes against (Australia, Canada, New Zealand and the United States) and 11 abstentions. The four states that voted against subsequently declared their support for the Declaration (including Australia in 2007, following the election of the Rudd Government).
    
    \vspace{\baselineskip}

    The Declaration mentions Treaties between Indigenous Peoples and States in its Preamble and in Article 37. The latter provides:
    \begin{enumerate}
        \item Indigenous peoples have the right to the recognition, observance and enforcement of treaties, agreements and other constructive arrangements concluded with States or their successors and to have States honour and respect such treaties, agreements and other constructive arrangements.
        \item Nothing in this Declaration may be interpreted as diminishing or eliminating the rights of indigenous peoples contained in treaties, agreements and other constructive arrangements.
    \end{enumerate}

    You are a legal advisor in the independent Treaty Authority established in Victoria to oversee negotiations between the Victorian Government and the First Peoples' Assembly of Victoria to ensure a fair process in the conclusion of a Treaty that delivers self-determination for Victoria's First Peoples. Victoria was the first State to commit to all three elements of the Uluru Statement from the Heart (Voice, Treaty and Truth). The relevance of the Declaration to the Treaty-making process has been considered in detail by Dr Harry Hobbs. Statewide Treaty negotiations began in Victoria in November 2024.

    \vspace{\baselineskip}

    You have been asked to provide legal advice addressing the following questions:

    \begin{enumerate}
        \item What is the status of the Declaration under international law?
        \item What is the status of the Declaration under Australian law?
        \item In \case{\textit{Love v Commonwealth} [2020] HCA 3}, Bell J (at [73]) cited the Declaration when making the following observation:
        \begin{quote}
            ``It is not offensive, in the context of contemporary international understanding, to recognise the cultural and spiritual dimensions of the distinctive connection between indigenous peoples and their traditional lands, and in light of that recognition to hold that the exercise of the sovereign power of this nation does not extend to the exclusion of the indigenous inhabitants from the Australian community.''
        \end{quote}

        What legal effect was Bell J ascribing to the Declaration under Aboriginal law?
        \item In what respects does the Advancing the Treaty Process with Aboriginal Victorians Act 2018 (Vic) implement the Declaration in Victorian law?
        \item What status will a Treaty or Treaties between Indigenous Victorians and the State of Victoria have as a matter of Victorian, Australian and international law?
    \end{enumerate}
\end{tutorialquestion}

\subsection*{Question 1}
\begin{itemize}
    \item UNGA Resolutions are not binding, but they can be evidence of state practice and \gls{opinio juris} for customary law (and may possibly be a piece of soft law)
    \item This declaration is not a treaty, as there is no objective evidence that the parties sought to be bound nor that they made any decisions
    \item To determine whether it is a source of law, \statute{Art 38(1)} must be looked at:
    \begin{itemize}
        \item It does not fit in with any of the sources mentioned in \statute{Art 38(1)}
        \item It might be possible to argue that it is customary international law under \statute{Art 38(1)(b)}; nonetheless, it would be hard to make out \gls{opinio juris} on the facts
        \item It is still possible that there is state practice present, as there are 143 states voting in favour of the declaration; moreover, the reversal in position by the states that voted against it, which all had high levels of Indigenous populations, indicates a strengthening of support for the declaration and hence may evidence state practice
        \item Moreover, there is some form of international consensus, but there is no specific consensus/practice present
    \end{itemize}
    \item Some aspects of UN Declarations have attained customary international law status (\case{\textit{Horta v Commonwealth} (1994) 181 CLR 183} (Page \pageref{case:Horta v Commonwealth})), but it is unlikely that this is the case here
    \item Given the declaration has not been signed and ratified, it is fundamentally not directly binding on states
    \item Moreover, the UN sets out legal principles; the consensual, non-binding status of its decisions makes it especially active to bring states together
    \item Declarations are a serious matter; since states do not wish to get on the wrong side of the international community, they are an easy way of getting them all on the same page
    \item Any detailed analysis of this point must begin with why the declaration is non-binding, but as soft law, it nonetheless evidences state practice and \gls{opinio juris}
    \begin{itemize}
        \item A link to \case{\textit{Horta v Commonwealth} (1994) 181 CLR 183} (Page \pageref{case:Horta v Commonwealth}) is then required to evidence how UN principles may become customary international law
        \item A connection to self-determination must then be made to explore why states continue to nonetheless accept these declarations
    \end{itemize}
\end{itemize}

\subsection*{Question 2}
\begin{itemize}
    \item There is no Act directly implementing the Declaration, but there is statute, such as the \statute{\textit{Native Title Act 1993}}; moreover, aspects of the declaration had already been implemented into Australian law (which had occurred before the signing of this declaration)
    \item As there is no specific Act in Australia that implements this declaration, it is not binding on Australia's domestic law (since Australia takes a hard transformation approach, it is only binding if the legislature enacts legislation that aligns with the declaration)
    \begin{itemize}
        \item For full marks, discuss the theories and process that Australia adopts (i.e., monism vs dualism, and the resultant transformation approach)
        \item It is possible for customary international law to be a source of domestic law, or otherwise influence Australian law (\case{\textit{Mabo}})
        \item Under these cases, a universally recognised principle of international law will be applied, but does not explicitly form a part of Australian law
    \end{itemize}
    \item The declaration is quite possibly a source of non-binding soft law
    \item Under \statute{\textit{Constitution} s 51}, international legislation can be implemented into domestic legislation
    \item In \case{\textit{Chow Hung Ching v R} (1949) 77 CLR 449} (Page \pageref{case:Chow Hung Ching}), Latham CJ held that a universally recognised principle could be applied by the courts
    \begin{itemize}
        \item This approach was followed in \case{\textit{Mabo}}, where it was held that international law is an influence on development; if it is customary, it would take effect, but not be directly applicable
    \end{itemize}
    \item The \textit{Polites} principle is of relevance here: in the absence of express words to the contrary, it is assumed that domestic legislation is intended to conform with international law and should be taken as such
    \item A treaty will only be ratified when legislation with respect to it has been passed in the Commonwealth parliament
    \item Customary international law can be applied in Australian Courts (\case{\textit{Chow Hung Ching v R} (1949) 77 CLR 449} (Page \pageref{case:Chow Hung Ching})), but when there is a conflict between domestic and international law, domestic law will prevail
\end{itemize}

\subsection*{Question 3}
\begin{itemize}
    \item Bell J referred to the Declaration's legal effect, and for the capacity for it to be used as a source of international law
    \item This is a demonstration of contemporary understanding of indigenous rights, which can be a source of evidence for the current global understanding around this area
    \item \case{\textit{Al-Kateb v Godwin} (2004) 219 CLR 562} (Page \pageref{case:Al-Kateb v Godwin}) holds that a direct application of international law to interpret/modify the \textit{Constitution} would contradict \statute{\textit{Constitution} s 128}, which holds that it can only be changed by Australia; thus, any external influence cannot be detrimental to this, nor can it be used to assist in the interpretation of Australia
    \begin{itemize}
        \item Bell J was using the declaration here to inform her understanding about whether Indigenous people have a connection to their land (thus, it was not a source of law, but rather a supplementary material to help understand the context)
        \item In this minority opinion in \case{\textit{Al-Kateb v Godwin} (2004) 219 CLR 562}, Kirby J stated that the law should be pragmatic and evolve with the times (``the complete isolation of the constitution from international law is not possible/desirable''; ``national courts and constitutional courts have a duty to interpret constitutional texts with respect to international sources'')
    \end{itemize}
    \item Moreover, the exploration of this treaty does not have any legal effect; only the legislature is able to enact treaties/declarations into binding law
    \item The main issue in \case{\textit{Love v Commonwealth} [2020] HCA 3} was whether indigenous peoples could be deported as aliens
    \begin{itemize}
        \item The common law recognises native title, and hence they cannot
        \item The use of the Declaration was purely for interpretive purposes
        \item The use of extrinsic material is permissible, but has no legal effect
    \end{itemize}
    \item Furthermore, the approach taken here was noe one of soft transformation
\end{itemize}

\subsection*{Question 4}
\begin{itemize}
    \item The Victorian parliament can implement the Declaration into Victoria law using its plenary power
    \item The Declaration was not used in the Act itself, but references were instead placed in the preamble to ensure its relevance, thereby making it highly influential
    \item The terminology of Art 37 of the declaration is also used to connect the Act to the Declaration (moreover, there are similarities in the Act and the Declaration, giving it substantial adoption but not making it binding)
    \item The Act does not implement the Declaration, but instead provides the Victorian government a framework to be consistent with the Declaration; it still does not have any binding effect (in any case, it would be the Federal Government who would enact the Declaration, as Federal Parliament has the external affairs power, per \statute{\textit{Constitution} s 51(xxix)})
    \item The Act is just a framework and shows good faith on the part of the Victorian government, rather than anything of important or of relevant substance
    \item The treaty upholds the representative process, and recognises the right to self-determination
    \item It does not directly implement the declaration in Victorian law, but continues to have a strong influence
\end{itemize}

\subsection*{Question 5}
\begin{itemize}
    \item The treaty will be binding in Victoria unless there is federal law that conflicts with it, or the commonwealth legislature legislates against it (\case{\textit{Commonwealth v Tasmania} (1983) 185 CLR 1}) (Page \pageref{case:Commonwealth v Tasmania})
    \item However, this is not binding on any other territories in Australia, following \statute{\textit{Constitution} s 109}
    \item Since the Commonwealth is not a party to this treaty, it is not binding upon the Commonwealth
    \item Moreover, this treaty is not binding internationally, as the state of Victoria (being a state of a State) and Indigenous Peoples both are not peoples of international law
    \begin{itemize}
        \item Under \statute{\textit{VCLT} Art 2(1)(a)}, only states have the power to enter into a treaty, as only they can consent to its principles
        \item There is no evidence of state practice, as Victoria is once again not an international state
    \end{itemize}
\end{itemize}

\section{Personality, Statehood and Self-Determination}
\begin{tutorialquestion}
    \flushleft
    
    Aquisita is a large state that acquired a number of colonies, including Bulgan, a mountainous
    region, in the early twentieth century. Bulgan is home to three nomadic tribes that ranged
    across the territory's expansive alpine plains during the summer months and then based
    themselves in three loosely defined areas known as Chinggis, Delkhii and Erenhot. All three
    nomadic groups are ethnically and racially similar and speak a common language (although
    each group also speaks a distinct dialogue unique to their group).

    \vspace{\baselineskip}

    In 1963, the Aquisita Parliament passed a law which defined the borders of Delkhii and
    established the territory as a `Special Delkhii Administrative Zone'. The Aquisita government
    described this measure as necessary to maintain public order given past histories of conflict
    between the Delkhii and other tribes. There was no consultation with the nomadic tribes in
    Bulgan prior to this law being passed. In 1978, Bulgan was granted independence by Aquisita
    and became a member of the United Nations, however the status of the `Special Delkhii
    Administrative Zone' was unchanged, and ethnic Delkhii peoples were excluded from holding
    public office or serving in the new Bulgan government.

    \vspace{\baselineskip}

    Over the past five years the nomadic people of Delkhii have been engaged in an armed
    uprising against the Bulgan administration in Delkhii, and in early 2025 were successful in
    seizing control of Delkhii's largest city and taking control of Bulgan government offices there.
    While the group had control of the city, Bulgan forces remained in command of several more
    remote parts of the Special Delkhii Administrative Zone. The Delkhii have received very
    extensive financial and other support from Imperia, another state in the region, over several
    years. Imperia has been accused by Asquisita and the Bulgan government of `controlling the
    Delkhii for its own purposes.'

    \vspace{\baselineskip}

    In a public broadcast from the Delkhii Television Centre in March 2025, the leaders of the
    Delkhii declared that the territory was now the Independent State of Delkhii (ISD), that new
    boundaries for the state would be announced in coming months, and that action will be taken
    against `unfriendly' governments that have not supported the right of the Delkhii to self-
    determination. Several governments, including Imperia, have recognised ISD, however others
    have not.

    \vspace{\baselineskip}

    The ISD wishes to apply for membership of the United Nations and seeks your advice on the
    main issues of international law arising from these facts.
    
\end{tutorialquestion}

\section{Title to Territory}
\begin{tutorialquestion}
    \flushleft
    In their original state, the Celestials were a cluster of five small reefs (submerged at high tide) and two rocks (above water at high tide) all within 100 nautical miles (M) of Arema's coast. 
    
    \vspace{\baselineskip}
    
    The Celestials are subject to competing claims by Arema and Binton. Arema asserts sovereignty over the Celestials on the basis that they are proximate to Arema's coast, were first discovered by a private explorer from Arema in the 1800s (who left a small brass plaque on one of the rocks), and were regularly visited until the 1950s by the Arema navy. As a result of a civil war in 1960 in Arema these naval visits ceased. Arema did not establish a permanent physical presence on Celestials, apart from constructing a small lighthouse on one of the rocks.
    
    \vspace{\baselineskip}

    From 1965, Binton regularly deployed naval vessels to the Celestials and the brass plaque and lighthouse were removed. Binton publicly claimed the Celestials in 1970, on the ground that they had been abandoned by Arema. Binton then began an extensive program of land reclamation which has transformed the small features into one island. Binton has constructed several military installations on the Celestials and stationed a garrison of troops there. Arema has repeatedly protested the statements and actions of Binton, although in 1980 the Foreign Minister of Arema did make a speech in which he stated that `the time has come to recognise that Arema no longer has any interest in the remote and inhospitable Celestials.'
    
    \vspace{\baselineskip}

    In January 2025, Binton asserted a 50 nautical mile `economic and security zone' around the Celestials, and the Parliament of Binton passed a law making it a criminal offence for `any foreign national to be present in the economic and security zone of Celestials unless in possession of a valid permit issued by the Binton Department of Homeland Integrity.'
    
    \vspace{\baselineskip}

    Assess the competing claims of Arema and Binton and the legality of Binton's assertion of sovereignty and jurisdiction over the Celestials.

\end{tutorialquestion}